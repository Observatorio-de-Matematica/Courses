\documentclass{article}

%% Created with wxMaxima 16.04.2

\setlength{\parskip}{\medskipamount}
\setlength{\parindent}{0pt}
\usepackage[utf8]{inputenc}
\DeclareUnicodeCharacter{00B5}{\ensuremath{\mu}}
\usepackage{graphicx}
\usepackage{color}
\usepackage{amsmath}
\usepackage{ifthen}
\newsavebox{\picturebox}
\newlength{\pictureboxwidth}
\newlength{\pictureboxheight}
\newcommand{\includeimage}[1]{
    \savebox{\picturebox}{\includegraphics{#1}}
    \settoheight{\pictureboxheight}{\usebox{\picturebox}}
    \settowidth{\pictureboxwidth}{\usebox{\picturebox}}
    \ifthenelse{\lengthtest{\pictureboxwidth > .95\linewidth}}
    {
        \includegraphics[width=.95\linewidth,height=.80\textheight,keepaspectratio]{#1}
    }
    {
        \ifthenelse{\lengthtest{\pictureboxheight>.80\textheight}}
        {
            \includegraphics[width=.95\linewidth,height=.80\textheight,keepaspectratio]{#1}
            
        }
        {
            \includegraphics{#1}
        }
    }
}
\newlength{\thislabelwidth}
\DeclareMathOperator{\abs}{abs}
\usepackage{animate} % This package is required because the wxMaxima configuration option
                      % "Export animations to TeX" was enabled when this file was generated.

\definecolor{labelcolor}{RGB}{100,0,0}

\usepackage{fullpage}
\usepackage{amssymb}
\usepackage{enumerate}
\usepackage[bookmarks=false,pdfstartview={FitH},colorlinks=true,urlcolor=blue]{hyperref}
\usepackage{bookmark}
\usepackage{mathtools}

\begin{document}

\pagebreak{}
{\Huge {\sc Chapter 2 curves and frames}}
\setcounter{section}{0}
\setcounter{subsection}{0}
\setcounter{figure}{0}


\hypersetup{pdfauthor={Daniel Volinski},
            pdftitle={Differential Geometry},
            pdfsubject={Differential Geometry},
            pdfkeywords={James Cook}}

\begin{verbatim}
Lecture Notes for Differential Geometry
James S. Cook
Liberty University
Department of Mathematics
Summer 2015
\end{verbatim}

Written by Daniel Volinski at \href{mailto:danielvolinski@yahoo.es}{danielvolinski@yahoo.es}



\noindent
%%%%%%%%%%%%%%%
%%% INPUT:
\begin{minipage}[t]{8ex}\color{red}\bf
(\%{}i2) 
\end{minipage}
\begin{minipage}[t]{\textwidth}\color{blue}\tt
info:build\_info()\$info\ensuremath{@}version;
\end{minipage}
%%% OUTPUT:
\[\displaystyle
\tag{\%{}o2}\label{o2} 
\mbox{}
\]5.38.1



\noindent
%%%%%%%%%%%%%%%
%%% INPUT:
\begin{minipage}[t]{8ex}\color{red}\bf
(\%{}i2) 
\end{minipage}
\begin{minipage}[t]{\textwidth}\color{blue}\tt
reset()\$kill(all)\$
\end{minipage}


\noindent
%%%%%%%%%%%%%%%
%%% INPUT:
\begin{minipage}[t]{8ex}\color{red}\bf
(\%{}i1) 
\end{minipage}
\begin{minipage}[t]{\textwidth}\color{blue}\tt
derivabbrev:true\$
\end{minipage}


\noindent
%%%%%%%%%%%%%%%
%%% INPUT:
\begin{minipage}[t]{8ex}\color{red}\bf
(\%{}i2) 
\end{minipage}
\begin{minipage}[t]{\textwidth}\color{blue}\tt
ratprint:false\$
\end{minipage}


\noindent
%%%%%%%%%%%%%%%
%%% INPUT:
\begin{minipage}[t]{8ex}\color{red}\bf
(\%{}i3) 
\end{minipage}
\begin{minipage}[t]{\textwidth}\color{blue}\tt
fpprintprec:5\$
\end{minipage}


\noindent
%%%%%%%%%%%%%%%
%%% INPUT:
\begin{minipage}[t]{8ex}\color{red}\bf
(\%{}i4) 
\end{minipage}
\begin{minipage}[t]{\textwidth}\color{blue}\tt
load(linearalgebra)\$
\end{minipage}


\noindent
%%%%%%%%%%%%%%%
%%% INPUT:
\begin{minipage}[t]{8ex}\color{red}\bf
(\%{}i5) 
\end{minipage}
\begin{minipage}[t]{\textwidth}\color{blue}\tt
if get('draw,'version)=false then load(draw)\$
\end{minipage}
%%% OUTPUT:
%%%%%%%%%%%%%%%


\noindent
%%%%%%%%%%%%%%%
%%% INPUT:
\begin{minipage}[t]{8ex}\color{red}\bf
(\%{}i6) 
\end{minipage}
\begin{minipage}[t]{\textwidth}\color{blue}\tt
wxplot\_size:[1024,768]\$
\end{minipage}


\noindent
%%%%%%%%%%%%%%%
%%% INPUT:
\begin{minipage}[t]{8ex}\color{red}\bf
(\%{}i7) 
\end{minipage}
\begin{minipage}[t]{\textwidth}\color{blue}\tt
if get('drawdf,'version)=false then load(drawdf)\$
\end{minipage}


\noindent
%%%%%%%%%%%%%%%
%%% INPUT:
\begin{minipage}[t]{8ex}\color{red}\bf
(\%{}i8) 
\end{minipage}
\begin{minipage}[t]{\textwidth}\color{blue}\tt
set\_draw\_defaults(xtics=1,ytics=1,ztics=1,xyplane=0,nticks=100,\\
                  xaxis=true,xaxis\_type=solid,xaxis\_width=3,\\
                  yaxis=true,yaxis\_type=solid,yaxis\_width=3,\\
                  zaxis=true,zaxis\_type=solid,zaxis\_width=3,\\
                  background\_color=light\_gray)\$
\end{minipage}


\noindent
%%%%%%%%%%%%%%%
%%% INPUT:
\begin{minipage}[t]{8ex}\color{red}\bf
(\%{}i9) 
\end{minipage}
\begin{minipage}[t]{\textwidth}\color{blue}\tt
if get('vect,'version)=false then load(vect)\$
\end{minipage}


\noindent
%%%%%%%%%%%%%%%
%%% INPUT:
\begin{minipage}[t]{8ex}\color{red}\bf
(\%{}i10) 
\end{minipage}
\begin{minipage}[t]{\textwidth}\color{blue}\tt
norm(u):=block(ratsimp(radcan(\ensuremath{\sqrt{}}(u.u))))\$
\end{minipage}


\noindent
%%%%%%%%%%%%%%%
%%% INPUT:
\begin{minipage}[t]{8ex}\color{red}\bf
(\%{}i11) 
\end{minipage}
\begin{minipage}[t]{\textwidth}\color{blue}\tt
normalize(v):=block(v/norm(v))\$
\end{minipage}


\noindent
%%%%%%%%%%%%%%%
%%% INPUT:
\begin{minipage}[t]{8ex}\color{red}\bf
(\%{}i12) 
\end{minipage}
\begin{minipage}[t]{\textwidth}\color{blue}\tt
angle(u,v):=block([junk:radcan(\ensuremath{\sqrt{}}((u.u)*(v.v)))],acos(u.v/junk))\$
\end{minipage}


\noindent
%%%%%%%%%%%%%%%
%%% INPUT:
\begin{minipage}[t]{8ex}\color{red}\bf
(\%{}i13) 
\end{minipage}
\begin{minipage}[t]{\textwidth}\color{blue}\tt
mycross(va,vb):=[va[2]*vb[3]-va[3]*vb[2],va[3]*vb[1]-va[1]*vb[3],va[1]*vb[2]-va[2]*vb[1]]\$
\end{minipage}


\noindent
%%%%%%%%%%%%%%%
%%% INPUT:
\begin{minipage}[t]{8ex}\color{red}\bf
(\%{}i14) 
\end{minipage}
\begin{minipage}[t]{\textwidth}\color{blue}\tt
if get('cartan,'version)=false then load(cartan)\$
\end{minipage}


\noindent
%%%%%%%%%%%%%%%
%%% INPUT:
\begin{minipage}[t]{8ex}\color{red}\bf
(\%{}i15) 
\end{minipage}
\begin{minipage}[t]{\textwidth}\color{blue}\tt
declare(trigsimp,evfun)\$
\end{minipage}
\pagebreak


\section{on distance in three dimensions}


\section{vectors and frames in three dimensions}


\textbf{Levi-Civita symbol}



\noindent
%%%%%%%%%%%%%%%
%%% INPUT:
\begin{minipage}[t]{8ex}\color{red}\bf
(\%{}i16) 
\end{minipage}
\begin{minipage}[t]{\textwidth}\color{blue}\tt
\ensuremath{\epsilon}(i,j,k):=\ensuremath{\frac{1}{2}}*(i-j)*(j-k)*(k-i)\$
\end{minipage}


\noindent
%%%%%%%%%%%%%%%
%%% INPUT:
\begin{minipage}[t]{8ex}\color{red}\bf
(\%{}i18) 
\end{minipage}
\begin{minipage}[t]{\textwidth}\color{blue}\tt
v:[v\_1,v\_2,v\_3]\$w:[w\_1,w\_2,w\_3]\$
\end{minipage}


\noindent
%%%%%%%%%%%%%%%
%%% INPUT:
\begin{minipage}[t]{8ex}\color{red}\bf
(\%{}i19) 
\end{minipage}
\begin{minipage}[t]{\textwidth}\color{blue}\tt
mycross(v,w);
\end{minipage}
%%% OUTPUT:
\[\displaystyle
\tag{\%{}o19}\label{o19} 
[{{v}_{2}}{{w}_{3}}-{{v}_{3}}{{w}_{2}},{{v}_{3}}{{w}_{1}}-{{v}_{1}}{{w}_{3}},{{v}_{1}}{{w}_{2}}-{{v}_{2}}{{w}_{1}}]\mbox{}
\]
%%%%%%%%%%%%%%%


\noindent
%%%%%%%%%%%%%%%
%%% INPUT:
\begin{minipage}[t]{8ex}\color{red}\bf
(\%{}i20) 
\end{minipage}
\begin{minipage}[t]{\textwidth}\color{blue}\tt
makelist(sum(sum(\ensuremath{\epsilon}(i,j,k)*v[i]*w[j],i,1,3),j,1,3),k,1,3);
\end{minipage}
%%% OUTPUT:
\[\displaystyle
\tag{\%{}o20}\label{o20} 
[{{v}_{2}}{{w}_{3}}-{{v}_{3}}{{w}_{2}},{{v}_{3}}{{w}_{1}}-{{v}_{1}}{{w}_{3}},{{v}_{1}}{{w}_{2}}-{{v}_{2}}{{w}_{1}}]\mbox{}
\]
%%%%%%%%%%%%%%%


\noindent
%%%%%%%%%%%%%%%
%%% INPUT:
\begin{minipage}[t]{8ex}\color{red}\bf
(\%{}i21) 
\end{minipage}
\begin{minipage}[t]{\textwidth}\color{blue}\tt
is(\%=\%th(2));
\end{minipage}
%%% OUTPUT:
\[\displaystyle
\tag{\%{}o21}\label{o21} 
\mbox{true}\mbox{}
\]
%%%%%%%%%%%%%%%


\noindent
%%%%%%%%%%%%%%%
%%% INPUT:
\begin{minipage}[t]{8ex}\color{red}\bf
(\%{}i22) 
\end{minipage}
\begin{minipage}[t]{\textwidth}\color{blue}\tt
v.w;
\end{minipage}
%%% OUTPUT:
\[\displaystyle
\tag{\%{}o22}\label{o22} 
{{v}_{3}}{{w}_{3}}+{{v}_{2}}{{w}_{2}}+{{v}_{1}}{{w}_{1}}\mbox{}
\]
%%%%%%%%%%%%%%%


\noindent
%%%%%%%%%%%%%%%
%%% INPUT:
\begin{minipage}[t]{8ex}\color{red}\bf
(\%{}i23) 
\end{minipage}
\begin{minipage}[t]{\textwidth}\color{blue}\tt
sum(v[i]*w[i],i,1,3);
\end{minipage}
%%% OUTPUT:
\[\displaystyle
\tag{\%{}o23}\label{o23} 
{{v}_{3}}{{w}_{3}}+{{v}_{2}}{{w}_{2}}+{{v}_{1}}{{w}_{1}}\mbox{}
\]
%%%%%%%%%%%%%%%


\noindent
%%%%%%%%%%%%%%%
%%% INPUT:
\begin{minipage}[t]{8ex}\color{red}\bf
(\%{}i24) 
\end{minipage}
\begin{minipage}[t]{\textwidth}\color{blue}\tt
is(\%=\%th(2));
\end{minipage}
%%% OUTPUT:
\[\displaystyle
\tag{\%{}o24}\label{o24} 
\mbox{true}\mbox{}
\]
%%%%%%%%%%%%%%%


\noindent
%%%%%%%%%%%%%%%
%%% INPUT:
\begin{minipage}[t]{8ex}\color{red}\bf
(\%{}i25) 
\end{minipage}
\begin{minipage}[t]{\textwidth}\color{blue}\tt
expand(norm(mycross(v,w))\ensuremath{^2});
\end{minipage}
%%% OUTPUT:
\[\displaystyle
\tag{\%{}o25}\label{o25} 
{{{{v}_{2}}}^{2}}{{{{w}_{3}}}^{2}}+{{{{v}_{1}}}^{2}}{{{{w}_{3}}}^{2}}-2{{v}_{2}}{{v}_{3}}{{w}_{2}}{{w}_{3}}-2{{v}_{1}}{{v}_{3}}{{w}_{1}}{{w}_{3}}+{{{{v}_{3}}}^{2}}{{{{w}_{2}}}^{2}}+{{{{v}_{1}}}^{2}}{{{{w}_{2}}}^{2}}-2{{v}_{1}}{{v}_{2}}{{w}_{1}}{{w}_{2}}+{{{{v}_{3}}}^{2}}{{{{w}_{1}}}^{2}}+{{{{v}_{2}}}^{2}}{{{{w}_{1}}}^{2}}\mbox{}
\]
%%%%%%%%%%%%%%%


\noindent
%%%%%%%%%%%%%%%
%%% INPUT:
\begin{minipage}[t]{8ex}\color{red}\bf
(\%{}i26) 
\end{minipage}
\begin{minipage}[t]{\textwidth}\color{blue}\tt
expand((v.v)*(w.w)-(v.w)\ensuremath{^2});
\end{minipage}
%%% OUTPUT:
\[\displaystyle
\tag{\%{}o26}\label{o26} 
{{{{v}_{2}}}^{2}}{{{{w}_{3}}}^{2}}+{{{{v}_{1}}}^{2}}{{{{w}_{3}}}^{2}}-2{{v}_{2}}{{v}_{3}}{{w}_{2}}{{w}_{3}}-2{{v}_{1}}{{v}_{3}}{{w}_{1}}{{w}_{3}}+{{{{v}_{3}}}^{2}}{{{{w}_{2}}}^{2}}+{{{{v}_{1}}}^{2}}{{{{w}_{2}}}^{2}}-2{{v}_{1}}{{v}_{2}}{{w}_{1}}{{w}_{2}}+{{{{v}_{3}}}^{2}}{{{{w}_{1}}}^{2}}+{{{{v}_{2}}}^{2}}{{{{w}_{1}}}^{2}}\mbox{}
\]
%%%%%%%%%%%%%%%


\noindent
%%%%%%%%%%%%%%%
%%% INPUT:
\begin{minipage}[t]{8ex}\color{red}\bf
(\%{}i27) 
\end{minipage}
\begin{minipage}[t]{\textwidth}\color{blue}\tt
is(\%=\%th(2));
\end{minipage}
%%% OUTPUT:
\[\displaystyle
\tag{\%{}o27}\label{o27} 
\mbox{true}\mbox{}
\]
%%%%%%%%%%%%%%%

\subsection{Example 2.2.7.}


Let $p\in\mathbb{R}^3$ then $E_1$, $E_2$, $E_3$ given below form a
frame at $p$ $$E_1=\dfrac{1}{\sqrt{3}}\left({\left.\dfrac{\partial}
{\partial x}\right\rvert_p+\left.\dfrac{\partial}{\partial y}\right
\rvert_p+\left.\dfrac{\partial}{\partial z}\right\rvert_p}\right)
\quad,\quad E_2=\dfrac{1}{\sqrt{2}}\left({\left.\dfrac{\partial}
{\partial x}\right\rvert_p-\left.\dfrac{\partial}{\partial z}\right
\rvert_p}\right)\quad,\quad E_3=\dfrac{1}{\sqrt{6}}\left({\left.
\dfrac{\partial}{\partial x}\right\rvert_p-2\left.\dfrac{\partial}
{\partial y}\right\rvert_p+\left.\dfrac{\partial}{\partial z}\right
\rvert_p}\right)$$



\noindent
%%%%%%%%%%%%%%%
%%% INPUT:
\begin{minipage}[t]{8ex}\color{red}\bf
(\%{}i28) 
\end{minipage}
\begin{minipage}[t]{\textwidth}\color{blue}\tt
ldisplay(E\_1:1/\ensuremath{\sqrt{}}(3)*[1,1,1])\$
\end{minipage}
%%% OUTPUT:
\[\displaystyle
\tag{\%{}t28}\label{t28} 
{{E}_{1}}=\left[\frac{1}{\sqrt{3}},\frac{1}{\sqrt{3}},\frac{1}{\sqrt{3}}\right]\mbox{}
\]
%%%%%%%%%%%%%%%


\noindent
%%%%%%%%%%%%%%%
%%% INPUT:
\begin{minipage}[t]{8ex}\color{red}\bf
(\%{}i29) 
\end{minipage}
\begin{minipage}[t]{\textwidth}\color{blue}\tt
ldisplay(E\_2:1/\ensuremath{\sqrt{}}(2)*[1,0,-1])\$
\end{minipage}
%%% OUTPUT:
\[\displaystyle
\tag{\%{}t29}\label{t29} 
{{E}_{2}}=\left[\frac{1}{\sqrt{2}},0,-\frac{1}{\sqrt{2}}\right]\mbox{}
\]
%%%%%%%%%%%%%%%


\noindent
%%%%%%%%%%%%%%%
%%% INPUT:
\begin{minipage}[t]{8ex}\color{red}\bf
(\%{}i30) 
\end{minipage}
\begin{minipage}[t]{\textwidth}\color{blue}\tt
ldisplay(E\_3:1/\ensuremath{\sqrt{}}(6)*[1,-2,1])\$
\end{minipage}
%%% OUTPUT:
\[\displaystyle
\tag{\%{}t30}\label{t30} 
{{E}_{3}}=\left[\frac{1}{\sqrt{6}},-\frac{2}{\sqrt{6}},\frac{1}{\sqrt{6}}\right]\mbox{}
\]
%%%%%%%%%%%%%%%


\noindent
%%%%%%%%%%%%%%%
%%% INPUT:
\begin{minipage}[t]{8ex}\color{red}\bf
(\%{}i31) 
\end{minipage}
\begin{minipage}[t]{\textwidth}\color{blue}\tt
rootscontract(mycross(E\_1,E\_2));
\end{minipage}
%%% OUTPUT:
\[\displaystyle
\tag{\%{}o31}\label{o31} 
\left[-\frac{1}{\sqrt{6}},\frac{\sqrt{2}}{\sqrt{3}},-\frac{1}{\sqrt{6}}\right]\mbox{}
\]
%%%%%%%%%%%%%%%

\subsection{Example 2.2.8.}


Observe $\lbrace{\partial_x,\partial_y,\partial_z}\rbrace$ forms the
\textbf{Cartesian coordinate frame} on $\mathbb{R}^3$. We sometimes
denote this frame by the standard notation $\lbrace{U_1,U_2,U_3}
\rbrace$. It is often useful to express a given frame in terms of the
\textbf{Euclidean frame}. For example, the frame of the preceding
example is written as:



\noindent
%%%%%%%%%%%%%%%
%%% INPUT:
\begin{minipage}[t]{8ex}\color{red}\bf
(\%{}i32) 
\end{minipage}
\begin{minipage}[t]{\textwidth}\color{blue}\tt
U:[U\_1,U\_2,U\_3]\$
\end{minipage}


\noindent
%%%%%%%%%%%%%%%
%%% INPUT:
\begin{minipage}[t]{8ex}\color{red}\bf
(\%{}i33) 
\end{minipage}
\begin{minipage}[t]{\textwidth}\color{blue}\tt
ldisplay(E\_1:1/\ensuremath{\sqrt{}}(3)*(U\_1+U\_2+U\_3))\$
\end{minipage}
%%% OUTPUT:
\[\displaystyle
\tag{\%{}t33}\label{t33} 
{{E}_{1}}=\frac{{{U}_{3}}+{{U}_{2}}+{{U}_{1}}}{\sqrt{3}}\mbox{}
\]
%%%%%%%%%%%%%%%


\noindent
%%%%%%%%%%%%%%%
%%% INPUT:
\begin{minipage}[t]{8ex}\color{red}\bf
(\%{}i34) 
\end{minipage}
\begin{minipage}[t]{\textwidth}\color{blue}\tt
ldisplay(E\_2:1/\ensuremath{\sqrt{}}(2)*(U\_1-U\_3))\$
\end{minipage}
%%% OUTPUT:
\[\displaystyle
\tag{\%{}t34}\label{t34} 
{{E}_{2}}=\frac{{{U}_{1}}-{{U}_{3}}}{\sqrt{2}}\mbox{}
\]
%%%%%%%%%%%%%%%


\noindent
%%%%%%%%%%%%%%%
%%% INPUT:
\begin{minipage}[t]{8ex}\color{red}\bf
(\%{}i35) 
\end{minipage}
\begin{minipage}[t]{\textwidth}\color{blue}\tt
ldisplay(E\_3:1/\ensuremath{\sqrt{}}(6)*(U\_1+2*U\_2+U\_3))\$
\end{minipage}
%%% OUTPUT:
\[\displaystyle
\tag{\%{}t35}\label{t35} 
{{E}_{3}}=\frac{{{U}_{3}}+2{{U}_{2}}+{{U}_{1}}}{\sqrt{6}}\mbox{}
\]
%%%%%%%%%%%%%%%

\subsection{Example 2.2.9.}


The \textbf{cylindrical coordinate frame} is given below:\\
$E_1=\cos\theta\,U_1+\sin\theta\,U_2$\\
$E_2=-\sin\theta\,U_1+\cos\theta\,U_2$\\
$E_3=U_3$



\noindent
%%%%%%%%%%%%%%%
%%% INPUT:
\begin{minipage}[t]{8ex}\color{red}\bf
(\%{}i36) 
\end{minipage}
\begin{minipage}[t]{\textwidth}\color{blue}\tt
ldisplay(E\_1:cos(\ensuremath{\theta})*U\_1+sin(\ensuremath{\theta})*U\_2)\$
\end{minipage}
%%% OUTPUT:
\[\displaystyle
\tag{\%{}t36}\label{t36} 
{{E}_{1}}={{U}_{2}}\sin{\left( \mathit{\ensuremath{\theta}}\right) }+{{U}_{1}}\cos{\left( \mathit{\ensuremath{\theta}}\right) }\mbox{}
\]
%%%%%%%%%%%%%%%


\noindent
%%%%%%%%%%%%%%%
%%% INPUT:
\begin{minipage}[t]{8ex}\color{red}\bf
(\%{}i37) 
\end{minipage}
\begin{minipage}[t]{\textwidth}\color{blue}\tt
ldisplay(E\_2:-sin(\ensuremath{\theta})*U\_1+cos(\ensuremath{\theta})*U\_2)\$
\end{minipage}
%%% OUTPUT:
\[\displaystyle
\tag{\%{}t37}\label{t37} 
{{E}_{2}}={{U}_{2}}\cos{\left( \mathit{\ensuremath{\theta}}\right) }-{{U}_{1}}\sin{\left( \mathit{\ensuremath{\theta}}\right) }\mbox{}
\]
%%%%%%%%%%%%%%%


\noindent
%%%%%%%%%%%%%%%
%%% INPUT:
\begin{minipage}[t]{8ex}\color{red}\bf
(\%{}i38) 
\end{minipage}
\begin{minipage}[t]{\textwidth}\color{blue}\tt
ldisplay(E\_3:U\_3)\$
\end{minipage}
%%% OUTPUT:
\[\displaystyle
\tag{\%{}t38}\label{t38} 
{{E}_{3}}={{U}_{3}}\mbox{}
\]
%%%%%%%%%%%%%%%

I often use the notation $E_1=\hat{r}$, $E_2=\hat{\theta}$ and
$E_3=\hat{z}$ in multivariate calculus. This frame is very useful
for simplifying calculations with cylindrical symmetry.


\subsection{Example 2.2.10.}


The \textbf{spherical coordinate frame} for the usual spherical
coordinates used in third-semester-American calculus is given below:



\noindent
%%%%%%%%%%%%%%%
%%% INPUT:
\begin{minipage}[t]{8ex}\color{red}\bf
(\%{}i39) 
\end{minipage}
\begin{minipage}[t]{\textwidth}\color{blue}\tt
ldisplay(E\_1:cos(\ensuremath{\theta})*sin(\ensuremath{\phi})*U\_1+sin(\ensuremath{\theta})*sin(\ensuremath{\phi})*U\_2+cos(\ensuremath{\phi})*U\_3)\$
\end{minipage}
%%% OUTPUT:
\[\displaystyle
\tag{\%{}t39}\label{t39} 
{{E}_{1}}={{U}_{2}}\sin{\left( \mathit{\ensuremath{\theta}}\right) }\,\sin{\left( \mathit{\ensuremath{\phi}}\right) }+{{U}_{1}}\cos{\left( \mathit{\ensuremath{\theta}}\right) }\,\sin{\left( \mathit{\ensuremath{\phi}}\right) }+{{U}_{3}}\cos{\left( \mathit{\ensuremath{\phi}}\right) }\mbox{}
\]
%%%%%%%%%%%%%%%


\noindent
%%%%%%%%%%%%%%%
%%% INPUT:
\begin{minipage}[t]{8ex}\color{red}\bf
(\%{}i40) 
\end{minipage}
\begin{minipage}[t]{\textwidth}\color{blue}\tt
ldisplay(E\_2:cos(\ensuremath{\theta})*cos(\ensuremath{\phi})*U\_1+sin(\ensuremath{\theta})*cos(\ensuremath{\phi})*U\_2-sin(\ensuremath{\phi})*U\_3)\$
\end{minipage}
%%% OUTPUT:
\[\displaystyle
\tag{\%{}t40}\label{t40} 
{{E}_{2}}=-{{U}_{3}}\sin{\left( \mathit{\ensuremath{\phi}}\right) }+{{U}_{2}}\sin{\left( \mathit{\ensuremath{\theta}}\right) }\,\cos{\left( \mathit{\ensuremath{\phi}}\right) }+{{U}_{1}}\cos{\left( \mathit{\ensuremath{\theta}}\right) }\,\cos{\left( \mathit{\ensuremath{\phi}}\right) }\mbox{}
\]
%%%%%%%%%%%%%%%


\noindent
%%%%%%%%%%%%%%%
%%% INPUT:
\begin{minipage}[t]{8ex}\color{red}\bf
(\%{}i41) 
\end{minipage}
\begin{minipage}[t]{\textwidth}\color{blue}\tt
ldisplay(E\_3:-sin(\ensuremath{\theta})*U\_1+cos(\ensuremath{\theta})*U\_2)\$
\end{minipage}
%%% OUTPUT:
\[\displaystyle
\tag{\%{}t41}\label{t41} 
{{E}_{3}}={{U}_{2}}\cos{\left( \mathit{\ensuremath{\theta}}\right) }-{{U}_{1}}\sin{\left( \mathit{\ensuremath{\theta}}\right) }\mbox{}
\]
%%%%%%%%%%%%%%%

I often use the notation $E_1=\hat{\rho}$, $E_2=\hat{\phi}$ and
$E_3=\hat{\theta}$ in multivariate calculus. This frame is very
useful for simplifying calculations with spherical symmetry.


I should warn the readers of O'neill, he uses a different choice of
\textbf{spherical coordinates} than we implicitly use in the example
above. In fact, the example is based on the formulas:



\noindent
%%%%%%%%%%%%%%%
%%% INPUT:
\begin{minipage}[t]{8ex}\color{red}\bf
(\%{}i42) 
\end{minipage}
\begin{minipage}[t]{\textwidth}\color{blue}\tt
\ensuremath{\xi}:[\ensuremath{\rho},\ensuremath{\phi},\ensuremath{\theta}]\$
\end{minipage}


\noindent
%%%%%%%%%%%%%%%
%%% INPUT:
\begin{minipage}[t]{8ex}\color{red}\bf
(\%{}i46) 
\end{minipage}
\begin{minipage}[t]{\textwidth}\color{blue}\tt
assume(0\ensuremath{\leq}\ensuremath{\rho})\$\\
assume(0\ensuremath{\leq}\ensuremath{\phi},\ensuremath{\phi}\ensuremath{\leq}\ensuremath{\pi})\$\\
assume(0\ensuremath{\leq}sin(\ensuremath{\phi}))\$\\
assume(0\ensuremath{\leq}\ensuremath{\theta},\ensuremath{\theta}\ensuremath{\leq}2*\ensuremath{\pi})\$
\end{minipage}


\noindent
%%%%%%%%%%%%%%%
%%% INPUT:
\begin{minipage}[t]{8ex}\color{red}\bf
(\%{}i47) 
\end{minipage}
\begin{minipage}[t]{\textwidth}\color{blue}\tt
ldisplay(Tr:[\ensuremath{\rho}*cos(\ensuremath{\theta})*sin(\ensuremath{\phi}),\ensuremath{\rho}*sin(\ensuremath{\theta})*sin(\ensuremath{\phi}),\ensuremath{\rho}*cos(\ensuremath{\phi})])\$
\end{minipage}
%%% OUTPUT:
\[\displaystyle
\tag{\%{}t47}\label{t47} 
\mathit{Tr}=[\cos{\left( \mathit{\ensuremath{\theta}}\right) }\mathit{\ensuremath{\rho}}\,\sin{\left( \mathit{\ensuremath{\phi}}\right) },\sin{\left( \mathit{\ensuremath{\theta}}\right) }\mathit{\ensuremath{\rho}}\,\sin{\left( \mathit{\ensuremath{\phi}}\right) },\mathit{\ensuremath{\rho}}\,\cos{\left( \mathit{\ensuremath{\phi}}\right) }]\mbox{}
\]
%%%%%%%%%%%%%%%


\noindent
%%%%%%%%%%%%%%%
%%% INPUT:
\begin{minipage}[t]{8ex}\color{red}\bf
(\%{}i48) 
\end{minipage}
\begin{minipage}[t]{\textwidth}\color{blue}\tt
scalefactors(append([Tr],\ensuremath{\xi}))\$
\end{minipage}


\noindent
%%%%%%%%%%%%%%%
%%% INPUT:
\begin{minipage}[t]{8ex}\color{red}\bf
(\%{}i49) 
\end{minipage}
\begin{minipage}[t]{\textwidth}\color{blue}\tt
sf;
\end{minipage}
%%% OUTPUT:
\[\displaystyle
\tag{\%{}o49}\label{o49} 
[1,\mathit{\ensuremath{\rho}},\mathit{\ensuremath{\rho}}\,\sin{\left( \mathit{\ensuremath{\phi}}\right) }]\mbox{}
\]
%%%%%%%%%%%%%%%


\noindent
%%%%%%%%%%%%%%%
%%% INPUT:
\begin{minipage}[t]{8ex}\color{red}\bf
(\%{}i50) 
\end{minipage}
\begin{minipage}[t]{\textwidth}\color{blue}\tt
sfprod;
\end{minipage}
%%% OUTPUT:
\[\displaystyle
\tag{\%{}o50}\label{o50} 
{{\mathit{\ensuremath{\rho}}}^{2}}\,\sin{\left( \mathit{\ensuremath{\phi}}\right) }\mbox{}
\]
%%%%%%%%%%%%%%%


\noindent
%%%%%%%%%%%%%%%
%%% INPUT:
\begin{minipage}[t]{8ex}\color{red}\bf
(\%{}i51) 
\end{minipage}
\begin{minipage}[t]{\textwidth}\color{blue}\tt
J:jacobian(Tr,\ensuremath{\xi});
\end{minipage}
%%% OUTPUT:
\[\displaystyle
\tag{J}\label{J}
\begin{pmatrix}\cos{\left( \mathit{\ensuremath{\theta}}\right) }\,\sin{\left( \mathit{\ensuremath{\phi}}\right) } & \cos{\left( \mathit{\ensuremath{\theta}}\right) }\mathit{\ensuremath{\rho}}\,\cos{\left( \mathit{\ensuremath{\phi}}\right) } & -\sin{\left( \mathit{\ensuremath{\theta}}\right) }\mathit{\ensuremath{\rho}}\,\sin{\left( \mathit{\ensuremath{\phi}}\right) }\\
\sin{\left( \mathit{\ensuremath{\theta}}\right) }\,\sin{\left( \mathit{\ensuremath{\phi}}\right) } & \sin{\left( \mathit{\ensuremath{\theta}}\right) }\mathit{\ensuremath{\rho}}\,\cos{\left( \mathit{\ensuremath{\phi}}\right) } & \cos{\left( \mathit{\ensuremath{\theta}}\right) }\mathit{\ensuremath{\rho}}\,\sin{\left( \mathit{\ensuremath{\phi}}\right) }\\
\cos{\left( \mathit{\ensuremath{\phi}}\right) } & -\mathit{\ensuremath{\rho}}\,\sin{\left( \mathit{\ensuremath{\phi}}\right) } & 0\end{pmatrix}\mbox{}
\]
%%%%%%%%%%%%%%%


\noindent
%%%%%%%%%%%%%%%
%%% INPUT:
\begin{minipage}[t]{8ex}\color{red}\bf
(\%{}i52) 
\end{minipage}
\begin{minipage}[t]{\textwidth}\color{blue}\tt
lg:trigsimp(transpose(J).J);
\end{minipage}
%%% OUTPUT:
\[\displaystyle
\tag{lg}\label{lg}
\begin{pmatrix}1 & 0 & 0\\
0 & {{\mathit{\ensuremath{\rho}}}^{2}} & 0\\
0 & 0 & {{\mathit{\ensuremath{\rho}}}^{2}}\,{{\sin{\left( \mathit{\ensuremath{\phi}}\right) }}^{2}}\end{pmatrix}\mbox{}
\]
%%%%%%%%%%%%%%%

These coordinates envision $\phi$ being zero on the positive $z$-axis
then sweeping down to $\pi$ on the negative $z$-axis. In contrast,
see Figure 2.20 on page 86, O'neill prefers to work with $\phi$ which
is zero on the $xy$-plane then sweeps up or down to $\pm\pi/2$.


\subsection{Definition 2.2.11.}


\textbf{Attitude matrix of a frame}



\noindent
%%%%%%%%%%%%%%%
%%% INPUT:
\begin{minipage}[t]{8ex}\color{red}\bf
(\%{}i53) 
\end{minipage}
\begin{minipage}[t]{\textwidth}\color{blue}\tt
A:matrix(\\
    makelist(coeff(E\_1,k),k,U),\\
    makelist(coeff(E\_2,k),k,U),\\
    makelist(coeff(E\_3,k),k,U));
\end{minipage}
%%% OUTPUT:
\[\displaystyle
\tag{A}\label{A}
\begin{pmatrix}\cos{\left( \mathit{\ensuremath{\theta}}\right) }\,\sin{\left( \mathit{\ensuremath{\phi}}\right) } & \sin{\left( \mathit{\ensuremath{\theta}}\right) }\,\sin{\left( \mathit{\ensuremath{\phi}}\right) } & \cos{\left( \mathit{\ensuremath{\phi}}\right) }\\
\cos{\left( \mathit{\ensuremath{\theta}}\right) }\,\cos{\left( \mathit{\ensuremath{\phi}}\right) } & \sin{\left( \mathit{\ensuremath{\theta}}\right) }\,\cos{\left( \mathit{\ensuremath{\phi}}\right) } & -\sin{\left( \mathit{\ensuremath{\phi}}\right) }\\
-\sin{\left( \mathit{\ensuremath{\theta}}\right) } & \cos{\left( \mathit{\ensuremath{\theta}}\right) } & 0\end{pmatrix}\mbox{}
\]
%%%%%%%%%%%%%%%


\noindent
%%%%%%%%%%%%%%%
%%% INPUT:
\begin{minipage}[t]{8ex}\color{red}\bf
(\%{}i54) 
\end{minipage}
\begin{minipage}[t]{\textwidth}\color{blue}\tt
trigsimp(transpose(A).A);
\end{minipage}
%%% OUTPUT:
\[\displaystyle
\tag{\%{}o54}\label{o54} 
\begin{pmatrix}1 & 0 & 0\\
0 & 1 & 0\\
0 & 0 & 1\end{pmatrix}\mbox{}
\]
%%%%%%%%%%%%%%%


\noindent
%%%%%%%%%%%%%%%
%%% INPUT:
\begin{minipage}[t]{8ex}\color{red}\bf
(\%{}i55) 
\end{minipage}
\begin{minipage}[t]{\textwidth}\color{blue}\tt
is(trigsimp(mycross(A[1],A[2]))=A[3]);
\end{minipage}
%%% OUTPUT:
\[\displaystyle
\tag{\%{}o55}\label{o55} 
\mbox{true}\mbox{}
\]
%%%%%%%%%%%%%%%

\subsection{Example 2.2.13.}


Following Example 2.2.7.



\noindent
%%%%%%%%%%%%%%%
%%% INPUT:
\begin{minipage}[t]{8ex}\color{red}\bf
(\%{}i56) 
\end{minipage}
\begin{minipage}[t]{\textwidth}\color{blue}\tt
ldisplay(E\_1:1/\ensuremath{\sqrt{}}(3)*[1,1,1])\$
\end{minipage}
%%% OUTPUT:
\[\displaystyle
\tag{\%{}t56}\label{t56} 
{{E}_{1}}=\left[\frac{1}{\sqrt{3}},\frac{1}{\sqrt{3}},\frac{1}{\sqrt{3}}\right]\mbox{}
\]
%%%%%%%%%%%%%%%


\noindent
%%%%%%%%%%%%%%%
%%% INPUT:
\begin{minipage}[t]{8ex}\color{red}\bf
(\%{}i57) 
\end{minipage}
\begin{minipage}[t]{\textwidth}\color{blue}\tt
ldisplay(E\_2:1/\ensuremath{\sqrt{}}(2)*[1,0,-1])\$
\end{minipage}
%%% OUTPUT:
\[\displaystyle
\tag{\%{}t57}\label{t57} 
{{E}_{2}}=\left[\frac{1}{\sqrt{2}},0,-\frac{1}{\sqrt{2}}\right]\mbox{}
\]
%%%%%%%%%%%%%%%


\noindent
%%%%%%%%%%%%%%%
%%% INPUT:
\begin{minipage}[t]{8ex}\color{red}\bf
(\%{}i58) 
\end{minipage}
\begin{minipage}[t]{\textwidth}\color{blue}\tt
ldisplay(E\_3:1/\ensuremath{\sqrt{}}(6)*[1,-2,1])\$
\end{minipage}
%%% OUTPUT:
\[\displaystyle
\tag{\%{}t58}\label{t58} 
{{E}_{3}}=\left[\frac{1}{\sqrt{6}},-\frac{2}{\sqrt{6}},\frac{1}{\sqrt{6}}\right]\mbox{}
\]
%%%%%%%%%%%%%%%


\noindent
%%%%%%%%%%%%%%%
%%% INPUT:
\begin{minipage}[t]{8ex}\color{red}\bf
(\%{}i59) 
\end{minipage}
\begin{minipage}[t]{\textwidth}\color{blue}\tt
A:matrix(E\_1,E\_2,E\_3);
\end{minipage}
%%% OUTPUT:
\[\displaystyle
\tag{A}\label{A}
\begin{pmatrix}\frac{1}{\sqrt{3}} & \frac{1}{\sqrt{3}} & \frac{1}{\sqrt{3}}\\
\frac{1}{\sqrt{2}} & 0 & -\frac{1}{\sqrt{2}}\\
\frac{1}{\sqrt{6}} & -\frac{2}{\sqrt{6}} & \frac{1}{\sqrt{6}}\end{pmatrix}\mbox{}
\]
%%%%%%%%%%%%%%%


\noindent
%%%%%%%%%%%%%%%
%%% INPUT:
\begin{minipage}[t]{8ex}\color{red}\bf
(\%{}i60) 
\end{minipage}
\begin{minipage}[t]{\textwidth}\color{blue}\tt
transpose(A).A;
\end{minipage}
%%% OUTPUT:
\[\displaystyle
\tag{\%{}o60}\label{o60} 
\begin{pmatrix}1 & 0 & 0\\
0 & 1 & 0\\
0 & 0 & 1\end{pmatrix}\mbox{}
\]
%%%%%%%%%%%%%%%

\subsection{Example 2.2.14.}


Following Example 2.2.8., the attitude of the \textbf{Cartesian frame}
is the identity matrix:



\noindent
%%%%%%%%%%%%%%%
%%% INPUT:
\begin{minipage}[t]{8ex}\color{red}\bf
(\%{}i61) 
\end{minipage}
\begin{minipage}[t]{\textwidth}\color{blue}\tt
ldisplay(E\_1:[1,0,0])\$
\end{minipage}
%%% OUTPUT:
\[\displaystyle
\tag{\%{}t61}\label{t61} 
{{E}_{1}}=[1,0,0]\mbox{}
\]
%%%%%%%%%%%%%%%


\noindent
%%%%%%%%%%%%%%%
%%% INPUT:
\begin{minipage}[t]{8ex}\color{red}\bf
(\%{}i62) 
\end{minipage}
\begin{minipage}[t]{\textwidth}\color{blue}\tt
ldisplay(E\_2:[0,1,0])\$
\end{minipage}
%%% OUTPUT:
\[\displaystyle
\tag{\%{}t62}\label{t62} 
{{E}_{2}}=[0,1,0]\mbox{}
\]
%%%%%%%%%%%%%%%


\noindent
%%%%%%%%%%%%%%%
%%% INPUT:
\begin{minipage}[t]{8ex}\color{red}\bf
(\%{}i63) 
\end{minipage}
\begin{minipage}[t]{\textwidth}\color{blue}\tt
ldisplay(E\_3:[0,0,1])\$
\end{minipage}
%%% OUTPUT:
\[\displaystyle
\tag{\%{}t63}\label{t63} 
{{E}_{3}}=[0,0,1]\mbox{}
\]
%%%%%%%%%%%%%%%


\noindent
%%%%%%%%%%%%%%%
%%% INPUT:
\begin{minipage}[t]{8ex}\color{red}\bf
(\%{}i64) 
\end{minipage}
\begin{minipage}[t]{\textwidth}\color{blue}\tt
is(mycross(E\_1,E\_2)=E\_3);
\end{minipage}
%%% OUTPUT:
\[\displaystyle
\tag{\%{}o64}\label{o64} 
\mbox{true}\mbox{}
\]
%%%%%%%%%%%%%%%


\noindent
%%%%%%%%%%%%%%%
%%% INPUT:
\begin{minipage}[t]{8ex}\color{red}\bf
(\%{}i65) 
\end{minipage}
\begin{minipage}[t]{\textwidth}\color{blue}\tt
A:matrix(E\_1,E\_2,E\_3);
\end{minipage}
%%% OUTPUT:
\[\displaystyle
\tag{A}\label{A}
\begin{pmatrix}1 & 0 & 0\\
0 & 1 & 0\\
0 & 0 & 1\end{pmatrix}\mbox{}
\]
%%%%%%%%%%%%%%%


\noindent
%%%%%%%%%%%%%%%
%%% INPUT:
\begin{minipage}[t]{8ex}\color{red}\bf
(\%{}i66) 
\end{minipage}
\begin{minipage}[t]{\textwidth}\color{blue}\tt
transpose(A).A;
\end{minipage}
%%% OUTPUT:
\[\displaystyle
\tag{\%{}o66}\label{o66} 
\begin{pmatrix}1 & 0 & 0\\
0 & 1 & 0\\
0 & 0 & 1\end{pmatrix}\mbox{}
\]
%%%%%%%%%%%%%%%


\noindent
%%%%%%%%%%%%%%%
%%% INPUT:
\begin{minipage}[t]{8ex}\color{red}\bf
(\%{}i67) 
\end{minipage}
\begin{minipage}[t]{\textwidth}\color{blue}\tt
is(mycross(A[1],A[2])=A[3]);
\end{minipage}
%%% OUTPUT:
\[\displaystyle
\tag{\%{}o67}\label{o67} 
\mbox{true}\mbox{}
\]
%%%%%%%%%%%%%%%

\subsection{Example 2.2.15.}


Following Example 2.2.9, the attitude of the \textbf{cylindrical
coordinate frame} is:



\noindent
%%%%%%%%%%%%%%%
%%% INPUT:
\begin{minipage}[t]{8ex}\color{red}\bf
(\%{}i68) 
\end{minipage}
\begin{minipage}[t]{\textwidth}\color{blue}\tt
ldisplay(E\_1:[cos(\ensuremath{\theta}),sin(\ensuremath{\theta}),0])\$
\end{minipage}
%%% OUTPUT:
\[\displaystyle
\tag{\%{}t68}\label{t68} 
{{E}_{1}}=[\cos{\left( \mathit{\ensuremath{\theta}}\right) },\sin{\left( \mathit{\ensuremath{\theta}}\right) },0]\mbox{}
\]
%%%%%%%%%%%%%%%


\noindent
%%%%%%%%%%%%%%%
%%% INPUT:
\begin{minipage}[t]{8ex}\color{red}\bf
(\%{}i69) 
\end{minipage}
\begin{minipage}[t]{\textwidth}\color{blue}\tt
ldisplay(E\_2:[-sin(\ensuremath{\theta}),cos(\ensuremath{\theta}),0])\$
\end{minipage}
%%% OUTPUT:
\[\displaystyle
\tag{\%{}t69}\label{t69} 
{{E}_{2}}=[-\sin{\left( \mathit{\ensuremath{\theta}}\right) },\cos{\left( \mathit{\ensuremath{\theta}}\right) },0]\mbox{}
\]
%%%%%%%%%%%%%%%


\noindent
%%%%%%%%%%%%%%%
%%% INPUT:
\begin{minipage}[t]{8ex}\color{red}\bf
(\%{}i70) 
\end{minipage}
\begin{minipage}[t]{\textwidth}\color{blue}\tt
ldisplay(E\_3:[0,0,1])\$
\end{minipage}
%%% OUTPUT:
\[\displaystyle
\tag{\%{}t70}\label{t70} 
{{E}_{3}}=[0,0,1]\mbox{}
\]
%%%%%%%%%%%%%%%


\noindent
%%%%%%%%%%%%%%%
%%% INPUT:
\begin{minipage}[t]{8ex}\color{red}\bf
(\%{}i71) 
\end{minipage}
\begin{minipage}[t]{\textwidth}\color{blue}\tt
is(trigsimp(mycross(E\_1,E\_2)=E\_3));
\end{minipage}
%%% OUTPUT:
\[\displaystyle
\tag{\%{}o71}\label{o71} 
\mbox{true}\mbox{}
\]
%%%%%%%%%%%%%%%


\noindent
%%%%%%%%%%%%%%%
%%% INPUT:
\begin{minipage}[t]{8ex}\color{red}\bf
(\%{}i72) 
\end{minipage}
\begin{minipage}[t]{\textwidth}\color{blue}\tt
A:matrix(E\_1,E\_2,E\_3);
\end{minipage}
%%% OUTPUT:
\[\displaystyle
\tag{A}\label{A}
\begin{pmatrix}\cos{\left( \mathit{\ensuremath{\theta}}\right) } & \sin{\left( \mathit{\ensuremath{\theta}}\right) } & 0\\
-\sin{\left( \mathit{\ensuremath{\theta}}\right) } & \cos{\left( \mathit{\ensuremath{\theta}}\right) } & 0\\
0 & 0 & 1\end{pmatrix}\mbox{}
\]
%%%%%%%%%%%%%%%


\noindent
%%%%%%%%%%%%%%%
%%% INPUT:
\begin{minipage}[t]{8ex}\color{red}\bf
(\%{}i73) 
\end{minipage}
\begin{minipage}[t]{\textwidth}\color{blue}\tt
trigsimp(transpose(A).A);
\end{minipage}
%%% OUTPUT:
\[\displaystyle
\tag{\%{}o73}\label{o73} 
\begin{pmatrix}1 & 0 & 0\\
0 & 1 & 0\\
0 & 0 & 1\end{pmatrix}\mbox{}
\]
%%%%%%%%%%%%%%%


\noindent
%%%%%%%%%%%%%%%
%%% INPUT:
\begin{minipage}[t]{8ex}\color{red}\bf
(\%{}i74) 
\end{minipage}
\begin{minipage}[t]{\textwidth}\color{blue}\tt
is(trigsimp(mycross(A[1],A[2]))=A[3]);
\end{minipage}
%%% OUTPUT:
\[\displaystyle
\tag{\%{}o74}\label{o74} 
\mbox{true}\mbox{}
\]
%%%%%%%%%%%%%%%

\subsection{Example 2.2.16.}


Following Example 2.2.9 and 2.2.15, the
\textbf{cylindrical coordinate frame} has attitude matrix:



\noindent
%%%%%%%%%%%%%%%
%%% INPUT:
\begin{minipage}[t]{8ex}\color{red}\bf
(\%{}i75) 
\end{minipage}
\begin{minipage}[t]{\textwidth}\color{blue}\tt
ldisplay(E\_1:[cos(\ensuremath{\theta})*sin(\ensuremath{\phi}),sin(\ensuremath{\theta})*sin(\ensuremath{\phi}),cos(\ensuremath{\phi})])\$
\end{minipage}
%%% OUTPUT:
\[\displaystyle
\tag{\%{}t75}\label{t75} 
{{E}_{1}}=[\cos{\left( \mathit{\ensuremath{\theta}}\right) }\,\sin{\left( \mathit{\ensuremath{\phi}}\right) },\sin{\left( \mathit{\ensuremath{\theta}}\right) }\,\sin{\left( \mathit{\ensuremath{\phi}}\right) },\cos{\left( \mathit{\ensuremath{\phi}}\right) }]\mbox{}
\]
%%%%%%%%%%%%%%%


\noindent
%%%%%%%%%%%%%%%
%%% INPUT:
\begin{minipage}[t]{8ex}\color{red}\bf
(\%{}i76) 
\end{minipage}
\begin{minipage}[t]{\textwidth}\color{blue}\tt
ldisplay(E\_2:[cos(\ensuremath{\theta})*cos(\ensuremath{\phi}),sin(\ensuremath{\theta})*cos(\ensuremath{\phi}),-sin(\ensuremath{\phi})])\$
\end{minipage}
%%% OUTPUT:
\[\displaystyle
\tag{\%{}t76}\label{t76} 
{{E}_{2}}=[\cos{\left( \mathit{\ensuremath{\theta}}\right) }\,\cos{\left( \mathit{\ensuremath{\phi}}\right) },\sin{\left( \mathit{\ensuremath{\theta}}\right) }\,\cos{\left( \mathit{\ensuremath{\phi}}\right) },-\sin{\left( \mathit{\ensuremath{\phi}}\right) }]\mbox{}
\]
%%%%%%%%%%%%%%%


\noindent
%%%%%%%%%%%%%%%
%%% INPUT:
\begin{minipage}[t]{8ex}\color{red}\bf
(\%{}i77) 
\end{minipage}
\begin{minipage}[t]{\textwidth}\color{blue}\tt
ldisplay(E\_3:[-sin(\ensuremath{\theta}),cos(\ensuremath{\theta}),0])\$
\end{minipage}
%%% OUTPUT:
\[\displaystyle
\tag{\%{}t77}\label{t77} 
{{E}_{3}}=[-\sin{\left( \mathit{\ensuremath{\theta}}\right) },\cos{\left( \mathit{\ensuremath{\theta}}\right) },0]\mbox{}
\]
%%%%%%%%%%%%%%%


\noindent
%%%%%%%%%%%%%%%
%%% INPUT:
\begin{minipage}[t]{8ex}\color{red}\bf
(\%{}i78) 
\end{minipage}
\begin{minipage}[t]{\textwidth}\color{blue}\tt
is(trigsimp(mycross(E\_1,E\_2)=E\_3));
\end{minipage}
%%% OUTPUT:
\[\displaystyle
\tag{\%{}o78}\label{o78} 
\mbox{true}\mbox{}
\]
%%%%%%%%%%%%%%%


\noindent
%%%%%%%%%%%%%%%
%%% INPUT:
\begin{minipage}[t]{8ex}\color{red}\bf
(\%{}i79) 
\end{minipage}
\begin{minipage}[t]{\textwidth}\color{blue}\tt
A:matrix(E\_1,E\_2,E\_3);
\end{minipage}
%%% OUTPUT:
\[\displaystyle
\tag{A}\label{A}
\begin{pmatrix}\cos{\left( \mathit{\ensuremath{\theta}}\right) }\,\sin{\left( \mathit{\ensuremath{\phi}}\right) } & \sin{\left( \mathit{\ensuremath{\theta}}\right) }\,\sin{\left( \mathit{\ensuremath{\phi}}\right) } & \cos{\left( \mathit{\ensuremath{\phi}}\right) }\\
\cos{\left( \mathit{\ensuremath{\theta}}\right) }\,\cos{\left( \mathit{\ensuremath{\phi}}\right) } & \sin{\left( \mathit{\ensuremath{\theta}}\right) }\,\cos{\left( \mathit{\ensuremath{\phi}}\right) } & -\sin{\left( \mathit{\ensuremath{\phi}}\right) }\\
-\sin{\left( \mathit{\ensuremath{\theta}}\right) } & \cos{\left( \mathit{\ensuremath{\theta}}\right) } & 0\end{pmatrix}\mbox{}
\]
%%%%%%%%%%%%%%%


\noindent
%%%%%%%%%%%%%%%
%%% INPUT:
\begin{minipage}[t]{8ex}\color{red}\bf
(\%{}i80) 
\end{minipage}
\begin{minipage}[t]{\textwidth}\color{blue}\tt
trigsimp(transpose(A).A);
\end{minipage}
%%% OUTPUT:
\[\displaystyle
\tag{\%{}o80}\label{o80} 
\begin{pmatrix}1 & 0 & 0\\
0 & 1 & 0\\
0 & 0 & 1\end{pmatrix}\mbox{}
\]
%%%%%%%%%%%%%%%


\noindent
%%%%%%%%%%%%%%%
%%% INPUT:
\begin{minipage}[t]{8ex}\color{red}\bf
(\%{}i81) 
\end{minipage}
\begin{minipage}[t]{\textwidth}\color{blue}\tt
is(trigsimp(mycross(A[1],A[2]))=A[3]);
\end{minipage}
%%% OUTPUT:
\[\displaystyle
\tag{\%{}o81}\label{o81} 
\mbox{true}\mbox{}
\]
%%%%%%%%%%%%%%%


\noindent
%%%%%%%%%%%%%%%
%%% INPUT:
\begin{minipage}[t]{8ex}\color{red}\bf
(\%{}i82) 
\end{minipage}
\begin{minipage}[t]{\textwidth}\color{blue}\tt
init\_cartan(\ensuremath{\xi})\$
\end{minipage}


\noindent
%%%%%%%%%%%%%%%
%%% INPUT:
\begin{minipage}[t]{8ex}\color{red}\bf
(\%{}i83) 
\end{minipage}
\begin{minipage}[t]{\textwidth}\color{blue}\tt
matrix\_element\_mult:"\ensuremath{\sim }"\$
\end{minipage}


\noindent
%%%%%%%%%%%%%%%
%%% INPUT:
\begin{minipage}[t]{8ex}\color{red}\bf
(\%{}i84) 
\end{minipage}
\begin{minipage}[t]{\textwidth}\color{blue}\tt
ldisplay(dA:trigsimp(ext\_diff(A)))\$
\end{minipage}
%%% OUTPUT:
\[\displaystyle
\tag{\%{}t84}\label{t84} 
\mathit{dA}=\begin{pmatrix}\mathit{d\ensuremath{\phi}}\,\cos{\left( \mathit{\ensuremath{\theta}}\right) }\,\cos{\left( \mathit{\ensuremath{\phi}}\right) }-\mathit{d\ensuremath{\theta}}\,\sin{\left( \mathit{\ensuremath{\theta}}\right) }\,\sin{\left( \mathit{\ensuremath{\phi}}\right) } & \mathit{d\ensuremath{\theta}}\,\cos{\left( \mathit{\ensuremath{\theta}}\right) }\,\sin{\left( \mathit{\ensuremath{\phi}}\right) }+\mathit{d\ensuremath{\phi}}\,\sin{\left( \mathit{\ensuremath{\theta}}\right) }\,\cos{\left( \mathit{\ensuremath{\phi}}\right) } & -\mathit{d\ensuremath{\phi}}\,\sin{\left( \mathit{\ensuremath{\phi}}\right) }\\
-\mathit{d\ensuremath{\phi}}\,\cos{\left( \mathit{\ensuremath{\theta}}\right) }\,\sin{\left( \mathit{\ensuremath{\phi}}\right) }-\mathit{d\ensuremath{\theta}}\,\sin{\left( \mathit{\ensuremath{\theta}}\right) }\,\cos{\left( \mathit{\ensuremath{\phi}}\right) } & \mathit{d\ensuremath{\theta}}\,\cos{\left( \mathit{\ensuremath{\theta}}\right) }\,\cos{\left( \mathit{\ensuremath{\phi}}\right) }-\mathit{d\ensuremath{\phi}}\,\sin{\left( \mathit{\ensuremath{\theta}}\right) }\,\sin{\left( \mathit{\ensuremath{\phi}}\right) } & -\mathit{d\ensuremath{\phi}}\,\cos{\left( \mathit{\ensuremath{\phi}}\right) }\\
-\mathit{d\ensuremath{\theta}}\,\cos{\left( \mathit{\ensuremath{\theta}}\right) } & -\mathit{d\ensuremath{\theta}}\,\sin{\left( \mathit{\ensuremath{\theta}}\right) } & 0\end{pmatrix}\mbox{}
\]
%%%%%%%%%%%%%%%


\noindent
%%%%%%%%%%%%%%%
%%% INPUT:
\begin{minipage}[t]{8ex}\color{red}\bf
(\%{}i85) 
\end{minipage}
\begin{minipage}[t]{\textwidth}\color{blue}\tt
ldisplay(\ensuremath{\omega}:trigsimp(dA.transpose(A)))\$
\end{minipage}
%%% OUTPUT:
\[\displaystyle
\tag{\%{}t85}\label{t85} 
\mathit{\ensuremath{\omega}}=\begin{pmatrix}0 & \mathit{d\ensuremath{\phi}} & \mathit{d\ensuremath{\theta}}\,\sin{\left( \mathit{\ensuremath{\phi}}\right) }\\
-\mathit{d\ensuremath{\phi}} & 0 & \mathit{d\ensuremath{\theta}}\,\cos{\left( \mathit{\ensuremath{\phi}}\right) }\\
-\mathit{d\ensuremath{\theta}}\,\sin{\left( \mathit{\ensuremath{\phi}}\right) } & -\mathit{d\ensuremath{\theta}}\,\cos{\left( \mathit{\ensuremath{\phi}}\right) } & 0\end{pmatrix}\mbox{}
\]
%%%%%%%%%%%%%%%


\noindent
%%%%%%%%%%%%%%%
%%% INPUT:
\begin{minipage}[t]{8ex}\color{red}\bf
(\%{}i86) 
\end{minipage}
\begin{minipage}[t]{\textwidth}\color{blue}\tt
matrix\_element\_mult:"*"\$
\end{minipage}
\pagebreak


\section{calculus of vectors fields along curves}


\subsection{Example 2.3.2.}


Let $\alpha=\left({t,t^2,t^3}\right)$ for $t\in\mathbb{R}$ and
$Y=x^2\partial_x+(y+sin(z))\partial_z$ then identify we have vector
field component functions: $$Y^1=x^2\quad,\quad Y^2=0\quad,\quad
Y^3=y+sin(z)$$ which give parametrized components on $\alpha=
\left({t,t^2,t^3}\right)$ of $$\left({Y^1\circ\alpha}\right)(t)=
t^2\quad,\quad\left({Y^2\circ\alpha}\right)(t)=0\quad,\quad
\left({Y^3\circ\alpha}\right)(t)=t^2+\sin(t^3)$$



\noindent
%%%%%%%%%%%%%%%
%%% INPUT:
\begin{minipage}[t]{8ex}\color{red}\bf
(\%{}i87) 
\end{minipage}
\begin{minipage}[t]{\textwidth}\color{blue}\tt
kill(labels,t,x,y,z)\$
\end{minipage}


\noindent
%%%%%%%%%%%%%%%
%%% INPUT:
\begin{minipage}[t]{8ex}\color{red}\bf
(\%{}i1) 
\end{minipage}
\begin{minipage}[t]{\textwidth}\color{blue}\tt
\ensuremath{\zeta}:[x,y,z]\$
\end{minipage}


\noindent
%%%%%%%%%%%%%%%
%%% INPUT:
\begin{minipage}[t]{8ex}\color{red}\bf
(\%{}i2) 
\end{minipage}
\begin{minipage}[t]{\textwidth}\color{blue}\tt
\ensuremath{\alpha}:[t,t\ensuremath{^2},t\ensuremath{^3}]\$
\end{minipage}


\noindent
%%%%%%%%%%%%%%%
%%% INPUT:
\begin{minipage}[t]{8ex}\color{red}\bf
(\%{}i3) 
\end{minipage}
\begin{minipage}[t]{\textwidth}\color{blue}\tt
Y:[x\ensuremath{^2},0,y+sin(z)]\$
\end{minipage}


\noindent
%%%%%%%%%%%%%%%
%%% INPUT:
\begin{minipage}[t]{8ex}\color{red}\bf
(\%{}i4) 
\end{minipage}
\begin{minipage}[t]{\textwidth}\color{blue}\tt
ldisplay(Yo\ensuremath{\alpha}:at(Y,map("=",\ensuremath{\zeta},\ensuremath{\alpha})))\$
\end{minipage}
%%% OUTPUT:
\[\displaystyle
\tag{\%{}t4}\label{t4} 
\mathit{Yo\ensuremath{\alpha}}=[{{t}^{2}},0,\sin{\left( {{t}^{3}}\right) }+{{t}^{2}}]\mbox{}
\]
%%%%%%%%%%%%%%%

\subsection{Example 2.3.4.}


Continuing Example 2.3.2., the vector field along $\alpha$ is given
by $$\left({Y\circ\alpha}\right)(t)=t^2\,U_1+\left({t^2+\sin(t^3)}
\right)\,U_3$$ thus $Y^\prime=2\,t\,U_1+\left({2\,t+3\,t^2\cos(t^3)}
\right)\in\,T_{(t,t^2,t^3)}\mathbb{R}^3$



\noindent
%%%%%%%%%%%%%%%
%%% INPUT:
\begin{minipage}[t]{8ex}\color{red}\bf
(\%{}i5) 
\end{minipage}
\begin{minipage}[t]{\textwidth}\color{blue}\tt
ldisplay(Yo\ensuremath{\alpha}\ensuremath{\backslash}':diff(Yo\ensuremath{\alpha},t))\$
\end{minipage}
%%% OUTPUT:
\[\displaystyle
\tag{\%{}t5}\label{t5} 
\mathit{Yo\ensuremath{\alpha}'}=[2t,0,3{{t}^{2}}\,\cos{\left( {{t}^{3}}\right) }+2t]\mbox{}
\]
%%%%%%%%%%%%%%%

\subsection{Example 2.3.7.}


Let $\alpha=\left({t,t^2,t^3}\right)$ for $t\in\mathbb{R}$. Then
$$\alpha^\prime(t)=U_1+2\,t\,U_2+3\,t^2\,U_3\quad,\quad
\alpha^{\prime\prime}(t)=2\,U_2+6\,t\,U_3$$ where both $\alpha^\prime$
and $\alpha^{\prime\prime}$ are in $T_{\alpha(t)}\mathbb{R}^3$.



\noindent
%%%%%%%%%%%%%%%
%%% INPUT:
\begin{minipage}[t]{8ex}\color{red}\bf
(\%{}i6) 
\end{minipage}
\begin{minipage}[t]{\textwidth}\color{blue}\tt
ldisplay(\ensuremath{\alpha}:[t,t\ensuremath{^2},t\ensuremath{^3}])\$
\end{minipage}
%%% OUTPUT:
\[\displaystyle
\tag{\%{}t6}\label{t6} 
\mathit{\ensuremath{\alpha}}=[t,{{t}^{2}},{{t}^{3}}]\mbox{}
\]
%%%%%%%%%%%%%%%


\noindent
%%%%%%%%%%%%%%%
%%% INPUT:
\begin{minipage}[t]{8ex}\color{red}\bf
(\%{}i7) 
\end{minipage}
\begin{minipage}[t]{\textwidth}\color{blue}\tt
ldisplay(\ensuremath{\alpha}\ensuremath{\backslash}':diff(\ensuremath{\alpha},t))\$
\end{minipage}
%%% OUTPUT:
\[\displaystyle
\tag{\%{}t7}\label{t7} 
\mathit{\ensuremath{\alpha}'}=[1,2t,3{{t}^{2}}]\mbox{}
\]
%%%%%%%%%%%%%%%


\noindent
%%%%%%%%%%%%%%%
%%% INPUT:
\begin{minipage}[t]{8ex}\color{red}\bf
(\%{}i8) 
\end{minipage}
\begin{minipage}[t]{\textwidth}\color{blue}\tt
ldisplay(\ensuremath{\alpha}\ensuremath{\backslash}'\ensuremath{\backslash}':diff(\ensuremath{\alpha}\ensuremath{\backslash}',t))\$
\end{minipage}
%%% OUTPUT:
\[\displaystyle
\tag{\%{}t8}\label{t8} 
\mathit{\ensuremath{\alpha}''}=[0,2,6t]\mbox{}
\]
%%%%%%%%%%%%%%%
\pagebreak


\section{Frenet Serret frame of a curve}


\subsection{Example 2.4.4.}


Consider the helix defined by $R,m>0$ and
$$\alpha(s)=\left({R\cos(k\,s),R\sin(k\,s),m\,k\,s}\right)$$
for $s\in\mathbb{R}$ and $k=1/\sqrt{R^2+m^2}$



\noindent
%%%%%%%%%%%%%%%
%%% INPUT:
\begin{minipage}[t]{8ex}\color{red}\bf
(\%{}i9) 
\end{minipage}
\begin{minipage}[t]{\textwidth}\color{blue}\tt
assume(R\ensuremath{>}0,m\ensuremath{>}0)\$
\end{minipage}


\noindent
%%%%%%%%%%%%%%%
%%% INPUT:
\begin{minipage}[t]{8ex}\color{red}\bf
(\%{}i10) 
\end{minipage}
\begin{minipage}[t]{\textwidth}\color{blue}\tt
declare([R,m],constant)\$
\end{minipage}


\noindent
%%%%%%%%%%%%%%%
%%% INPUT:
\begin{minipage}[t]{8ex}\color{red}\bf
(\%{}i11) 
\end{minipage}
\begin{minipage}[t]{\textwidth}\color{blue}\tt
paramk:k=1/\ensuremath{\sqrt{}}(R\ensuremath{^2}+m\ensuremath{^2})\$
\end{minipage}


\noindent
%%%%%%%%%%%%%%%
%%% INPUT:
\begin{minipage}[t]{8ex}\color{red}\bf
(\%{}i12) 
\end{minipage}
\begin{minipage}[t]{\textwidth}\color{blue}\tt
ldisplay(\ensuremath{\alpha}:[R*cos(k*s),R*sin(k*s),m*k*s])\$
\end{minipage}
%%% OUTPUT:
\[\displaystyle
\tag{\%{}t12}\label{t12} 
\mathit{\ensuremath{\alpha}}=[R\,\cos{(ks)},R\,\sin{(ks)},mks]\mbox{}
\]
%%%%%%%%%%%%%%%

Calculate



\noindent
%%%%%%%%%%%%%%%
%%% INPUT:
\begin{minipage}[t]{8ex}\color{red}\bf
(\%{}i13) 
\end{minipage}
\begin{minipage}[t]{\textwidth}\color{blue}\tt
ldisplay(T:\ensuremath{\alpha}\ensuremath{\backslash}':diff(\ensuremath{\alpha},s))\$
\end{minipage}
%%% OUTPUT:
\[\displaystyle
\tag{\%{}t13}\label{t13} 
T=[-Rk\,\sin{(ks)},Rk\,\cos{(ks)},mk]\mbox{}
\]
%%%%%%%%%%%%%%%


\noindent
%%%%%%%%%%%%%%%
%%% INPUT:
\begin{minipage}[t]{8ex}\color{red}\bf
(\%{}i14) 
\end{minipage}
\begin{minipage}[t]{\textwidth}\color{blue}\tt
at(trigsimp(norm(\ensuremath{\alpha}\ensuremath{\backslash}')),paramk);
\end{minipage}
%%% OUTPUT:
\[\displaystyle
\tag{\%{}o14}\label{o14} 
1\mbox{}
\]
%%%%%%%%%%%%%%%

It follows $T=\alpha^\prime$. Differentiate $\alpha^\prime$ to obtain:



\noindent
%%%%%%%%%%%%%%%
%%% INPUT:
\begin{minipage}[t]{8ex}\color{red}\bf
(\%{}i15) 
\end{minipage}
\begin{minipage}[t]{\textwidth}\color{blue}\tt
ldisplay(T\ensuremath{\backslash}':\ensuremath{\alpha}\ensuremath{\backslash}'\ensuremath{\backslash}':diff(\ensuremath{\alpha}\ensuremath{\backslash}',s))\$
\end{minipage}
%%% OUTPUT:
\[\displaystyle
\tag{\%{}t15}\label{t15} 
\mathit{T'}=[-R\,{{k}^{2}}\,\cos{(ks)},-R\,{{k}^{2}}\,\sin{(ks)},0]\mbox{}
\]
%%%%%%%%%%%%%%%


\noindent
%%%%%%%%%%%%%%%
%%% INPUT:
\begin{minipage}[t]{8ex}\color{red}\bf
(\%{}i16) 
\end{minipage}
\begin{minipage}[t]{\textwidth}\color{blue}\tt
ldisplay(\ensuremath{\kappa}:at(trigsimp(norm(\ensuremath{\alpha}\ensuremath{\backslash}'\ensuremath{\backslash}')),paramk))\$
\end{minipage}
%%% OUTPUT:
\[\displaystyle
\tag{\%{}t16}\label{t16} 
\mathit{\ensuremath{\kappa}}=\frac{R}{{{m}^{2}}+{{R}^{2}}}\mbox{}
\]
%%%%%%%%%%%%%%%


\noindent
%%%%%%%%%%%%%%%
%%% INPUT:
\begin{minipage}[t]{8ex}\color{red}\bf
(\%{}i17) 
\end{minipage}
\begin{minipage}[t]{\textwidth}\color{blue}\tt
trigsimp(norm(T\ensuremath{\backslash}'));
\end{minipage}
%%% OUTPUT:
\[\displaystyle
\tag{\%{}o17}\label{o17} 
R\,{{k}^{2}}\mbox{}
\]
%%%%%%%%%%%%%%%


\noindent
%%%%%%%%%%%%%%%
%%% INPUT:
\begin{minipage}[t]{8ex}\color{red}\bf
(\%{}i18) 
\end{minipage}
\begin{minipage}[t]{\textwidth}\color{blue}\tt
ldisplay(N:[-cos(k*s),-sin(k*s),0])\$
\end{minipage}
%%% OUTPUT:
\[\displaystyle
\tag{\%{}t18}\label{t18} 
N=[-\cos{(ks)},-\sin{(ks)},0]\mbox{}
\]
%%%%%%%%%%%%%%%


\noindent
%%%%%%%%%%%%%%%
%%% INPUT:
\begin{minipage}[t]{8ex}\color{red}\bf
(\%{}i19) 
\end{minipage}
\begin{minipage}[t]{\textwidth}\color{blue}\tt
ldisplay(B:trigsimp(mycross(T,N)))\$
\end{minipage}
%%% OUTPUT:
\[\displaystyle
\tag{\%{}t19}\label{t19} 
B=[mk\,\sin{(ks)},-mk\,\cos{(ks)},Rk]\mbox{}
\]
%%%%%%%%%%%%%%%

As a quick check on the calculation, notice
$\mathbf{B}\cdot\mathbf{N}=0$ and $\mathbf{B}\cdot\mathbf{T}=0$.



\noindent
%%%%%%%%%%%%%%%
%%% INPUT:
\begin{minipage}[t]{8ex}\color{red}\bf
(\%{}i20) 
\end{minipage}
\begin{minipage}[t]{\textwidth}\color{blue}\tt
trigsimp(B.N);
\end{minipage}
%%% OUTPUT:
\[\displaystyle
\tag{\%{}o20}\label{o20} 
0\mbox{}
\]
%%%%%%%%%%%%%%%


\noindent
%%%%%%%%%%%%%%%
%%% INPUT:
\begin{minipage}[t]{8ex}\color{red}\bf
(\%{}i21) 
\end{minipage}
\begin{minipage}[t]{\textwidth}\color{blue}\tt
trigsimp(B.T);
\end{minipage}
%%% OUTPUT:
\[\displaystyle
\tag{\%{}o21}\label{o21} 
0\mbox{}
\]
%%%%%%%%%%%%%%%

Calculate:



\noindent
%%%%%%%%%%%%%%%
%%% INPUT:
\begin{minipage}[t]{8ex}\color{red}\bf
(\%{}i22) 
\end{minipage}
\begin{minipage}[t]{\textwidth}\color{blue}\tt
ldisplay(B\ensuremath{\backslash}':diff(B,s))\$
\end{minipage}
%%% OUTPUT:
\[\displaystyle
\tag{\%{}t22}\label{t22} 
\mathit{B'}=[m\,{{k}^{2}}\,\cos{(ks)},m\,{{k}^{2}}\,\sin{(ks)},0]\mbox{}
\]
%%%%%%%%%%%%%%%

thus:



\noindent
%%%%%%%%%%%%%%%
%%% INPUT:
\begin{minipage}[t]{8ex}\color{red}\bf
(\%{}i23) 
\end{minipage}
\begin{minipage}[t]{\textwidth}\color{blue}\tt
ldisplay(\ensuremath{\tau}:-at(trigsimp(B\ensuremath{\backslash}'.N),paramk))\$
\end{minipage}
%%% OUTPUT:
\[\displaystyle
\tag{\%{}t23}\label{t23} 
\mathit{\ensuremath{\tau}}=\frac{m}{{{m}^{2}}+{{R}^{2}}}\mbox{}
\]
%%%%%%%%%%%%%%%

\subsection{Example 2.4.7.}


If a curve is on a sphere then it is at least as curved as a great
circle on the sphere. To see this, consider $\alpha:I\rightarrow
\mathbb{R}^3$ a unit-speed regular curve on the sphere with center
$C$ and radius $R$. We are given $\Vert{\alpha(s)-C}\Vert=R$ for all
$s\in\,I$.


\subsection{Theorem 2.4.8.}


\textbf{Frenet Serret Equations for non-unit speed curves}\\
Let $\alpha$ be a non-linear regular smooth curve with speed
$v=\lVert{\alpha^\prime}\rVert$ and $\mathbf{T},\mathbf{N},\mathbf{B},
\kappa$ and $\tau$ as defined through the unit-speed
reparameterization then:
$$\dfrac{\mathrm{d}\mathbf{T}}{\mathrm{d}t}=v\kappa\mathbf{N}$$
$$\dfrac{\mathrm{d}\mathbf{N}}{\mathrm{d}t}=-v\kappa\mathbf{T}+v\tau\mathbf{B}$$
$$\dfrac{\mathrm{d}\mathbf{B}}{\mathrm{d}t}=-v\tau\mathbf{N}$$
Moreover, $\kappa=\frac{1}{v}\lVert{\mathbf{T}^\prime}\rVert$ and
$\tau=-\frac{1}{v}\mathbf{B}^\prime\cdot\mathbf{N}$


\subsection{Proposition 2.4.9.}


\textbf{Acceleration in terms of curvature and speed}\\
Let $\alpha$ be a non-linear regular smooth curve with speed
$v=\lVert{\alpha^\prime}\rVert$ then $\alpha^{\prime\prime}=
\frac{\mathrm{d}v}{\mathrm{d}t}\mathbf{T}+\kappa\,v^2\mathbf{N}$


\subsection{Example 2.4.10.}


Suppose $\alpha(t)=\left({t,t^2,t^3}\right)$ Calculate the Frenet
apparatus or at least try.
\href{https://en.wikipedia.org/wiki/Frenet%E2%80%93Serret_formulas#Other_expressions_of_the_frame}
{Other expressions of the frame}



\noindent
%%%%%%%%%%%%%%%
%%% INPUT:
\begin{minipage}[t]{8ex}\color{red}\bf
(\%{}i24) 
\end{minipage}
\begin{minipage}[t]{\textwidth}\color{blue}\tt
ldisplay(\ensuremath{\alpha}:[t,t\ensuremath{^2},t\ensuremath{^2}])\$
\end{minipage}
%%% OUTPUT:
\[\displaystyle
\tag{\%{}t24}\label{t24} 
\mathit{\ensuremath{\alpha}}=[t,{{t}^{2}},{{t}^{2}}]\mbox{}
\]
%%%%%%%%%%%%%%%


\noindent
%%%%%%%%%%%%%%%
%%% INPUT:
\begin{minipage}[t]{8ex}\color{red}\bf
(\%{}i25) 
\end{minipage}
\begin{minipage}[t]{\textwidth}\color{blue}\tt
ldisplay(\ensuremath{\alpha}\ensuremath{\backslash}':diff(\ensuremath{\alpha},t))\$
\end{minipage}
%%% OUTPUT:
\[\displaystyle
\tag{\%{}t25}\label{t25} 
\mathit{\ensuremath{\alpha}'}=[1,2t,2t]\mbox{}
\]
%%%%%%%%%%%%%%%


\noindent
%%%%%%%%%%%%%%%
%%% INPUT:
\begin{minipage}[t]{8ex}\color{red}\bf
(\%{}i26) 
\end{minipage}
\begin{minipage}[t]{\textwidth}\color{blue}\tt
ldisplay(\ensuremath{\alpha}\ensuremath{\backslash}'\ensuremath{\backslash}':diff(\ensuremath{\alpha}\ensuremath{\backslash}',t))\$
\end{minipage}
%%% OUTPUT:
\[\displaystyle
\tag{\%{}t26}\label{t26} 
\mathit{\ensuremath{\alpha}''}=[0,2,2]\mbox{}
\]
%%%%%%%%%%%%%%%


\noindent
%%%%%%%%%%%%%%%
%%% INPUT:
\begin{minipage}[t]{8ex}\color{red}\bf
(\%{}i27) 
\end{minipage}
\begin{minipage}[t]{\textwidth}\color{blue}\tt
ldisplay(\ensuremath{\alpha}\ensuremath{\backslash}'\ensuremath{\backslash}'\ensuremath{\backslash}':diff(\ensuremath{\alpha}\ensuremath{\backslash}'\ensuremath{\backslash}',t))\$
\end{minipage}
%%% OUTPUT:
\[\displaystyle
\tag{\%{}t27}\label{t27} 
\mathit{\ensuremath{\alpha}'''}=[0,0,0]\mbox{}
\]
%%%%%%%%%%%%%%%


\noindent
%%%%%%%%%%%%%%%
%%% INPUT:
\begin{minipage}[t]{8ex}\color{red}\bf
(\%{}i28) 
\end{minipage}
\begin{minipage}[t]{\textwidth}\color{blue}\tt
ldisplay(T:normalize(\ensuremath{\alpha}\ensuremath{\backslash}'))\$
\end{minipage}
%%% OUTPUT:
\[\displaystyle
\tag{\%{}t28}\label{t28} 
T=\left[\frac{1}{\sqrt{8{{t}^{2}}+1}},\frac{2t}{\sqrt{8{{t}^{2}}+1}},\frac{2t}{\sqrt{8{{t}^{2}}+1}}\right]\mbox{}
\]
%%%%%%%%%%%%%%%


\noindent
%%%%%%%%%%%%%%%
%%% INPUT:
\begin{minipage}[t]{8ex}\color{red}\bf
(\%{}i29) 
\end{minipage}
\begin{minipage}[t]{\textwidth}\color{blue}\tt
ldisplay(T\ensuremath{\backslash}':factor(diff(T,t)))\$
\end{minipage}
%%% OUTPUT:
\[\displaystyle
\tag{\%{}t29}\label{t29} 
\mathit{T'}=\left[-\frac{8t}{{{\left( 8{{t}^{2}}+1\right) }^{\frac{3}{2}}}},\frac{2}{{{\left( 8{{t}^{2}}+1\right) }^{\frac{3}{2}}}},\frac{2}{{{\left( 8{{t}^{2}}+1\right) }^{\frac{3}{2}}}}\right]\mbox{}
\]
%%%%%%%%%%%%%%%


\noindent
%%%%%%%%%%%%%%%
%%% INPUT:
\begin{minipage}[t]{8ex}\color{red}\bf
(\%{}i30) 
\end{minipage}
\begin{minipage}[t]{\textwidth}\color{blue}\tt
ldisplay(N:rootscontract(normalize(T\ensuremath{\backslash}')))\$
\end{minipage}
%%% OUTPUT:
\[\displaystyle
\tag{\%{}t30}\label{t30} 
N=\left[-t\,\sqrt{\frac{8}{8{{t}^{2}}+1}},\frac{1}{\sqrt{16{{t}^{2}}+2}},\frac{1}{\sqrt{16{{t}^{2}}+2}}\right]\mbox{}
\]
%%%%%%%%%%%%%%%


\noindent
%%%%%%%%%%%%%%%
%%% INPUT:
\begin{minipage}[t]{8ex}\color{red}\bf
(\%{}i31) 
\end{minipage}
\begin{minipage}[t]{\textwidth}\color{blue}\tt
ldisplay(B:factor(mycross(T,N)))\$
\end{minipage}
%%% OUTPUT:
\[\displaystyle
\tag{\%{}t31}\label{t31} 
B=\left[0,-\frac{1}{\sqrt{2}},\frac{1}{\sqrt{2}}\right]\mbox{}
\]
%%%%%%%%%%%%%%%


\noindent
%%%%%%%%%%%%%%%
%%% INPUT:
\begin{minipage}[t]{8ex}\color{red}\bf
(\%{}i32) 
\end{minipage}
\begin{minipage}[t]{\textwidth}\color{blue}\tt
ldisplay(S:mycross(\ensuremath{\alpha}\ensuremath{\backslash}',\ensuremath{\alpha}\ensuremath{\backslash}'\ensuremath{\backslash}'))\$
\end{minipage}
%%% OUTPUT:
\[\displaystyle
\tag{\%{}t32}\label{t32} 
S=[0,-2,2]\mbox{}
\]
%%%%%%%%%%%%%%%


\noindent
%%%%%%%%%%%%%%%
%%% INPUT:
\begin{minipage}[t]{8ex}\color{red}\bf
(\%{}i33) 
\end{minipage}
\begin{minipage}[t]{\textwidth}\color{blue}\tt
ldisplay(\ensuremath{\kappa}:norm(S)/norm(\ensuremath{\alpha}\ensuremath{\backslash}')\ensuremath{^3})\$
\end{minipage}
%%% OUTPUT:
\[\displaystyle
\tag{\%{}t33}\label{t33} 
\mathit{\ensuremath{\kappa}}=\frac{{{2}^{\frac{3}{2}}}}{{{\left( 8{{t}^{2}}+1\right) }^{\frac{3}{2}}}}\mbox{}
\]
%%%%%%%%%%%%%%%


\noindent
%%%%%%%%%%%%%%%
%%% INPUT:
\begin{minipage}[t]{8ex}\color{red}\bf
(\%{}i34) 
\end{minipage}
\begin{minipage}[t]{\textwidth}\color{blue}\tt
ldisplay(\ensuremath{\tau}:(S.\ensuremath{\alpha}\ensuremath{\backslash}'\ensuremath{\backslash}'\ensuremath{\backslash}')/norm(S)\ensuremath{^2})\$
\end{minipage}
%%% OUTPUT:
\[\displaystyle
\tag{\%{}t34}\label{t34} 
\mathit{\ensuremath{\tau}}=0\mbox{}
\]
%%%%%%%%%%%%%%%

$\alpha^{\prime\prime}$ in the Frenet-Serret frame



\noindent
%%%%%%%%%%%%%%%
%%% INPUT:
\begin{minipage}[t]{8ex}\color{red}\bf
(\%{}i35) 
\end{minipage}
\begin{minipage}[t]{\textwidth}\color{blue}\tt
ldisplay(\ensuremath{\alpha}\ensuremath{\backslash}'\ensuremath{\backslash}':[(\ensuremath{\alpha}\ensuremath{\backslash}'\ensuremath{\backslash}'.T),(\ensuremath{\alpha}\ensuremath{\backslash}'\ensuremath{\backslash}'.N),(\ensuremath{\alpha}\ensuremath{\backslash}'\ensuremath{\backslash}'.B)])\$
\end{minipage}
%%% OUTPUT:
\[\displaystyle
\tag{\%{}t35}\label{t35} 
\mathit{\ensuremath{\alpha}''}=\left[\frac{8t}{\sqrt{8{{t}^{2}}+1}},\frac{4}{\sqrt{16{{t}^{2}}+2}},0\right]\mbox{}
\]
%%%%%%%%%%%%%%%

\subsection{Theorem 2.4.11.}


\textbf{Slick formulas for the Frenet apparatus}\\
Let $\alpha$ be a non-linear regular smooth curve with speed
$v=\lVert{\alpha^\prime}\rVert$ and $\mathbf{T},\mathbf{N},\mathbf{B},
\kappa$ and $\tau$ as defined through the unit-speed
reparameterization then: $$\mathbf{T}=\dfrac{\alpha^\prime}{\lVert{
\alpha^\prime}\rVert},\quad\mathbf{B}=\dfrac{\alpha^\prime\times
\alpha^{\prime\prime}}{\lVert{\alpha^\prime\times\alpha^{\prime
\prime}}\rVert},\quad\mathbf{N}=\mathbf{B}\times\mathbf{T},$$
$$\kappa=\dfrac{\lVert{\alpha^\prime\times\alpha^{\prime\prime}}
\rVert}{\lVert{\alpha^\prime}\rVert^3},\quad\tau=\dfrac{(\alpha^
\prime\times\alpha^{\prime\prime})\cdot\alpha^{\prime\prime\prime}}
{\lVert{\alpha^\prime\times\alpha^{\prime\prime}}\rVert^2}$$

\pagebreak


\section{covariant derivatives}


\subsection{Example 2.5.2.}


If $W(p)=a\,U_1+b\,U_2+c\,U_3$ for constants $a,b,c\in\mathbb{R}$ for
all $p\in\mathbb{R}^3$ then $W(p+t\,v)=a\,U_1+b\,U_2+c\,U_3$ hence
$W^\prime(p+t\,v)=0$ thus $(\nabla_v\,W)(p)=0$ for all $p\in\mathbb{R}^3$
hence $\nabla_v\,W=0$ for any choice of $V\in\mathfrak{X}(\mathbb{R}^3)$
as the calculation held for arbitrary $v$ at each $p$.



\noindent
%%%%%%%%%%%%%%%
%%% INPUT:
\begin{minipage}[t]{8ex}\color{red}\bf
(\%{}i36) 
\end{minipage}
\begin{minipage}[t]{\textwidth}\color{blue}\tt
kill(t,x,v,z)\$
\end{minipage}

\textbf{Define the space} $\mathbb{R}^3$



\noindent
%%%%%%%%%%%%%%%
%%% INPUT:
\begin{minipage}[t]{8ex}\color{red}\bf
(\%{}i37) 
\end{minipage}
\begin{minipage}[t]{\textwidth}\color{blue}\tt
\ensuremath{\zeta}:[x,y,z]\$
\end{minipage}


\noindent
%%%%%%%%%%%%%%%
%%% INPUT:
\begin{minipage}[t]{8ex}\color{red}\bf
(\%{}i38) 
\end{minipage}
\begin{minipage}[t]{\textwidth}\color{blue}\tt
scalefactors(\ensuremath{\zeta})\$
\end{minipage}


\noindent
%%%%%%%%%%%%%%%
%%% INPUT:
\begin{minipage}[t]{8ex}\color{red}\bf
(\%{}i39) 
\end{minipage}
\begin{minipage}[t]{\textwidth}\color{blue}\tt
init\_cartan(\ensuremath{\zeta})\$
\end{minipage}

\textbf{Define frame} $\mathbf{U}\in\mathbb{R}^3$



\noindent
%%%%%%%%%%%%%%%
%%% INPUT:
\begin{minipage}[t]{8ex}\color{red}\bf
(\%{}i40) 
\end{minipage}
\begin{minipage}[t]{\textwidth}\color{blue}\tt
U:[U\_1,U\_2,U\_3]\$
\end{minipage}

\textbf{Define point} $P=(1,2,3)\in\mathbb{R}^3$



\noindent
%%%%%%%%%%%%%%%
%%% INPUT:
\begin{minipage}[t]{8ex}\color{red}\bf
(\%{}i41) 
\end{minipage}
\begin{minipage}[t]{\textwidth}\color{blue}\tt
declare([a,b,c],constant)\$
\end{minipage}


\noindent
%%%%%%%%%%%%%%%
%%% INPUT:
\begin{minipage}[t]{8ex}\color{red}\bf
(\%{}i42) 
\end{minipage}
\begin{minipage}[t]{\textwidth}\color{blue}\tt
P:[a,b,c]\$
\end{minipage}


\noindent
%%%%%%%%%%%%%%%
%%% INPUT:
\begin{minipage}[t]{8ex}\color{red}\bf
(\%{}i43) 
\end{minipage}
\begin{minipage}[t]{\textwidth}\color{blue}\tt
v:[v\_1,v\_2,v\_3]\$
\end{minipage}


\noindent
%%%%%%%%%%%%%%%
%%% INPUT:
\begin{minipage}[t]{8ex}\color{red}\bf
(\%{}i44) 
\end{minipage}
\begin{minipage}[t]{\textwidth}\color{blue}\tt
ldisplay(W:x*U\_1+y*U\_2+z*U\_3)\$
\end{minipage}
%%% OUTPUT:
\[\displaystyle
\tag{\%{}t44}\label{t44} 
W={{U}_{3}}z+{{U}_{2}}y+{{U}_{1}}x\mbox{}
\]
%%%%%%%%%%%%%%%


\noindent
%%%%%%%%%%%%%%%
%%% INPUT:
\begin{minipage}[t]{8ex}\color{red}\bf
(\%{}i45) 
\end{minipage}
\begin{minipage}[t]{\textwidth}\color{blue}\tt
ldisplay(W:makelist(coeff(W,i),i,U))\$
\end{minipage}
%%% OUTPUT:
\[\displaystyle
\tag{\%{}t45}\label{t45} 
W=[x,y,z]\mbox{}
\]
%%%%%%%%%%%%%%%


\noindent
%%%%%%%%%%%%%%%
%%% INPUT:
\begin{minipage}[t]{8ex}\color{red}\bf
(\%{}i46) 
\end{minipage}
\begin{minipage}[t]{\textwidth}\color{blue}\tt
ldisplay(W\_P:at(W,map("=",\ensuremath{\zeta},P)))\$
\end{minipage}
%%% OUTPUT:
\[\displaystyle
\tag{\%{}t46}\label{t46} 
{{W}_{P}}=[a,b,c]\mbox{}
\]
%%%%%%%%%%%%%%%


\noindent
%%%%%%%%%%%%%%%
%%% INPUT:
\begin{minipage}[t]{8ex}\color{red}\bf
(\%{}i47) 
\end{minipage}
\begin{minipage}[t]{\textwidth}\color{blue}\tt
ldisplay(W\_t:at(W,map("=",\ensuremath{\zeta},P+t*v)))\$
\end{minipage}
%%% OUTPUT:
\[\displaystyle
\tag{\%{}t47}\label{t47} 
{{W}_{t}}=[t\,{{v}_{1}}+a,t\,{{v}_{2}}+b,t\,{{v}_{3}}+c]\mbox{}
\]
%%%%%%%%%%%%%%%

\subsection{Example 2.5.3.}


What about the change of $W=x^2\,U_1+y\,U_3$ along $v=2\,U_2+U_3$
at $p=(1.2.3)$?\\ Calculate, $$W(p+t\,v)=W(1+2\,t,2,3+t)=(1+2\,t)^2
\,U_1+(3+t)\,U_3$$ thus, $$W^\prime(p+t\,v)=4(1+2\,t)\,U_1+U_3
\Rightarrow W^\prime(p+t\,v)(0)=4\,U_1+U_3.$$
Therefore,$(\nabla_v\,W)(1,2,3)=4\,U_1+U_3$



\noindent
%%%%%%%%%%%%%%%
%%% INPUT:
\begin{minipage}[t]{8ex}\color{red}\bf
(\%{}i48) 
\end{minipage}
\begin{minipage}[t]{\textwidth}\color{blue}\tt
kill(t,x,y,z)\$
\end{minipage}

\textbf{Define the space} $\mathbb{R}^3$



\noindent
%%%%%%%%%%%%%%%
%%% INPUT:
\begin{minipage}[t]{8ex}\color{red}\bf
(\%{}i49) 
\end{minipage}
\begin{minipage}[t]{\textwidth}\color{blue}\tt
\ensuremath{\zeta}:[x,y,z]\$
\end{minipage}


\noindent
%%%%%%%%%%%%%%%
%%% INPUT:
\begin{minipage}[t]{8ex}\color{red}\bf
(\%{}i50) 
\end{minipage}
\begin{minipage}[t]{\textwidth}\color{blue}\tt
scalefactors(\ensuremath{\zeta})\$
\end{minipage}


\noindent
%%%%%%%%%%%%%%%
%%% INPUT:
\begin{minipage}[t]{8ex}\color{red}\bf
(\%{}i51) 
\end{minipage}
\begin{minipage}[t]{\textwidth}\color{blue}\tt
init\_cartan(\ensuremath{\zeta})\$
\end{minipage}

\textbf{Define frame} $\mathbf{U}\in\mathbb{R}^3$



\noindent
%%%%%%%%%%%%%%%
%%% INPUT:
\begin{minipage}[t]{8ex}\color{red}\bf
(\%{}i52) 
\end{minipage}
\begin{minipage}[t]{\textwidth}\color{blue}\tt
U:[U\_1,U\_2,U\_3]\$
\end{minipage}

\textbf{Define} $\mathbf{W}\in\mathfrak{X}(\mathbb{R}^3)$



\noindent
%%%%%%%%%%%%%%%
%%% INPUT:
\begin{minipage}[t]{8ex}\color{red}\bf
(\%{}i53) 
\end{minipage}
\begin{minipage}[t]{\textwidth}\color{blue}\tt
ldisplay(W:x\ensuremath{^2}*U\_1-2*z\ensuremath{^3}*U\_2+y*U\_3)\$
\end{minipage}
%%% OUTPUT:
\[\displaystyle
\tag{\%{}t53}\label{t53} 
W=-2{{U}_{2}}{{z}^{3}}+{{U}_{3}}y+{{U}_{1}}{{x}^{2}}\mbox{}
\]
%%%%%%%%%%%%%%%


\noindent
%%%%%%%%%%%%%%%
%%% INPUT:
\begin{minipage}[t]{8ex}\color{red}\bf
(\%{}i54) 
\end{minipage}
\begin{minipage}[t]{\textwidth}\color{blue}\tt
ldisplay(W:makelist(coeff(W,k),k,U))\$
\end{minipage}
%%% OUTPUT:
\[\displaystyle
\tag{\%{}t54}\label{t54} 
W=[{{x}^{2}},-2{{z}^{3}},y]\mbox{}
\]
%%%%%%%%%%%%%%%

\textbf{Define} $\mathbf{v}\in\mathfrak{X}(\mathbb{R}^3)$



\noindent
%%%%%%%%%%%%%%%
%%% INPUT:
\begin{minipage}[t]{8ex}\color{red}\bf
(\%{}i55) 
\end{minipage}
\begin{minipage}[t]{\textwidth}\color{blue}\tt
ldisplay(v:2*U\_2-U\_1)\$
\end{minipage}
%%% OUTPUT:
\[\displaystyle
\tag{\%{}t55}\label{t55} 
v=2{{U}_{2}}-{{U}_{1}}\mbox{}
\]
%%%%%%%%%%%%%%%


\noindent
%%%%%%%%%%%%%%%
%%% INPUT:
\begin{minipage}[t]{8ex}\color{red}\bf
(\%{}i56) 
\end{minipage}
\begin{minipage}[t]{\textwidth}\color{blue}\tt
ldisplay(v:makelist(coeff(v,k),k,U))\$
\end{minipage}
%%% OUTPUT:
\[\displaystyle
\tag{\%{}t56}\label{t56} 
v=[-1,2,0]\mbox{}
\]
%%%%%%%%%%%%%%%

\textbf{Define} $\mathbf{P}\in\mathbb{R}^3$



\noindent
%%%%%%%%%%%%%%%
%%% INPUT:
\begin{minipage}[t]{8ex}\color{red}\bf
(\%{}i57) 
\end{minipage}
\begin{minipage}[t]{\textwidth}\color{blue}\tt
ldisplay(P:[1,2,3])\$
\end{minipage}
%%% OUTPUT:
\[\displaystyle
\tag{\%{}t57}\label{t57} 
P=[1,2,3]\mbox{}
\]
%%%%%%%%%%%%%%%

\textbf{Using definition}


\textbf{Calculate} $\mathbf{DW}\in\mathfrak{X}(\mathbb{R}^3)$



\noindent
%%%%%%%%%%%%%%%
%%% INPUT:
\begin{minipage}[t]{8ex}\color{red}\bf
(\%{}i58) 
\end{minipage}
\begin{minipage}[t]{\textwidth}\color{blue}\tt
ldisplay(DW:at(diff(at(W,map("=",\ensuremath{\zeta},P+t*v)),t),t=0))\$
\end{minipage}
%%% OUTPUT:
\[\displaystyle
\tag{\%{}t58}\label{t58} 
\mathit{DW}=[-2,0,2]\mbox{}
\]
%%%%%%%%%%%%%%%


\noindent
%%%%%%%%%%%%%%%
%%% INPUT:
\begin{minipage}[t]{8ex}\color{red}\bf
(\%{}i59) 
\end{minipage}
\begin{minipage}[t]{\textwidth}\color{blue}\tt
ldisplay(DW:DW.U)\$
\end{minipage}
%%% OUTPUT:
\[\displaystyle
\tag{\%{}t59}\label{t59} 
\mathit{DW}=2{{U}_{3}}-2{{U}_{1}}\mbox{}
\]
%%%%%%%%%%%%%%%

\textbf{Using grad}



\noindent
%%%%%%%%%%%%%%%
%%% INPUT:
\begin{minipage}[t]{8ex}\color{red}\bf
(\%{}i60) 
\end{minipage}
\begin{minipage}[t]{\textwidth}\color{blue}\tt
gradW:apply('matrix,ev(express(grad(W)),diff));
\end{minipage}
%%% OUTPUT:
\[\displaystyle
\tag{gradW}\label{gradW}
\begin{pmatrix}2x & 0 & 0\\
0 & 0 & 1\\
0 & -6{{z}^{2}} & 0\end{pmatrix}\mbox{}
\]
%%%%%%%%%%%%%%%

\textbf{Calculate} $\mathbf{DW1}\in\mathfrak{X}(\mathbb{R}^3)$



\noindent
%%%%%%%%%%%%%%%
%%% INPUT:
\begin{minipage}[t]{8ex}\color{red}\bf
(\%{}i61) 
\end{minipage}
\begin{minipage}[t]{\textwidth}\color{blue}\tt
ldisplay(DW1:list\_matrix\_entries(v.at(gradW,map("=",\ensuremath{\zeta},P))))\$
\end{minipage}
%%% OUTPUT:
\[\displaystyle
\tag{\%{}t61}\label{t61} 
\mathit{DW1}=[-2,0,2]\mbox{}
\]
%%%%%%%%%%%%%%%


\noindent
%%%%%%%%%%%%%%%
%%% INPUT:
\begin{minipage}[t]{8ex}\color{red}\bf
(\%{}i62) 
\end{minipage}
\begin{minipage}[t]{\textwidth}\color{blue}\tt
ldisplay(DW1:DW1.U)\$
\end{minipage}
%%% OUTPUT:
\[\displaystyle
\tag{\%{}t62}\label{t62} 
\mathit{DW1}=2{{U}_{3}}-2{{U}_{1}}\mbox{}
\]
%%%%%%%%%%%%%%%


\noindent
%%%%%%%%%%%%%%%
%%% INPUT:
\begin{minipage}[t]{8ex}\color{red}\bf
(\%{}i63) 
\end{minipage}
\begin{minipage}[t]{\textwidth}\color{blue}\tt
is(DW1=DW);
\end{minipage}
%%% OUTPUT:
\[\displaystyle
\tag{\%{}o63}\label{o63} 
\mbox{true}\mbox{}
\]
%%%%%%%%%%%%%%%

\subsection{Proposition 2.5.4.}


\textbf{Coordinate derivative formula for covariant derivative}\\
Let $\mathbf{V},\mathbf{W}\in\mathfrak{X}(\mathbb{R}^3)$ then
$$\nabla_V\mathbf{W}=\sum_{j=1}^3\mathbf{V}\left[{W^j}\right]U_j$$
    


\noindent
%%%%%%%%%%%%%%%
%%% INPUT:
\begin{minipage}[t]{8ex}\color{red}\bf
(\%{}i64) 
\end{minipage}
\begin{minipage}[t]{\textwidth}\color{blue}\tt
ldisplay(DW2:list\_matrix\_entries(v.gradW))\$
\end{minipage}
%%% OUTPUT:
\[\displaystyle
\tag{\%{}t64}\label{t64} 
\mathit{DW2}=[-2x,0,2]\mbox{}
\]
%%%%%%%%%%%%%%%


\noindent
%%%%%%%%%%%%%%%
%%% INPUT:
\begin{minipage}[t]{8ex}\color{red}\bf
(\%{}i65) 
\end{minipage}
\begin{minipage}[t]{\textwidth}\color{blue}\tt
ldisplay(DW2:DW2.U)\$
\end{minipage}
%%% OUTPUT:
\[\displaystyle
\tag{\%{}t65}\label{t65} 
\mathit{DW2}=2{{U}_{3}}-2{{U}_{1}}x\mbox{}
\]
%%%%%%%%%%%%%%%

\subsection{Example 2.5.6.}


Let $V=x\,U_1+y^2\,U_2+z^3\,U_3$ and $W=y\,z\,U_1+x\,y\,U_3$. Recall
our notation $U_1,U_2,U_3$ masks the fact that these are derivations;
$U_1=\partial_x$, $U_2=\partial_y$ and $U_3=\partial_z$ thus:



\noindent
%%%%%%%%%%%%%%%
%%% INPUT:
\begin{minipage}[t]{8ex}\color{red}\bf
(\%{}i66) 
\end{minipage}
\begin{minipage}[t]{\textwidth}\color{blue}\tt
kill(t,x,v,z)\$
\end{minipage}

\textbf{Define the space} $\mathbb{R}^3$



\noindent
%%%%%%%%%%%%%%%
%%% INPUT:
\begin{minipage}[t]{8ex}\color{red}\bf
(\%{}i67) 
\end{minipage}
\begin{minipage}[t]{\textwidth}\color{blue}\tt
\ensuremath{\zeta}:[x,y,z]\$
\end{minipage}


\noindent
%%%%%%%%%%%%%%%
%%% INPUT:
\begin{minipage}[t]{8ex}\color{red}\bf
(\%{}i68) 
\end{minipage}
\begin{minipage}[t]{\textwidth}\color{blue}\tt
scalefactors(\ensuremath{\zeta})\$
\end{minipage}


\noindent
%%%%%%%%%%%%%%%
%%% INPUT:
\begin{minipage}[t]{8ex}\color{red}\bf
(\%{}i69) 
\end{minipage}
\begin{minipage}[t]{\textwidth}\color{blue}\tt
init\_cartan(\ensuremath{\zeta})\$
\end{minipage}

\textbf{Define frame} $\mathbf{U}\in\mathbb{R}^3$



\noindent
%%%%%%%%%%%%%%%
%%% INPUT:
\begin{minipage}[t]{8ex}\color{red}\bf
(\%{}i70) 
\end{minipage}
\begin{minipage}[t]{\textwidth}\color{blue}\tt
U:[U\_1,U\_2,U\_3]\$
\end{minipage}

\textbf{Define} $\mathbf{W}\in\mathfrak{X}(\mathbb{R}^3)$



\noindent
%%%%%%%%%%%%%%%
%%% INPUT:
\begin{minipage}[t]{8ex}\color{red}\bf
(\%{}i71) 
\end{minipage}
\begin{minipage}[t]{\textwidth}\color{blue}\tt
ldisplay(W:y*z*U\_1+x*y*U\_3)\$
\end{minipage}
%%% OUTPUT:
\[\displaystyle
\tag{\%{}t71}\label{t71} 
W={{U}_{1}}yz+{{U}_{3}}xy\mbox{}
\]
%%%%%%%%%%%%%%%


\noindent
%%%%%%%%%%%%%%%
%%% INPUT:
\begin{minipage}[t]{8ex}\color{red}\bf
(\%{}i72) 
\end{minipage}
\begin{minipage}[t]{\textwidth}\color{blue}\tt
ldisplay(W:makelist(coeff(W,k),k,U))\$
\end{minipage}
%%% OUTPUT:
\[\displaystyle
\tag{\%{}t72}\label{t72} 
W=[yz,0,xy]\mbox{}
\]
%%%%%%%%%%%%%%%

\textbf{Define} $\mathbf{v}\in\mathfrak{X}(\mathbb{R}^3)$



\noindent
%%%%%%%%%%%%%%%
%%% INPUT:
\begin{minipage}[t]{8ex}\color{red}\bf
(\%{}i73) 
\end{minipage}
\begin{minipage}[t]{\textwidth}\color{blue}\tt
ldisplay(v:x*U\_1+y\ensuremath{^2}*U\_2+z\ensuremath{^3}*U\_3)\$
\end{minipage}
%%% OUTPUT:
\[\displaystyle
\tag{\%{}t73}\label{t73} 
v={{U}_{3}}{{z}^{3}}+{{U}_{2}}{{y}^{2}}+{{U}_{1}}x\mbox{}
\]
%%%%%%%%%%%%%%%


\noindent
%%%%%%%%%%%%%%%
%%% INPUT:
\begin{minipage}[t]{8ex}\color{red}\bf
(\%{}i74) 
\end{minipage}
\begin{minipage}[t]{\textwidth}\color{blue}\tt
ldisplay(v:makelist(coeff(v,k),k,U))\$
\end{minipage}
%%% OUTPUT:
\[\displaystyle
\tag{\%{}t74}\label{t74} 
v=[x,{{y}^{2}},{{z}^{3}}]\mbox{}
\]
%%%%%%%%%%%%%%%

\textbf{Calculate} $\nabla\mathbf{W}$ Matrix



\noindent
%%%%%%%%%%%%%%%
%%% INPUT:
\begin{minipage}[t]{8ex}\color{red}\bf
(\%{}i75) 
\end{minipage}
\begin{minipage}[t]{\textwidth}\color{blue}\tt
gradW:apply('matrix,ev(express(grad(W)),diff));
\end{minipage}
%%% OUTPUT:
\[\displaystyle
\tag{gradW}\label{gradW}
\begin{pmatrix}0 & 0 & y\\
z & 0 & x\\
y & 0 & 0\end{pmatrix}\mbox{}
\]
%%%%%%%%%%%%%%%

\textbf{Calculate} $\mathbf{DW1}\in\mathfrak{X}(\mathbb{R}^3)$



\noindent
%%%%%%%%%%%%%%%
%%% INPUT:
\begin{minipage}[t]{8ex}\color{red}\bf
(\%{}i76) 
\end{minipage}
\begin{minipage}[t]{\textwidth}\color{blue}\tt
ldisplay(DW:list\_matrix\_entries(v.gradW))\$
\end{minipage}
%%% OUTPUT:
\[\displaystyle
\tag{\%{}t76}\label{t76} 
\mathit{DW}=[y\,{{z}^{3}}+{{y}^{2}}z,0,x\,{{y}^{2}}+xy]\mbox{}
\]
%%%%%%%%%%%%%%%


\noindent
%%%%%%%%%%%%%%%
%%% INPUT:
\begin{minipage}[t]{8ex}\color{red}\bf
(\%{}i77) 
\end{minipage}
\begin{minipage}[t]{\textwidth}\color{blue}\tt
ldisplay(DW:DW.U)\$
\end{minipage}
%%% OUTPUT:
\[\displaystyle
\tag{\%{}t77}\label{t77} 
\mathit{DW}={{U}_{1}}\left( y\,{{z}^{3}}+{{y}^{2}}z\right) +{{U}_{3}}\left( x\,{{y}^{2}}+xy\right) \mbox{}
\]
%%%%%%%%%%%%%%%

\subsection{Example 2.5.7.}


Calculate $\nabla_V\,V$.



\noindent
%%%%%%%%%%%%%%%
%%% INPUT:
\begin{minipage}[t]{8ex}\color{red}\bf
(\%{}i78) 
\end{minipage}
\begin{minipage}[t]{\textwidth}\color{blue}\tt
gradV:apply('matrix,ev(express(grad(v)),diff));
\end{minipage}
%%% OUTPUT:
\[\displaystyle
\tag{gradV}\label{gradV}
\begin{pmatrix}1 & 0 & 0\\
0 & 2y & 0\\
0 & 0 & 3{{z}^{2}}\end{pmatrix}\mbox{}
\]
%%%%%%%%%%%%%%%


\noindent
%%%%%%%%%%%%%%%
%%% INPUT:
\begin{minipage}[t]{8ex}\color{red}\bf
(\%{}i79) 
\end{minipage}
\begin{minipage}[t]{\textwidth}\color{blue}\tt
ldisplay(DV:list\_matrix\_entries(v.gradV))\$
\end{minipage}
%%% OUTPUT:
\[\displaystyle
\tag{\%{}t79}\label{t79} 
\mathit{DV}=[x,2{{y}^{3}},3{{z}^{5}}]\mbox{}
\]
%%%%%%%%%%%%%%%


\noindent
%%%%%%%%%%%%%%%
%%% INPUT:
\begin{minipage}[t]{8ex}\color{red}\bf
(\%{}i80) 
\end{minipage}
\begin{minipage}[t]{\textwidth}\color{blue}\tt
ldisplay(DV:DV.U)\$
\end{minipage}
%%% OUTPUT:
\[\displaystyle
\tag{\%{}t80}\label{t80} 
\mathit{DV}=3{{U}_{3}}{{z}^{5}}+2{{U}_{2}}{{y}^{3}}+{{U}_{1}}x\mbox{}
\]
%%%%%%%%%%%%%%%


\noindent
%%%%%%%%%%%%%%%
%%% INPUT:
\begin{minipage}[t]{8ex}\color{red}\bf
(\%{}i81) 
\end{minipage}
\begin{minipage}[t]{\textwidth}\color{blue}\tt
v.ev(express(grad(norm(v))),diff);
\end{minipage}
%%% OUTPUT:
\[\displaystyle
\tag{\%{}o81}\label{o81} 
\frac{3{{z}^{8}}}{\sqrt{{{z}^{6}}+{{y}^{4}}+{{x}^{2}}}}+\frac{2{{y}^{5}}}{\sqrt{{{z}^{6}}+{{y}^{4}}+{{x}^{2}}}}+\frac{{{x}^{2}}}{\sqrt{{{z}^{6}}+{{y}^{4}}+{{x}^{2}}}}\mbox{}
\]
%%%%%%%%%%%%%%%
\pagebreak


\section{frames and connection forms}


\subsection{Definition 2.6.1.}


\textbf{Connection forms}\\
If $E_1,E_2,E_3$ is a frame for $\mathbb{R}^3$ then define
$\omega_{i\,j}(p)\in\left(T_p\mathbb{R}^3\right)^\ast$ by
$$\omega_{i\,j}(v)=\left({\nabla_v\,E_i}\right)\cdot\,E_j(p)$$
for each $v\in\,T_p\mathbb{R}^3$. That is, $\omega_{i\,j}$ is a
differential one-form on $\mathbb{R}^3$ defined by the assignment
$p\rightarrow\omega_{i\,j}(p)$ for each $p\in\mathbb{R}^3$.


\subsection{Proposition 2.6.2.}


\textbf{Properties of the covariant derivative on} $\mathbb{R}^3$\\
Let $\lbrace{E_1,E_2,E_3}\rbrace$ be a frame on $\mathbb{R}^3$ then
$\omega_{i\,j}=-\omega_{j\,i}$ and $\nabla_v\,E_i=\sum_{j=1}^3
\omega_{i\,j}(V)\,E_j$


\subsection{Example 2.6.4.}


Let $A=\begin{pmatrix} \mathrm{d}x & \mathrm{d}y \\
\mathrm{d}z & z^2\,\mathrm{d}y+y^2\,\mathrm{d}z \end{pmatrix}$ and
$B=\begin{pmatrix} \mathrm{d}x+\mathrm{d}y & 0 \\ z^2\,\mathrm{d}y
& \,\mathrm{d}x+\mathrm{d}z \end{pmatrix}$


\textbf{Define the space} $\mathbb{R}^3$



\noindent
%%%%%%%%%%%%%%%
%%% INPUT:
\begin{minipage}[t]{8ex}\color{red}\bf
(\%{}i82) 
\end{minipage}
\begin{minipage}[t]{\textwidth}\color{blue}\tt
\ensuremath{\zeta}:[x,y,z]\$
\end{minipage}


\noindent
%%%%%%%%%%%%%%%
%%% INPUT:
\begin{minipage}[t]{8ex}\color{red}\bf
(\%{}i83) 
\end{minipage}
\begin{minipage}[t]{\textwidth}\color{blue}\tt
scalefactors(\ensuremath{\zeta})\$
\end{minipage}


\noindent
%%%%%%%%%%%%%%%
%%% INPUT:
\begin{minipage}[t]{8ex}\color{red}\bf
(\%{}i84) 
\end{minipage}
\begin{minipage}[t]{\textwidth}\color{blue}\tt
init\_cartan(\ensuremath{\zeta})\$
\end{minipage}


\noindent
%%%%%%%%%%%%%%%
%%% INPUT:
\begin{minipage}[t]{8ex}\color{red}\bf
(\%{}i85) 
\end{minipage}
\begin{minipage}[t]{\textwidth}\color{blue}\tt
matrix\_element\_mult:"\ensuremath{\sim }"\$
\end{minipage}


\noindent
%%%%%%%%%%%%%%%
%%% INPUT:
\begin{minipage}[t]{8ex}\color{red}\bf
(\%{}i86) 
\end{minipage}
\begin{minipage}[t]{\textwidth}\color{blue}\tt
ldisplay(A:matrix([dx,dy],[dz,z\ensuremath{^2}*dy+y\ensuremath{^2}*dz]))\$
\end{minipage}
%%% OUTPUT:
\[\displaystyle
\tag{\%{}t86}\label{t86} 
A=\begin{pmatrix}\mathit{dx} & \mathit{dy}\\
\mathit{dz} & {{y}^{2}}\,\mathit{dz}+{{z}^{2}}\,\mathit{dy}\end{pmatrix}\mbox{}
\]
%%%%%%%%%%%%%%%


\noindent
%%%%%%%%%%%%%%%
%%% INPUT:
\begin{minipage}[t]{8ex}\color{red}\bf
(\%{}i87) 
\end{minipage}
\begin{minipage}[t]{\textwidth}\color{blue}\tt
ldisplay(B:matrix([dx+dy,0],[z\ensuremath{^2}*dy,dx+dz]))\$
\end{minipage}
%%% OUTPUT:
\[\displaystyle
\tag{\%{}t87}\label{t87} 
B=\begin{pmatrix}\mathit{dy}+\mathit{dx} & 0\\
{{z}^{2}}\,\mathit{dy} & \mathit{dz}+\mathit{dx}\end{pmatrix}\mbox{}
\]
%%%%%%%%%%%%%%%

\textbf{Calculate} $A\wedge B$



\noindent
%%%%%%%%%%%%%%%
%%% INPUT:
\begin{minipage}[t]{8ex}\color{red}\bf
(\%{}i88) 
\end{minipage}
\begin{minipage}[t]{\textwidth}\color{blue}\tt
A.B;
\end{minipage}
%%% OUTPUT:
\[\displaystyle
\tag{\%{}o88}\label{o88} 
\begin{pmatrix}\mathit{dx}\,\mathit{dy} & \mathit{dy}\,\mathit{dz}-\mathit{dx}\,\mathit{dy}\\
-{{y}^{2}}\,{{z}^{2}}\,\mathit{dy}\,\mathit{dz}-\mathit{dy}\,\mathit{dz}-\mathit{dx}\,\mathit{dz} & {{z}^{2}}\,\mathit{dy}\,\mathit{dz}-{{y}^{2}}\,\mathit{dx}\,\mathit{dz}-{{z}^{2}}\,\mathit{dx}\,\mathit{dy}\end{pmatrix}\mbox{}
\]
%%%%%%%%%%%%%%%


\noindent
%%%%%%%%%%%%%%%
%%% INPUT:
\begin{minipage}[t]{8ex}\color{red}\bf
(\%{}i89) 
\end{minipage}
\begin{minipage}[t]{\textwidth}\color{blue}\tt
ldisplay(dA:factor(ext\_diff(A)))\$
\end{minipage}
%%% OUTPUT:
\[\displaystyle
\tag{\%{}t89}\label{t89} 
\mathit{dA}=\begin{pmatrix}0 & 0\\
0 & -2\left( z-y\right) \,\mathit{dy}\,\mathit{dz}\end{pmatrix}\mbox{}
\]
%%%%%%%%%%%%%%%


\noindent
%%%%%%%%%%%%%%%
%%% INPUT:
\begin{minipage}[t]{8ex}\color{red}\bf
(\%{}i90) 
\end{minipage}
\begin{minipage}[t]{\textwidth}\color{blue}\tt
ldisplay(dB:factor(ext\_diff(B)))\$
\end{minipage}
%%% OUTPUT:
\[\displaystyle
\tag{\%{}t90}\label{t90} 
\mathit{dB}=\begin{pmatrix}0 & 0\\
-2z\,\mathit{dy}\,\mathit{dz} & 0\end{pmatrix}\mbox{}
\]
%%%%%%%%%%%%%%%

\subsection{Proposition 2.6.5.}


\textbf{Product rule for matrices of forms}\\
Let $A$ be a matrix of $p$-forma and $B$ a matrix of $q$-forms and
suppose that $A,B$ are multipliable then $$\mathrm{d}(A\wedge B)=
\mathrm{d}A\wedge B+(-1)^p A\wedge\mathrm{d}B$$


\subsection{Proposition 2.6.6.}


\textbf{Attitude matrix}\\
Let $A$ be the attitude matrix of a given frame then
$$\mathrm{d}A^T\wedge A=-A^T\wedge\mathrm{d}A,\quad
\mathrm{d}A\wedge A^T=-A\wedge\mathrm{d}A^T$$
Moreover, $\mathrm{d}A=-A\wedge\mathrm{d}A^T\wedge A$


\subsection{Example 2.6.8.}


Following Examples 2.2.9 and 2.2.15, the
\textbf{cylindrical coordinate frame} has attitude matrix:



\noindent
%%%%%%%%%%%%%%%
%%% INPUT:
\begin{minipage}[t]{8ex}\color{red}\bf
(\%{}i91) 
\end{minipage}
\begin{minipage}[t]{\textwidth}\color{blue}\tt
kill(labels,t,x,y,z,r,\ensuremath{\theta})\$
\end{minipage}

\textbf{Define the space} $\mathbb{R}^3$



\noindent
%%%%%%%%%%%%%%%
%%% INPUT:
\begin{minipage}[t]{8ex}\color{red}\bf
(\%{}i1) 
\end{minipage}
\begin{minipage}[t]{\textwidth}\color{blue}\tt
\ensuremath{\xi}:[r,\ensuremath{\theta},z]\$
\end{minipage}


\noindent
%%%%%%%%%%%%%%%
%%% INPUT:
\begin{minipage}[t]{8ex}\color{red}\bf
(\%{}i2) 
\end{minipage}
\begin{minipage}[t]{\textwidth}\color{blue}\tt
init\_cartan(\ensuremath{\xi})\$
\end{minipage}


\noindent
%%%%%%%%%%%%%%%
%%% INPUT:
\begin{minipage}[t]{8ex}\color{red}\bf
(\%{}i3) 
\end{minipage}
\begin{minipage}[t]{\textwidth}\color{blue}\tt
matrix\_element\_mult:"\ensuremath{\sim }"\$
\end{minipage}


\noindent
%%%%%%%%%%%%%%%
%%% INPUT:
\begin{minipage}[t]{8ex}\color{red}\bf
(\%{}i4) 
\end{minipage}
\begin{minipage}[t]{\textwidth}\color{blue}\tt
A:matrix([cos(\ensuremath{\theta}),sin(\ensuremath{\theta}),0],[-sin(\ensuremath{\theta}),cos(\ensuremath{\theta}),0],[0,0,1]);
\end{minipage}
%%% OUTPUT:
\[\displaystyle
\tag{A}\label{A}
\begin{pmatrix}\cos{\left( \mathit{\ensuremath{\theta}}\right) } & \sin{\left( \mathit{\ensuremath{\theta}}\right) } & 0\\
-\sin{\left( \mathit{\ensuremath{\theta}}\right) } & \cos{\left( \mathit{\ensuremath{\theta}}\right) } & 0\\
0 & 0 & 1\end{pmatrix}\mbox{}
\]
%%%%%%%%%%%%%%%


\noindent
%%%%%%%%%%%%%%%
%%% INPUT:
\begin{minipage}[t]{8ex}\color{red}\bf
(\%{}i5) 
\end{minipage}
\begin{minipage}[t]{\textwidth}\color{blue}\tt
ldisplay(dA:ext\_diff(A))\$
\end{minipage}
%%% OUTPUT:
\[\displaystyle
\tag{\%{}t5}\label{t5} 
\mathit{dA}=\begin{pmatrix}-\mathit{d\ensuremath{\theta}}\,\sin{\left( \mathit{\ensuremath{\theta}}\right) } & \mathit{d\ensuremath{\theta}}\,\cos{\left( \mathit{\ensuremath{\theta}}\right) } & 0\\
-\mathit{d\ensuremath{\theta}}\,\cos{\left( \mathit{\ensuremath{\theta}}\right) } & -\mathit{d\ensuremath{\theta}}\,\sin{\left( \mathit{\ensuremath{\theta}}\right) } & 0\\
0 & 0 & 0\end{pmatrix}\mbox{}
\]
%%%%%%%%%%%%%%%


\noindent
%%%%%%%%%%%%%%%
%%% INPUT:
\begin{minipage}[t]{8ex}\color{red}\bf
(\%{}i6) 
\end{minipage}
\begin{minipage}[t]{\textwidth}\color{blue}\tt
ldisplay(\ensuremath{\omega}:trigsimp(dA.transpose(A)))\$
\end{minipage}
%%% OUTPUT:
\[\displaystyle
\tag{\%{}t6}\label{t6} 
\mathit{\ensuremath{\omega}}=\begin{pmatrix}0 & \mathit{d\ensuremath{\theta}} & 0\\
-\mathit{d\ensuremath{\theta}} & 0 & 0\\
0 & 0 & 0\end{pmatrix}\mbox{}
\]
%%%%%%%%%%%%%%%


\noindent
%%%%%%%%%%%%%%%
%%% INPUT:
\begin{minipage}[t]{8ex}\color{red}\bf
(\%{}i7) 
\end{minipage}
\begin{minipage}[t]{\textwidth}\color{blue}\tt
matrix\_element\_mult:"*"\$
\end{minipage}

\subsection{Example 2.6.9.}



\noindent
%%%%%%%%%%%%%%%
%%% INPUT:
\begin{minipage}[t]{8ex}\color{red}\bf
(\%{}i8) 
\end{minipage}
\begin{minipage}[t]{\textwidth}\color{blue}\tt
kill(labels,t,x,y,z,\ensuremath{\rho},\ensuremath{\theta},\ensuremath{\phi})\$
\end{minipage}

Following Example 2.2.10 and 2.2.16, the
\textbf{spherical coordinate frame} has attitude matrix:



\noindent
%%%%%%%%%%%%%%%
%%% INPUT:
\begin{minipage}[t]{8ex}\color{red}\bf
(\%{}i1) 
\end{minipage}
\begin{minipage}[t]{\textwidth}\color{blue}\tt
\ensuremath{\xi}:[\ensuremath{\rho},\ensuremath{\theta},\ensuremath{\phi}]\$
\end{minipage}


\noindent
%%%%%%%%%%%%%%%
%%% INPUT:
\begin{minipage}[t]{8ex}\color{red}\bf
(\%{}i2) 
\end{minipage}
\begin{minipage}[t]{\textwidth}\color{blue}\tt
init\_cartan(\ensuremath{\xi})\$
\end{minipage}


\noindent
%%%%%%%%%%%%%%%
%%% INPUT:
\begin{minipage}[t]{8ex}\color{red}\bf
(\%{}i3) 
\end{minipage}
\begin{minipage}[t]{\textwidth}\color{blue}\tt
matrix\_element\_mult:"\ensuremath{\sim }"\$
\end{minipage}


\noindent
%%%%%%%%%%%%%%%
%%% INPUT:
\begin{minipage}[t]{8ex}\color{red}\bf
(\%{}i4) 
\end{minipage}
\begin{minipage}[t]{\textwidth}\color{blue}\tt
ldisplay(A:matrix([cos(\ensuremath{\theta})*sin(\ensuremath{\phi}),sin(\ensuremath{\theta})*sin(\ensuremath{\phi}),cos(\ensuremath{\phi})],
                  [cos(\ensuremath{\theta})*cos(\ensuremath{\phi}),sin(\ensuremath{\theta})*cos(\ensuremath{\phi}),-sin(\ensuremath{\phi})],
                  [-sin(\ensuremath{\theta}),cos(\ensuremath{\theta}),0]))\$
\end{minipage}
%%% OUTPUT:
\[\displaystyle
\tag{\%{}t4}\label{t4} 
A=\begin{pmatrix}\cos{\left( \mathit{\ensuremath{\theta}}\right) }\,\sin{\left( \mathit{\ensuremath{\phi}}\right) } & \sin{\left( \mathit{\ensuremath{\theta}}\right) }\,\sin{\left( \mathit{\ensuremath{\phi}}\right) } & \cos{\left( \mathit{\ensuremath{\phi}}\right) }\\
\cos{\left( \mathit{\ensuremath{\theta}}\right) }\,\cos{\left( \mathit{\ensuremath{\phi}}\right) } & \sin{\left( \mathit{\ensuremath{\theta}}\right) }\,\cos{\left( \mathit{\ensuremath{\phi}}\right) } & -\sin{\left( \mathit{\ensuremath{\phi}}\right) }\\
-\sin{\left( \mathit{\ensuremath{\theta}}\right) } & \cos{\left( \mathit{\ensuremath{\theta}}\right) } & 0\end{pmatrix}\mbox{}
\]
%%%%%%%%%%%%%%%


\noindent
%%%%%%%%%%%%%%%
%%% INPUT:
\begin{minipage}[t]{8ex}\color{red}\bf
(\%{}i5) 
\end{minipage}
\begin{minipage}[t]{\textwidth}\color{blue}\tt
ldisplay(dA:ext\_diff(A))\$
\end{minipage}
%%% OUTPUT:
\[\displaystyle
\tag{\%{}t5}\label{t5} 
\mathit{dA}=\begin{pmatrix}\mathit{d\ensuremath{\phi}}\,\cos{\left( \mathit{\ensuremath{\theta}}\right) }\,\cos{\left( \mathit{\ensuremath{\phi}}\right) }-\mathit{d\ensuremath{\theta}}\,\sin{\left( \mathit{\ensuremath{\theta}}\right) }\,\sin{\left( \mathit{\ensuremath{\phi}}\right) } & \mathit{d\ensuremath{\theta}}\,\cos{\left( \mathit{\ensuremath{\theta}}\right) }\,\sin{\left( \mathit{\ensuremath{\phi}}\right) }+\mathit{d\ensuremath{\phi}}\,\sin{\left( \mathit{\ensuremath{\theta}}\right) }\,\cos{\left( \mathit{\ensuremath{\phi}}\right) } & -\mathit{d\ensuremath{\phi}}\,\sin{\left( \mathit{\ensuremath{\phi}}\right) }\\
-\mathit{d\ensuremath{\phi}}\,\cos{\left( \mathit{\ensuremath{\theta}}\right) }\,\sin{\left( \mathit{\ensuremath{\phi}}\right) }-\mathit{d\ensuremath{\theta}}\,\sin{\left( \mathit{\ensuremath{\theta}}\right) }\,\cos{\left( \mathit{\ensuremath{\phi}}\right) } & \mathit{d\ensuremath{\theta}}\,\cos{\left( \mathit{\ensuremath{\theta}}\right) }\,\cos{\left( \mathit{\ensuremath{\phi}}\right) }-\mathit{d\ensuremath{\phi}}\,\sin{\left( \mathit{\ensuremath{\theta}}\right) }\,\sin{\left( \mathit{\ensuremath{\phi}}\right) } & -\mathit{d\ensuremath{\phi}}\,\cos{\left( \mathit{\ensuremath{\phi}}\right) }\\
-\mathit{d\ensuremath{\theta}}\,\cos{\left( \mathit{\ensuremath{\theta}}\right) } & -\mathit{d\ensuremath{\theta}}\,\sin{\left( \mathit{\ensuremath{\theta}}\right) } & 0\end{pmatrix}\mbox{}
\]
%%%%%%%%%%%%%%%


\noindent
%%%%%%%%%%%%%%%
%%% INPUT:
\begin{minipage}[t]{8ex}\color{red}\bf
(\%{}i6) 
\end{minipage}
\begin{minipage}[t]{\textwidth}\color{blue}\tt
coeff(dA,d\ensuremath{\theta});
\end{minipage}
%%% OUTPUT:
\[\displaystyle
\tag{\%{}o6}\label{o6} 
\begin{pmatrix}-\sin{\left( \mathit{\ensuremath{\theta}}\right) }\,\sin{\left( \mathit{\ensuremath{\phi}}\right) } & \cos{\left( \mathit{\ensuremath{\theta}}\right) }\,\sin{\left( \mathit{\ensuremath{\phi}}\right) } & 0\\
-\sin{\left( \mathit{\ensuremath{\theta}}\right) }\,\cos{\left( \mathit{\ensuremath{\phi}}\right) } & \cos{\left( \mathit{\ensuremath{\theta}}\right) }\,\cos{\left( \mathit{\ensuremath{\phi}}\right) } & 0\\
-\cos{\left( \mathit{\ensuremath{\theta}}\right) } & -\sin{\left( \mathit{\ensuremath{\theta}}\right) } & 0\end{pmatrix}\mbox{}
\]
%%%%%%%%%%%%%%%


\noindent
%%%%%%%%%%%%%%%
%%% INPUT:
\begin{minipage}[t]{8ex}\color{red}\bf
(\%{}i7) 
\end{minipage}
\begin{minipage}[t]{\textwidth}\color{blue}\tt
coeff(dA,d\ensuremath{\phi});
\end{minipage}
%%% OUTPUT:
\[\displaystyle
\tag{\%{}o7}\label{o7} 
\begin{pmatrix}\cos{\left( \mathit{\ensuremath{\theta}}\right) }\,\cos{\left( \mathit{\ensuremath{\phi}}\right) } & \sin{\left( \mathit{\ensuremath{\theta}}\right) }\,\cos{\left( \mathit{\ensuremath{\phi}}\right) } & -\sin{\left( \mathit{\ensuremath{\phi}}\right) }\\
-\cos{\left( \mathit{\ensuremath{\theta}}\right) }\,\sin{\left( \mathit{\ensuremath{\phi}}\right) } & -\sin{\left( \mathit{\ensuremath{\theta}}\right) }\,\sin{\left( \mathit{\ensuremath{\phi}}\right) } & -\cos{\left( \mathit{\ensuremath{\phi}}\right) }\\
0 & 0 & 0\end{pmatrix}\mbox{}
\]
%%%%%%%%%%%%%%%


\noindent
%%%%%%%%%%%%%%%
%%% INPUT:
\begin{minipage}[t]{8ex}\color{red}\bf
(\%{}i8) 
\end{minipage}
\begin{minipage}[t]{\textwidth}\color{blue}\tt
ldisplay(\ensuremath{\omega}:trigsimp(dA.transpose(A)))\$
\end{minipage}
%%% OUTPUT:
\[\displaystyle
\tag{\%{}t8}\label{t8} 
\mathit{\ensuremath{\omega}}=\begin{pmatrix}0 & \mathit{d\ensuremath{\phi}} & \mathit{d\ensuremath{\theta}}\,\sin{\left( \mathit{\ensuremath{\phi}}\right) }\\
-\mathit{d\ensuremath{\phi}} & 0 & \mathit{d\ensuremath{\theta}}\,\cos{\left( \mathit{\ensuremath{\phi}}\right) }\\
-\mathit{d\ensuremath{\theta}}\,\sin{\left( \mathit{\ensuremath{\phi}}\right) } & -\mathit{d\ensuremath{\theta}}\,\cos{\left( \mathit{\ensuremath{\phi}}\right) } & 0\end{pmatrix}\mbox{}
\]
%%%%%%%%%%%%%%%


\noindent
%%%%%%%%%%%%%%%
%%% INPUT:
\begin{minipage}[t]{8ex}\color{red}\bf
(\%{}i9) 
\end{minipage}
\begin{minipage}[t]{\textwidth}\color{blue}\tt
matrix\_element\_mult:"*"\$
\end{minipage}
\pagebreak


\section{coframes and Cartan Structure Equations}


\subsection{Definition 2.7.1.}


\textbf{Coframe on} $\mathbb{R}^3$\\
Suppose $\lbrace{E_1,E_2,E_3}\rbrace$ is a frame on $\mathbb{R}^3$
then we say a set of differential one-forms $\lbrace{\theta^1,
\theta_2,\theta_3}\rbrace$ on $\mathbb{R}^3$ is a \textbf{coframe}
if $\theta^i(E_j)=\delta_{i\,j}$ for all $i,j$.


\subsection{Example 2.7.2.}



\noindent
%%%%%%%%%%%%%%%
%%% INPUT:
\begin{minipage}[t]{8ex}\color{red}\bf
(\%{}i10) 
\end{minipage}
\begin{minipage}[t]{\textwidth}\color{blue}\tt
kill(labels,x,y,z)\$
\end{minipage}

\textbf{Define the space} $\mathbb{R}^3$



\noindent
%%%%%%%%%%%%%%%
%%% INPUT:
\begin{minipage}[t]{8ex}\color{red}\bf
(\%{}i1) 
\end{minipage}
\begin{minipage}[t]{\textwidth}\color{blue}\tt
\ensuremath{\zeta}:[x,y,z]\$
\end{minipage}


\noindent
%%%%%%%%%%%%%%%
%%% INPUT:
\begin{minipage}[t]{8ex}\color{red}\bf
(\%{}i2) 
\end{minipage}
\begin{minipage}[t]{\textwidth}\color{blue}\tt
scalefactors(\ensuremath{\zeta})\$
\end{minipage}


\noindent
%%%%%%%%%%%%%%%
%%% INPUT:
\begin{minipage}[t]{8ex}\color{red}\bf
(\%{}i3) 
\end{minipage}
\begin{minipage}[t]{\textwidth}\color{blue}\tt
init\_cartan(\ensuremath{\zeta})\$
\end{minipage}


\noindent
%%%%%%%%%%%%%%%
%%% INPUT:
\begin{minipage}[t]{8ex}\color{red}\bf
(\%{}i4) 
\end{minipage}
\begin{minipage}[t]{\textwidth}\color{blue}\tt
U:[U\_1,U\_2,U\_3]\$
\end{minipage}

\textbf{Define array function of two arguments} $h$



\noindent
%%%%%%%%%%%%%%%
%%% INPUT:
\begin{minipage}[t]{8ex}\color{red}\bf
(\%{}i5) 
\end{minipage}
\begin{minipage}[t]{\textwidth}\color{blue}\tt
h[i,j]:=diff(\ensuremath{\zeta},\ensuremath{\zeta}[i])|concat(d,\ensuremath{\zeta}[j])\$
\end{minipage}


\noindent
%%%%%%%%%%%%%%%
%%% INPUT:
\begin{minipage}[t]{8ex}\color{red}\bf
(\%{}i6) 
\end{minipage}
\begin{minipage}[t]{\textwidth}\color{blue}\tt
genmatrix(h,cartan\_dim,cartan\_dim);
\end{minipage}
%%% OUTPUT:
\[\displaystyle
\tag{\%{}o6}\label{o6} 
\begin{pmatrix}1 & 0 & 0\\
0 & 1 & 0\\
0 & 0 & 1\end{pmatrix}\mbox{}
\]
%%%%%%%%%%%%%%%

\textbf{Define array function of two arguments} $g$



\noindent
%%%%%%%%%%%%%%%
%%% INPUT:
\begin{minipage}[t]{8ex}\color{red}\bf
(\%{}i7) 
\end{minipage}
\begin{minipage}[t]{\textwidth}\color{blue}\tt
g[i,j]:=diff(cartan\_coords,cartan\_coords[i])|cartan\_basis[j]\$
\end{minipage}


\noindent
%%%%%%%%%%%%%%%
%%% INPUT:
\begin{minipage}[t]{8ex}\color{red}\bf
(\%{}i8) 
\end{minipage}
\begin{minipage}[t]{\textwidth}\color{blue}\tt
genmatrix(g,cartan\_dim,cartan\_dim);
\end{minipage}
%%% OUTPUT:
\[\displaystyle
\tag{\%{}o8}\label{o8} 
\begin{pmatrix}1 & 0 & 0\\
0 & 1 & 0\\
0 & 0 & 1\end{pmatrix}\mbox{}
\]
%%%%%%%%%%%%%%%

\textbf{Define generic vector} $\mathbf{V}\in\mathfrak{X}(\mathbb{R}^3)$



\noindent
%%%%%%%%%%%%%%%
%%% INPUT:
\begin{minipage}[t]{8ex}\color{red}\bf
(\%{}i9) 
\end{minipage}
\begin{minipage}[t]{\textwidth}\color{blue}\tt
V:[V\_1,V\_2,V\_3]\$
\end{minipage}


\noindent
%%%%%%%%%%%%%%%
%%% INPUT:
\begin{minipage}[t]{8ex}\color{red}\bf
(\%{}i12) 
\end{minipage}
\begin{minipage}[t]{\textwidth}\color{blue}\tt
V|dx;
V|dy;
V|dz;
\end{minipage}
%%% OUTPUT:
\[\displaystyle
\tag{\%{}o10}\label{o10} 
{{V}_{1}}\mbox{}\]
\[\tag{\%{}o11}\label{o11} 
{{V}_{2}}\mbox{}\]
\[\tag{\%{}o12}\label{o12} 
{{V}_{3}}\mbox{}
\]
%%%%%%%%%%%%%%%


\noindent
%%%%%%%%%%%%%%%
%%% INPUT:
\begin{minipage}[t]{8ex}\color{red}\bf
(\%{}i13) 
\end{minipage}
\begin{minipage}[t]{\textwidth}\color{blue}\tt
V:[V|dx,V|dy,V|dz];
\end{minipage}
%%% OUTPUT:
\[\displaystyle
\tag{V}\label{V}
[{{V}_{1}},{{V}_{2}},{{V}_{3}}]\mbox{}
\]
%%%%%%%%%%%%%%%

\subsection{Proposition 2.7.3.}


\textbf{Components with respect to frame and coframe}\\
If $\lbrace{E_1,E_2,E_3}\rbrace$ is a frame with coframe
$\lbrace{\theta^1,\theta^2,\theta^3}\rbrace$ if $Y\in\mathfrak{X}
(\mathbb{R}^3)$ and $\alpha\in\mathcal{A}^1(\mathbb{R}^3)$ then
$$Y=\sum_{j=1}^3\theta^j(Y)E_j,\quad
\alpha=\sum_{j=1}^3\alpha(E_j)\theta^j$$


\subsection{Proposition 2.7.4.}


\textbf{Attitude of coframe}\\
If $\lbrace{E_1,E_2,E_3}\rbrace$ is a frame with coframe
$\lbrace{\theta^1,\theta^2,\theta^3}\rbrace$ and
\ $\lbrace{U_1,U_2,U_3}\rbrace$ is the Cartesian frame with coframe
$\lbrace{\mathrm{d}x^1,\mathrm{d}x^2,\mathrm{d}x^3}\rbrace$ on
$\mathbb{R}^3$ then $$E_i=\sum_j A_{i\,j}U_j\Leftrightarrow
\theta^i=\sum_j A_{i\,j}\mathrm{d}x^j$$


\subsection{Theorem 2.7.5.}


\textbf{Cartan Structure Equations for} $\mathbb{R}^3$
If $E_i$ is a frame with coframe $theta^i$ and $\omega$ is the
connection form for the given frame then:
$$\mathrm{d}\theta^i=\sum_j\omega_{i\,j}\wedge\theta^j,\quad
\mathrm{d}\omega_{i\,j}=\sum_k\omega_{i\,k}\wedge\omega_{k,j}$$


\subsection{Example 2.7.6.}



\noindent
%%%%%%%%%%%%%%%
%%% INPUT:
\begin{minipage}[t]{8ex}\color{red}\bf
(\%{}i14) 
\end{minipage}
\begin{minipage}[t]{\textwidth}\color{blue}\tt
kill(labels,t,x,y,z,r,\ensuremath{\theta},U\_1,U\_2,U\_3,\ensuremath{\theta}\_1,\ensuremath{\theta}\_2,\ensuremath{\theta}\_3,E\_1,E\_2,E\_3)\$
\end{minipage}


\noindent
%%%%%%%%%%%%%%%
%%% INPUT:
\begin{minipage}[t]{8ex}\color{red}\bf
(\%{}i1) 
\end{minipage}
\begin{minipage}[t]{\textwidth}\color{blue}\tt
\ensuremath{\zeta}:[x,y,z]\$
\end{minipage}


\noindent
%%%%%%%%%%%%%%%
%%% INPUT:
\begin{minipage}[t]{8ex}\color{red}\bf
(\%{}i3) 
\end{minipage}
\begin{minipage}[t]{\textwidth}\color{blue}\tt
assume(0\ensuremath{\leq}r)\$\\
assume(0\ensuremath{\leq}\ensuremath{\theta},\ensuremath{\theta}\ensuremath{\leq}2*\ensuremath{\pi})\$
\end{minipage}


\noindent
%%%%%%%%%%%%%%%
%%% INPUT:
\begin{minipage}[t]{8ex}\color{red}\bf
(\%{}i4) 
\end{minipage}
\begin{minipage}[t]{\textwidth}\color{blue}\tt
\ensuremath{\xi}:[r,\ensuremath{\theta},z]\$
\end{minipage}

\textbf{Cartesian frame}



\noindent
%%%%%%%%%%%%%%%
%%% INPUT:
\begin{minipage}[t]{8ex}\color{red}\bf
(\%{}i5) 
\end{minipage}
\begin{minipage}[t]{\textwidth}\color{blue}\tt
U:[U\_1,U\_2,U\_3]\$
\end{minipage}

\textbf{Initialize cartan package}



\noindent
%%%%%%%%%%%%%%%
%%% INPUT:
\begin{minipage}[t]{8ex}\color{red}\bf
(\%{}i6) 
\end{minipage}
\begin{minipage}[t]{\textwidth}\color{blue}\tt
init\_cartan(\ensuremath{\xi})\$
\end{minipage}


\noindent
%%%%%%%%%%%%%%%
%%% INPUT:
\begin{minipage}[t]{8ex}\color{red}\bf
(\%{}i7) 
\end{minipage}
\begin{minipage}[t]{\textwidth}\color{blue}\tt
cartan\_basis;
\end{minipage}
%%% OUTPUT:
\[\displaystyle
\tag{\%{}o7}\label{o7} 
[\mathit{dr},\mathit{d\ensuremath{\theta}},\mathit{dz}]\mbox{}
\]
%%%%%%%%%%%%%%%


\noindent
%%%%%%%%%%%%%%%
%%% INPUT:
\begin{minipage}[t]{8ex}\color{red}\bf
(\%{}i8) 
\end{minipage}
\begin{minipage}[t]{\textwidth}\color{blue}\tt
cartan\_coords;
\end{minipage}
%%% OUTPUT:
\[\displaystyle
\tag{\%{}o8}\label{o8} 
[r,\mathit{\ensuremath{\theta}},z]\mbox{}
\]
%%%%%%%%%%%%%%%


\noindent
%%%%%%%%%%%%%%%
%%% INPUT:
\begin{minipage}[t]{8ex}\color{red}\bf
(\%{}i9) 
\end{minipage}
\begin{minipage}[t]{\textwidth}\color{blue}\tt
cartan\_dim;
\end{minipage}
%%% OUTPUT:
\[\displaystyle
\tag{\%{}o9}\label{o9} 
3\mbox{}
\]
%%%%%%%%%%%%%%%


\noindent
%%%%%%%%%%%%%%%
%%% INPUT:
\begin{minipage}[t]{8ex}\color{red}\bf
(\%{}i10) 
\end{minipage}
\begin{minipage}[t]{\textwidth}\color{blue}\tt
extdim;
\end{minipage}
%%% OUTPUT:
\[\displaystyle
\tag{\%{}o10}\label{o10} 
3\mbox{}
\]
%%%%%%%%%%%%%%%

\textbf{Transformation formulas}



\noindent
%%%%%%%%%%%%%%%
%%% INPUT:
\begin{minipage}[t]{8ex}\color{red}\bf
(\%{}i11) 
\end{minipage}
\begin{minipage}[t]{\textwidth}\color{blue}\tt
ldisplay(Tr:[r*cos(\ensuremath{\theta}),r*sin(\ensuremath{\theta}),z])\$
\end{minipage}
%%% OUTPUT:
\[\displaystyle
\tag{\%{}t11}\label{t11} 
\mathit{Tr}=[r\,\cos{\left( \mathit{\ensuremath{\theta}}\right) },r\,\sin{\left( \mathit{\ensuremath{\theta}}\right) },z]\mbox{}
\]
%%%%%%%%%%%%%%%

\textbf{Jacobian matrix}



\noindent
%%%%%%%%%%%%%%%
%%% INPUT:
\begin{minipage}[t]{8ex}\color{red}\bf
(\%{}i12) 
\end{minipage}
\begin{minipage}[t]{\textwidth}\color{blue}\tt
ldisplay(J:jacobian(Tr,\ensuremath{\xi}))\$
\end{minipage}
%%% OUTPUT:
\[\displaystyle
\tag{\%{}t12}\label{t12} 
J=\begin{pmatrix}\cos{\left( \mathit{\ensuremath{\theta}}\right) } & -r\,\sin{\left( \mathit{\ensuremath{\theta}}\right) } & 0\\
\sin{\left( \mathit{\ensuremath{\theta}}\right) } & r\,\cos{\left( \mathit{\ensuremath{\theta}}\right) } & 0\\
0 & 0 & 1\end{pmatrix}\mbox{}
\]
%%%%%%%%%%%%%%%

\textbf{Metric tensor}



\noindent
%%%%%%%%%%%%%%%
%%% INPUT:
\begin{minipage}[t]{8ex}\color{red}\bf
(\%{}i13) 
\end{minipage}
\begin{minipage}[t]{\textwidth}\color{blue}\tt
ldisplay(lg:trigsimp(transpose(J).J))\$
\end{minipage}
%%% OUTPUT:
\[\displaystyle
\tag{\%{}t13}\label{t13} 
\mathit{lg}=\begin{pmatrix}1 & 0 & 0\\
0 & {{r}^{2}} & 0\\
0 & 0 & 1\end{pmatrix}\mbox{}
\]
%%%%%%%%%%%%%%%

\textbf{Jacobian}



\noindent
%%%%%%%%%%%%%%%
%%% INPUT:
\begin{minipage}[t]{8ex}\color{red}\bf
(\%{}i14) 
\end{minipage}
\begin{minipage}[t]{\textwidth}\color{blue}\tt
ldisplay(Jdet:trigsimp(determinant(J)))\$
\end{minipage}
%%% OUTPUT:
\[\displaystyle
\tag{\%{}t14}\label{t14} 
\mathit{Jdet}=r\mbox{}
\]
%%%%%%%%%%%%%%%

\textbf{Initialize vect package}



\noindent
%%%%%%%%%%%%%%%
%%% INPUT:
\begin{minipage}[t]{8ex}\color{red}\bf
(\%{}i15) 
\end{minipage}
\begin{minipage}[t]{\textwidth}\color{blue}\tt
scalefactors(append([Tr],\ensuremath{\xi}))\$
\end{minipage}


\noindent
%%%%%%%%%%%%%%%
%%% INPUT:
\begin{minipage}[t]{8ex}\color{red}\bf
(\%{}i16) 
\end{minipage}
\begin{minipage}[t]{\textwidth}\color{blue}\tt
sf;
\end{minipage}
%%% OUTPUT:
\[\displaystyle
\tag{\%{}o16}\label{o16} 
[1,r,1]\mbox{}
\]
%%%%%%%%%%%%%%%


\noindent
%%%%%%%%%%%%%%%
%%% INPUT:
\begin{minipage}[t]{8ex}\color{red}\bf
(\%{}i17) 
\end{minipage}
\begin{minipage}[t]{\textwidth}\color{blue}\tt
sfprod;
\end{minipage}
%%% OUTPUT:
\[\displaystyle
\tag{\%{}o17}\label{o17} 
r\mbox{}
\]
%%%%%%%%%%%%%%%

\pagebreak
\textbf{Volume}



\noindent
%%%%%%%%%%%%%%%
%%% INPUT:
\begin{minipage}[t]{8ex}\color{red}\bf
(\%{}i18) 
\end{minipage}
\begin{minipage}[t]{\textwidth}\color{blue}\tt
[dx,dy,dz]:list\_matrix\_entries(J.cartan\_basis)\$
\end{minipage}


\noindent
%%%%%%%%%%%%%%%
%%% INPUT:
\begin{minipage}[t]{8ex}\color{red}\bf
(\%{}i19) 
\end{minipage}
\begin{minipage}[t]{\textwidth}\color{blue}\tt
dv:trigsimp(dx\ensuremath{\sim }dy\ensuremath{\sim }dz);
\end{minipage}
%%% OUTPUT:
\[\displaystyle
\tag{dv}\label{dv}
r\,\mathit{dr}\,\mathit{dz}\,\mathit{d\ensuremath{\theta}}\mbox{}
\]
%%%%%%%%%%%%%%%


\noindent
%%%%%%%%%%%%%%%
%%% INPUT:
\begin{minipage}[t]{8ex}\color{red}\bf
(\%{}i20) 
\end{minipage}
\begin{minipage}[t]{\textwidth}\color{blue}\tt
diff(\ensuremath{\xi},z)|(diff(\ensuremath{\xi},\ensuremath{\theta})|(diff(\ensuremath{\xi},r)|dv));
\end{minipage}
%%% OUTPUT:
\[\displaystyle
\tag{\%{}o20}\label{o20} 
r\mbox{}
\]
%%%%%%%%%%%%%%%


\noindent
%%%%%%%%%%%%%%%
%%% INPUT:
\begin{minipage}[t]{8ex}\color{red}\bf
(\%{}i21) 
\end{minipage}
\begin{minipage}[t]{\textwidth}\color{blue}\tt
ldisplay(d\ensuremath{\zeta}:trigsimp(ext\_diff(at(\ensuremath{\zeta},map("=",\ensuremath{\zeta},Tr)))))\$
\end{minipage}
%%% OUTPUT:
\[\displaystyle
\tag{\%{}t21}\label{t21} 
\mathit{d\ensuremath{\zeta}}=[\mathit{dr}\,\cos{\left( \mathit{\ensuremath{\theta}}\right) }-r\,\mathit{d\ensuremath{\theta}}\,\sin{\left( \mathit{\ensuremath{\theta}}\right) },\mathit{dr}\,\sin{\left( \mathit{\ensuremath{\theta}}\right) }+r\,\mathit{d\ensuremath{\theta}}\,\cos{\left( \mathit{\ensuremath{\theta}}\right) },\mathit{dz}]\mbox{}
\]
%%%%%%%%%%%%%%%

\textbf{Attitude matrix}



\noindent
%%%%%%%%%%%%%%%
%%% INPUT:
\begin{minipage}[t]{8ex}\color{red}\bf
(\%{}i22) 
\end{minipage}
\begin{minipage}[t]{\textwidth}\color{blue}\tt
ldisplay(A:apply('matrix,makelist(trigsimp(normalize(k)),k,args(transpose(J)))))\$
\end{minipage}
%%% OUTPUT:
\[\displaystyle
\tag{\%{}t22}\label{t22} 
A=\begin{pmatrix}\cos{\left( \mathit{\ensuremath{\theta}}\right) } & \sin{\left( \mathit{\ensuremath{\theta}}\right) } & 0\\
-\sin{\left( \mathit{\ensuremath{\theta}}\right) } & \cos{\left( \mathit{\ensuremath{\theta}}\right) } & 0\\
0 & 0 & 1\end{pmatrix}\mbox{}
\]
%%%%%%%%%%%%%%%

\textbf{Frame}



\noindent
%%%%%%%%%%%%%%%
%%% INPUT:
\begin{minipage}[t]{8ex}\color{red}\bf
(\%{}i23) 
\end{minipage}
\begin{minipage}[t]{\textwidth}\color{blue}\tt
E:[E\_1,E\_2,E\_3]:trigsimp(list\_matrix\_entries(A.U))\$
\end{minipage}


\noindent
%%%%%%%%%%%%%%%
%%% INPUT:
\begin{minipage}[t]{8ex}\color{red}\bf
(\%{}i24) 
\end{minipage}
\begin{minipage}[t]{\textwidth}\color{blue}\tt
map(ldisp,E)\$
\end{minipage}
%%% OUTPUT:
\[\displaystyle
\tag{\%{}t24}\label{t24} 
{{U}_{2}}\sin{\left( \mathit{\ensuremath{\theta}}\right) }+{{U}_{1}}\cos{\left( \mathit{\ensuremath{\theta}}\right) }\mbox{}\]
\[\tag{\%{}t25}\label{t25} 
{{U}_{2}}\cos{\left( \mathit{\ensuremath{\theta}}\right) }-{{U}_{1}}\sin{\left( \mathit{\ensuremath{\theta}}\right) }\mbox{}\]
\[\tag{\%{}t26}\label{t26} 
{{U}_{3}}\mbox{}
\]
%%%%%%%%%%%%%%%

\textbf{Coframe}



\noindent
%%%%%%%%%%%%%%%
%%% INPUT:
\begin{minipage}[t]{8ex}\color{red}\bf
(\%{}i27) 
\end{minipage}
\begin{minipage}[t]{\textwidth}\color{blue}\tt
ldisplay(\ensuremath{\Theta}:[\ensuremath{\theta}\_1,\ensuremath{\theta}\_2,\ensuremath{\theta}\_3]:list\_matrix\_entries(trigsimp(A.[dx,dy,dz])))\$
\end{minipage}
%%% OUTPUT:
\[\displaystyle
\tag{\%{}t27}\label{t27} 
\mathit{\ensuremath{\Theta}}=[\mathit{dr},r\,\mathit{d\ensuremath{\theta}},\mathit{dz}]\mbox{}
\]
%%%%%%%%%%%%%%%


\noindent
%%%%%%%%%%%%%%%
%%% INPUT:
\begin{minipage}[t]{8ex}\color{red}\bf
(\%{}i28) 
\end{minipage}
\begin{minipage}[t]{\textwidth}\color{blue}\tt
ldisplay(\ensuremath{\Theta}:list\_matrix\_entries(trigsimp(A.d\ensuremath{\zeta})))\$
\end{minipage}
%%% OUTPUT:
\[\displaystyle
\tag{\%{}t28}\label{t28} 
\mathit{\ensuremath{\Theta}}=[\mathit{dr},r\,\mathit{d\ensuremath{\theta}},\mathit{dz}]\mbox{}
\]
%%%%%%%%%%%%%%%


\noindent
%%%%%%%%%%%%%%%
%%% INPUT:
\begin{minipage}[t]{8ex}\color{red}\bf
(\%{}i29) 
\end{minipage}
\begin{minipage}[t]{\textwidth}\color{blue}\tt
ldisplay(\ensuremath{\Theta}:sf*cartan\_basis)\$
\end{minipage}
%%% OUTPUT:
\[\displaystyle
\tag{\%{}t29}\label{t29} 
\mathit{\ensuremath{\Theta}}=[\mathit{dr},r\,\mathit{d\ensuremath{\theta}},\mathit{dz}]\mbox{}
\]
%%%%%%%%%%%%%%%

\pagebreak
$\mathrm{d}A$



\noindent
%%%%%%%%%%%%%%%
%%% INPUT:
\begin{minipage}[t]{8ex}\color{red}\bf
(\%{}i30) 
\end{minipage}
\begin{minipage}[t]{\textwidth}\color{blue}\tt
ldisplay(dA:ext\_diff(A))\$
\end{minipage}
%%% OUTPUT:
\[\displaystyle
\tag{\%{}t30}\label{t30} 
\mathit{dA}=\begin{pmatrix}-\mathit{d\ensuremath{\theta}}\,\sin{\left( \mathit{\ensuremath{\theta}}\right) } & \mathit{d\ensuremath{\theta}}\,\cos{\left( \mathit{\ensuremath{\theta}}\right) } & 0\\
-\mathit{d\ensuremath{\theta}}\,\cos{\left( \mathit{\ensuremath{\theta}}\right) } & -\mathit{d\ensuremath{\theta}}\,\sin{\left( \mathit{\ensuremath{\theta}}\right) } & 0\\
0 & 0 & 0\end{pmatrix}\mbox{}
\]
%%%%%%%%%%%%%%%

\textbf{Change matrix multiplication operator}



\noindent
%%%%%%%%%%%%%%%
%%% INPUT:
\begin{minipage}[t]{8ex}\color{red}\bf
(\%{}i31) 
\end{minipage}
\begin{minipage}[t]{\textwidth}\color{blue}\tt
matrix\_element\_mult:"\ensuremath{\sim }"\$
\end{minipage}

\textbf{Connection form} $\omega\in\mathcal{A}^1(\mathbb{R}^3)$



\noindent
%%%%%%%%%%%%%%%
%%% INPUT:
\begin{minipage}[t]{8ex}\color{red}\bf
(\%{}i32) 
\end{minipage}
\begin{minipage}[t]{\textwidth}\color{blue}\tt
ldisplay(\ensuremath{\omega}:trigsimp(dA.transpose(A)))\$
\end{minipage}
%%% OUTPUT:
\[\displaystyle
\tag{\%{}t32}\label{t32} 
\mathit{\ensuremath{\omega}}=\begin{pmatrix}0 & \mathit{d\ensuremath{\theta}} & 0\\
-\mathit{d\ensuremath{\theta}} & 0 & 0\\
0 & 0 & 0\end{pmatrix}\mbox{}
\]
%%%%%%%%%%%%%%%


\noindent
%%%%%%%%%%%%%%%
%%% INPUT:
\begin{minipage}[t]{8ex}\color{red}\bf
(\%{}i33) 
\end{minipage}
\begin{minipage}[t]{\textwidth}\color{blue}\tt
ldisplay(d\ensuremath{\omega}:ext\_diff(\ensuremath{\omega}))\$
\end{minipage}
%%% OUTPUT:
\[\displaystyle
\tag{\%{}t33}\label{t33} 
\mathit{d\ensuremath{\omega}}=\begin{pmatrix}0 & 0 & 0\\
0 & 0 & 0\\
0 & 0 & 0\end{pmatrix}\mbox{}
\]
%%%%%%%%%%%%%%%


\noindent
%%%%%%%%%%%%%%%
%%% INPUT:
\begin{minipage}[t]{8ex}\color{red}\bf
(\%{}i34) 
\end{minipage}
\begin{minipage}[t]{\textwidth}\color{blue}\tt
trigsimp(\ensuremath{\omega}.\ensuremath{\omega});
\end{minipage}
%%% OUTPUT:
\[\displaystyle
\tag{\%{}o34}\label{o34} 
\begin{pmatrix}0 & 0 & 0\\
0 & 0 & 0\\
0 & 0 & 0\end{pmatrix}\mbox{}
\]
%%%%%%%%%%%%%%%

\textbf{First Cartan structure equation}



\noindent
%%%%%%%%%%%%%%%
%%% INPUT:
\begin{minipage}[t]{8ex}\color{red}\bf
(\%{}i35) 
\end{minipage}
\begin{minipage}[t]{\textwidth}\color{blue}\tt
ldisplay(d\ensuremath{\Theta}:ext\_diff(\ensuremath{\Theta}))\$
\end{minipage}
%%% OUTPUT:
\[\displaystyle
\tag{\%{}t35}\label{t35} 
\mathit{d\ensuremath{\Theta}}=[0,\mathit{dr}\,\mathit{d\ensuremath{\theta}},0]\mbox{}
\]
%%%%%%%%%%%%%%%


\noindent
%%%%%%%%%%%%%%%
%%% INPUT:
\begin{minipage}[t]{8ex}\color{red}\bf
(\%{}i36) 
\end{minipage}
\begin{minipage}[t]{\textwidth}\color{blue}\tt
list\_matrix\_entries(\ensuremath{\omega}.\ensuremath{\Theta});
\end{minipage}
%%% OUTPUT:
\[\displaystyle
\tag{\%{}o36}\label{o36} 
[0,\mathit{dr}\,\mathit{d\ensuremath{\theta}},0]\mbox{}
\]
%%%%%%%%%%%%%%%

\textbf{Second Cartan structure equation}



\noindent
%%%%%%%%%%%%%%%
%%% INPUT:
\begin{minipage}[t]{8ex}\color{red}\bf
(\%{}i37) 
\end{minipage}
\begin{minipage}[t]{\textwidth}\color{blue}\tt
ldisplay(d\ensuremath{\omega}:ext\_diff(\ensuremath{\omega}))\$
\end{minipage}
%%% OUTPUT:
\[\displaystyle
\tag{\%{}t37}\label{t37} 
\mathit{d\ensuremath{\omega}}=\begin{pmatrix}0 & 0 & 0\\
0 & 0 & 0\\
0 & 0 & 0\end{pmatrix}\mbox{}
\]
%%%%%%%%%%%%%%%


\noindent
%%%%%%%%%%%%%%%
%%% INPUT:
\begin{minipage}[t]{8ex}\color{red}\bf
(\%{}i38) 
\end{minipage}
\begin{minipage}[t]{\textwidth}\color{blue}\tt
trigsimp(\ensuremath{\omega}.\ensuremath{\omega});
\end{minipage}
%%% OUTPUT:
\[\displaystyle
\tag{\%{}o38}\label{o38} 
\begin{pmatrix}0 & 0 & 0\\
0 & 0 & 0\\
0 & 0 & 0\end{pmatrix}\mbox{}
\]
%%%%%%%%%%%%%%%

\textbf{Restore matrix multiplication operator}



\noindent
%%%%%%%%%%%%%%%
%%% INPUT:
\begin{minipage}[t]{8ex}\color{red}\bf
(\%{}i39) 
\end{minipage}
\begin{minipage}[t]{\textwidth}\color{blue}\tt
matrix\_element\_mult:"*"\$
\end{minipage}

\subsection{Example 2.7.7.}



\noindent
%%%%%%%%%%%%%%%
%%% INPUT:
\begin{minipage}[t]{8ex}\color{red}\bf
(\%{}i40) 
\end{minipage}
\begin{minipage}[t]{\textwidth}\color{blue}\tt
kill(labels,t,x,y,z,r,\ensuremath{\theta},U\_1,U\_2,U\_3,\ensuremath{\theta}\_1,\ensuremath{\theta}\_2,\ensuremath{\theta}\_3,E\_1,E\_2,E\_3)\$
\end{minipage}


\noindent
%%%%%%%%%%%%%%%
%%% INPUT:
\begin{minipage}[t]{8ex}\color{red}\bf
(\%{}i1) 
\end{minipage}
\begin{minipage}[t]{\textwidth}\color{blue}\tt
kill(labels,x,y,z,r,\ensuremath{\theta})\$
\end{minipage}


\noindent
%%%%%%%%%%%%%%%
%%% INPUT:
\begin{minipage}[t]{8ex}\color{red}\bf
(\%{}i1) 
\end{minipage}
\begin{minipage}[t]{\textwidth}\color{blue}\tt
\ensuremath{\zeta}:[x,y,z]\$
\end{minipage}


\noindent
%%%%%%%%%%%%%%%
%%% INPUT:
\begin{minipage}[t]{8ex}\color{red}\bf
(\%{}i5) 
\end{minipage}
\begin{minipage}[t]{\textwidth}\color{blue}\tt
assume(0\ensuremath{\leq}r)\$\\
assume(0\ensuremath{\leq}\ensuremath{\theta},\ensuremath{\theta}\ensuremath{\leq}\ensuremath{\pi})\$\\
assume(0\ensuremath{\leq}sin(\ensuremath{\theta}))\$\\
assume(0\ensuremath{\leq}\ensuremath{\phi},\ensuremath{\phi}\ensuremath{\leq}2*\ensuremath{\pi})\$
\end{minipage}


\noindent
%%%%%%%%%%%%%%%
%%% INPUT:
\begin{minipage}[t]{8ex}\color{red}\bf
(\%{}i6) 
\end{minipage}
\begin{minipage}[t]{\textwidth}\color{blue}\tt
\ensuremath{\xi}:[r,\ensuremath{\theta},\ensuremath{\phi}]\$
\end{minipage}

\textbf{Cartesian frame}



\noindent
%%%%%%%%%%%%%%%
%%% INPUT:
\begin{minipage}[t]{8ex}\color{red}\bf
(\%{}i7) 
\end{minipage}
\begin{minipage}[t]{\textwidth}\color{blue}\tt
U:[U\_1,U\_2,U\_3]\$
\end{minipage}

\textbf{Initialize cartan package}



\noindent
%%%%%%%%%%%%%%%
%%% INPUT:
\begin{minipage}[t]{8ex}\color{red}\bf
(\%{}i8) 
\end{minipage}
\begin{minipage}[t]{\textwidth}\color{blue}\tt
init\_cartan(\ensuremath{\xi})\$
\end{minipage}


\noindent
%%%%%%%%%%%%%%%
%%% INPUT:
\begin{minipage}[t]{8ex}\color{red}\bf
(\%{}i9) 
\end{minipage}
\begin{minipage}[t]{\textwidth}\color{blue}\tt
cartan\_basis;
\end{minipage}
%%% OUTPUT:
\[\displaystyle
\tag{\%{}o9}\label{o9} 
[\mathit{dr},\mathit{d\ensuremath{\theta}},\mathit{d\ensuremath{\phi}}]\mbox{}
\]
%%%%%%%%%%%%%%%


\noindent
%%%%%%%%%%%%%%%
%%% INPUT:
\begin{minipage}[t]{8ex}\color{red}\bf
(\%{}i10) 
\end{minipage}
\begin{minipage}[t]{\textwidth}\color{blue}\tt
cartan\_coords;
\end{minipage}
%%% OUTPUT:
\[\displaystyle
\tag{\%{}o10}\label{o10} 
[r,\mathit{\ensuremath{\theta}},\mathit{\ensuremath{\phi}}]\mbox{}
\]
%%%%%%%%%%%%%%%


\noindent
%%%%%%%%%%%%%%%
%%% INPUT:
\begin{minipage}[t]{8ex}\color{red}\bf
(\%{}i11) 
\end{minipage}
\begin{minipage}[t]{\textwidth}\color{blue}\tt
cartan\_dim;
\end{minipage}
%%% OUTPUT:
\[\displaystyle
\tag{\%{}o11}\label{o11} 
3\mbox{}
\]
%%%%%%%%%%%%%%%


\noindent
%%%%%%%%%%%%%%%
%%% INPUT:
\begin{minipage}[t]{8ex}\color{red}\bf
(\%{}i12) 
\end{minipage}
\begin{minipage}[t]{\textwidth}\color{blue}\tt
extdim;
\end{minipage}
%%% OUTPUT:
\[\displaystyle
\tag{\%{}o12}\label{o12} 
3\mbox{}
\]
%%%%%%%%%%%%%%%

\textbf{Transformation formulas}



\noindent
%%%%%%%%%%%%%%%
%%% INPUT:
\begin{minipage}[t]{8ex}\color{red}\bf
(\%{}i13) 
\end{minipage}
\begin{minipage}[t]{\textwidth}\color{blue}\tt
ldisplay(Tr:[r*sin(\ensuremath{\theta})*cos(\ensuremath{\phi}),r*sin(\ensuremath{\theta})*sin(\ensuremath{\phi}),r*cos(\ensuremath{\theta})])\$
\end{minipage}
%%% OUTPUT:
\[\displaystyle
\tag{\%{}t13}\label{t13} 
\mathit{Tr}=[r\,\sin{\left( \mathit{\ensuremath{\theta}}\right) }\,\cos{\left( \mathit{\ensuremath{\phi}}\right) },r\,\sin{\left( \mathit{\ensuremath{\theta}}\right) }\,\sin{\left( \mathit{\ensuremath{\phi}}\right) },r\,\cos{\left( \mathit{\ensuremath{\theta}}\right) }]\mbox{}
\]
%%%%%%%%%%%%%%%

\textbf{Jacobian matrix}



\noindent
%%%%%%%%%%%%%%%
%%% INPUT:
\begin{minipage}[t]{8ex}\color{red}\bf
(\%{}i14) 
\end{minipage}
\begin{minipage}[t]{\textwidth}\color{blue}\tt
ldisplay(J:jacobian(Tr,\ensuremath{\xi}))\$
\end{minipage}
%%% OUTPUT:
\[\displaystyle
\tag{\%{}t14}\label{t14} 
J=\begin{pmatrix}\sin{\left( \mathit{\ensuremath{\theta}}\right) }\,\cos{\left( \mathit{\ensuremath{\phi}}\right) } & r\,\cos{\left( \mathit{\ensuremath{\theta}}\right) }\,\cos{\left( \mathit{\ensuremath{\phi}}\right) } & -r\,\sin{\left( \mathit{\ensuremath{\theta}}\right) }\,\sin{\left( \mathit{\ensuremath{\phi}}\right) }\\
\sin{\left( \mathit{\ensuremath{\theta}}\right) }\,\sin{\left( \mathit{\ensuremath{\phi}}\right) } & r\,\cos{\left( \mathit{\ensuremath{\theta}}\right) }\,\sin{\left( \mathit{\ensuremath{\phi}}\right) } & r\,\sin{\left( \mathit{\ensuremath{\theta}}\right) }\,\cos{\left( \mathit{\ensuremath{\phi}}\right) }\\
\cos{\left( \mathit{\ensuremath{\theta}}\right) } & -r\,\sin{\left( \mathit{\ensuremath{\theta}}\right) } & 0\end{pmatrix}\mbox{}
\]
%%%%%%%%%%%%%%%

\pagebreak
\textbf{Metric tensor}



\noindent
%%%%%%%%%%%%%%%
%%% INPUT:
\begin{minipage}[t]{8ex}\color{red}\bf
(\%{}i15) 
\end{minipage}
\begin{minipage}[t]{\textwidth}\color{blue}\tt
ldisplay(lg:trigsimp(transpose(J).J))\$
\end{minipage}
%%% OUTPUT:
\[\displaystyle
\tag{\%{}t15}\label{t15} 
\mathit{lg}=\begin{pmatrix}1 & 0 & 0\\
0 & {{r}^{2}} & 0\\
0 & 0 & {{r}^{2}}\,{{\sin{\left( \mathit{\ensuremath{\theta}}\right) }}^{2}}\end{pmatrix}\mbox{}
\]
%%%%%%%%%%%%%%%

\textbf{Jacobian}



\noindent
%%%%%%%%%%%%%%%
%%% INPUT:
\begin{minipage}[t]{8ex}\color{red}\bf
(\%{}i16) 
\end{minipage}
\begin{minipage}[t]{\textwidth}\color{blue}\tt
ldisplay(Jdet:trigsimp(determinant(J)))\$
\end{minipage}
%%% OUTPUT:
\[\displaystyle
\tag{\%{}t16}\label{t16} 
\mathit{Jdet}={{r}^{2}}\,\sin{\left( \mathit{\ensuremath{\theta}}\right) }\mbox{}
\]
%%%%%%%%%%%%%%%

\textbf{Initialize vect package}



\noindent
%%%%%%%%%%%%%%%
%%% INPUT:
\begin{minipage}[t]{8ex}\color{red}\bf
(\%{}i17) 
\end{minipage}
\begin{minipage}[t]{\textwidth}\color{blue}\tt
scalefactors(append([Tr],\ensuremath{\xi}))\$
\end{minipage}


\noindent
%%%%%%%%%%%%%%%
%%% INPUT:
\begin{minipage}[t]{8ex}\color{red}\bf
(\%{}i18) 
\end{minipage}
\begin{minipage}[t]{\textwidth}\color{blue}\tt
sf;
\end{minipage}
%%% OUTPUT:
\[\displaystyle
\tag{\%{}o18}\label{o18} 
[1,r,r\,\sin{\left( \mathit{\ensuremath{\theta}}\right) }]\mbox{}
\]
%%%%%%%%%%%%%%%


\noindent
%%%%%%%%%%%%%%%
%%% INPUT:
\begin{minipage}[t]{8ex}\color{red}\bf
(\%{}i19) 
\end{minipage}
\begin{minipage}[t]{\textwidth}\color{blue}\tt
sfprod;
\end{minipage}
%%% OUTPUT:
\[\displaystyle
\tag{\%{}o19}\label{o19} 
{{r}^{2}}\,\sin{\left( \mathit{\ensuremath{\theta}}\right) }\mbox{}
\]
%%%%%%%%%%%%%%%

\textbf{Volume}



\noindent
%%%%%%%%%%%%%%%
%%% INPUT:
\begin{minipage}[t]{8ex}\color{red}\bf
(\%{}i20) 
\end{minipage}
\begin{minipage}[t]{\textwidth}\color{blue}\tt
[dx,dy,dz]:list\_matrix\_entries(J.cartan\_basis)\$
\end{minipage}


\noindent
%%%%%%%%%%%%%%%
%%% INPUT:
\begin{minipage}[t]{8ex}\color{red}\bf
(\%{}i21) 
\end{minipage}
\begin{minipage}[t]{\textwidth}\color{blue}\tt
dv:trigsimp(dx\ensuremath{\sim }dy\ensuremath{\sim }dz);
\end{minipage}
%%% OUTPUT:
\[\displaystyle
\tag{dv}\label{dv}
{{r}^{2}}\,\mathit{dr}\,\mathit{d\ensuremath{\theta}}\,\mathit{d\ensuremath{\phi}}\,\sin{\left( \mathit{\ensuremath{\theta}}\right) }\mbox{}
\]
%%%%%%%%%%%%%%%


\noindent
%%%%%%%%%%%%%%%
%%% INPUT:
\begin{minipage}[t]{8ex}\color{red}\bf
(\%{}i22) 
\end{minipage}
\begin{minipage}[t]{\textwidth}\color{blue}\tt
diff(\ensuremath{\xi},\ensuremath{\phi})|(diff(\ensuremath{\xi},\ensuremath{\theta})|(diff(\ensuremath{\xi},r)|dv));
\end{minipage}
%%% OUTPUT:
\[\displaystyle
\tag{\%{}o22}\label{o22} 
{{r}^{2}}\,\sin{\left( \mathit{\ensuremath{\theta}}\right) }\mbox{}
\]
%%%%%%%%%%%%%%%


\noindent
%%%%%%%%%%%%%%%
%%% INPUT:
\begin{minipage}[t]{8ex}\color{red}\bf
(\%{}i23) 
\end{minipage}
\begin{minipage}[t]{\textwidth}\color{blue}\tt
ldisplay(d\ensuremath{\zeta}:trigsimp(ext\_diff(at(\ensuremath{\zeta},map("=",\ensuremath{\zeta},Tr)))))\$
\end{minipage}
%%% OUTPUT:
\[\displaystyle
\tag{\%{}t23}\label{t23} 
\mathit{d\ensuremath{\zeta}}=[-r\,\mathit{d\ensuremath{\phi}}\,\sin{\left( \mathit{\ensuremath{\theta}}\right) }\,\sin{\left( \mathit{\ensuremath{\phi}}\right) }+\mathit{dr}\,\sin{\left( \mathit{\ensuremath{\theta}}\right) }\,\cos{\left( \mathit{\ensuremath{\phi}}\right) }+r\,\mathit{d\ensuremath{\theta}}\,\cos{\left( \mathit{\ensuremath{\theta}}\right) }\,\cos{\left( \mathit{\ensuremath{\phi}}\right) },\mathit{dr}\mbox{}\\\,\sin{\left( \mathit{\ensuremath{\theta}}\right) }\,\sin{\left( \mathit{\ensuremath{\phi}}\right) }+r\,\mathit{d\ensuremath{\theta}}\,\cos{\left( \mathit{\ensuremath{\theta}}\right) }\,\sin{\left( \mathit{\ensuremath{\phi}}\right) }+r\,\mathit{d\ensuremath{\phi}}\,\sin{\left( \mathit{\ensuremath{\theta}}\right) }\,\cos{\left( \mathit{\ensuremath{\phi}}\right) },\mathit{dr}\,\cos{\left( \mathit{\ensuremath{\theta}}\right) }-r\,\mathit{d\ensuremath{\theta}}\,\sin{\left( \mathit{\ensuremath{\theta}}\right) }]\mbox{}
\]
%%%%%%%%%%%%%%%

\textbf{Attitude matrix}



\noindent
%%%%%%%%%%%%%%%
%%% INPUT:
\begin{minipage}[t]{8ex}\color{red}\bf
(\%{}i24) 
\end{minipage}
\begin{minipage}[t]{\textwidth}\color{blue}\tt
ldisplay(A:apply('matrix,makelist(trigsimp(normalize(k)),k,args(transpose(J)))))\$
\end{minipage}
%%% OUTPUT:
\[\displaystyle
\tag{\%{}t24}\label{t24} 
A=\begin{pmatrix}\sin{\left( \mathit{\ensuremath{\theta}}\right) }\,\cos{\left( \mathit{\ensuremath{\phi}}\right) } & \sin{\left( \mathit{\ensuremath{\theta}}\right) }\,\sin{\left( \mathit{\ensuremath{\phi}}\right) } & \cos{\left( \mathit{\ensuremath{\theta}}\right) }\\
\cos{\left( \mathit{\ensuremath{\theta}}\right) }\,\cos{\left( \mathit{\ensuremath{\phi}}\right) } & \cos{\left( \mathit{\ensuremath{\theta}}\right) }\,\sin{\left( \mathit{\ensuremath{\phi}}\right) } & -\sin{\left( \mathit{\ensuremath{\theta}}\right) }\\
-\sin{\left( \mathit{\ensuremath{\phi}}\right) } & \cos{\left( \mathit{\ensuremath{\phi}}\right) } & 0\end{pmatrix}\mbox{}
\]
%%%%%%%%%%%%%%%

\textbf{Frame}



\noindent
%%%%%%%%%%%%%%%
%%% INPUT:
\begin{minipage}[t]{8ex}\color{red}\bf
(\%{}i25) 
\end{minipage}
\begin{minipage}[t]{\textwidth}\color{blue}\tt
E:[E\_1,E\_2,E\_3]:trigsimp(list\_matrix\_entries(A.U))\$
\end{minipage}


\noindent
%%%%%%%%%%%%%%%
%%% INPUT:
\begin{minipage}[t]{8ex}\color{red}\bf
(\%{}i26) 
\end{minipage}
\begin{minipage}[t]{\textwidth}\color{blue}\tt
map(ldisp,E)\$
\end{minipage}
%%% OUTPUT:
\[\displaystyle
\tag{\%{}t26}\label{t26} 
{{U}_{2}}\sin{\left( \mathit{\ensuremath{\theta}}\right) }\,\sin{\left( \mathit{\ensuremath{\phi}}\right) }+{{U}_{1}}\sin{\left( \mathit{\ensuremath{\theta}}\right) }\,\cos{\left( \mathit{\ensuremath{\phi}}\right) }+{{U}_{3}}\cos{\left( \mathit{\ensuremath{\theta}}\right) }\mbox{}\]
\[\tag{\%{}t27}\label{t27} 
{{U}_{2}}\cos{\left( \mathit{\ensuremath{\theta}}\right) }\,\sin{\left( \mathit{\ensuremath{\phi}}\right) }+{{U}_{1}}\cos{\left( \mathit{\ensuremath{\theta}}\right) }\,\cos{\left( \mathit{\ensuremath{\phi}}\right) }-{{U}_{3}}\sin{\left( \mathit{\ensuremath{\theta}}\right) }\mbox{}\]
\[\tag{\%{}t28}\label{t28} 
{{U}_{2}}\cos{\left( \mathit{\ensuremath{\phi}}\right) }-{{U}_{1}}\sin{\left( \mathit{\ensuremath{\phi}}\right) }\mbox{}
\]
%%%%%%%%%%%%%%%

\textbf{Coframe}



\noindent
%%%%%%%%%%%%%%%
%%% INPUT:
\begin{minipage}[t]{8ex}\color{red}\bf
(\%{}i29) 
\end{minipage}
\begin{minipage}[t]{\textwidth}\color{blue}\tt
ldisplay(\ensuremath{\Theta}:[\ensuremath{\theta}\_1,\ensuremath{\theta}\_2,\ensuremath{\theta}\_3]:list\_matrix\_entries(trigsimp(A.[dx,dy,dz])))\$
\end{minipage}
%%% OUTPUT:
\[\displaystyle
\tag{\%{}t29}\label{t29} 
\mathit{\ensuremath{\Theta}}=[\mathit{dr},r\,\mathit{d\ensuremath{\theta}},r\,\mathit{d\ensuremath{\phi}}\,\sin{\left( \mathit{\ensuremath{\theta}}\right) }]\mbox{}
\]
%%%%%%%%%%%%%%%


\noindent
%%%%%%%%%%%%%%%
%%% INPUT:
\begin{minipage}[t]{8ex}\color{red}\bf
(\%{}i30) 
\end{minipage}
\begin{minipage}[t]{\textwidth}\color{blue}\tt
ldisplay(\ensuremath{\Theta}:list\_matrix\_entries(trigsimp(A.d\ensuremath{\zeta})))\$
\end{minipage}
%%% OUTPUT:
\[\displaystyle
\tag{\%{}t30}\label{t30} 
\mathit{\ensuremath{\Theta}}=[\mathit{dr},r\,\mathit{d\ensuremath{\theta}},r\,\mathit{d\ensuremath{\phi}}\,\sin{\left( \mathit{\ensuremath{\theta}}\right) }]\mbox{}
\]
%%%%%%%%%%%%%%%


\noindent
%%%%%%%%%%%%%%%
%%% INPUT:
\begin{minipage}[t]{8ex}\color{red}\bf
(\%{}i31) 
\end{minipage}
\begin{minipage}[t]{\textwidth}\color{blue}\tt
ldisplay(\ensuremath{\Theta}:sf*cartan\_basis)\$
\end{minipage}
%%% OUTPUT:
\[\displaystyle
\tag{\%{}t31}\label{t31} 
\mathit{\ensuremath{\Theta}}=[\mathit{dr},r\,\mathit{d\ensuremath{\theta}},r\,\mathit{d\ensuremath{\phi}}\,\sin{\left( \mathit{\ensuremath{\theta}}\right) }]\mbox{}
\]
%%%%%%%%%%%%%%%

$\mathrm{d}A$



\noindent
%%%%%%%%%%%%%%%
%%% INPUT:
\begin{minipage}[t]{8ex}\color{red}\bf
(\%{}i32) 
\end{minipage}
\begin{minipage}[t]{\textwidth}\color{blue}\tt
ldisplay(dA:ext\_diff(A))\$
\end{minipage}
%%% OUTPUT:
\[\displaystyle
\tag{\%{}t32}\label{t32} 
\mathit{dA}=\begin{pmatrix}\mathit{d\ensuremath{\theta}}\,\cos{\left( \mathit{\ensuremath{\theta}}\right) }\,\cos{\left( \mathit{\ensuremath{\phi}}\right) }-\mathit{d\ensuremath{\phi}}\,\sin{\left( \mathit{\ensuremath{\theta}}\right) }\,\sin{\left( \mathit{\ensuremath{\phi}}\right) } & \mathit{d\ensuremath{\theta}}\,\cos{\left( \mathit{\ensuremath{\theta}}\right) }\,\sin{\left( \mathit{\ensuremath{\phi}}\right) }+\mathit{d\ensuremath{\phi}}\,\sin{\left( \mathit{\ensuremath{\theta}}\right) }\,\cos{\left( \mathit{\ensuremath{\phi}}\right) } & -\mathit{d\ensuremath{\theta}}\,\sin{\left( \mathit{\ensuremath{\theta}}\right) }\\
-\mathit{d\ensuremath{\phi}}\,\cos{\left( \mathit{\ensuremath{\theta}}\right) }\,\sin{\left( \mathit{\ensuremath{\phi}}\right) }-\mathit{d\ensuremath{\theta}}\,\sin{\left( \mathit{\ensuremath{\theta}}\right) }\,\cos{\left( \mathit{\ensuremath{\phi}}\right) } & \mathit{d\ensuremath{\phi}}\,\cos{\left( \mathit{\ensuremath{\theta}}\right) }\,\cos{\left( \mathit{\ensuremath{\phi}}\right) }-\mathit{d\ensuremath{\theta}}\,\sin{\left( \mathit{\ensuremath{\theta}}\right) }\,\sin{\left( \mathit{\ensuremath{\phi}}\right) } & -\mathit{d\ensuremath{\theta}}\,\cos{\left( \mathit{\ensuremath{\theta}}\right) }\\
-\mathit{d\ensuremath{\phi}}\,\cos{\left( \mathit{\ensuremath{\phi}}\right) } & -\mathit{d\ensuremath{\phi}}\,\sin{\left( \mathit{\ensuremath{\phi}}\right) } & 0\end{pmatrix}\mbox{}
\]
%%%%%%%%%%%%%%%

\textbf{Change matrix multiplication operator}



\noindent
%%%%%%%%%%%%%%%
%%% INPUT:
\begin{minipage}[t]{8ex}\color{red}\bf
(\%{}i33) 
\end{minipage}
\begin{minipage}[t]{\textwidth}\color{blue}\tt
matrix\_element\_mult:"\ensuremath{\sim }"\$
\end{minipage}

\textbf{Connection form} $\omega=\mathrm{d}A\wedge A^T\in\mathcal{A}^1(\mathbb{R}^3)$



\noindent
%%%%%%%%%%%%%%%
%%% INPUT:
\begin{minipage}[t]{8ex}\color{red}\bf
(\%{}i34) 
\end{minipage}
\begin{minipage}[t]{\textwidth}\color{blue}\tt
ldisplay(\ensuremath{\omega}:trigsimp(dA.transpose(A)))\$
\end{minipage}
%%% OUTPUT:
\[\displaystyle
\tag{\%{}t34}\label{t34} 
\mathit{\ensuremath{\omega}}=\begin{pmatrix}0 & \mathit{d\ensuremath{\theta}} & \mathit{d\ensuremath{\phi}}\,\sin{\left( \mathit{\ensuremath{\theta}}\right) }\\
-\mathit{d\ensuremath{\theta}} & 0 & \mathit{d\ensuremath{\phi}}\,\cos{\left( \mathit{\ensuremath{\theta}}\right) }\\
-\mathit{d\ensuremath{\phi}}\,\sin{\left( \mathit{\ensuremath{\theta}}\right) } & -\mathit{d\ensuremath{\phi}}\,\cos{\left( \mathit{\ensuremath{\theta}}\right) } & 0\end{pmatrix}\mbox{}
\]
%%%%%%%%%%%%%%%

\textbf{First Cartan structure equation}



\noindent
%%%%%%%%%%%%%%%
%%% INPUT:
\begin{minipage}[t]{8ex}\color{red}\bf
(\%{}i35) 
\end{minipage}
\begin{minipage}[t]{\textwidth}\color{blue}\tt
ldisplay(d\ensuremath{\Theta}:ext\_diff(\ensuremath{\Theta}))\$
\end{minipage}
%%% OUTPUT:
\[\displaystyle
\tag{\%{}t35}\label{t35} 
\mathit{d\ensuremath{\Theta}}=[0,\mathit{dr}\,\mathit{d\ensuremath{\theta}},\mathit{dr}\,\mathit{d\ensuremath{\phi}}\,\sin{\left( \mathit{\ensuremath{\theta}}\right) }+r\,\mathit{d\ensuremath{\theta}}\,\mathit{d\ensuremath{\phi}}\,\cos{\left( \mathit{\ensuremath{\theta}}\right) }]\mbox{}
\]
%%%%%%%%%%%%%%%


\noindent
%%%%%%%%%%%%%%%
%%% INPUT:
\begin{minipage}[t]{8ex}\color{red}\bf
(\%{}i36) 
\end{minipage}
\begin{minipage}[t]{\textwidth}\color{blue}\tt
list\_matrix\_entries(\ensuremath{\omega}.\ensuremath{\Theta});
\end{minipage}
%%% OUTPUT:
\[\displaystyle
\tag{\%{}o36}\label{o36} 
[0,\mathit{dr}\,\mathit{d\ensuremath{\theta}},\mathit{dr}\,\mathit{d\ensuremath{\phi}}\,\sin{\left( \mathit{\ensuremath{\theta}}\right) }+r\,\mathit{d\ensuremath{\theta}}\,\mathit{d\ensuremath{\phi}}\,\cos{\left( \mathit{\ensuremath{\theta}}\right) }]\mbox{}
\]
%%%%%%%%%%%%%%%

\textbf{Second Cartan structure equation}



\noindent
%%%%%%%%%%%%%%%
%%% INPUT:
\begin{minipage}[t]{8ex}\color{red}\bf
(\%{}i37) 
\end{minipage}
\begin{minipage}[t]{\textwidth}\color{blue}\tt
ldisplay(d\ensuremath{\omega}:ext\_diff(\ensuremath{\omega}))\$
\end{minipage}
%%% OUTPUT:
\[\displaystyle
\tag{\%{}t37}\label{t37} 
\mathit{d\ensuremath{\omega}}=\begin{pmatrix}0 & 0 & \mathit{d\ensuremath{\theta}}\,\mathit{d\ensuremath{\phi}}\,\cos{\left( \mathit{\ensuremath{\theta}}\right) }\\
0 & 0 & -\mathit{d\ensuremath{\theta}}\,\mathit{d\ensuremath{\phi}}\,\sin{\left( \mathit{\ensuremath{\theta}}\right) }\\
-\mathit{d\ensuremath{\theta}}\,\mathit{d\ensuremath{\phi}}\,\cos{\left( \mathit{\ensuremath{\theta}}\right) } & \mathit{d\ensuremath{\theta}}\,\mathit{d\ensuremath{\phi}}\,\sin{\left( \mathit{\ensuremath{\theta}}\right) } & 0\end{pmatrix}\mbox{}
\]
%%%%%%%%%%%%%%%


\noindent
%%%%%%%%%%%%%%%
%%% INPUT:
\begin{minipage}[t]{8ex}\color{red}\bf
(\%{}i38) 
\end{minipage}
\begin{minipage}[t]{\textwidth}\color{blue}\tt
trigsimp(\ensuremath{\omega}.\ensuremath{\omega});
\end{minipage}
%%% OUTPUT:
\[\displaystyle
\tag{\%{}o38}\label{o38} 
\begin{pmatrix}0 & 0 & \mathit{d\ensuremath{\theta}}\,\mathit{d\ensuremath{\phi}}\,\cos{\left( \mathit{\ensuremath{\theta}}\right) }\\
0 & 0 & -\mathit{d\ensuremath{\theta}}\,\mathit{d\ensuremath{\phi}}\,\sin{\left( \mathit{\ensuremath{\theta}}\right) }\\
-\mathit{d\ensuremath{\theta}}\,\mathit{d\ensuremath{\phi}}\,\cos{\left( \mathit{\ensuremath{\theta}}\right) } & \mathit{d\ensuremath{\theta}}\,\mathit{d\ensuremath{\phi}}\,\sin{\left( \mathit{\ensuremath{\theta}}\right) } & 0\end{pmatrix}\mbox{}
\]
%%%%%%%%%%%%%%%

\textbf{Restore matrix multiplication operator}



\noindent
%%%%%%%%%%%%%%%
%%% INPUT:
\begin{minipage}[t]{8ex}\color{red}\bf
(\%{}i39) 
\end{minipage}
\begin{minipage}[t]{\textwidth}\color{blue}\tt
matrix\_element\_mult:"*"\$
\end{minipage}
\end{document}
