\documentclass{article}

%% Created with wxMaxima 16.04.2

\setlength{\parskip}{\medskipamount}
\setlength{\parindent}{0pt}
\usepackage[utf8]{inputenc}
\DeclareUnicodeCharacter{00B5}{\ensuremath{\mu}}
\usepackage{graphicx}
\usepackage{color}
\usepackage{amsmath}
\usepackage{ifthen}
\newsavebox{\picturebox}
\newlength{\pictureboxwidth}
\newlength{\pictureboxheight}
\newcommand{\includeimage}[1]{
    \savebox{\picturebox}{\includegraphics{#1}}
    \settoheight{\pictureboxheight}{\usebox{\picturebox}}
    \settowidth{\pictureboxwidth}{\usebox{\picturebox}}
    \ifthenelse{\lengthtest{\pictureboxwidth > .95\linewidth}}
    {
        \includegraphics[width=.95\linewidth,height=.80\textheight,keepaspectratio]{#1}
    }
    {
        \ifthenelse{\lengthtest{\pictureboxheight>.80\textheight}}
        {
            \includegraphics[width=.95\linewidth,height=.80\textheight,keepaspectratio]{#1}
            
        }
        {
            \includegraphics{#1}
        }
    }
}
\newlength{\thislabelwidth}
\DeclareMathOperator{\abs}{abs}
\usepackage{animate} % This package is required because the wxMaxima configuration option
                      % "Export animations to TeX" was enabled when this file was generated.

\definecolor{labelcolor}{RGB}{100,0,0}

\usepackage{fullpage}
\usepackage{amssymb}
\usepackage{enumerate}
\usepackage[bookmarks=false,pdfstartview={FitH},colorlinks=true,urlcolor=blue]{hyperref}
\usepackage{bookmark}
\usepackage{mathtools}

\begin{document}

\pagebreak{}
{\Huge {\sc Chapter 1 introduction}}
\setcounter{section}{0}
\setcounter{subsection}{0}
\setcounter{figure}{0}


\hypersetup{pdfauthor={Daniel Volinski},
            pdftitle={Differential Geometry},
            pdfsubject={Differential Geometry},
            pdfkeywords={James Cook}}

\begin{verbatim}
Lecture Notes for Differential Geometry
James S. Cook
Liberty University
Department of Mathematics
Summer 2015
\end{verbatim}

Written by Daniel Volinski at \href{mailto:danielvolinski@yahoo.es}{danielvolinski@yahoo.es}



\noindent
%%%%%%%%%%%%%%%
%%% INPUT:
\begin{minipage}[t]{8ex}\color{red}\bf
(\%{}i2) 
\end{minipage}
\begin{minipage}[t]{\textwidth}\color{blue}\tt
info:build\_info()\$info\ensuremath{@}version;
\end{minipage}
%%% OUTPUT:
\[\displaystyle
\tag{\%{}o2}\label{o2} 
\mbox{}
\]5.38.1



\noindent
%%%%%%%%%%%%%%%
%%% INPUT:
\begin{minipage}[t]{8ex}\color{red}\bf
(\%{}i2) 
\end{minipage}
\begin{minipage}[t]{\textwidth}\color{blue}\tt
reset()\$kill(all)\$
\end{minipage}


\noindent
%%%%%%%%%%%%%%%
%%% INPUT:
\begin{minipage}[t]{8ex}\color{red}\bf
(\%{}i1) 
\end{minipage}
\begin{minipage}[t]{\textwidth}\color{blue}\tt
derivabbrev:true\$
\end{minipage}


\noindent
%%%%%%%%%%%%%%%
%%% INPUT:
\begin{minipage}[t]{8ex}\color{red}\bf
(\%{}i2) 
\end{minipage}
\begin{minipage}[t]{\textwidth}\color{blue}\tt
ratprint:false\$
\end{minipage}


\noindent
%%%%%%%%%%%%%%%
%%% INPUT:
\begin{minipage}[t]{8ex}\color{red}\bf
(\%{}i3) 
\end{minipage}
\begin{minipage}[t]{\textwidth}\color{blue}\tt
fpprintprec:5\$
\end{minipage}


\noindent
%%%%%%%%%%%%%%%
%%% INPUT:
\begin{minipage}[t]{8ex}\color{red}\bf
(\%{}i4) 
\end{minipage}
\begin{minipage}[t]{\textwidth}\color{blue}\tt
load(linearalgebra)\$
\end{minipage}


\noindent
%%%%%%%%%%%%%%%
%%% INPUT:
\begin{minipage}[t]{8ex}\color{red}\bf
(\%{}i5) 
\end{minipage}
\begin{minipage}[t]{\textwidth}\color{blue}\tt
if get('draw,'version)=false then load(draw)\$
\end{minipage}
%%% OUTPUT:
%%%%%%%%%%%%%%%


\noindent
%%%%%%%%%%%%%%%
%%% INPUT:
\begin{minipage}[t]{8ex}\color{red}\bf
(\%{}i6) 
\end{minipage}
\begin{minipage}[t]{\textwidth}\color{blue}\tt
wxplot\_size:[1024,768]\$
\end{minipage}


\noindent
%%%%%%%%%%%%%%%
%%% INPUT:
\begin{minipage}[t]{8ex}\color{red}\bf
(\%{}i7) 
\end{minipage}
\begin{minipage}[t]{\textwidth}\color{blue}\tt
if get('drawdf,'version)=false then load(drawdf)\$
\end{minipage}


\noindent
%%%%%%%%%%%%%%%
%%% INPUT:
\begin{minipage}[t]{8ex}\color{red}\bf
(\%{}i8) 
\end{minipage}
\begin{minipage}[t]{\textwidth}\color{blue}\tt
set\_draw\_defaults(xtics=1,ytics=1,ztics=1,xyplane=0,nticks=100,\\
                  xaxis=true,xaxis\_type=solid,xaxis\_width=3,\\
                  yaxis=true,yaxis\_type=solid,yaxis\_width=3,\\
                  zaxis=true,zaxis\_type=solid,zaxis\_width=3,\\
                  background\_color=light\_gray)\$
\end{minipage}


\noindent
%%%%%%%%%%%%%%%
%%% INPUT:
\begin{minipage}[t]{8ex}\color{red}\bf
(\%{}i9) 
\end{minipage}
\begin{minipage}[t]{\textwidth}\color{blue}\tt
if get('vect,'version)=false then load(vect)\$
\end{minipage}


\noindent
%%%%%%%%%%%%%%%
%%% INPUT:
\begin{minipage}[t]{8ex}\color{red}\bf
(\%{}i10) 
\end{minipage}
\begin{minipage}[t]{\textwidth}\color{blue}\tt
norm(u):=block(ratsimp(radcan(\ensuremath{\sqrt{}}(u.u))))\$
\end{minipage}


\noindent
%%%%%%%%%%%%%%%
%%% INPUT:
\begin{minipage}[t]{8ex}\color{red}\bf
(\%{}i11) 
\end{minipage}
\begin{minipage}[t]{\textwidth}\color{blue}\tt
normalize(v):=block(v/norm(v))\$
\end{minipage}


\noindent
%%%%%%%%%%%%%%%
%%% INPUT:
\begin{minipage}[t]{8ex}\color{red}\bf
(\%{}i12) 
\end{minipage}
\begin{minipage}[t]{\textwidth}\color{blue}\tt
angle(u,v):=block([junk:radcan(\ensuremath{\sqrt{}}((u.u)*(v.v)))],acos(u.v/junk))\$
\end{minipage}


\noindent
%%%%%%%%%%%%%%%
%%% INPUT:
\begin{minipage}[t]{8ex}\color{red}\bf
(\%{}i13) 
\end{minipage}
\begin{minipage}[t]{\textwidth}\color{blue}\tt
mycross(va,vb):=[va[2]*vb[3]-va[3]*vb[2],va[3]*vb[1]-va[1]*vb[3],va[1]*vb[2]-va[2]*vb[1]]\$
\end{minipage}


\noindent
%%%%%%%%%%%%%%%
%%% INPUT:
\begin{minipage}[t]{8ex}\color{red}\bf
(\%{}i14) 
\end{minipage}
\begin{minipage}[t]{\textwidth}\color{blue}\tt
if get('cartan,'version)=false then load(cartan)\$
\end{minipage}


\noindent
%%%%%%%%%%%%%%%
%%% INPUT:
\begin{minipage}[t]{8ex}\color{red}\bf
(\%{}i15) 
\end{minipage}
\begin{minipage}[t]{\textwidth}\color{blue}\tt
declare(trigsimp,evfun)\$
\end{minipage}
\pagebreak


\section{points and vectors}


\section{on tangent and cotangent spaces and bundles}


\subsection{Example 1.2.2.}


Define $f=2\,x^1-3\,x^4+(x_n)^2$ then $f(p)=\left({2\,x^1-3\,x^4+
(x_n)^2}\right)(p)$ and by the usual addition of functions $f(p)=2\,
p^1-3 p^4+(p_n)^2$.


\subsection{Example 1.2.3.}


Define $f=x+y^2+z^3$. Observe $f(a,b,c)=a+b^2+c^3$.



\noindent
%%%%%%%%%%%%%%%
%%% INPUT:
\begin{minipage}[t]{8ex}\color{red}\bf
(\%{}i16) 
\end{minipage}
\begin{minipage}[t]{\textwidth}\color{blue}\tt
ldisplay(f:x+y\ensuremath{^2}+z\ensuremath{^3})\$
\end{minipage}
%%% OUTPUT:
\[\displaystyle
\tag{\%{}t16}\label{t16} 
f={{z}^{3}}+{{y}^{2}}+x\mbox{}
\]
%%%%%%%%%%%%%%%


\noindent
%%%%%%%%%%%%%%%
%%% INPUT:
\begin{minipage}[t]{8ex}\color{red}\bf
(\%{}i17) 
\end{minipage}
\begin{minipage}[t]{\textwidth}\color{blue}\tt
at(f,[x=a,y=b,z=c]);
\end{minipage}
%%% OUTPUT:
\[\displaystyle
\tag{\%{}o17}\label{o17} 
{{c}^{3}}+{{b}^{2}}+a\mbox{}
\]
%%%%%%%%%%%%%%%

\subsection{Example 1.2.13.}


Let $f=\sqrt{(x^1)^2+\dots+(x^n)^2}$ clearly $f^2=(x^1)^2+\dots+(x^n)^2$
and thus $\mathrm{d}f^2=2 f\,\mathrm{d}f$ and $\mathrm{d}(f^2)=2 x^1\,
\mathrm{d}x^1+\dots+2 x^n\,\mathrm{d}x^n$ thus $$\mathrm{d}f=\dfrac{x^1\,
\mathrm{d}x^1+\dots+x^n\,\mathrm{d}x^n}{\sqrt{(x^1)^2+\dots+(x^n)^2}}$$

\pagebreak


\section{the wedge product and differential forms}


\subsection{Example 1.3.4.}


A force $\vec{F}$ is conservative if there exists $f$ such that $\vec{F}=
-\nabla\phi$. In the language of differential forms, this means the one-form
$\omega_{\vec{F}}$ represents a conservative force if $\omega_{\vec{F}}=
\omega_{-\nabla\phi}=-\mathrm{d}\phi$. Observe, $\omega_{\vec{F}}=
-\mathrm{d}\phi$ implies $\mathrm{d}\omega_{\vec{F}}=-\mathrm{d}^2\phi=0$.
As an application, consider $\omega_{\vec{F}}=-y\,\mathrm{d}x+x\,
\mathrm{d}y+\mathrm{d}z$, is $\vec{F}$ conservative? 


Calculate $$\mathrm{d}\omega_{\vec{F}}=-\mathrm{d}y\wedge\mathrm{d}x+
\mathrm{d}x\wedge\mathrm{d}y+\mathrm{d}(\mathrm{d}z)=2\,\mathrm{d}x\wedge
\mathrm{d}y\neq 0$$



\noindent
%%%%%%%%%%%%%%%
%%% INPUT:
\begin{minipage}[t]{8ex}\color{red}\bf
(\%{}i18) 
\end{minipage}
\begin{minipage}[t]{\textwidth}\color{blue}\tt
kill(labels,x,y,z)\$
\end{minipage}


\noindent
%%%%%%%%%%%%%%%
%%% INPUT:
\begin{minipage}[t]{8ex}\color{red}\bf
(\%{}i1) 
\end{minipage}
\begin{minipage}[t]{\textwidth}\color{blue}\tt
\ensuremath{\zeta}:[x,y,z]\$
\end{minipage}


\noindent
%%%%%%%%%%%%%%%
%%% INPUT:
\begin{minipage}[t]{8ex}\color{red}\bf
(\%{}i2) 
\end{minipage}
\begin{minipage}[t]{\textwidth}\color{blue}\tt
init\_cartan(\ensuremath{\zeta})\$
\end{minipage}


\noindent
%%%%%%%%%%%%%%%
%%% INPUT:
\begin{minipage}[t]{8ex}\color{red}\bf
(\%{}i3) 
\end{minipage}
\begin{minipage}[t]{\textwidth}\color{blue}\tt
ldisplay(\ensuremath{\omega}\_F:-y*dx+x*dy+dz)\$
\end{minipage}
%%% OUTPUT:
\[\displaystyle
\tag{\%{}t3}\label{t3} 
{{\mathit{\ensuremath{\omega}}}_{F}}=\mathit{dz}+x\,\mathit{dy}-y\,\mathit{dx}\mbox{}
\]
%%%%%%%%%%%%%%%


\noindent
%%%%%%%%%%%%%%%
%%% INPUT:
\begin{minipage}[t]{8ex}\color{red}\bf
(\%{}i4) 
\end{minipage}
\begin{minipage}[t]{\textwidth}\color{blue}\tt
ldisplay(d\ensuremath{\omega}\_F:edit(ext\_diff(\ensuremath{\omega}\_F)))\$
\end{minipage}
%%% OUTPUT:
\[\displaystyle
\tag{\%{}t4}\label{t4} 
{{\mathit{d\ensuremath{\omega}}}_{F}}=2\mathit{dx}\,\mathit{dy}\mbox{}
\]
%%%%%%%%%%%%%%%

thus $\vec{F}$ is not conservative.


\subsection{Example 1.3.5.}


It turns out that Maxwell's equations can be expressed in terms of the
exterior derivative of the potential one-form $A$. The one-form contains
both the voltage function and the magnetic vector-potential from which the
time-derivative and gradient derive the electric and magnetic Fields.
In spacetime the relation between the potentials and fields is simply
$\vec{F}=\mathrm{d}A$. The choice of $A$ is far from unique. There is a
\textbf{gauge freedom}. In particular, we can add an exterior derivative
function of spacetime $\lambda$ and create $A^\prime=A+\mathrm{d}\lambda$.
Note, $\mathrm{d}A^\prime=\mathrm{d}A+\mathrm{d}^2\lambda$ hence $A$ and
$A^\prime$ generate the same electric and magnetic fields (which make up
the Faraday tensor $F$)


\subsection{Definition 1.3.6.}


Work and Flux form correspondance



\noindent
%%%%%%%%%%%%%%%
%%% INPUT:
\begin{minipage}[t]{8ex}\color{red}\bf
(\%{}i5) 
\end{minipage}
\begin{minipage}[t]{\textwidth}\color{blue}\tt
\ensuremath{\zeta}:[x,y,z]\$
\end{minipage}


\noindent
%%%%%%%%%%%%%%%
%%% INPUT:
\begin{minipage}[t]{8ex}\color{red}\bf
(\%{}i6) 
\end{minipage}
\begin{minipage}[t]{\textwidth}\color{blue}\tt
init\_cartan(\ensuremath{\zeta})\$
\end{minipage}


\noindent
%%%%%%%%%%%%%%%
%%% INPUT:
\begin{minipage}[t]{8ex}\color{red}\bf
(\%{}i7) 
\end{minipage}
\begin{minipage}[t]{\textwidth}\color{blue}\tt
F:[P,Q,R]\$
\end{minipage}


\noindent
%%%%%%%%%%%%%%%
%%% INPUT:
\begin{minipage}[t]{8ex}\color{red}\bf
(\%{}i8) 
\end{minipage}
\begin{minipage}[t]{\textwidth}\color{blue}\tt
ldisplay(\ensuremath{\omega}\_F:F.cartan\_basis)\$
\end{minipage}
%%% OUTPUT:
\[\displaystyle
\tag{\%{}t8}\label{t8} 
{{\mathit{\ensuremath{\omega}}}_{F}}=R\,\mathit{dz}+Q\,\mathit{dy}+P\,\mathit{dx}\mbox{}
\]
%%%%%%%%%%%%%%%


\noindent
%%%%%%%%%%%%%%%
%%% INPUT:
\begin{minipage}[t]{8ex}\color{red}\bf
(\%{}i10) 
\end{minipage}
\begin{minipage}[t]{\textwidth}\color{blue}\tt
cartan\_flow:makelist((dx\ensuremath{\sim }dy\ensuremath{\sim }dz)/k,k,cartan\_basis)\$\\
cartan\_sign:makelist((-1)\^{}(k+1),k,cartan\_dim)\$
\end{minipage}


\noindent
%%%%%%%%%%%%%%%
%%% INPUT:
\begin{minipage}[t]{8ex}\color{red}\bf
(\%{}i11) 
\end{minipage}
\begin{minipage}[t]{\textwidth}\color{blue}\tt
ldisplay(\ensuremath{\Phi}\_F:F.(cartan\_flow*cartan\_sign))\$
\end{minipage}
%%% OUTPUT:
\[\displaystyle
\tag{\%{}t11}\label{t11} 
{{\mathit{\ensuremath{\Phi}}}_{F}}=P\,\mathit{dy}\,\mathit{dz}-Q\,\mathit{dx}\,\mathit{dz}+R\,\mathit{dx}\,\mathit{dy}\mbox{}
\]
%%%%%%%%%%%%%%%

\subsection{Proposition 1.3.7.}


Vector algebra in Differential form.



\noindent
%%%%%%%%%%%%%%%
%%% INPUT:
\begin{minipage}[t]{8ex}\color{red}\bf
(\%{}i12) 
\end{minipage}
\begin{minipage}[t]{\textwidth}\color{blue}\tt
kill(t,x,y,z)\$
\end{minipage}


\noindent
%%%%%%%%%%%%%%%
%%% INPUT:
\begin{minipage}[t]{8ex}\color{red}\bf
(\%{}i13) 
\end{minipage}
\begin{minipage}[t]{\textwidth}\color{blue}\tt
\ensuremath{\zeta}:[x,y,z]\$
\end{minipage}


\noindent
%%%%%%%%%%%%%%%
%%% INPUT:
\begin{minipage}[t]{8ex}\color{red}\bf
(\%{}i14) 
\end{minipage}
\begin{minipage}[t]{\textwidth}\color{blue}\tt
init\_cartan(\ensuremath{\zeta})\$
\end{minipage}


\noindent
%%%%%%%%%%%%%%%
%%% INPUT:
\begin{minipage}[t]{8ex}\color{red}\bf
(\%{}i17) 
\end{minipage}
\begin{minipage}[t]{\textwidth}\color{blue}\tt
A:[A\_1,A\_2,A\_3]\$\\
B:[B\_1,B\_2,B\_3]\$\\
C:[C\_1,C\_2,C\_3]\$
\end{minipage}


\noindent
%%%%%%%%%%%%%%%
%%% INPUT:
\begin{minipage}[t]{8ex}\color{red}\bf
(\%{}i20) 
\end{minipage}
\begin{minipage}[t]{\textwidth}\color{blue}\tt
ldisplay(\ensuremath{\omega}\_A:A.cartan\_basis)\$\\
ldisplay(\ensuremath{\omega}\_B:B.cartan\_basis)\$\\
ldisplay(\ensuremath{\omega}\_C:C.cartan\_basis)\$
\end{minipage}
%%% OUTPUT:
\[\displaystyle
\tag{\%{}t18}\label{t18} 
{{\mathit{\ensuremath{\omega}}}_{A}}={{A}_{3}}\mathit{dz}+{{A}_{2}}\mathit{dy}+{{A}_{1}}\mathit{dx}\mbox{}\]
\[\tag{\%{}t19}\label{t19} 
{{\mathit{\ensuremath{\omega}}}_{B}}={{B}_{3}}\mathit{dz}+{{B}_{2}}\mathit{dy}+{{B}_{1}}\mathit{dx}\mbox{}\]
\[\tag{\%{}t20}\label{t20} 
{{\mathit{\ensuremath{\omega}}}_{C}}={{C}_{3}}\mathit{dz}+{{C}_{2}}\mathit{dy}+{{C}_{1}}\mathit{dx}\mbox{}
\]
%%%%%%%%%%%%%%%


\noindent
%%%%%%%%%%%%%%%
%%% INPUT:
\begin{minipage}[t]{8ex}\color{red}\bf
(\%{}i22) 
\end{minipage}
\begin{minipage}[t]{\textwidth}\color{blue}\tt
cartan\_flow:makelist((dx\ensuremath{\sim }dy\ensuremath{\sim }dz)/k,k,cartan\_basis)\$\\
cartan\_sign:makelist((-1)\^{}(k+1),k,cartan\_dim)\$
\end{minipage}


\noindent
%%%%%%%%%%%%%%%
%%% INPUT:
\begin{minipage}[t]{8ex}\color{red}\bf
(\%{}i25) 
\end{minipage}
\begin{minipage}[t]{\textwidth}\color{blue}\tt
ldisplay(\ensuremath{\Phi}\_A:A.(cartan\_flow*cartan\_sign))\$\\
ldisplay(\ensuremath{\Phi}\_B:B.(cartan\_flow*cartan\_sign))\$\\
ldisplay(\ensuremath{\Phi}\_C:C.(cartan\_flow*cartan\_sign))\$
\end{minipage}
%%% OUTPUT:
\[\displaystyle
\tag{\%{}t23}\label{t23} 
{{\mathit{\ensuremath{\Phi}}}_{A}}={{A}_{1}}\mathit{dy}\,\mathit{dz}-{{A}_{2}}\mathit{dx}\,\mathit{dz}+{{A}_{3}}\mathit{dx}\,\mathit{dy}\mbox{}\]
\[\tag{\%{}t24}\label{t24} 
{{\mathit{\ensuremath{\Phi}}}_{B}}={{B}_{1}}\mathit{dy}\,\mathit{dz}-{{B}_{2}}\mathit{dx}\,\mathit{dz}+{{B}_{3}}\mathit{dx}\,\mathit{dy}\mbox{}\]
\[\tag{\%{}t25}\label{t25} 
{{\mathit{\ensuremath{\Phi}}}_{C}}={{C}_{1}}\mathit{dy}\,\mathit{dz}-{{C}_{2}}\mathit{dx}\,\mathit{dz}+{{C}_{3}}\mathit{dx}\,\mathit{dy}\mbox{}
\]
%%%%%%%%%%%%%%%

$\omega_A\wedge\omega_B=\Phi_{A\times B}$



\noindent
%%%%%%%%%%%%%%%
%%% INPUT:
\begin{minipage}[t]{8ex}\color{red}\bf
(\%{}i26) 
\end{minipage}
\begin{minipage}[t]{\textwidth}\color{blue}\tt
edit(\ensuremath{\omega}\_A\ensuremath{\sim }\ensuremath{\omega}\_B);
\end{minipage}
%%% OUTPUT:
\[\displaystyle
\tag{\%{}o26}\label{o26} 
\left( {{A}_{2}}{{B}_{3}}-{{A}_{3}}{{B}_{2}}\right) \mathit{dy}\,\mathit{dz}+\left( {{A}_{1}}{{B}_{3}}-{{A}_{3}}{{B}_{1}}\right) \mathit{dx}\,\mathit{dz}+\left( {{A}_{1}}{{B}_{2}}-{{A}_{2}}{{B}_{1}}\right) \mathit{dx}\,\mathit{dy}\mbox{}
\]
%%%%%%%%%%%%%%%


\noindent
%%%%%%%%%%%%%%%
%%% INPUT:
\begin{minipage}[t]{8ex}\color{red}\bf
(\%{}i27) 
\end{minipage}
\begin{minipage}[t]{\textwidth}\color{blue}\tt
edit(mycross(A,B).(cartan\_flow*cartan\_sign));
\end{minipage}
%%% OUTPUT:
\[\displaystyle
\tag{\%{}o27}\label{o27} 
\left( {{A}_{2}}{{B}_{3}}-{{A}_{3}}{{B}_{2}}\right) \mathit{dy}\,\mathit{dz}+\left( {{A}_{1}}{{B}_{3}}-{{A}_{3}}{{B}_{1}}\right) \mathit{dx}\,\mathit{dz}+\left( {{A}_{1}}{{B}_{2}}-{{A}_{2}}{{B}_{1}}\right) \mathit{dx}\,\mathit{dy}\mbox{}
\]
%%%%%%%%%%%%%%%


\noindent
%%%%%%%%%%%%%%%
%%% INPUT:
\begin{minipage}[t]{8ex}\color{red}\bf
(\%{}i28) 
\end{minipage}
\begin{minipage}[t]{\textwidth}\color{blue}\tt
is(\%=\%th(2));
\end{minipage}
%%% OUTPUT:
\[\displaystyle
\tag{\%{}o28}\label{o28} 
\mbox{true}\mbox{}
\]
%%%%%%%%%%%%%%%

$\omega_A\wedge\omega_B\wedge\omega_C=A\cdot(B\times C)
\,\mathrm{d}x\wedge\,\mathrm{d}y\wedge\,\mathrm{d}z$



\noindent
%%%%%%%%%%%%%%%
%%% INPUT:
\begin{minipage}[t]{8ex}\color{red}\bf
(\%{}i29) 
\end{minipage}
\begin{minipage}[t]{\textwidth}\color{blue}\tt
edit(\ensuremath{\omega}\_A\ensuremath{\sim }\ensuremath{\omega}\_B\ensuremath{\sim }\ensuremath{\omega}\_C);
\end{minipage}
%%% OUTPUT:
\[\displaystyle
\tag{\%{}o29}\label{o29} 
\left( {{A}_{1}}{{B}_{2}}{{C}_{3}}-{{A}_{2}}{{B}_{1}}{{C}_{3}}-{{A}_{1}}{{B}_{3}}{{C}_{2}}+{{A}_{3}}{{B}_{1}}{{C}_{2}}+{{A}_{2}}{{B}_{3}}{{C}_{1}}-{{A}_{3}}{{B}_{2}}{{C}_{1}}\right) \mathit{dx}\,\mathit{dy}\,\mathit{dz}\mbox{}
\]
%%%%%%%%%%%%%%%


\noindent
%%%%%%%%%%%%%%%
%%% INPUT:
\begin{minipage}[t]{8ex}\color{red}\bf
(\%{}i30) 
\end{minipage}
\begin{minipage}[t]{\textwidth}\color{blue}\tt
edit(A.(mycross(B,C))*(dx\ensuremath{\sim }dy\ensuremath{\sim }dz));
\end{minipage}
%%% OUTPUT:
\[\displaystyle
\tag{\%{}o30}\label{o30} 
\left( {{A}_{1}}{{B}_{2}}{{C}_{3}}-{{A}_{2}}{{B}_{1}}{{C}_{3}}-{{A}_{1}}{{B}_{3}}{{C}_{2}}+{{A}_{3}}{{B}_{1}}{{C}_{2}}+{{A}_{2}}{{B}_{3}}{{C}_{1}}-{{A}_{3}}{{B}_{2}}{{C}_{1}}\right) \mathit{dx}\,\mathit{dy}\,\mathit{dz}\mbox{}
\]
%%%%%%%%%%%%%%%


\noindent
%%%%%%%%%%%%%%%
%%% INPUT:
\begin{minipage}[t]{8ex}\color{red}\bf
(\%{}i31) 
\end{minipage}
\begin{minipage}[t]{\textwidth}\color{blue}\tt
is(\%=\%th(2));
\end{minipage}
%%% OUTPUT:
\[\displaystyle
\tag{\%{}o31}\label{o31} 
\mbox{true}\mbox{}
\]
%%%%%%%%%%%%%%%

$\omega_A\wedge\Phi_B=(A\cdot B)
\,\mathrm{d}x\wedge\,\mathrm{d}y\wedge\,\mathrm{d}z$



\noindent
%%%%%%%%%%%%%%%
%%% INPUT:
\begin{minipage}[t]{8ex}\color{red}\bf
(\%{}i32) 
\end{minipage}
\begin{minipage}[t]{\textwidth}\color{blue}\tt
edit(\ensuremath{\omega}\_A\ensuremath{\sim }\ensuremath{\Phi}\_B);
\end{minipage}
%%% OUTPUT:
\[\displaystyle
\tag{\%{}o32}\label{o32} 
\left( {{A}_{3}}{{B}_{3}}+{{A}_{2}}{{B}_{2}}+{{A}_{1}}{{B}_{1}}\right) \mathit{dx}\,\mathit{dy}\,\mathit{dz}\mbox{}
\]
%%%%%%%%%%%%%%%


\noindent
%%%%%%%%%%%%%%%
%%% INPUT:
\begin{minipage}[t]{8ex}\color{red}\bf
(\%{}i33) 
\end{minipage}
\begin{minipage}[t]{\textwidth}\color{blue}\tt
edit((A.B)*(dx\ensuremath{\sim }dy\ensuremath{\sim }dz));
\end{minipage}
%%% OUTPUT:
\[\displaystyle
\tag{\%{}o33}\label{o33} 
\left( {{A}_{3}}{{B}_{3}}+{{A}_{2}}{{B}_{2}}+{{A}_{1}}{{B}_{1}}\right) \mathit{dx}\,\mathit{dy}\,\mathit{dz}\mbox{}
\]
%%%%%%%%%%%%%%%


\noindent
%%%%%%%%%%%%%%%
%%% INPUT:
\begin{minipage}[t]{8ex}\color{red}\bf
(\%{}i34) 
\end{minipage}
\begin{minipage}[t]{\textwidth}\color{blue}\tt
is(\%=\%th(2));
\end{minipage}
%%% OUTPUT:
\[\displaystyle
\tag{\%{}o34}\label{o34} 
\mbox{true}\mbox{}
\]
%%%%%%%%%%%%%%%

Differential vector calculus in differential form.


\subsection{Proposition 1.3.8.}



\noindent
%%%%%%%%%%%%%%%
%%% INPUT:
\begin{minipage}[t]{8ex}\color{red}\bf
(\%{}i35) 
\end{minipage}
\begin{minipage}[t]{\textwidth}\color{blue}\tt
kill(t,x,y,z,f)\$
\end{minipage}


\noindent
%%%%%%%%%%%%%%%
%%% INPUT:
\begin{minipage}[t]{8ex}\color{red}\bf
(\%{}i36) 
\end{minipage}
\begin{minipage}[t]{\textwidth}\color{blue}\tt
\ensuremath{\zeta}:[x,y,z]\$
\end{minipage}


\noindent
%%%%%%%%%%%%%%%
%%% INPUT:
\begin{minipage}[t]{8ex}\color{red}\bf
(\%{}i37) 
\end{minipage}
\begin{minipage}[t]{\textwidth}\color{blue}\tt
init\_cartan(\ensuremath{\zeta})\$
\end{minipage}


\noindent
%%%%%%%%%%%%%%%
%%% INPUT:
\begin{minipage}[t]{8ex}\color{red}\bf
(\%{}i39) 
\end{minipage}
\begin{minipage}[t]{\textwidth}\color{blue}\tt
F:[F\_1,F\_2,F\_3]\$\\
G:[G\_1,G\_2,G\_3]\$
\end{minipage}


\noindent
%%%%%%%%%%%%%%%
%%% INPUT:
\begin{minipage}[t]{8ex}\color{red}\bf
(\%{}i42) 
\end{minipage}
\begin{minipage}[t]{\textwidth}\color{blue}\tt
depends(f,\ensuremath{\zeta})\$\\
depends(F,\ensuremath{\zeta})\$\\
depends(G,\ensuremath{\zeta})\$
\end{minipage}


\noindent
%%%%%%%%%%%%%%%
%%% INPUT:
\begin{minipage}[t]{8ex}\color{red}\bf
(\%{}i44) 
\end{minipage}
\begin{minipage}[t]{\textwidth}\color{blue}\tt
ldisplay(\ensuremath{\omega}\_F:F.cartan\_basis)\$\\
ldisplay(\ensuremath{\omega}\_G:G.cartan\_basis)\$
\end{minipage}
%%% OUTPUT:
\[\displaystyle
\tag{\%{}t43}\label{t43} 
{{\mathit{\ensuremath{\omega}}}_{F}}={{F}_{3}}\mathit{dz}+{{F}_{2}}\mathit{dy}+{{F}_{1}}\mathit{dx}\mbox{}\]
\[\tag{\%{}t44}\label{t44} 
{{\mathit{\ensuremath{\omega}}}_{G}}={{G}_{3}}\mathit{dz}+{{G}_{2}}\mathit{dy}+{{G}_{1}}\mathit{dx}\mbox{}
\]
%%%%%%%%%%%%%%%


\noindent
%%%%%%%%%%%%%%%
%%% INPUT:
\begin{minipage}[t]{8ex}\color{red}\bf
(\%{}i46) 
\end{minipage}
\begin{minipage}[t]{\textwidth}\color{blue}\tt
cartan\_flow:makelist((dx\ensuremath{\sim }dy\ensuremath{\sim }dz)/k,k,cartan\_basis)\$\\
cartan\_sign:makelist((-1)\^{}(k+1),k,cartan\_dim)\$
\end{minipage}


\noindent
%%%%%%%%%%%%%%%
%%% INPUT:
\begin{minipage}[t]{8ex}\color{red}\bf
(\%{}i48) 
\end{minipage}
\begin{minipage}[t]{\textwidth}\color{blue}\tt
ldisplay(\ensuremath{\Phi}\_F:F.(cartan\_flow*cartan\_sign))\$\\
ldisplay(\ensuremath{\Phi}\_G:G.(cartan\_flow*cartan\_sign))\$
\end{minipage}
%%% OUTPUT:
\[\displaystyle
\tag{\%{}t47}\label{t47} 
{{\mathit{\ensuremath{\Phi}}}_{F}}={{F}_{1}}\mathit{dy}\,\mathit{dz}-{{F}_{2}}\mathit{dx}\,\mathit{dz}+{{F}_{3}}\mathit{dx}\,\mathit{dy}\mbox{}\]
\[\tag{\%{}t48}\label{t48} 
{{\mathit{\ensuremath{\Phi}}}_{G}}={{G}_{1}}\mathit{dy}\,\mathit{dz}-{{G}_{2}}\mathit{dx}\,\mathit{dz}+{{G}_{3}}\mathit{dx}\,\mathit{dy}\mbox{}
\]
%%%%%%%%%%%%%%%

$\mathrm{d}f=\omega_{\nabla f}$



\noindent
%%%%%%%%%%%%%%%
%%% INPUT:
\begin{minipage}[t]{8ex}\color{red}\bf
(\%{}i49) 
\end{minipage}
\begin{minipage}[t]{\textwidth}\color{blue}\tt
ext\_diff(f);
\end{minipage}
%%% OUTPUT:
\[\displaystyle
\tag{\%{}o49}\label{o49} 
\left( {{f}_{z}}\right) \,\mathit{dz}+\left( {{f}_{y}}\right) \,\mathit{dy}+\left( {{f}_{x}}\right) \,\mathit{dx}\mbox{}
\]
%%%%%%%%%%%%%%%


\noindent
%%%%%%%%%%%%%%%
%%% INPUT:
\begin{minipage}[t]{8ex}\color{red}\bf
(\%{}i50) 
\end{minipage}
\begin{minipage}[t]{\textwidth}\color{blue}\tt
ev(express(grad(f)),diff).cartan\_basis;
\end{minipage}
%%% OUTPUT:
\[\displaystyle
\tag{\%{}o50}\label{o50} 
\left( {{f}_{z}}\right) \,\mathit{dz}+\left( {{f}_{y}}\right) \,\mathit{dy}+\left( {{f}_{x}}\right) \,\mathit{dx}\mbox{}
\]
%%%%%%%%%%%%%%%


\noindent
%%%%%%%%%%%%%%%
%%% INPUT:
\begin{minipage}[t]{8ex}\color{red}\bf
(\%{}i51) 
\end{minipage}
\begin{minipage}[t]{\textwidth}\color{blue}\tt
is(\%=\%th(2));
\end{minipage}
%%% OUTPUT:
\[\displaystyle
\tag{\%{}o51}\label{o51} 
\mbox{true}\mbox{}
\]
%%%%%%%%%%%%%%%

$\mathrm{d}\omega_F=\Phi_{\nabla\times\,F}$



\noindent
%%%%%%%%%%%%%%%
%%% INPUT:
\begin{minipage}[t]{8ex}\color{red}\bf
(\%{}i52) 
\end{minipage}
\begin{minipage}[t]{\textwidth}\color{blue}\tt
edit(ext\_diff(\ensuremath{\omega}\_F));
\end{minipage}
%%% OUTPUT:
\[\displaystyle
\tag{\%{}o52}\label{o52} 
\left( {{{{F}_{3}}}_{y}}-{{{{F}_{2}}}_{z}}\right) \mathit{dy}\,\mathit{dz}+\left( {{{{F}_{3}}}_{x}}-{{{{F}_{1}}}_{z}}\right) \mathit{dx}\,\mathit{dz}+\left( {{{{F}_{2}}}_{x}}-{{{{F}_{1}}}_{y}}\right) \mathit{dx}\,\mathit{dy}\mbox{}
\]
%%%%%%%%%%%%%%%


\noindent
%%%%%%%%%%%%%%%
%%% INPUT:
\begin{minipage}[t]{8ex}\color{red}\bf
(\%{}i53) 
\end{minipage}
\begin{minipage}[t]{\textwidth}\color{blue}\tt
edit(ev(express(curl(F)),diff).(cartan\_flow*cartan\_sign));
\end{minipage}
%%% OUTPUT:
\[\displaystyle
\tag{\%{}o53}\label{o53} 
\left( {{{{F}_{3}}}_{y}}-{{{{F}_{2}}}_{z}}\right) \mathit{dy}\,\mathit{dz}+\left( {{{{F}_{3}}}_{x}}-{{{{F}_{1}}}_{z}}\right) \mathit{dx}\,\mathit{dz}+\left( {{{{F}_{2}}}_{x}}-{{{{F}_{1}}}_{y}}\right) \mathit{dx}\,\mathit{dy}\mbox{}
\]
%%%%%%%%%%%%%%%


\noindent
%%%%%%%%%%%%%%%
%%% INPUT:
\begin{minipage}[t]{8ex}\color{red}\bf
(\%{}i54) 
\end{minipage}
\begin{minipage}[t]{\textwidth}\color{blue}\tt
is(\%=\%th(2));
\end{minipage}
%%% OUTPUT:
\[\displaystyle
\tag{\%{}o54}\label{o54} 
\mbox{true}\mbox{}
\]
%%%%%%%%%%%%%%%

$\mathrm{d}\Phi_G=(\nabla\cdot\,G)
\,\mathrm{d}x\wedge\,\mathrm{d}y\wedge\,\mathrm{d}z$



\noindent
%%%%%%%%%%%%%%%
%%% INPUT:
\begin{minipage}[t]{8ex}\color{red}\bf
(\%{}i55) 
\end{minipage}
\begin{minipage}[t]{\textwidth}\color{blue}\tt
edit(ext\_diff(\ensuremath{\Phi}\_G));
\end{minipage}
%%% OUTPUT:
\[\displaystyle
\tag{\%{}o55}\label{o55} 
\left( {{{{G}_{3}}}_{z}}+{{{{G}_{2}}}_{y}}+{{{{G}_{1}}}_{x}}\right) \mathit{dx}\,\mathit{dy}\,\mathit{dz}\mbox{}
\]
%%%%%%%%%%%%%%%


\noindent
%%%%%%%%%%%%%%%
%%% INPUT:
\begin{minipage}[t]{8ex}\color{red}\bf
(\%{}i56) 
\end{minipage}
\begin{minipage}[t]{\textwidth}\color{blue}\tt
ev(express(div(G)),diff)*(dx\ensuremath{\sim }dy\ensuremath{\sim }dz);
\end{minipage}
%%% OUTPUT:
\[\displaystyle
\tag{\%{}o56}\label{o56} 
\left( {{{{G}_{3}}}_{z}}+{{{{G}_{2}}}_{y}}+{{{{G}_{1}}}_{x}}\right) \mathit{dx}\,\mathit{dy}\,\mathit{dz}\mbox{}
\]
%%%%%%%%%%%%%%%


\noindent
%%%%%%%%%%%%%%%
%%% INPUT:
\begin{minipage}[t]{8ex}\color{red}\bf
(\%{}i57) 
\end{minipage}
\begin{minipage}[t]{\textwidth}\color{blue}\tt
is(\%=\%th(2));
\end{minipage}
%%% OUTPUT:
\[\displaystyle
\tag{\%{}o57}\label{o57} 
\mbox{true}\mbox{}
\]
%%%%%%%%%%%%%%%
\pagebreak


\section{paths and curves}



\noindent
%%%%%%%%%%%%%%%
%%% INPUT:
\begin{minipage}[t]{8ex}\color{red}\bf
(\%{}i58) 
\end{minipage}
\begin{minipage}[t]{\textwidth}\color{blue}\tt
kill(labels)\$
\end{minipage}

\subsection{Example 1.4.4.}


Let $\alpha(t)=p+t(q-p)$ for a given pair of distinct points
$p,q\in\mathbb{R}^n$. You should identify $\alpha$ as the line connecting
point $p=\alpha(0)$ and $q=\alpha(1)$. If we define $v=q-p$ then the
velocity of $\alpha$ is given by: $$\alpha^\prime(t)=\left.{v^1
\dfrac{\partial}{\partial x^1}}\right\vert_{\alpha(t)}+\left.{v^2
\dfrac{\partial}{\partial x^2}}\right\vert_{\alpha(t)}+\dots+\left.{v^n
\dfrac{\partial}{\partial x^n}}\right\vert_{\alpha(t)}$$


Specializing to $n=2$ and $v=\langle{a,b}\rangle$ we have $\alpha(t)=
(p^1+t a,p^2+t b)$ and $$\alpha^\prime(t)=\left.{a\dfrac{\partial}
{\partial x}}\right\vert_{\alpha(t)}+\left.{b\dfrac{\partial}
{\partial y}}\right\vert_{\alpha(t)}$$


As an easy to check case, take $p=(0,0)$ hence $p^1=0$ and $p^2=0$ hence
$\alpha^\prime(t)[f]=2 t(a^2+b^2)$. For $t>0$ we see $f$ is increasing as
we travel away from the origin along the line $\alpha(t)$. But, $f$ is just
the distance from the origin squared so the rate of change is quite
reasonable. If we were to impose $a^2+b^2=1$ then $t$ represents the
distance from the origin and the result reduces to $\alpha^\prime[f]=2 t$
which makes sense as $f(\alpha(t))=(t a)^2+(t b)^2=t^2(a^2+b^2)=t^2$


Notice that $\alpha^\prime(t)[f]$ gives the usual third-semester-American
calculus directional derivative in the direction of $\alpha^\prime(t)[f]$
only if we choose a parameter $t$ for which $\lVert\alpha^\prime(t)\rVert=1$.
This choice of parametrization is known as the \textit{arclength} or
\textit{unit-speed} parametrization.



\noindent
%%%%%%%%%%%%%%%
%%% INPUT:
\begin{minipage}[t]{8ex}\color{red}\bf
(\%{}i1) 
\end{minipage}
\begin{minipage}[t]{\textwidth}\color{blue}\tt
kill(t,x,y,z)\$
\end{minipage}


\noindent
%%%%%%%%%%%%%%%
%%% INPUT:
\begin{minipage}[t]{8ex}\color{red}\bf
(\%{}i2) 
\end{minipage}
\begin{minipage}[t]{\textwidth}\color{blue}\tt
\ensuremath{\zeta}:[x,y]\$
\end{minipage}


\noindent
%%%%%%%%%%%%%%%
%%% INPUT:
\begin{minipage}[t]{8ex}\color{red}\bf
(\%{}i3) 
\end{minipage}
\begin{minipage}[t]{\textwidth}\color{blue}\tt
scalefactors(\ensuremath{\zeta})\$
\end{minipage}


\noindent
%%%%%%%%%%%%%%%
%%% INPUT:
\begin{minipage}[t]{8ex}\color{red}\bf
(\%{}i4) 
\end{minipage}
\begin{minipage}[t]{\textwidth}\color{blue}\tt
init\_cartan(\ensuremath{\zeta})\$
\end{minipage}


\noindent
%%%%%%%%%%%%%%%
%%% INPUT:
\begin{minipage}[t]{8ex}\color{red}\bf
(\%{}i5) 
\end{minipage}
\begin{minipage}[t]{\textwidth}\color{blue}\tt
declare([a,b,p\_1,p\_2],constant)\$
\end{minipage}


\noindent
%%%%%%%%%%%%%%%
%%% INPUT:
\begin{minipage}[t]{8ex}\color{red}\bf
(\%{}i6) 
\end{minipage}
\begin{minipage}[t]{\textwidth}\color{blue}\tt
v:[a,b]\$
\end{minipage}


\noindent
%%%%%%%%%%%%%%%
%%% INPUT:
\begin{minipage}[t]{8ex}\color{red}\bf
(\%{}i7) 
\end{minipage}
\begin{minipage}[t]{\textwidth}\color{blue}\tt
P:[p\_1,p\_2]\$
\end{minipage}


\noindent
%%%%%%%%%%%%%%%
%%% INPUT:
\begin{minipage}[t]{8ex}\color{red}\bf
(\%{}i8) 
\end{minipage}
\begin{minipage}[t]{\textwidth}\color{blue}\tt
ldisplay(\ensuremath{\alpha}:P+t*v)\$
\end{minipage}
%%% OUTPUT:
\[\displaystyle
\tag{\%{}t8}\label{t8} 
\mathit{\ensuremath{\alpha}}=[at+{{p}_{1}},bt+{{p}_{2}}]\mbox{}
\]
%%%%%%%%%%%%%%%


\noindent
%%%%%%%%%%%%%%%
%%% INPUT:
\begin{minipage}[t]{8ex}\color{red}\bf
(\%{}i9) 
\end{minipage}
\begin{minipage}[t]{\textwidth}\color{blue}\tt
ldisplay(\ensuremath{\alpha}\ensuremath{\backslash}':diff(\ensuremath{\alpha},t))\$
\end{minipage}
%%% OUTPUT:
\[\displaystyle
\tag{\%{}t9}\label{t9} 
\mathit{\ensuremath{\alpha}'}=[a,b]\mbox{}
\]
%%%%%%%%%%%%%%%


\noindent
%%%%%%%%%%%%%%%
%%% INPUT:
\begin{minipage}[t]{8ex}\color{red}\bf
(\%{}i10) 
\end{minipage}
\begin{minipage}[t]{\textwidth}\color{blue}\tt
norm(\ensuremath{\alpha}\ensuremath{\backslash}');
\end{minipage}
%%% OUTPUT:
\[\displaystyle
\tag{\%{}o10}\label{o10} 
\sqrt{{{b}^{2}}+{{a}^{2}}}\mbox{}
\]
%%%%%%%%%%%%%%%


\noindent
%%%%%%%%%%%%%%%
%%% INPUT:
\begin{minipage}[t]{8ex}\color{red}\bf
(\%{}i11) 
\end{minipage}
\begin{minipage}[t]{\textwidth}\color{blue}\tt
f:x\ensuremath{^2}+y\ensuremath{^2}\$
\end{minipage}


\noindent
%%%%%%%%%%%%%%%
%%% INPUT:
\begin{minipage}[t]{8ex}\color{red}\bf
(\%{}i12) 
\end{minipage}
\begin{minipage}[t]{\textwidth}\color{blue}\tt
ldisplay(gradf:ev(express(grad(f)),diff))\$
\end{minipage}
%%% OUTPUT:
\[\displaystyle
\tag{\%{}t12}\label{t12} 
\mathit{gradf}=[2x,2y]\mbox{}
\]
%%%%%%%%%%%%%%%


\noindent
%%%%%%%%%%%%%%%
%%% INPUT:
\begin{minipage}[t]{8ex}\color{red}\bf
(\%{}i13) 
\end{minipage}
\begin{minipage}[t]{\textwidth}\color{blue}\tt
at(\ensuremath{\alpha}\ensuremath{\backslash}'.gradf,map("=",\ensuremath{\zeta},\ensuremath{\alpha}));
\end{minipage}
%%% OUTPUT:
\[\displaystyle
\tag{\%{}o13}\label{o13} 
2b\,\left( bt+{{p}_{2}}\right) +2a\,\left( at+{{p}_{1}}\right) \mbox{}
\]
%%%%%%%%%%%%%%%

\subsection{Example 1.4.5.}


Let $R,m>0$ be constants and $\alpha(t)=\langle{R\cos(t),R\sin(t),m t}
\rangle$ for $t\in\mathbb{R}$. We say $\alpha$ is a helix with slope $m$
and radius $R$. Notice $\alpha(t)$ falls on the cylinder $x^2+y^2=R^2$.
Of course, we could define helices around other circular cylinders.
The velocity vector field for $\alpha$ is given by: $$\alpha^\prime=\left.
{\left({-R\sin(t)\dfrac{\partial}{\partial x}+R\cos(t)\dfrac{\partial}
{\partial y}+m\dfrac{\partial}{\partial z}}\right)}\right\vert_{\alpha(t)}$$


Then, $f(x,y,z)=x^2+y^2$ has $$\alpha^\prime(t)[f]=\left.{\left({-2 x
R\sin(t)+2 y R\cos(t)}\right)}\right\rvert_{\alpha(t)}=-2 R^2\cos(t)
\sin{t}+2 R^2\sin(t)\cos(t)=0.$$



\noindent
%%%%%%%%%%%%%%%
%%% INPUT:
\begin{minipage}[t]{8ex}\color{red}\bf
(\%{}i14) 
\end{minipage}
\begin{minipage}[t]{\textwidth}\color{blue}\tt
kill(t,x,y,z,R,m)\$
\end{minipage}


\noindent
%%%%%%%%%%%%%%%
%%% INPUT:
\begin{minipage}[t]{8ex}\color{red}\bf
(\%{}i15) 
\end{minipage}
\begin{minipage}[t]{\textwidth}\color{blue}\tt
\ensuremath{\zeta}:[x,y,z]\$
\end{minipage}


\noindent
%%%%%%%%%%%%%%%
%%% INPUT:
\begin{minipage}[t]{8ex}\color{red}\bf
(\%{}i16) 
\end{minipage}
\begin{minipage}[t]{\textwidth}\color{blue}\tt
scalefactors(\ensuremath{\zeta})\$
\end{minipage}


\noindent
%%%%%%%%%%%%%%%
%%% INPUT:
\begin{minipage}[t]{8ex}\color{red}\bf
(\%{}i17) 
\end{minipage}
\begin{minipage}[t]{\textwidth}\color{blue}\tt
init\_cartan(\ensuremath{\zeta})\$
\end{minipage}


\noindent
%%%%%%%%%%%%%%%
%%% INPUT:
\begin{minipage}[t]{8ex}\color{red}\bf
(\%{}i18) 
\end{minipage}
\begin{minipage}[t]{\textwidth}\color{blue}\tt
orderless(m,R)\$
\end{minipage}


\noindent
%%%%%%%%%%%%%%%
%%% INPUT:
\begin{minipage}[t]{8ex}\color{red}\bf
(\%{}i19) 
\end{minipage}
\begin{minipage}[t]{\textwidth}\color{blue}\tt
declare([R,m],constant)\$
\end{minipage}


\noindent
%%%%%%%%%%%%%%%
%%% INPUT:
\begin{minipage}[t]{8ex}\color{red}\bf
(\%{}i20) 
\end{minipage}
\begin{minipage}[t]{\textwidth}\color{blue}\tt
assume(R\ensuremath{>}0,m\ensuremath{>}0)\$
\end{minipage}


\noindent
%%%%%%%%%%%%%%%
%%% INPUT:
\begin{minipage}[t]{8ex}\color{red}\bf
(\%{}i21) 
\end{minipage}
\begin{minipage}[t]{\textwidth}\color{blue}\tt
ldisplay(\ensuremath{\alpha}:[R*cos(t),R*sin(t),m*t])\$
\end{minipage}
%%% OUTPUT:
\[\displaystyle
\tag{\%{}t21}\label{t21} 
\mathit{\ensuremath{\alpha}}=[R\,\cos{(t)},R\,\sin{(t)},mt]\mbox{}
\]
%%%%%%%%%%%%%%%


\noindent
%%%%%%%%%%%%%%%
%%% INPUT:
\begin{minipage}[t]{8ex}\color{red}\bf
(\%{}i22) 
\end{minipage}
\begin{minipage}[t]{\textwidth}\color{blue}\tt
ldisplay(\ensuremath{\alpha}\ensuremath{\backslash}':diff(\ensuremath{\alpha},t))\$
\end{minipage}
%%% OUTPUT:
\[\displaystyle
\tag{\%{}t22}\label{t22} 
\mathit{\ensuremath{\alpha}'}=[-R\,\sin{(t)},R\,\cos{(t)},m]\mbox{}
\]
%%%%%%%%%%%%%%%


\noindent
%%%%%%%%%%%%%%%
%%% INPUT:
\begin{minipage}[t]{8ex}\color{red}\bf
(\%{}i23) 
\end{minipage}
\begin{minipage}[t]{\textwidth}\color{blue}\tt
trigsimp(norm(\ensuremath{\alpha}\ensuremath{\backslash}'));
\end{minipage}
%%% OUTPUT:
\[\displaystyle
\tag{\%{}o23}\label{o23} 
\sqrt{{{R}^{2}}+{{m}^{2}}}\mbox{}
\]
%%%%%%%%%%%%%%%


\noindent
%%%%%%%%%%%%%%%
%%% INPUT:
\begin{minipage}[t]{8ex}\color{red}\bf
(\%{}i24) 
\end{minipage}
\begin{minipage}[t]{\textwidth}\color{blue}\tt
f:x\ensuremath{^2}+y\ensuremath{^2}\$
\end{minipage}


\noindent
%%%%%%%%%%%%%%%
%%% INPUT:
\begin{minipage}[t]{8ex}\color{red}\bf
(\%{}i25) 
\end{minipage}
\begin{minipage}[t]{\textwidth}\color{blue}\tt
ldisplay(gradf:ev(express(grad(f)),diff))\$
\end{minipage}
%%% OUTPUT:
\[\displaystyle
\tag{\%{}t25}\label{t25} 
\mathit{gradf}=[2x,2y,0]\mbox{}
\]
%%%%%%%%%%%%%%%


\noindent
%%%%%%%%%%%%%%%
%%% INPUT:
\begin{minipage}[t]{8ex}\color{red}\bf
(\%{}i26) 
\end{minipage}
\begin{minipage}[t]{\textwidth}\color{blue}\tt
at(\ensuremath{\alpha}\ensuremath{\backslash}'.gradf,map("=",\ensuremath{\zeta},\ensuremath{\alpha}));
\end{minipage}
%%% OUTPUT:
\[\displaystyle
\tag{\%{}o26}\label{o26} 
0\mbox{}
\]
%%%%%%%%%%%%%%%

\subsection{Example 1.4.8.}


Consider the helix defined by $R,m > 0$ and $\alpha(t)=\left({R
\cos(t),R\sin(t),m\,t}\right)$ for $t\in\mathbb{R}$. The speed of
this helix is simply $\Vert{\alpha^\prime(t)}\Vert=\sqrt{R^2+m^2}$.
Let $h(s)=\frac{s}{\sqrt{R^2+m^2}}$ then if $\beta$ is $\alpha$
reparametrized by $h$ we calculate by Preposition 1.4.7



\noindent
%%%%%%%%%%%%%%%
%%% INPUT:
\begin{minipage}[t]{8ex}\color{red}\bf
(\%{}i27) 
\end{minipage}
\begin{minipage}[t]{\textwidth}\color{blue}\tt
ldisplay(h:s/\ensuremath{\sqrt{}}(R\ensuremath{^2}+m\ensuremath{^2}))\$
\end{minipage}
%%% OUTPUT:
\[\displaystyle
\tag{\%{}t27}\label{t27} 
h=\frac{s}{\sqrt{{{R}^{2}}+{{m}^{2}}}}\mbox{}
\]
%%%%%%%%%%%%%%%


\noindent
%%%%%%%%%%%%%%%
%%% INPUT:
\begin{minipage}[t]{8ex}\color{red}\bf
(\%{}i28) 
\end{minipage}
\begin{minipage}[t]{\textwidth}\color{blue}\tt
ldisplay(\ensuremath{\beta}\ensuremath{\backslash}':diff(h,s)*at(\ensuremath{\alpha}\ensuremath{\backslash}',t=h))\$
\end{minipage}
%%% OUTPUT:
\[\displaystyle
\tag{\%{}t28}\label{t28} 
\mathit{\ensuremath{\beta}'}=\left[-\frac{R\,\sin{\left( \frac{s}{\sqrt{{{R}^{2}}+{{m}^{2}}}}\right) }}{\sqrt{{{R}^{2}}+{{m}^{2}}}},\frac{R\,\cos{\left( \frac{s}{\sqrt{{{R}^{2}}+{{m}^{2}}}}\right) }}{\sqrt{{{R}^{2}}+{{m}^{2}}}},\frac{m}{\sqrt{{{R}^{2}}+{{m}^{2}}}}\right]\mbox{}
\]
%%%%%%%%%%%%%%%


\noindent
%%%%%%%%%%%%%%%
%%% INPUT:
\begin{minipage}[t]{8ex}\color{red}\bf
(\%{}i29) 
\end{minipage}
\begin{minipage}[t]{\textwidth}\color{blue}\tt
trigsimp(norm(\ensuremath{\beta}\ensuremath{\backslash}'));
\end{minipage}
%%% OUTPUT:
\[\displaystyle
\tag{\%{}o29}\label{o29} 
1\mbox{}
\]
%%%%%%%%%%%%%%%


\noindent
%%%%%%%%%%%%%%%
%%% INPUT:
\begin{minipage}[t]{8ex}\color{red}\bf
(\%{}i30) 
\end{minipage}
\begin{minipage}[t]{\textwidth}\color{blue}\tt
g:z\$
\end{minipage}


\noindent
%%%%%%%%%%%%%%%
%%% INPUT:
\begin{minipage}[t]{8ex}\color{red}\bf
(\%{}i31) 
\end{minipage}
\begin{minipage}[t]{\textwidth}\color{blue}\tt
ldisplay(gradg:ev(express(grad(g)),diff))\$
\end{minipage}
%%% OUTPUT:
\[\displaystyle
\tag{\%{}t31}\label{t31} 
\mathit{gradg}=[0,0,1]\mbox{}
\]
%%%%%%%%%%%%%%%


\noindent
%%%%%%%%%%%%%%%
%%% INPUT:
\begin{minipage}[t]{8ex}\color{red}\bf
(\%{}i32) 
\end{minipage}
\begin{minipage}[t]{\textwidth}\color{blue}\tt
at(\ensuremath{\beta}\ensuremath{\backslash}'.gradg,map("=",\ensuremath{\zeta},\ensuremath{\alpha}));
\end{minipage}
%%% OUTPUT:
\[\displaystyle
\tag{\%{}o32}\label{o32} 
\frac{m}{\sqrt{{{R}^{2}}+{{m}^{2}}}}\mbox{}
\]
%%%%%%%%%%%%%%%


\noindent
%%%%%%%%%%%%%%%
%%% INPUT:
\begin{minipage}[t]{8ex}\color{red}\bf
(\%{}i33) 
\end{minipage}
\begin{minipage}[t]{\textwidth}\color{blue}\tt
unorder()\$
\end{minipage}
\pagebreak


\section{the push-forward or differential of a map}


\subsection{Example 1.5.1.}



\noindent
%%%%%%%%%%%%%%%
%%% INPUT:
\begin{minipage}[t]{8ex}\color{red}\bf
(\%{}i34) 
\end{minipage}
\begin{minipage}[t]{\textwidth}\color{blue}\tt
kill(t,x\_1,x\_2,a,b)\$
\end{minipage}

Let $F(x^1,x^2)=\left({x^1+x^2,x^1-x^2}\right)$. Consider a
parametrized curve $\alpha(t)=\left({a(t),b(t)}\right)$. The image of
$\alpha$  under $F$ is: $$\left({F\circ\alpha(t)}\right)=
F\left({a(t),b(t)}\right)=\left({a(t)+b(t),a(t)-b(t)}\right)$$



\noindent
%%%%%%%%%%%%%%%
%%% INPUT:
\begin{minipage}[t]{8ex}\color{red}\bf
(\%{}i35) 
\end{minipage}
\begin{minipage}[t]{\textwidth}\color{blue}\tt
\ensuremath{\zeta}:[x\_1,x\_2]\$
\end{minipage}


\noindent
%%%%%%%%%%%%%%%
%%% INPUT:
\begin{minipage}[t]{8ex}\color{red}\bf
(\%{}i36) 
\end{minipage}
\begin{minipage}[t]{\textwidth}\color{blue}\tt
depends(\ensuremath{\zeta},t)\$
\end{minipage}


\noindent
%%%%%%%%%%%%%%%
%%% INPUT:
\begin{minipage}[t]{8ex}\color{red}\bf
(\%{}i37) 
\end{minipage}
\begin{minipage}[t]{\textwidth}\color{blue}\tt
F:[x\_1+x\_2,x\_1-x\_2]\$
\end{minipage}


\noindent
%%%%%%%%%%%%%%%
%%% INPUT:
\begin{minipage}[t]{8ex}\color{red}\bf
(\%{}i38) 
\end{minipage}
\begin{minipage}[t]{\textwidth}\color{blue}\tt
ldisplay(\ensuremath{\alpha}:[a,b])\$
\end{minipage}
%%% OUTPUT:
\[\displaystyle
\tag{\%{}t38}\label{t38} 
\mathit{\ensuremath{\alpha}}=[a,b]\mbox{}
\]
%%%%%%%%%%%%%%%


\noindent
%%%%%%%%%%%%%%%
%%% INPUT:
\begin{minipage}[t]{8ex}\color{red}\bf
(\%{}i39) 
\end{minipage}
\begin{minipage}[t]{\textwidth}\color{blue}\tt
depends(\ensuremath{\alpha},t)\$
\end{minipage}


\noindent
%%%%%%%%%%%%%%%
%%% INPUT:
\begin{minipage}[t]{8ex}\color{red}\bf
(\%{}i40) 
\end{minipage}
\begin{minipage}[t]{\textwidth}\color{blue}\tt
ldisplay(\ensuremath{\alpha}\ensuremath{\backslash}':diff(\ensuremath{\alpha},t))\$
\end{minipage}
%%% OUTPUT:
\[\displaystyle
\tag{\%{}t40}\label{t40} 
\mathit{\ensuremath{\alpha}'}=[{{a}_{t}},{{b}_{t}}]\mbox{}
\]
%%%%%%%%%%%%%%%


\noindent
%%%%%%%%%%%%%%%
%%% INPUT:
\begin{minipage}[t]{8ex}\color{red}\bf
(\%{}i41) 
\end{minipage}
\begin{minipage}[t]{\textwidth}\color{blue}\tt
at(F,map("=",\ensuremath{\zeta},\ensuremath{\alpha}));
\end{minipage}
%%% OUTPUT:
\[\displaystyle
\tag{\%{}o41}\label{o41} 
[b+a,a-b]\mbox{}
\]
%%%%%%%%%%%%%%%


\noindent
%%%%%%%%%%%%%%%
%%% INPUT:
\begin{minipage}[t]{8ex}\color{red}\bf
(\%{}i42) 
\end{minipage}
\begin{minipage}[t]{\textwidth}\color{blue}\tt
diff(\%,t);
\end{minipage}
%%% OUTPUT:
\[\displaystyle
\tag{\%{}o42}\label{o42} 
[{{b}_{t}}+{{a}_{t}},{{a}_{t}}-{{b}_{t}}]\mbox{}
\]
%%%%%%%%%%%%%%%


\noindent
%%%%%%%%%%%%%%%
%%% INPUT:
\begin{minipage}[t]{8ex}\color{red}\bf
(\%{}i43) 
\end{minipage}
\begin{minipage}[t]{\textwidth}\color{blue}\tt
list\_matrix\_entries(jacobian(F,\ensuremath{\zeta}).\ensuremath{\alpha}\ensuremath{\backslash}');
\end{minipage}
%%% OUTPUT:
\[\displaystyle
\tag{\%{}o43}\label{o43} 
[{{b}_{t}}+{{a}_{t}},{{a}_{t}}-{{b}_{t}}]\mbox{}
\]
%%%%%%%%%%%%%%%


\noindent
%%%%%%%%%%%%%%%
%%% INPUT:
\begin{minipage}[t]{8ex}\color{red}\bf
(\%{}i44) 
\end{minipage}
\begin{minipage}[t]{\textwidth}\color{blue}\tt
is(\%=\%th(2));
\end{minipage}
%%% OUTPUT:
\[\displaystyle
\tag{\%{}o44}\label{o44} 
\mbox{true}\mbox{}
\]
%%%%%%%%%%%%%%%

\pagebreak
\subsection{Example 1.5.2.}


Another example, $F(x^1,x^2)=\left({e^{x^1+x^2},\sin(x^2),\cos(x^2)}
\right)$. Once more, consider the curve $\alpha=(a,b)$ hence
$\alpha^\prime=\langle{a^\prime,b^\prime}\rangle$ and
$$(F\circ\alpha)^\prime=\langle{e^{a+b}\,(a^\prime)+b^\prime),\cos(b)
\,b^\prime,(-\sin(b))\,b^\prime}\rangle$$


\subsection{Example 1.5.4.}



\noindent
%%%%%%%%%%%%%%%
%%% INPUT:
\begin{minipage}[t]{8ex}\color{red}\bf
(\%{}i45) 
\end{minipage}
\begin{minipage}[t]{\textwidth}\color{blue}\tt
kill(t,x,y)\$
\end{minipage}

Let $F(x,y)=x^2+y^2$ then $F(x,y)=R^2$ is a circle and



\noindent
%%%%%%%%%%%%%%%
%%% INPUT:
\begin{minipage}[t]{8ex}\color{red}\bf
(\%{}i46) 
\end{minipage}
\begin{minipage}[t]{\textwidth}\color{blue}\tt
\ensuremath{\zeta}:[x,y]\$
\end{minipage}


\noindent
%%%%%%%%%%%%%%%
%%% INPUT:
\begin{minipage}[t]{8ex}\color{red}\bf
(\%{}i47) 
\end{minipage}
\begin{minipage}[t]{\textwidth}\color{blue}\tt
scalefactors(\ensuremath{\zeta})\$
\end{minipage}


\noindent
%%%%%%%%%%%%%%%
%%% INPUT:
\begin{minipage}[t]{8ex}\color{red}\bf
(\%{}i48) 
\end{minipage}
\begin{minipage}[t]{\textwidth}\color{blue}\tt
F:x\ensuremath{^2}+y\ensuremath{^2}\$
\end{minipage}


\noindent
%%%%%%%%%%%%%%%
%%% INPUT:
\begin{minipage}[t]{8ex}\color{red}\bf
(\%{}i49) 
\end{minipage}
\begin{minipage}[t]{\textwidth}\color{blue}\tt
ldisplay(J\_F:ev(express(grad(F)),diff))\$
\end{minipage}
%%% OUTPUT:
\[\displaystyle
\tag{\%{}t49}\label{t49} 
{{J}_{F}}=[2x,2y]\mbox{}
\]
%%%%%%%%%%%%%%%

We see $y\neq 0$ implies the last column is nonzero hence we may
solve for $y$ near such points. In this case, $G(x)=\pm\sqrt{R^2-x^2}$
where we choose $\pm$ appropriate to the location of the local solution.


\subsection{Example 1.5.5.}



\noindent
%%%%%%%%%%%%%%%
%%% INPUT:
\begin{minipage}[t]{8ex}\color{red}\bf
(\%{}i50) 
\end{minipage}
\begin{minipage}[t]{\textwidth}\color{blue}\tt
kill(t,x,y,z)\$
\end{minipage}

Let $F(x,y,z)=cos(x)+y+z^2$ then



\noindent
%%%%%%%%%%%%%%%
%%% INPUT:
\begin{minipage}[t]{8ex}\color{red}\bf
(\%{}i51) 
\end{minipage}
\begin{minipage}[t]{\textwidth}\color{blue}\tt
\ensuremath{\zeta}:[x,y,z]\$
\end{minipage}


\noindent
%%%%%%%%%%%%%%%
%%% INPUT:
\begin{minipage}[t]{8ex}\color{red}\bf
(\%{}i52) 
\end{minipage}
\begin{minipage}[t]{\textwidth}\color{blue}\tt
scalefactors(\ensuremath{\zeta})\$
\end{minipage}


\noindent
%%%%%%%%%%%%%%%
%%% INPUT:
\begin{minipage}[t]{8ex}\color{red}\bf
(\%{}i53) 
\end{minipage}
\begin{minipage}[t]{\textwidth}\color{blue}\tt
F:cos(x)+y+z\ensuremath{^2}\$
\end{minipage}


\noindent
%%%%%%%%%%%%%%%
%%% INPUT:
\begin{minipage}[t]{8ex}\color{red}\bf
(\%{}i54) 
\end{minipage}
\begin{minipage}[t]{\textwidth}\color{blue}\tt
ldisplay(J\_F:ev(express(grad(F)),diff))\$
\end{minipage}
%%% OUTPUT:
\[\displaystyle
\tag{\%{}t54}\label{t54} 
{{J}_{F}}=[-\sin{(x)},1,2z]\mbox{}
\]
%%%%%%%%%%%%%%%

this tells me I can solve for $z=z(x,y)$ when $z\neq 0$, or I can
solve for $y=y(x,z)$ anywhere on $F(x,y,z)=c$, or I can solve for
$x=x(y,z)$ when $x\neq n\pi$ for $n\in\mathbb{Z}$. Notice we can
rearrange coordinates to put $x$ or $y$ as the last coordinate.

\end{document}
