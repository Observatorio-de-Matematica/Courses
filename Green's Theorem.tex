\documentclass{article}

%% Created with wxMaxima 16.04.2

\setlength{\parskip}{\medskipamount}
\setlength{\parindent}{0pt}
\usepackage[utf8]{inputenc}
\DeclareUnicodeCharacter{00B5}{\ensuremath{\mu}}
\usepackage{graphicx}
\usepackage{color}
\usepackage{amsmath}
\usepackage{ifthen}
\newsavebox{\picturebox}
\newlength{\pictureboxwidth}
\newlength{\pictureboxheight}
\newcommand{\includeimage}[1]{
    \savebox{\picturebox}{\includegraphics{#1}}
    \settoheight{\pictureboxheight}{\usebox{\picturebox}}
    \settowidth{\pictureboxwidth}{\usebox{\picturebox}}
    \ifthenelse{\lengthtest{\pictureboxwidth > .95\linewidth}}
    {
        \includegraphics[width=.95\linewidth,height=.80\textheight,keepaspectratio]{#1}
    }
    {
        \ifthenelse{\lengthtest{\pictureboxheight>.80\textheight}}
        {
            \includegraphics[width=.95\linewidth,height=.80\textheight,keepaspectratio]{#1}
            
        }
        {
            \includegraphics{#1}
        }
    }
}
\newlength{\thislabelwidth}
\DeclareMathOperator{\abs}{abs}
\usepackage{animate} % This package is required because the wxMaxima configuration option
                      % "Export animations to TeX" was enabled when this file was generated.

\definecolor{labelcolor}{RGB}{100,0,0}

\usepackage{fullpage}
\usepackage{amssymb}
\usepackage{enumerate}
\usepackage[bookmarks=false,pdfstartview={FitH},colorlinks=true,urlcolor=blue]{hyperref}
\usepackage{bookmark}
\usepackage{mathtools}

\begin{document}

\pagebreak{}
{\Huge {\sc Green's Theorem}}
\setcounter{section}{0}
\setcounter{subsection}{0}
\setcounter{figure}{0}


\hypersetup{pdfauthor={Daniel Volinski},
            pdftitle={Green's Theorem},
            pdfsubject={Multivariable Calculus},
            pdfkeywords={Michael Penn}}

Written by Daniel Volinski at \href{mailto:danielvolinski@yahoo.es}{danielvolinski@yahoo.es}



\noindent
%%%%%%%%%%%%%%%
%%% INPUT:
\begin{minipage}[t]{8ex}\color{red}\bf
(\%{}i2) 
\end{minipage}
\begin{minipage}[t]{\textwidth}\color{blue}\tt
info:build\_info()\$info\ensuremath{@}version;
\end{minipage}
%%% OUTPUT:
\[\displaystyle
\tag{\%{}o2}\label{o2} 
\mbox{}
\]5.38.1



\noindent
%%%%%%%%%%%%%%%
%%% INPUT:
\begin{minipage}[t]{8ex}\color{red}\bf
(\%{}i2) 
\end{minipage}
\begin{minipage}[t]{\textwidth}\color{blue}\tt
reset()\$kill(all)\$
\end{minipage}


\noindent
%%%%%%%%%%%%%%%
%%% INPUT:
\begin{minipage}[t]{8ex}\color{red}\bf
(\%{}i1) 
\end{minipage}
\begin{minipage}[t]{\textwidth}\color{blue}\tt
derivabbrev:true\$
\end{minipage}


\noindent
%%%%%%%%%%%%%%%
%%% INPUT:
\begin{minipage}[t]{8ex}\color{red}\bf
(\%{}i2) 
\end{minipage}
\begin{minipage}[t]{\textwidth}\color{blue}\tt
ratprint:false\$
\end{minipage}


\noindent
%%%%%%%%%%%%%%%
%%% INPUT:
\begin{minipage}[t]{8ex}\color{red}\bf
(\%{}i3) 
\end{minipage}
\begin{minipage}[t]{\textwidth}\color{blue}\tt
fpprintprec:5\$
\end{minipage}


\noindent
%%%%%%%%%%%%%%%
%%% INPUT:
\begin{minipage}[t]{8ex}\color{red}\bf
(\%{}i4) 
\end{minipage}
\begin{minipage}[t]{\textwidth}\color{blue}\tt
load(linearalgebra)\$
\end{minipage}


\noindent
%%%%%%%%%%%%%%%
%%% INPUT:
\begin{minipage}[t]{8ex}\color{red}\bf
(\%{}i5) 
\end{minipage}
\begin{minipage}[t]{\textwidth}\color{blue}\tt
if get('draw,'version)=false then load(draw)\$
\end{minipage}
%%% OUTPUT:
%%%%%%%%%%%%%%%


\noindent
%%%%%%%%%%%%%%%
%%% INPUT:
\begin{minipage}[t]{8ex}\color{red}\bf
(\%{}i6) 
\end{minipage}
\begin{minipage}[t]{\textwidth}\color{blue}\tt
wxplot\_size:[1024,768]\$
\end{minipage}


\noindent
%%%%%%%%%%%%%%%
%%% INPUT:
\begin{minipage}[t]{8ex}\color{red}\bf
(\%{}i7) 
\end{minipage}
\begin{minipage}[t]{\textwidth}\color{blue}\tt
if get('drawdf,'version)=false then load(drawdf)\$
\end{minipage}


\noindent
%%%%%%%%%%%%%%%
%%% INPUT:
\begin{minipage}[t]{8ex}\color{red}\bf
(\%{}i8) 
\end{minipage}
\begin{minipage}[t]{\textwidth}\color{blue}\tt
set\_draw\_defaults(xtics=1,ytics=1,ztics=1,xyplane=0,nticks=100)\$
\end{minipage}


\noindent
%%%%%%%%%%%%%%%
%%% INPUT:
\begin{minipage}[t]{8ex}\color{red}\bf
(\%{}i9) 
\end{minipage}
\begin{minipage}[t]{\textwidth}\color{blue}\tt
if get('vect,'version)=false then load(vect)\$
\end{minipage}


\noindent
%%%%%%%%%%%%%%%
%%% INPUT:
\begin{minipage}[t]{8ex}\color{red}\bf
(\%{}i10) 
\end{minipage}
\begin{minipage}[t]{\textwidth}\color{blue}\tt
norm(u):=block(ratsimp(radcan(\ensuremath{\sqrt{}}(u.u))))\$
\end{minipage}


\noindent
%%%%%%%%%%%%%%%
%%% INPUT:
\begin{minipage}[t]{8ex}\color{red}\bf
(\%{}i11) 
\end{minipage}
\begin{minipage}[t]{\textwidth}\color{blue}\tt
normalize(v):=block(v/norm(v))\$
\end{minipage}


\noindent
%%%%%%%%%%%%%%%
%%% INPUT:
\begin{minipage}[t]{8ex}\color{red}\bf
(\%{}i12) 
\end{minipage}
\begin{minipage}[t]{\textwidth}\color{blue}\tt
angle(u,v):=block([junk:radcan(\ensuremath{\sqrt{}}((u.u)*(v.v)))],acos(u.v/junk))\$
\end{minipage}


\noindent
%%%%%%%%%%%%%%%
%%% INPUT:
\begin{minipage}[t]{8ex}\color{red}\bf
(\%{}i13) 
\end{minipage}
\begin{minipage}[t]{\textwidth}\color{blue}\tt
mycross(va,vb):=[va[2]*vb[3]-va[3]*vb[2],va[3]*vb[1]-va[1]*vb[3],va[1]*vb[2]-va[2]*vb[1]]\$
\end{minipage}


\noindent
%%%%%%%%%%%%%%%
%%% INPUT:
\begin{minipage}[t]{8ex}\color{red}\bf
(\%{}i14) 
\end{minipage}
\begin{minipage}[t]{\textwidth}\color{blue}\tt
if get('cartan,'version)=false then load(cartan)\$
\end{minipage}


\noindent
%%%%%%%%%%%%%%%
%%% INPUT:
\begin{minipage}[t]{8ex}\color{red}\bf
(\%{}i15) 
\end{minipage}
\begin{minipage}[t]{\textwidth}\color{blue}\tt
declare(trigsimp,evfun)\$
\end{minipage}
\pagebreak


\section{Green's Theorem}


Based on Michael Penn Video
\href{https://www.youtube.com/watch?v=dCgHWc0m8U8}
{Green's Theorem}


Let $C$ be a positively oriented, piecewise smooth, simple closed curve
bounding the region $D$. If $P$ and $Q$ have continuous partial derivatives
on an open region containing $D$ then,

$$\oint_C P\,\mathrm{d}x+Q\,\mathrm{d}y=
\iint_D\left({\dfrac{\partial Q}{\partial x}-
\dfrac{\partial P}{\partial y}}\right)\,\mathrm{d}A$$

\pagebreak


\section{Verification Example 1}


Based on Michael Penn Video
\href{https://www.youtube.com/watch?v=_wToO33okAI}
{Verification Example 1}


Calculate $\oint_C x\,\mathrm{d}x+y\,\mathrm{d}y$ where $C$ is the
line segment $(0,1)\rightarrow(0,0)\rightarrow(1,0)$, parabola
$y=1-x^2$ $(1,0)\rightarrow(0,1)$



\noindent
%%%%%%%%%%%%%%%
%%% INPUT:
\begin{minipage}[t]{8ex}\color{red}\bf
(\%{}i16) 
\end{minipage}
\begin{minipage}[t]{\textwidth}\color{blue}\tt
kill(labels,t,x,y,z)\$
\end{minipage}

\textbf{Define the space} $\mathbb{R}^2$



\noindent
%%%%%%%%%%%%%%%
%%% INPUT:
\begin{minipage}[t]{8ex}\color{red}\bf
(\%{}i1) 
\end{minipage}
\begin{minipage}[t]{\textwidth}\color{blue}\tt
\ensuremath{\zeta}:[x,y]\$
\end{minipage}


\noindent
%%%%%%%%%%%%%%%
%%% INPUT:
\begin{minipage}[t]{8ex}\color{red}\bf
(\%{}i2) 
\end{minipage}
\begin{minipage}[t]{\textwidth}\color{blue}\tt
scalefactors(\ensuremath{\zeta})\$
\end{minipage}


\noindent
%%%%%%%%%%%%%%%
%%% INPUT:
\begin{minipage}[t]{8ex}\color{red}\bf
(\%{}i3) 
\end{minipage}
\begin{minipage}[t]{\textwidth}\color{blue}\tt
init\_cartan(\ensuremath{\zeta})\$
\end{minipage}
\pagebreak


\textbf{Vector field} $\vec{F}\in\mathbb{R}^2$



\noindent
%%%%%%%%%%%%%%%
%%% INPUT:
\begin{minipage}[t]{8ex}\color{red}\bf
(\%{}i4) 
\end{minipage}
\begin{minipage}[t]{\textwidth}\color{blue}\tt
ldisplay(F:[x,y])\$
\end{minipage}
%%% OUTPUT:
\[\displaystyle
\tag{\%{}t4}\label{t4} 
F=[x,y]\mbox{}
\]
%%%%%%%%%%%%%%%

\textbf{2D Direction field}



\noindent
%%%%%%%%%%%%%%%
%%% INPUT:
\begin{minipage}[t]{8ex}\color{red}\bf
(\%{}i5) 
\end{minipage}
\begin{minipage}[t]{\textwidth}\color{blue}\tt
wxdrawdf(F,[x,-5,5],[y,-5,5])\$
\end{minipage}
%%% OUTPUT:
\[\displaystyle
\tag{\%{}t5}\label{t5} 
\includegraphics[width=.95\linewidth,height=.80\textheight,keepaspectratio]{Green's Theorem_img/Green's Theorem_1}\mbox{}
\]
%%%%%%%%%%%%%%%
\pagebreak


$\nabla\times\vec{F}\in\mathbb{R}^2$



\noindent
%%%%%%%%%%%%%%%
%%% INPUT:
\begin{minipage}[t]{8ex}\color{red}\bf
(\%{}i6) 
\end{minipage}
\begin{minipage}[t]{\textwidth}\color{blue}\tt
ldisplay(curlF:ev(express(curl(F)),diff))\$
\end{minipage}
%%% OUTPUT:
\[\displaystyle
\tag{\%{}t6}\label{t6} 
\mathit{curlF}=0\mbox{}
\]
%%%%%%%%%%%%%%%

\textbf{Work form} $\alpha\in\mathcal{A}^1(\mathbb{R}^2)$



\noindent
%%%%%%%%%%%%%%%
%%% INPUT:
\begin{minipage}[t]{8ex}\color{red}\bf
(\%{}i7) 
\end{minipage}
\begin{minipage}[t]{\textwidth}\color{blue}\tt
ldisplay(\ensuremath{\alpha}:F.cartan\_basis)\$
\end{minipage}
%%% OUTPUT:
\[\displaystyle
\tag{\%{}t7}\label{t7} 
\mathit{\ensuremath{\alpha}}=y\,\mathit{dy}+x\,\mathit{dx}\mbox{}
\]
%%%%%%%%%%%%%%%

$\mathrm{d}\alpha\in\mathcal{A}^2(\mathbb{R}^2)$



\noindent
%%%%%%%%%%%%%%%
%%% INPUT:
\begin{minipage}[t]{8ex}\color{red}\bf
(\%{}i8) 
\end{minipage}
\begin{minipage}[t]{\textwidth}\color{blue}\tt
ldisplay(d\ensuremath{\alpha}:edit(ext\_diff(\ensuremath{\alpha})))\$
\end{minipage}
%%% OUTPUT:
\[\displaystyle
\tag{\%{}t8}\label{t8} 
\mathit{d\ensuremath{\alpha}}=0\mbox{}
\]
%%%%%%%%%%%%%%%

$\nabla\cdot\vec{F}\in\mathbb{R}$



\noindent
%%%%%%%%%%%%%%%
%%% INPUT:
\begin{minipage}[t]{8ex}\color{red}\bf
(\%{}i9) 
\end{minipage}
\begin{minipage}[t]{\textwidth}\color{blue}\tt
ldisplay(divF:ev(express(div(F)),diff))\$
\end{minipage}
%%% OUTPUT:
\[\displaystyle
\tag{\%{}t9}\label{t9} 
\mathit{divF}=2\mbox{}
\]
%%%%%%%%%%%%%%%

\textbf{Flux form} $\beta\in\mathcal{A}^1(\mathbb{R}^2)$



\noindent
%%%%%%%%%%%%%%%
%%% INPUT:
\begin{minipage}[t]{8ex}\color{red}\bf
(\%{}i10) 
\end{minipage}
\begin{minipage}[t]{\textwidth}\color{blue}\tt
ldisplay(\ensuremath{\beta}:F[1]*cartan\_basis[2]-F[2]*cartan\_basis[1])\$
\end{minipage}
%%% OUTPUT:
\[\displaystyle
\tag{\%{}t10}\label{t10} 
\mathit{\ensuremath{\beta}}=x\,\mathit{dy}-y\,\mathit{dx}\mbox{}
\]
%%%%%%%%%%%%%%%

$\mathrm{d}\beta\in\mathcal{A}^2(\mathbb{R}^2)$



\noindent
%%%%%%%%%%%%%%%
%%% INPUT:
\begin{minipage}[t]{8ex}\color{red}\bf
(\%{}i11) 
\end{minipage}
\begin{minipage}[t]{\textwidth}\color{blue}\tt
ldisplay(d\ensuremath{\beta}:ext\_diff(\ensuremath{\beta}))\$
\end{minipage}
%%% OUTPUT:
\[\displaystyle
\tag{\%{}t11}\label{t11} 
\mathit{d\ensuremath{\beta}}=2\mathit{dx}\,\mathit{dy}\mbox{}
\]
%%%%%%%%%%%%%%%


\noindent
%%%%%%%%%%%%%%%
%%% INPUT:
\begin{minipage}[t]{8ex}\color{red}\bf
(\%{}i12) 
\end{minipage}
\begin{minipage}[t]{\textwidth}\color{blue}\tt
d\ensuremath{\beta}/apply("*",cartan\_basis);
\end{minipage}
%%% OUTPUT:
\[\displaystyle
\tag{\%{}o12}\label{o12} 
2\mbox{}
\]
%%%%%%%%%%%%%%%
\pagebreak


\textbf{End points}



\noindent
%%%%%%%%%%%%%%%
%%% INPUT:
\begin{minipage}[t]{8ex}\color{red}\bf
(\%{}i15) 
\end{minipage}
\begin{minipage}[t]{\textwidth}\color{blue}\tt
A:[0,1]\$B:[0,0]\$C:[1,0]\$
\end{minipage}

\textbf{Curve} $\vec{C}_1\in\mathbb{R}^2$



\noindent
%%%%%%%%%%%%%%%
%%% INPUT:
\begin{minipage}[t]{8ex}\color{red}\bf
(\%{}i16) 
\end{minipage}
\begin{minipage}[t]{\textwidth}\color{blue}\tt
ldisplay(C\_1:t*B+(1-t)*A)\$
\end{minipage}
%%% OUTPUT:
\[\displaystyle
\tag{\%{}t16}\label{t16} 
{{C}_{1}}=[0,1-t]\mbox{}
\]
%%%%%%%%%%%%%%%

\textbf{Derivative of the curve} $\vec{C}_1$



\noindent
%%%%%%%%%%%%%%%
%%% INPUT:
\begin{minipage}[t]{8ex}\color{red}\bf
(\%{}i17) 
\end{minipage}
\begin{minipage}[t]{\textwidth}\color{blue}\tt
ldisplay(C\ensuremath{\backslash}'\_1:diff(C\_1,t))\$
\end{minipage}
%%% OUTPUT:
\[\displaystyle
\tag{\%{}t17}\label{t17} 
{{\mathit{C'}}_{1}}=[0,-1]\mbox{}
\]
%%%%%%%%%%%%%%%

$\vec{F}\circ\vec{C_1}$



\noindent
%%%%%%%%%%%%%%%
%%% INPUT:
\begin{minipage}[t]{8ex}\color{red}\bf
(\%{}i18) 
\end{minipage}
\begin{minipage}[t]{\textwidth}\color{blue}\tt
ldisplay(FoC\_1:subst(map("=",\ensuremath{\zeta},C\_1),F))\$
\end{minipage}
%%% OUTPUT:
\[\displaystyle
\tag{\%{}t18}\label{t18} 
{{\mathit{FoC}}_{1}}=[0,1-t]\mbox{}
\]
%%%%%%%%%%%%%%%

$\vec{F}\cdot\vec{C}^{\prime}_1\in\mathbb{R}$



\noindent
%%%%%%%%%%%%%%%
%%% INPUT:
\begin{minipage}[t]{8ex}\color{red}\bf
(\%{}i19) 
\end{minipage}
\begin{minipage}[t]{\textwidth}\color{blue}\tt
ldisplay(T\_1:FoC\_1.C\ensuremath{\backslash}'\_1)\$
\end{minipage}
%%% OUTPUT:
\[\displaystyle
\tag{\%{}t19}\label{t19} 
{{T}_{1}}=t-1\mbox{}
\]
%%%%%%%%%%%%%%%

\textbf{Pullback} $\vec{C}^\ast_1\alpha\in\mathcal{A}^1(\mathbb{R}^2)$



\noindent
%%%%%%%%%%%%%%%
%%% INPUT:
\begin{minipage}[t]{8ex}\color{red}\bf
(\%{}i20) 
\end{minipage}
\begin{minipage}[t]{\textwidth}\color{blue}\tt
ldisplay(P\_1:C\ensuremath{\backslash}'\_1|subst(map("=",\ensuremath{\zeta},C\_1),\ensuremath{\alpha}))\$
\end{minipage}
%%% OUTPUT:
\[\displaystyle
\tag{\%{}t20}\label{t20} 
{{P}_{1}}=t-1\mbox{}
\]
%%%%%%%%%%%%%%%

\textbf{Line integral} $I_1$



\noindent
%%%%%%%%%%%%%%%
%%% INPUT:
\begin{minipage}[t]{8ex}\color{red}\bf
(\%{}i21) 
\end{minipage}
\begin{minipage}[t]{\textwidth}\color{blue}\tt
I\_1:'integrate(T\_1,t,0,1)\$
\end{minipage}


\noindent
%%%%%%%%%%%%%%%
%%% INPUT:
\begin{minipage}[t]{8ex}\color{red}\bf
(\%{}i22) 
\end{minipage}
\begin{minipage}[t]{\textwidth}\color{blue}\tt
ldisplay(I\_1=box(ev(I\_1,integrate)))\$
\end{minipage}
%%% OUTPUT:
\[\displaystyle
\tag{\%{}t22}\label{t22} 
\int_{0}^{1}{\left. t-1dt\right.}=\left( -\frac{1}{2}\right) \mbox{}
\]
%%%%%%%%%%%%%%%
\pagebreak


\textbf{Curve} $\vec{C}_2\in\mathbb{R}^2$



\noindent
%%%%%%%%%%%%%%%
%%% INPUT:
\begin{minipage}[t]{8ex}\color{red}\bf
(\%{}i23) 
\end{minipage}
\begin{minipage}[t]{\textwidth}\color{blue}\tt
ldisplay(C\_2:t*C+(1-t)*B)\$
\end{minipage}
%%% OUTPUT:
\[\displaystyle
\tag{\%{}t23}\label{t23} 
{{C}_{2}}=[t,0]\mbox{}
\]
%%%%%%%%%%%%%%%

\textbf{Derivative of the curve} $\vec{C}_2$



\noindent
%%%%%%%%%%%%%%%
%%% INPUT:
\begin{minipage}[t]{8ex}\color{red}\bf
(\%{}i24) 
\end{minipage}
\begin{minipage}[t]{\textwidth}\color{blue}\tt
ldisplay(C\ensuremath{\backslash}'\_2:diff(C\_2,t))\$
\end{minipage}
%%% OUTPUT:
\[\displaystyle
\tag{\%{}t24}\label{t24} 
{{\mathit{C'}}_{2}}=[1,0]\mbox{}
\]
%%%%%%%%%%%%%%%

$\vec{F}\circ\vec{C_2}$



\noindent
%%%%%%%%%%%%%%%
%%% INPUT:
\begin{minipage}[t]{8ex}\color{red}\bf
(\%{}i25) 
\end{minipage}
\begin{minipage}[t]{\textwidth}\color{blue}\tt
ldisplay(FoC\_2:subst(map("=",\ensuremath{\zeta},C\_2),F))\$
\end{minipage}
%%% OUTPUT:
\[\displaystyle
\tag{\%{}t25}\label{t25} 
{{\mathit{FoC}}_{2}}=[t,0]\mbox{}
\]
%%%%%%%%%%%%%%%

$\vec{F}\cdot\vec{C}^{\prime}_2\in\mathbb{R}$



\noindent
%%%%%%%%%%%%%%%
%%% INPUT:
\begin{minipage}[t]{8ex}\color{red}\bf
(\%{}i26) 
\end{minipage}
\begin{minipage}[t]{\textwidth}\color{blue}\tt
ldisplay(T\_2:FoC\_2.C\ensuremath{\backslash}'\_2)\$
\end{minipage}
%%% OUTPUT:
\[\displaystyle
\tag{\%{}t26}\label{t26} 
{{T}_{2}}=t\mbox{}
\]
%%%%%%%%%%%%%%%

\textbf{Pullback} $\vec{C}^\ast_2\alpha\in\mathcal{A}^1(\mathbb{R}^2)$



\noindent
%%%%%%%%%%%%%%%
%%% INPUT:
\begin{minipage}[t]{8ex}\color{red}\bf
(\%{}i27) 
\end{minipage}
\begin{minipage}[t]{\textwidth}\color{blue}\tt
ldisplay(P\_2:C\ensuremath{\backslash}'\_2|subst(map("=",\ensuremath{\zeta},C\_2),\ensuremath{\alpha}))\$
\end{minipage}
%%% OUTPUT:
\[\displaystyle
\tag{\%{}t27}\label{t27} 
{{P}_{2}}=t\mbox{}
\]
%%%%%%%%%%%%%%%

\textbf{Line integral} $I_2$



\noindent
%%%%%%%%%%%%%%%
%%% INPUT:
\begin{minipage}[t]{8ex}\color{red}\bf
(\%{}i28) 
\end{minipage}
\begin{minipage}[t]{\textwidth}\color{blue}\tt
I\_2:'integrate(T\_2,t,0,1)\$
\end{minipage}


\noindent
%%%%%%%%%%%%%%%
%%% INPUT:
\begin{minipage}[t]{8ex}\color{red}\bf
(\%{}i29) 
\end{minipage}
\begin{minipage}[t]{\textwidth}\color{blue}\tt
ldisplay(I\_2=box(ev(I\_2,integrate)))\$
\end{minipage}
%%% OUTPUT:
\[\displaystyle
\tag{\%{}t29}\label{t29} 
\int_{0}^{1}{\left. tdt\right.}=\left( \frac{1}{2}\right) \mbox{}
\]
%%%%%%%%%%%%%%%
\pagebreak


\textbf{Curve} $\vec{C}_3\in\mathbb{R}^2$



\noindent
%%%%%%%%%%%%%%%
%%% INPUT:
\begin{minipage}[t]{8ex}\color{red}\bf
(\%{}i30) 
\end{minipage}
\begin{minipage}[t]{\textwidth}\color{blue}\tt
ldisplay(C\_3:[-t,1-t\ensuremath{^2}])\$
\end{minipage}
%%% OUTPUT:
\[\displaystyle
\tag{\%{}t30}\label{t30} 
{{C}_{3}}=[-t,1-{{t}^{2}}]\mbox{}
\]
%%%%%%%%%%%%%%%

\textbf{Derivative of the curve} $\vec{C}_3$



\noindent
%%%%%%%%%%%%%%%
%%% INPUT:
\begin{minipage}[t]{8ex}\color{red}\bf
(\%{}i31) 
\end{minipage}
\begin{minipage}[t]{\textwidth}\color{blue}\tt
ldisplay(C\ensuremath{\backslash}'\_3:diff(C\_3,t))\$
\end{minipage}
%%% OUTPUT:
\[\displaystyle
\tag{\%{}t31}\label{t31} 
{{\mathit{C'}}_{3}}=[-1,-2t]\mbox{}
\]
%%%%%%%%%%%%%%%

$\vec{F}\circ\vec{C_3}$



\noindent
%%%%%%%%%%%%%%%
%%% INPUT:
\begin{minipage}[t]{8ex}\color{red}\bf
(\%{}i32) 
\end{minipage}
\begin{minipage}[t]{\textwidth}\color{blue}\tt
ldisplay(FoC\_3:subst(map("=",\ensuremath{\zeta},C\_3),F))\$
\end{minipage}
%%% OUTPUT:
\[\displaystyle
\tag{\%{}t32}\label{t32} 
{{\mathit{FoC}}_{3}}=[-t,1-{{t}^{2}}]\mbox{}
\]
%%%%%%%%%%%%%%%

$\vec{F}\cdot\vec{C}^{\prime}_3\in\mathbb{R}$



\noindent
%%%%%%%%%%%%%%%
%%% INPUT:
\begin{minipage}[t]{8ex}\color{red}\bf
(\%{}i33) 
\end{minipage}
\begin{minipage}[t]{\textwidth}\color{blue}\tt
ldisplay(T\_3:expand(FoC\_3.C\ensuremath{\backslash}'\_3))\$
\end{minipage}
%%% OUTPUT:
\[\displaystyle
\tag{\%{}t33}\label{t33} 
{{T}_{3}}=2{{t}^{3}}-t\mbox{}
\]
%%%%%%%%%%%%%%%

\textbf{Pullback} $\vec{C}^\ast_3\alpha\in\mathcal{A}^1(\mathbb{R}^2)$



\noindent
%%%%%%%%%%%%%%%
%%% INPUT:
\begin{minipage}[t]{8ex}\color{red}\bf
(\%{}i34) 
\end{minipage}
\begin{minipage}[t]{\textwidth}\color{blue}\tt
ldisplay(P\_3:C\ensuremath{\backslash}'\_3|subst(map("=",\ensuremath{\zeta},C\_3),\ensuremath{\alpha}))\$
\end{minipage}
%%% OUTPUT:
\[\displaystyle
\tag{\%{}t34}\label{t34} 
{{P}_{3}}=2{{t}^{3}}-t\mbox{}
\]
%%%%%%%%%%%%%%%

\textbf{Line integral} $I_3$



\noindent
%%%%%%%%%%%%%%%
%%% INPUT:
\begin{minipage}[t]{8ex}\color{red}\bf
(\%{}i35) 
\end{minipage}
\begin{minipage}[t]{\textwidth}\color{blue}\tt
I\_3:'integrate(T\_3,t,0,1)\$
\end{minipage}


\noindent
%%%%%%%%%%%%%%%
%%% INPUT:
\begin{minipage}[t]{8ex}\color{red}\bf
(\%{}i36) 
\end{minipage}
\begin{minipage}[t]{\textwidth}\color{blue}\tt
ldisplay(I\_3=box(ev(I\_3,integrate)))\$
\end{minipage}
%%% OUTPUT:
\[\displaystyle
\tag{\%{}t36}\label{t36} 
\int_{0}^{1}{\left. 2{{t}^{3}}-tdt\right.}=(0)\mbox{}
\]
%%%%%%%%%%%%%%%

\textbf{Total line integral} $I_1+I_2+I_3$



\noindent
%%%%%%%%%%%%%%%
%%% INPUT:
\begin{minipage}[t]{8ex}\color{red}\bf
(\%{}i37) 
\end{minipage}
\begin{minipage}[t]{\textwidth}\color{blue}\tt
ldisplay(I\_1+I\_2+I\_3=box(ev(I\_1+I\_2+I\_3,integrate)))\$
\end{minipage}
%%% OUTPUT:
\[\displaystyle
\tag{\%{}t37}\label{t37} 
\int_{0}^{1}{\left. 2{{t}^{3}}-tdt\right.}+\int_{0}^{1}{\left. tdt\right.}+\int_{0}^{1}{\left. t-1dt\right.}=(0)\mbox{}
\]
%%%%%%%%%%%%%%%
\pagebreak


\textbf{Use Green's Theorem}



\noindent
%%%%%%%%%%%%%%%
%%% INPUT:
\begin{minipage}[t]{8ex}\color{red}\bf
(\%{}i38) 
\end{minipage}
\begin{minipage}[t]{\textwidth}\color{blue}\tt
I:'integrate('integrate(curlF,y,0,1-x\ensuremath{^2}),x,0,1);
\end{minipage}
%%% OUTPUT:
\[\displaystyle
\tag{I}\label{I}
0\mbox{}
\]
%%%%%%%%%%%%%%%

\textbf{Graphics}



\noindent
%%%%%%%%%%%%%%%
%%% INPUT:
\begin{minipage}[t]{8ex}\color{red}\bf
(\%{}i39) 
\end{minipage}
\begin{minipage}[t]{\textwidth}\color{blue}\tt
wxdraw2d(proportional\_axes=xy,xrange=[-0.2,1.2],yrange=[-0.2,1.2],\\
         color=red,apply(parametric,append(C\_1,[t,0,1])),\\
         color=blue,apply(parametric,append(C\_2,[t,0,1])),\\
         color=green,apply(parametric,append(C\_3,[t,-1,0])),\\
         color=black,label(["C\_1",-0.05,0.5]),\\
         label(["C\_2",0.5,-0.05],["C\_3",0.65,0.65]),\\
         color=black,label(["(0,1)",-0.05,1]),\\
         label(["(0,0)",-0.05,-0.05],["(1,0)",1,-0.05]))\$
\end{minipage}
%%% OUTPUT:
\[\displaystyle
\tag{\%{}t39}\label{t39} 
\includegraphics[width=.95\linewidth,height=.80\textheight,keepaspectratio]{Green's Theorem_img/Green's Theorem_2}\mbox{}
\]
%%%%%%%%%%%%%%%
\pagebreak


\section{Verification Example 2}


Based on Michael Penn Video
\href{https://www.youtube.com/watch?v=Vj9uBEFZw_Y}
{Verification Example 2}


Verify Green's theorem where:
$$D=\left\lbrace{(x,y)\mid 1 \leq x^2+y^2 \leq 4}\right\rbrace\quad,
\quad\oint_C x y^2\,\mathrm{d}y-x^2 y\,\mathrm{d}x$$



\noindent
%%%%%%%%%%%%%%%
%%% INPUT:
\begin{minipage}[t]{8ex}\color{red}\bf
(\%{}i40) 
\end{minipage}
\begin{minipage}[t]{\textwidth}\color{blue}\tt
kill(labels,t,x,y,z)\$
\end{minipage}

\textbf{Define the space} $\mathbb{R}^2$



\noindent
%%%%%%%%%%%%%%%
%%% INPUT:
\begin{minipage}[t]{8ex}\color{red}\bf
(\%{}i1) 
\end{minipage}
\begin{minipage}[t]{\textwidth}\color{blue}\tt
\ensuremath{\zeta}:[x,y]\$
\end{minipage}


\noindent
%%%%%%%%%%%%%%%
%%% INPUT:
\begin{minipage}[t]{8ex}\color{red}\bf
(\%{}i2) 
\end{minipage}
\begin{minipage}[t]{\textwidth}\color{blue}\tt
scalefactors(\ensuremath{\zeta})\$
\end{minipage}


\noindent
%%%%%%%%%%%%%%%
%%% INPUT:
\begin{minipage}[t]{8ex}\color{red}\bf
(\%{}i3) 
\end{minipage}
\begin{minipage}[t]{\textwidth}\color{blue}\tt
init\_cartan(\ensuremath{\zeta})\$
\end{minipage}
\pagebreak


\textbf{Vector field} $\vec{F}\in\mathbb{R}^2$



\noindent
%%%%%%%%%%%%%%%
%%% INPUT:
\begin{minipage}[t]{8ex}\color{red}\bf
(\%{}i4) 
\end{minipage}
\begin{minipage}[t]{\textwidth}\color{blue}\tt
ldisplay(F:[-x\ensuremath{^2}*y,x*y\ensuremath{^2}])\$
\end{minipage}
%%% OUTPUT:
\[\displaystyle
\tag{\%{}t4}\label{t4} 
F=[-{{x}^{2}}y,x\,{{y}^{2}}]\mbox{}
\]
%%%%%%%%%%%%%%%

\textbf{2D Direction field}



\noindent
%%%%%%%%%%%%%%%
%%% INPUT:
\begin{minipage}[t]{8ex}\color{red}\bf
(\%{}i5) 
\end{minipage}
\begin{minipage}[t]{\textwidth}\color{blue}\tt
wxdrawdf(F,[x,-5,5],[y,-5,5])\$
\end{minipage}
%%% OUTPUT:
\[\displaystyle
\tag{\%{}t5}\label{t5} 
\includegraphics[width=.95\linewidth,height=.80\textheight,keepaspectratio]{Green's Theorem_img/Green's Theorem_3}\mbox{}
\]
%%%%%%%%%%%%%%%
\pagebreak


$\nabla\times\vec{F}\in\mathbb{R}^2$



\noindent
%%%%%%%%%%%%%%%
%%% INPUT:
\begin{minipage}[t]{8ex}\color{red}\bf
(\%{}i6) 
\end{minipage}
\begin{minipage}[t]{\textwidth}\color{blue}\tt
ldisplay(curlF:ev(express(curl(F)),diff))\$
\end{minipage}
%%% OUTPUT:
\[\displaystyle
\tag{\%{}t6}\label{t6} 
\mathit{curlF}={{y}^{2}}+{{x}^{2}}\mbox{}
\]
%%%%%%%%%%%%%%%

\textbf{Work form} $\alpha\in\mathcal{A}^1(\mathbb{R}^2)$



\noindent
%%%%%%%%%%%%%%%
%%% INPUT:
\begin{minipage}[t]{8ex}\color{red}\bf
(\%{}i7) 
\end{minipage}
\begin{minipage}[t]{\textwidth}\color{blue}\tt
ldisplay(\ensuremath{\alpha}:F.cartan\_basis)\$
\end{minipage}
%%% OUTPUT:
\[\displaystyle
\tag{\%{}t7}\label{t7} 
\mathit{\ensuremath{\alpha}}=x\,{{y}^{2}}\,\mathit{dy}-{{x}^{2}}y\,\mathit{dx}\mbox{}
\]
%%%%%%%%%%%%%%%

$\mathrm{d}\alpha\in\mathcal{A}^2(\mathbb{R}^2)$



\noindent
%%%%%%%%%%%%%%%
%%% INPUT:
\begin{minipage}[t]{8ex}\color{red}\bf
(\%{}i8) 
\end{minipage}
\begin{minipage}[t]{\textwidth}\color{blue}\tt
ldisplay(d\ensuremath{\alpha}:edit(ext\_diff(\ensuremath{\alpha})))\$
\end{minipage}
%%% OUTPUT:
\[\displaystyle
\tag{\%{}t8}\label{t8} 
\mathit{d\ensuremath{\alpha}}=\left( {{y}^{2}}+{{x}^{2}}\right) \,\mathit{dx}\,\mathit{dy}\mbox{}
\]
%%%%%%%%%%%%%%%

$\nabla\cdot\vec{F}\in\mathbb{R}$



\noindent
%%%%%%%%%%%%%%%
%%% INPUT:
\begin{minipage}[t]{8ex}\color{red}\bf
(\%{}i9) 
\end{minipage}
\begin{minipage}[t]{\textwidth}\color{blue}\tt
ldisplay(divF:ev(express(div(F)),diff))\$
\end{minipage}
%%% OUTPUT:
\[\displaystyle
\tag{\%{}t9}\label{t9} 
\mathit{divF}=0\mbox{}
\]
%%%%%%%%%%%%%%%

\textbf{Flux form} $\beta\in\mathcal{A}^1(\mathbb{R}^2)$



\noindent
%%%%%%%%%%%%%%%
%%% INPUT:
\begin{minipage}[t]{8ex}\color{red}\bf
(\%{}i10) 
\end{minipage}
\begin{minipage}[t]{\textwidth}\color{blue}\tt
ldisplay(\ensuremath{\beta}:F[1]*cartan\_basis[2]-F[2]*cartan\_basis[1])\$
\end{minipage}
%%% OUTPUT:
\[\displaystyle
\tag{\%{}t10}\label{t10} 
\mathit{\ensuremath{\beta}}=-{{x}^{2}}y\,\mathit{dy}-x\,{{y}^{2}}\,\mathit{dx}\mbox{}
\]
%%%%%%%%%%%%%%%

$\mathrm{d}\beta\in\mathcal{A}^2(\mathbb{R}^2)$



\noindent
%%%%%%%%%%%%%%%
%%% INPUT:
\begin{minipage}[t]{8ex}\color{red}\bf
(\%{}i11) 
\end{minipage}
\begin{minipage}[t]{\textwidth}\color{blue}\tt
ldisplay(d\ensuremath{\beta}:ext\_diff(\ensuremath{\beta}))\$
\end{minipage}
%%% OUTPUT:
\[\displaystyle
\tag{\%{}t11}\label{t11} 
\mathit{d\ensuremath{\beta}}=0\mbox{}
\]
%%%%%%%%%%%%%%%


\noindent
%%%%%%%%%%%%%%%
%%% INPUT:
\begin{minipage}[t]{8ex}\color{red}\bf
(\%{}i12) 
\end{minipage}
\begin{minipage}[t]{\textwidth}\color{blue}\tt
d\ensuremath{\beta}/apply("*",cartan\_basis);
\end{minipage}
%%% OUTPUT:
\[\displaystyle
\tag{\%{}o12}\label{o12} 
0\mbox{}
\]
%%%%%%%%%%%%%%%
\pagebreak


\textbf{Curve} $\vec{C}_1\in\mathbb{R}^2$



\noindent
%%%%%%%%%%%%%%%
%%% INPUT:
\begin{minipage}[t]{8ex}\color{red}\bf
(\%{}i13) 
\end{minipage}
\begin{minipage}[t]{\textwidth}\color{blue}\tt
ldisplay(C\_1:[2*cos(t),2*sin(t)])\$
\end{minipage}
%%% OUTPUT:
\[\displaystyle
\tag{\%{}t13}\label{t13} 
{{C}_{1}}=[2\cos{(t)},2\sin{(t)}]\mbox{}
\]
%%%%%%%%%%%%%%%

\textbf{Derivative of the curve} $\vec{C}_1$



\noindent
%%%%%%%%%%%%%%%
%%% INPUT:
\begin{minipage}[t]{8ex}\color{red}\bf
(\%{}i14) 
\end{minipage}
\begin{minipage}[t]{\textwidth}\color{blue}\tt
ldisplay(C\ensuremath{\backslash}'\_1:diff(C\_1,t))\$
\end{minipage}
%%% OUTPUT:
\[\displaystyle
\tag{\%{}t14}\label{t14} 
{{\mathit{C'}}_{1}}=[-2\sin{(t)},2\cos{(t)}]\mbox{}
\]
%%%%%%%%%%%%%%%

$\vec{F}\circ\vec{C_1}$



\noindent
%%%%%%%%%%%%%%%
%%% INPUT:
\begin{minipage}[t]{8ex}\color{red}\bf
(\%{}i15) 
\end{minipage}
\begin{minipage}[t]{\textwidth}\color{blue}\tt
ldisplay(FoC\_1:subst(map("=",\ensuremath{\zeta},C\_1),F))\$
\end{minipage}
%%% OUTPUT:
\[\displaystyle
\tag{\%{}t15}\label{t15} 
{{\mathit{FoC}}_{1}}=[-8{{\cos{(t)}}^{2}}\,\sin{(t)},8\cos{(t)}\,{{\sin{(t)}}^{2}}]\mbox{}
\]
%%%%%%%%%%%%%%%

$\vec{F}\cdot\vec{C}^{\prime}_1\in\mathbb{R}$



\noindent
%%%%%%%%%%%%%%%
%%% INPUT:
\begin{minipage}[t]{8ex}\color{red}\bf
(\%{}i16) 
\end{minipage}
\begin{minipage}[t]{\textwidth}\color{blue}\tt
ldisplay(T\_1:FoC\_1.C\ensuremath{\backslash}'\_1)\$
\end{minipage}
%%% OUTPUT:
\[\displaystyle
\tag{\%{}t16}\label{t16} 
{{T}_{1}}=32{{\cos{(t)}}^{2}}\,{{\sin{(t)}}^{2}}\mbox{}
\]
%%%%%%%%%%%%%%%

\textbf{Pullback} $\vec{C}^\ast_1\alpha\in\mathcal{A}^1(\mathbb{R}^2)$



\noindent
%%%%%%%%%%%%%%%
%%% INPUT:
\begin{minipage}[t]{8ex}\color{red}\bf
(\%{}i17) 
\end{minipage}
\begin{minipage}[t]{\textwidth}\color{blue}\tt
ldisplay(P\_1:C\ensuremath{\backslash}'\_1|subst(map("=",\ensuremath{\zeta},C\_1),\ensuremath{\alpha}))\$
\end{minipage}
%%% OUTPUT:
\[\displaystyle
\tag{\%{}t17}\label{t17} 
{{P}_{1}}=32{{\cos{(t)}}^{2}}\,{{\sin{(t)}}^{2}}\mbox{}
\]
%%%%%%%%%%%%%%%

\textbf{Line integral} $I_1$



\noindent
%%%%%%%%%%%%%%%
%%% INPUT:
\begin{minipage}[t]{8ex}\color{red}\bf
(\%{}i18) 
\end{minipage}
\begin{minipage}[t]{\textwidth}\color{blue}\tt
I\_1:'integrate(T\_1,t,0,2*\ensuremath{\pi})\$
\end{minipage}


\noindent
%%%%%%%%%%%%%%%
%%% INPUT:
\begin{minipage}[t]{8ex}\color{red}\bf
(\%{}i19) 
\end{minipage}
\begin{minipage}[t]{\textwidth}\color{blue}\tt
ldisplay(I\_1=box(ev(I\_1,integrate)))\$
\end{minipage}
%%% OUTPUT:
\[\displaystyle
\tag{\%{}t19}\label{t19} 
32\int_{0}^{2\ensuremath{\pi} }{\left. {{\cos{(t)}}^{2}}\,{{\sin{(t)}}^{2}}dt\right.}=\left( 8\ensuremath{\pi} \right) \mbox{}
\]
%%%%%%%%%%%%%%%
\pagebreak


\textbf{Curve} $\vec{C}_2\in\mathbb{R}^2$



\noindent
%%%%%%%%%%%%%%%
%%% INPUT:
\begin{minipage}[t]{8ex}\color{red}\bf
(\%{}i20) 
\end{minipage}
\begin{minipage}[t]{\textwidth}\color{blue}\tt
ldisplay(C\_2:[cos(t),sin(t)])\$
\end{minipage}
%%% OUTPUT:
\[\displaystyle
\tag{\%{}t20}\label{t20} 
{{C}_{2}}=[\cos{(t)},\sin{(t)}]\mbox{}
\]
%%%%%%%%%%%%%%%

\textbf{Derivative of the curve} $\vec{C}_2$



\noindent
%%%%%%%%%%%%%%%
%%% INPUT:
\begin{minipage}[t]{8ex}\color{red}\bf
(\%{}i21) 
\end{minipage}
\begin{minipage}[t]{\textwidth}\color{blue}\tt
ldisplay(C\ensuremath{\backslash}'\_2:diff(C\_2,t))\$
\end{minipage}
%%% OUTPUT:
\[\displaystyle
\tag{\%{}t21}\label{t21} 
{{\mathit{C'}}_{2}}=[-\sin{(t)},\cos{(t)}]\mbox{}
\]
%%%%%%%%%%%%%%%

$\vec{F}\circ\vec{C_2}$



\noindent
%%%%%%%%%%%%%%%
%%% INPUT:
\begin{minipage}[t]{8ex}\color{red}\bf
(\%{}i22) 
\end{minipage}
\begin{minipage}[t]{\textwidth}\color{blue}\tt
ldisplay(FoC\_2:subst(map("=",\ensuremath{\zeta},C\_2),F))\$
\end{minipage}
%%% OUTPUT:
\[\displaystyle
\tag{\%{}t22}\label{t22} 
{{\mathit{FoC}}_{2}}=[-{{\cos{(t)}}^{2}}\,\sin{(t)},\cos{(t)}\,{{\sin{(t)}}^{2}}]\mbox{}
\]
%%%%%%%%%%%%%%%

$\vec{F}\cdot\vec{C}^{\prime}_2\in\mathbb{R}$



\noindent
%%%%%%%%%%%%%%%
%%% INPUT:
\begin{minipage}[t]{8ex}\color{red}\bf
(\%{}i23) 
\end{minipage}
\begin{minipage}[t]{\textwidth}\color{blue}\tt
ldisplay(T\_2:FoC\_2.C\ensuremath{\backslash}'\_2)\$
\end{minipage}
%%% OUTPUT:
\[\displaystyle
\tag{\%{}t23}\label{t23} 
{{T}_{2}}=2{{\cos{(t)}}^{2}}\,{{\sin{(t)}}^{2}}\mbox{}
\]
%%%%%%%%%%%%%%%

\textbf{Pullback} $\vec{C}^\ast_2\alpha\in\mathcal{A}^1(\mathbb{R}^2)$



\noindent
%%%%%%%%%%%%%%%
%%% INPUT:
\begin{minipage}[t]{8ex}\color{red}\bf
(\%{}i24) 
\end{minipage}
\begin{minipage}[t]{\textwidth}\color{blue}\tt
ldisplay(P\_2:C\ensuremath{\backslash}'\_2|subst(map("=",\ensuremath{\zeta},C\_2),\ensuremath{\alpha}))\$
\end{minipage}
%%% OUTPUT:
\[\displaystyle
\tag{\%{}t24}\label{t24} 
{{P}_{2}}=2{{\cos{(t)}}^{2}}\,{{\sin{(t)}}^{2}}\mbox{}
\]
%%%%%%%%%%%%%%%

\textbf{Line integral} $I_2$



\noindent
%%%%%%%%%%%%%%%
%%% INPUT:
\begin{minipage}[t]{8ex}\color{red}\bf
(\%{}i25) 
\end{minipage}
\begin{minipage}[t]{\textwidth}\color{blue}\tt
I\_2:'integrate(T\_2,t,0,2*\ensuremath{\pi})\$
\end{minipage}


\noindent
%%%%%%%%%%%%%%%
%%% INPUT:
\begin{minipage}[t]{8ex}\color{red}\bf
(\%{}i26) 
\end{minipage}
\begin{minipage}[t]{\textwidth}\color{blue}\tt
ldisplay(I\_2=box(ev(I\_2,integrate)))\$
\end{minipage}
%%% OUTPUT:
\[\displaystyle
\tag{\%{}t26}\label{t26} 
2\int_{0}^{2\ensuremath{\pi} }{\left. {{\cos{(t)}}^{2}}\,{{\sin{(t)}}^{2}}dt\right.}=\left( \frac{\ensuremath{\pi} }{2}\right) \mbox{}
\]
%%%%%%%%%%%%%%%

\textbf{Total line integral} $I_2-I_1$



\noindent
%%%%%%%%%%%%%%%
%%% INPUT:
\begin{minipage}[t]{8ex}\color{red}\bf
(\%{}i27) 
\end{minipage}
\begin{minipage}[t]{\textwidth}\color{blue}\tt
ldisplay(I\_1-I\_2=box(ev(I\_1-I\_2,integrate)))\$
\end{minipage}
%%% OUTPUT:
\[\displaystyle
\tag{\%{}t27}\label{t27} 
30\int_{0}^{2\ensuremath{\pi} }{\left. {{\cos{(t)}}^{2}}\,{{\sin{(t)}}^{2}}dt\right.}=\left( \frac{15\ensuremath{\pi} }{2}\right) \mbox{}
\]
%%%%%%%%%%%%%%%
\pagebreak


\textbf{Graphics}



\noindent
%%%%%%%%%%%%%%%
%%% INPUT:
\begin{minipage}[t]{8ex}\color{red}\bf
(\%{}i28) 
\end{minipage}
\begin{minipage}[t]{\textwidth}\color{blue}\tt
wxdraw2d(proportional\_axes=xy,xrange=[-2.5,2.5],yrange=[-2.5,2.5],\\
         color=red,apply(parametric,append(C\_1,[t,0,2*\ensuremath{\pi}])),\\
         color=blue,apply(parametric,append(C\_2,[t,0,2*\ensuremath{\pi}])),\\
         color=black,label(["C\_1",1.5,1.5],["C\_2",0.8,0.8]))\$
\end{minipage}
%%% OUTPUT:
\[\displaystyle
\tag{\%{}t28}\label{t28} 
\includegraphics[width=.95\linewidth,height=.80\textheight,keepaspectratio]{Green's Theorem_img/Green's Theorem_4}\mbox{}
\]
%%%%%%%%%%%%%%%
\pagebreak


\textbf{Use Green's Theorem}


\textbf{Change to polar coordinates}



\noindent
%%%%%%%%%%%%%%%
%%% INPUT:
\begin{minipage}[t]{8ex}\color{red}\bf
(\%{}i30) 
\end{minipage}
\begin{minipage}[t]{\textwidth}\color{blue}\tt
assume(r\ensuremath{\geq}0)\$\\
assume(0\ensuremath{\leq}t,t\ensuremath{\leq}2*\ensuremath{\pi})\$
\end{minipage}


\noindent
%%%%%%%%%%%%%%%
%%% INPUT:
\begin{minipage}[t]{8ex}\color{red}\bf
(\%{}i31) 
\end{minipage}
\begin{minipage}[t]{\textwidth}\color{blue}\tt
\ensuremath{\xi}:[r,t]\$
\end{minipage}


\noindent
%%%%%%%%%%%%%%%
%%% INPUT:
\begin{minipage}[t]{8ex}\color{red}\bf
(\%{}i32) 
\end{minipage}
\begin{minipage}[t]{\textwidth}\color{blue}\tt
ldisplay(Tr:[r*cos(t),r*sin(t)])\$
\end{minipage}
%%% OUTPUT:
\[\displaystyle
\tag{\%{}t32}\label{t32} 
\mathit{Tr}=[r\,\cos{(t)},r\,\sin{(t)}]\mbox{}
\]
%%%%%%%%%%%%%%%


\noindent
%%%%%%%%%%%%%%%
%%% INPUT:
\begin{minipage}[t]{8ex}\color{red}\bf
(\%{}i33) 
\end{minipage}
\begin{minipage}[t]{\textwidth}\color{blue}\tt
ldisplay(J:jacobian(Tr,\ensuremath{\xi}))\$
\end{minipage}
%%% OUTPUT:
\[\displaystyle
\tag{\%{}t33}\label{t33} 
J=\begin{pmatrix}\cos{(t)} & -r\,\sin{(t)}\\
\sin{(t)} & r\,\cos{(t)}\end{pmatrix}\mbox{}
\]
%%%%%%%%%%%%%%%


\noindent
%%%%%%%%%%%%%%%
%%% INPUT:
\begin{minipage}[t]{8ex}\color{red}\bf
(\%{}i34) 
\end{minipage}
\begin{minipage}[t]{\textwidth}\color{blue}\tt
ldisplay(lg:trigsimp(transpose(J).J))\$
\end{minipage}
%%% OUTPUT:
\[\displaystyle
\tag{\%{}t34}\label{t34} 
\mathit{lg}=\begin{pmatrix}1 & 0\\
0 & {{r}^{2}}\end{pmatrix}\mbox{}
\]
%%%%%%%%%%%%%%%


\noindent
%%%%%%%%%%%%%%%
%%% INPUT:
\begin{minipage}[t]{8ex}\color{red}\bf
(\%{}i35) 
\end{minipage}
\begin{minipage}[t]{\textwidth}\color{blue}\tt
ldisplay(Jdet:trigsimp(determinant(J)))\$
\end{minipage}
%%% OUTPUT:
\[\displaystyle
\tag{\%{}t35}\label{t35} 
\mathit{Jdet}=r\mbox{}
\]
%%%%%%%%%%%%%%%

$\left({\nabla\times\vec{F}}\right)\circ\vec{Tr}\in\mathbb{R}$



\noindent
%%%%%%%%%%%%%%%
%%% INPUT:
\begin{minipage}[t]{8ex}\color{red}\bf
(\%{}i36) 
\end{minipage}
\begin{minipage}[t]{\textwidth}\color{blue}\tt
ldisplay(T:trigsimp(subst(map("=",\ensuremath{\zeta},Tr),curlF)))\$
\end{minipage}
%%% OUTPUT:
\[\displaystyle
\tag{\%{}t36}\label{t36} 
T={{r}^{2}}\mbox{}
\]
%%%%%%%%%%%%%%%

\textbf{Pullback} $\vec{Tr}^\ast\,\mathrm{d}\alpha\in\mathcal{A}^2(\mathbb{R}^2)$



\noindent
%%%%%%%%%%%%%%%
%%% INPUT:
\begin{minipage}[t]{8ex}\color{red}\bf
(\%{}i37) 
\end{minipage}
\begin{minipage}[t]{\textwidth}\color{blue}\tt
ldisplay(P:trigsimp(diff(Tr,t)|(diff(Tr,r)|subst(map("=",\ensuremath{\zeta},Tr),d\ensuremath{\alpha}))))\$
\end{minipage}
%%% OUTPUT:
\[\displaystyle
\tag{\%{}t37}\label{t37} 
P={{r}^{3}}\mbox{}
\]
%%%%%%%%%%%%%%%

\textbf{Surface integral}



\noindent
%%%%%%%%%%%%%%%
%%% INPUT:
\begin{minipage}[t]{8ex}\color{red}\bf
(\%{}i38) 
\end{minipage}
\begin{minipage}[t]{\textwidth}\color{blue}\tt
I:'integrate('integrate(T*Jdet,r,1,2),t,0,2*\ensuremath{\pi})\$
\end{minipage}


\noindent
%%%%%%%%%%%%%%%
%%% INPUT:
\begin{minipage}[t]{8ex}\color{red}\bf
(\%{}i39) 
\end{minipage}
\begin{minipage}[t]{\textwidth}\color{blue}\tt
ldisplay(I=box(ev(I,integrate)))\$
\end{minipage}
%%% OUTPUT:
\[\displaystyle
\tag{\%{}t39}\label{t39} 
2\ensuremath{\pi} \int_{1}^{2}{\left. {{r}^{3}}dr\right.}=\left( \frac{15\ensuremath{\pi} }{2}\right) \mbox{}
\]
%%%%%%%%%%%%%%%

\textbf{Clean up}



\noindent
%%%%%%%%%%%%%%%
%%% INPUT:
\begin{minipage}[t]{8ex}\color{red}\bf
(\%{}i41) 
\end{minipage}
\begin{minipage}[t]{\textwidth}\color{blue}\tt
forget(r\ensuremath{\geq}0)\$\\
forget(0\ensuremath{\leq}t,t\ensuremath{\leq}2*\ensuremath{\pi})\$
\end{minipage}
\pagebreak


\section{Two more Green's theorem examples}


Based on Michael Penn Video
\href{https://www.youtube.com/watch?v=rMisi24o78I}
{Two more Green's theorem examples}


$C$ is a right angle triangle with vertices $(-1,2),(4,2),(4,5)$\\
Calculate $$\oint_C\sin(x^2)\,\mathrm{d}x+(3 x-y)\,\mathrm{d}y$$



\noindent
%%%%%%%%%%%%%%%
%%% INPUT:
\begin{minipage}[t]{8ex}\color{red}\bf
(\%{}i42) 
\end{minipage}
\begin{minipage}[t]{\textwidth}\color{blue}\tt
kill(labels,t,x,y,z)\$
\end{minipage}

\textbf{Define the space} $\mathbb{R}^2$



\noindent
%%%%%%%%%%%%%%%
%%% INPUT:
\begin{minipage}[t]{8ex}\color{red}\bf
(\%{}i1) 
\end{minipage}
\begin{minipage}[t]{\textwidth}\color{blue}\tt
\ensuremath{\zeta}:[x,y]\$
\end{minipage}


\noindent
%%%%%%%%%%%%%%%
%%% INPUT:
\begin{minipage}[t]{8ex}\color{red}\bf
(\%{}i2) 
\end{minipage}
\begin{minipage}[t]{\textwidth}\color{blue}\tt
scalefactors(\ensuremath{\zeta})\$
\end{minipage}


\noindent
%%%%%%%%%%%%%%%
%%% INPUT:
\begin{minipage}[t]{8ex}\color{red}\bf
(\%{}i3) 
\end{minipage}
\begin{minipage}[t]{\textwidth}\color{blue}\tt
init\_cartan(\ensuremath{\zeta})\$
\end{minipage}
\pagebreak


\textbf{Vector field} $\vec{F}\in\mathbb{R}^2$



\noindent
%%%%%%%%%%%%%%%
%%% INPUT:
\begin{minipage}[t]{8ex}\color{red}\bf
(\%{}i4) 
\end{minipage}
\begin{minipage}[t]{\textwidth}\color{blue}\tt
ldisplay(F:[sin(x\ensuremath{^2}),3*x-y])\$
\end{minipage}
%%% OUTPUT:
\[\displaystyle
\tag{\%{}t4}\label{t4} 
F=[\sin{\left( {{x}^{2}}\right) },3x-y]\mbox{}
\]
%%%%%%%%%%%%%%%

\textbf{2D Direction field}



\noindent
%%%%%%%%%%%%%%%
%%% INPUT:
\begin{minipage}[t]{8ex}\color{red}\bf
(\%{}i5) 
\end{minipage}
\begin{minipage}[t]{\textwidth}\color{blue}\tt
wxdrawdf(F,[x,-5,5],[y,-5,5])\$
\end{minipage}
%%% OUTPUT:
\[\displaystyle
\tag{\%{}t5}\label{t5} 
\includegraphics[width=.95\linewidth,height=.80\textheight,keepaspectratio]{Green's Theorem_img/Green's Theorem_5}\mbox{}
\]
%%%%%%%%%%%%%%%
\pagebreak


$\nabla\times\vec{F}\in\mathbb{R}^2$



\noindent
%%%%%%%%%%%%%%%
%%% INPUT:
\begin{minipage}[t]{8ex}\color{red}\bf
(\%{}i6) 
\end{minipage}
\begin{minipage}[t]{\textwidth}\color{blue}\tt
ldisplay(curlF:ev(express(curl(F)),diff))\$
\end{minipage}
%%% OUTPUT:
\[\displaystyle
\tag{\%{}t6}\label{t6} 
\mathit{curlF}=3\mbox{}
\]
%%%%%%%%%%%%%%%

\textbf{Work form} $\alpha\in\mathcal{A}^1(\mathbb{R}^2)$



\noindent
%%%%%%%%%%%%%%%
%%% INPUT:
\begin{minipage}[t]{8ex}\color{red}\bf
(\%{}i7) 
\end{minipage}
\begin{minipage}[t]{\textwidth}\color{blue}\tt
ldisplay(\ensuremath{\alpha}:F.cartan\_basis)\$
\end{minipage}
%%% OUTPUT:
\[\displaystyle
\tag{\%{}t7}\label{t7} 
\mathit{\ensuremath{\alpha}}=\left( 3x-y\right) \,\mathit{dy}+\sin{\left( {{x}^{2}}\right) }\,\mathit{dx}\mbox{}
\]
%%%%%%%%%%%%%%%

$\mathrm{d}\alpha\in\mathcal{A}^2(\mathbb{R}^2)$



\noindent
%%%%%%%%%%%%%%%
%%% INPUT:
\begin{minipage}[t]{8ex}\color{red}\bf
(\%{}i8) 
\end{minipage}
\begin{minipage}[t]{\textwidth}\color{blue}\tt
ldisplay(d\ensuremath{\alpha}:edit(ext\_diff(\ensuremath{\alpha})))\$
\end{minipage}
%%% OUTPUT:
\[\displaystyle
\tag{\%{}t8}\label{t8} 
\mathit{d\ensuremath{\alpha}}=3\mathit{dx}\,\mathit{dy}\mbox{}
\]
%%%%%%%%%%%%%%%


\noindent
%%%%%%%%%%%%%%%
%%% INPUT:
\begin{minipage}[t]{8ex}\color{red}\bf
(\%{}i9) 
\end{minipage}
\begin{minipage}[t]{\textwidth}\color{blue}\tt
d\ensuremath{\alpha}/apply("*",cartan\_basis);
\end{minipage}
%%% OUTPUT:
\[\displaystyle
\tag{\%{}o9}\label{o9} 
3\mbox{}
\]
%%%%%%%%%%%%%%%

$\nabla\cdot\vec{F}\in\mathbb{R}$



\noindent
%%%%%%%%%%%%%%%
%%% INPUT:
\begin{minipage}[t]{8ex}\color{red}\bf
(\%{}i10) 
\end{minipage}
\begin{minipage}[t]{\textwidth}\color{blue}\tt
ldisplay(divF:ev(express(div(F)),diff))\$
\end{minipage}
%%% OUTPUT:
\[\displaystyle
\tag{\%{}t10}\label{t10} 
\mathit{divF}=2x\,\cos{\left( {{x}^{2}}\right) }-1\mbox{}
\]
%%%%%%%%%%%%%%%

\textbf{Flux form} $\beta\in\mathcal{A}^1(\mathbb{R}^2)$



\noindent
%%%%%%%%%%%%%%%
%%% INPUT:
\begin{minipage}[t]{8ex}\color{red}\bf
(\%{}i11) 
\end{minipage}
\begin{minipage}[t]{\textwidth}\color{blue}\tt
ldisplay(\ensuremath{\beta}:F[1]*cartan\_basis[2]-F[2]*cartan\_basis[1])\$
\end{minipage}
%%% OUTPUT:
\[\displaystyle
\tag{\%{}t11}\label{t11} 
\mathit{\ensuremath{\beta}}=\sin{\left( {{x}^{2}}\right) }\,\mathit{dy}-\left( 3x-y\right) \,\mathit{dx}\mbox{}
\]
%%%%%%%%%%%%%%%

$\mathrm{d}\beta\in\mathcal{A}^2(\mathbb{R}^2)$



\noindent
%%%%%%%%%%%%%%%
%%% INPUT:
\begin{minipage}[t]{8ex}\color{red}\bf
(\%{}i12) 
\end{minipage}
\begin{minipage}[t]{\textwidth}\color{blue}\tt
ldisplay(d\ensuremath{\beta}:edit(ext\_diff(\ensuremath{\beta})))\$
\end{minipage}
%%% OUTPUT:
\[\displaystyle
\tag{\%{}t12}\label{t12} 
\mathit{d\ensuremath{\beta}}=\left( 2x\,\cos{\left( {{x}^{2}}\right) }-1\right) \,\mathit{dx}\,\mathit{dy}\mbox{}
\]
%%%%%%%%%%%%%%%


\noindent
%%%%%%%%%%%%%%%
%%% INPUT:
\begin{minipage}[t]{8ex}\color{red}\bf
(\%{}i13) 
\end{minipage}
\begin{minipage}[t]{\textwidth}\color{blue}\tt
d\ensuremath{\beta}/apply("*",cartan\_basis);
\end{minipage}
%%% OUTPUT:
\[\displaystyle
\tag{\%{}o13}\label{o13} 
2x\,\cos{\left( {{x}^{2}}\right) }-1\mbox{}
\]
%%%%%%%%%%%%%%%
\pagebreak


\textbf{End points}



\noindent
%%%%%%%%%%%%%%%
%%% INPUT:
\begin{minipage}[t]{8ex}\color{red}\bf
(\%{}i16) 
\end{minipage}
\begin{minipage}[t]{\textwidth}\color{blue}\tt
A:[-1,2]\$B:[4,2]\$C:[4,5]\$
\end{minipage}

\textbf{Curve} $\vec{C}_1\in\mathbb{R}^2$



\noindent
%%%%%%%%%%%%%%%
%%% INPUT:
\begin{minipage}[t]{8ex}\color{red}\bf
(\%{}i17) 
\end{minipage}
\begin{minipage}[t]{\textwidth}\color{blue}\tt
ldisplay(C\_1:ratsimp(t*B+(1-t)*A))\$
\end{minipage}
%%% OUTPUT:
\[\displaystyle
\tag{\%{}t17}\label{t17} 
{{C}_{1}}=[5t-1,2]\mbox{}
\]
%%%%%%%%%%%%%%%

\textbf{Derivative of the curve} $\vec{C}_1$



\noindent
%%%%%%%%%%%%%%%
%%% INPUT:
\begin{minipage}[t]{8ex}\color{red}\bf
(\%{}i18) 
\end{minipage}
\begin{minipage}[t]{\textwidth}\color{blue}\tt
ldisplay(C\ensuremath{\backslash}'\_1:diff(C\_1,t))\$
\end{minipage}
%%% OUTPUT:
\[\displaystyle
\tag{\%{}t18}\label{t18} 
{{\mathit{C'}}_{1}}=[5,0]\mbox{}
\]
%%%%%%%%%%%%%%%

$\vec{F}\circ\vec{C_1}$



\noindent
%%%%%%%%%%%%%%%
%%% INPUT:
\begin{minipage}[t]{8ex}\color{red}\bf
(\%{}i19) 
\end{minipage}
\begin{minipage}[t]{\textwidth}\color{blue}\tt
ldisplay(FoC\_1:subst(map("=",\ensuremath{\zeta},C\_1),F))\$
\end{minipage}
%%% OUTPUT:
\[\displaystyle
\tag{\%{}t19}\label{t19} 
{{\mathit{FoC}}_{1}}=[\sin{\left( {{\left( 5t-1\right) }^{2}}\right) },3\left( 5t-1\right) -2]\mbox{}
\]
%%%%%%%%%%%%%%%

$\vec{F}\cdot\vec{C}^{\prime}_1\in\mathbb{R}$



\noindent
%%%%%%%%%%%%%%%
%%% INPUT:
\begin{minipage}[t]{8ex}\color{red}\bf
(\%{}i20) 
\end{minipage}
\begin{minipage}[t]{\textwidth}\color{blue}\tt
ldisplay(T\_1:expand(FoC\_1.C\ensuremath{\backslash}'\_1))\$
\end{minipage}
%%% OUTPUT:
\[\displaystyle
\tag{\%{}t20}\label{t20} 
{{T}_{1}}=5\sin{\left( 25{{t}^{2}}-10t+1\right) }\mbox{}
\]
%%%%%%%%%%%%%%%

\textbf{Pullback} $\vec{C}^\ast_1\alpha\in\mathcal{A}^1(\mathbb{R}^2)$



\noindent
%%%%%%%%%%%%%%%
%%% INPUT:
\begin{minipage}[t]{8ex}\color{red}\bf
(\%{}i21) 
\end{minipage}
\begin{minipage}[t]{\textwidth}\color{blue}\tt
ldisplay(P\_1:C\ensuremath{\backslash}'\_1|subst(map("=",\ensuremath{\zeta},C\_1),\ensuremath{\alpha}))\$
\end{minipage}
%%% OUTPUT:
\[\displaystyle
\tag{\%{}t21}\label{t21} 
{{P}_{1}}=5\sin{\left( 25{{t}^{2}}-10t+1\right) }\mbox{}
\]
%%%%%%%%%%%%%%%

\textbf{Line integral} $I_1$



\noindent
%%%%%%%%%%%%%%%
%%% INPUT:
\begin{minipage}[t]{8ex}\color{red}\bf
(\%{}i22) 
\end{minipage}
\begin{minipage}[t]{\textwidth}\color{blue}\tt
I\_1:'integrate(T\_1,t,0,1)\$
\end{minipage}
\pagebreak


\textbf{Curve} $\vec{C}_2\in\mathbb{R}^2$



\noindent
%%%%%%%%%%%%%%%
%%% INPUT:
\begin{minipage}[t]{8ex}\color{red}\bf
(\%{}i23) 
\end{minipage}
\begin{minipage}[t]{\textwidth}\color{blue}\tt
ldisplay(C\_2:ratsimp(t*C+(1-t)*B))\$
\end{minipage}
%%% OUTPUT:
\[\displaystyle
\tag{\%{}t23}\label{t23} 
{{C}_{2}}=[4,3t+2]\mbox{}
\]
%%%%%%%%%%%%%%%

\textbf{Derivative of the curve} $\vec{C}_2$



\noindent
%%%%%%%%%%%%%%%
%%% INPUT:
\begin{minipage}[t]{8ex}\color{red}\bf
(\%{}i24) 
\end{minipage}
\begin{minipage}[t]{\textwidth}\color{blue}\tt
ldisplay(C\ensuremath{\backslash}'\_2:diff(C\_2,t))\$
\end{minipage}
%%% OUTPUT:
\[\displaystyle
\tag{\%{}t24}\label{t24} 
{{\mathit{C'}}_{2}}=[0,3]\mbox{}
\]
%%%%%%%%%%%%%%%

$\vec{F}\circ\vec{C_2}$



\noindent
%%%%%%%%%%%%%%%
%%% INPUT:
\begin{minipage}[t]{8ex}\color{red}\bf
(\%{}i25) 
\end{minipage}
\begin{minipage}[t]{\textwidth}\color{blue}\tt
ldisplay(FoC\_2:subst(map("=",\ensuremath{\zeta},C\_2),F))\$
\end{minipage}
%%% OUTPUT:
\[\displaystyle
\tag{\%{}t25}\label{t25} 
{{\mathit{FoC}}_{2}}=[\sin{(16)},10-3t]\mbox{}
\]
%%%%%%%%%%%%%%%

$\vec{F}\cdot\vec{C}^{\prime}_2\in\mathbb{R}$



\noindent
%%%%%%%%%%%%%%%
%%% INPUT:
\begin{minipage}[t]{8ex}\color{red}\bf
(\%{}i26) 
\end{minipage}
\begin{minipage}[t]{\textwidth}\color{blue}\tt
ldisplay(T\_2:expand(FoC\_2.C\ensuremath{\backslash}'\_2))\$
\end{minipage}
%%% OUTPUT:
\[\displaystyle
\tag{\%{}t26}\label{t26} 
{{T}_{2}}=30-9t\mbox{}
\]
%%%%%%%%%%%%%%%

\textbf{Pullback} $\vec{C}^\ast_2\alpha\in\mathcal{A}^1(\mathbb{R}^2)$



\noindent
%%%%%%%%%%%%%%%
%%% INPUT:
\begin{minipage}[t]{8ex}\color{red}\bf
(\%{}i27) 
\end{minipage}
\begin{minipage}[t]{\textwidth}\color{blue}\tt
ldisplay(P\_2:C\ensuremath{\backslash}'\_2|subst(map("=",\ensuremath{\zeta},C\_2),\ensuremath{\alpha}))\$
\end{minipage}
%%% OUTPUT:
\[\displaystyle
\tag{\%{}t27}\label{t27} 
{{P}_{2}}=30-9t\mbox{}
\]
%%%%%%%%%%%%%%%

\textbf{Line integral} $I_2$



\noindent
%%%%%%%%%%%%%%%
%%% INPUT:
\begin{minipage}[t]{8ex}\color{red}\bf
(\%{}i28) 
\end{minipage}
\begin{minipage}[t]{\textwidth}\color{blue}\tt
I\_2:'integrate(T\_2,t,0,1)\$
\end{minipage}


\noindent
%%%%%%%%%%%%%%%
%%% INPUT:
\begin{minipage}[t]{8ex}\color{red}\bf
(\%{}i29) 
\end{minipage}
\begin{minipage}[t]{\textwidth}\color{blue}\tt
ldisplay(I\_2=box(ev(I\_2,integrate)))\$
\end{minipage}
%%% OUTPUT:
\[\displaystyle
\tag{\%{}t29}\label{t29} 
\int_{0}^{1}{\left. 30-9tdt\right.}=\left( \frac{51}{2}\right) \mbox{}
\]
%%%%%%%%%%%%%%%
\pagebreak


\textbf{Curve} $\vec{C}_3\in\mathbb{R}^2$



\noindent
%%%%%%%%%%%%%%%
%%% INPUT:
\begin{minipage}[t]{8ex}\color{red}\bf
(\%{}i30) 
\end{minipage}
\begin{minipage}[t]{\textwidth}\color{blue}\tt
ldisplay(C\_3:ratsimp(t*A+(1-t)*C))\$
\end{minipage}
%%% OUTPUT:
\[\displaystyle
\tag{\%{}t30}\label{t30} 
{{C}_{3}}=[4-5t,5-3t]\mbox{}
\]
%%%%%%%%%%%%%%%

\textbf{Derivative of the curve} $\vec{C}_3$



\noindent
%%%%%%%%%%%%%%%
%%% INPUT:
\begin{minipage}[t]{8ex}\color{red}\bf
(\%{}i31) 
\end{minipage}
\begin{minipage}[t]{\textwidth}\color{blue}\tt
ldisplay(C\ensuremath{\backslash}'\_3:diff(C\_3,t))\$
\end{minipage}
%%% OUTPUT:
\[\displaystyle
\tag{\%{}t31}\label{t31} 
{{\mathit{C'}}_{3}}=[-5,-3]\mbox{}
\]
%%%%%%%%%%%%%%%

$\vec{F}\circ\vec{C_3}$



\noindent
%%%%%%%%%%%%%%%
%%% INPUT:
\begin{minipage}[t]{8ex}\color{red}\bf
(\%{}i32) 
\end{minipage}
\begin{minipage}[t]{\textwidth}\color{blue}\tt
ldisplay(FoC\_3:ratsimp(subst(map("=",\ensuremath{\zeta},C\_3),F)))\$
\end{minipage}
%%% OUTPUT:
\[\displaystyle
\tag{\%{}t32}\label{t32} 
{{\mathit{FoC}}_{3}}=[\sin{\left( 25{{t}^{2}}-40t+16\right) },7-12t]\mbox{}
\]
%%%%%%%%%%%%%%%

$\vec{F}\cdot\vec{C}^{\prime}_3\in\mathbb{R}$



\noindent
%%%%%%%%%%%%%%%
%%% INPUT:
\begin{minipage}[t]{8ex}\color{red}\bf
(\%{}i33) 
\end{minipage}
\begin{minipage}[t]{\textwidth}\color{blue}\tt
ldisplay(T\_3:expand(FoC\_3.C\ensuremath{\backslash}'\_3))\$
\end{minipage}
%%% OUTPUT:
\[\displaystyle
\tag{\%{}t33}\label{t33} 
{{T}_{3}}=-5\sin{\left( 25{{t}^{2}}-40t+16\right) }+36t-21\mbox{}
\]
%%%%%%%%%%%%%%%

\textbf{Pullback} $\vec{C}^\ast_3\alpha\in\mathcal{A}^1(\mathbb{R}^2)$



\noindent
%%%%%%%%%%%%%%%
%%% INPUT:
\begin{minipage}[t]{8ex}\color{red}\bf
(\%{}i34) 
\end{minipage}
\begin{minipage}[t]{\textwidth}\color{blue}\tt
ldisplay(P\_3:C\ensuremath{\backslash}'\_3|subst(map("=",\ensuremath{\zeta},C\_3),\ensuremath{\alpha}))\$
\end{minipage}
%%% OUTPUT:
\[\displaystyle
\tag{\%{}t34}\label{t34} 
{{P}_{3}}=-5\sin{\left( 25{{t}^{2}}-40t+16\right) }+36t-21\mbox{}
\]
%%%%%%%%%%%%%%%

\textbf{Line integral} $I_3$



\noindent
%%%%%%%%%%%%%%%
%%% INPUT:
\begin{minipage}[t]{8ex}\color{red}\bf
(\%{}i35) 
\end{minipage}
\begin{minipage}[t]{\textwidth}\color{blue}\tt
I\_3:'integrate(T\_3,t,0,1)\$
\end{minipage}

\textbf{Total line integral} $I_1+I_2+I_3$



\noindent
%%%%%%%%%%%%%%%
%%% INPUT:
\begin{minipage}[t]{8ex}\color{red}\bf
(\%{}i36) 
\end{minipage}
\begin{minipage}[t]{\textwidth}\color{blue}\tt
ldisplay(I\_1+I\_2+I\_3=box(ev(I\_1+I\_2+I\_3,integrate,ratsimp)))\$
\end{minipage}
%%% OUTPUT:
\[\displaystyle
\tag{\%{}t36}\label{t36} 
5\int_{0}^{1}{\left. \sin{\left( 25{{t}^{2}}-10t+1\right) }dt\right.}+\int_{0}^{1}{\left. -5\sin{\left( 25{{t}^{2}}-40t+16\right) }+36t-21dt\right.}+\int_{0}^{1}{\left. 30-9tdt\right.}=\left( \frac{45}{2}\right) \mbox{}
\]
%%%%%%%%%%%%%%%
\pagebreak


\textbf{Use Green's Theorem}



\noindent
%%%%%%%%%%%%%%%
%%% INPUT:
\begin{minipage}[t]{8ex}\color{red}\bf
(\%{}i37) 
\end{minipage}
\begin{minipage}[t]{\textwidth}\color{blue}\tt
rhs(first(solve(eliminate(map("=",\ensuremath{\zeta},C\_3),[t]),y)));
\end{minipage}
%%% OUTPUT:
\[\displaystyle
\tag{\%{}o37}\label{o37} 
\frac{3x+13}{5}\mbox{}
\]
%%%%%%%%%%%%%%%


\noindent
%%%%%%%%%%%%%%%
%%% INPUT:
\begin{minipage}[t]{8ex}\color{red}\bf
(\%{}i38) 
\end{minipage}
\begin{minipage}[t]{\textwidth}\color{blue}\tt
I:'integrate('integrate(curlF,y,2,\%),x,-1,4)\$
\end{minipage}


\noindent
%%%%%%%%%%%%%%%
%%% INPUT:
\begin{minipage}[t]{8ex}\color{red}\bf
(\%{}i39) 
\end{minipage}
\begin{minipage}[t]{\textwidth}\color{blue}\tt
ldisplay(I=box(ev(I,integrate)))\$
\end{minipage}
%%% OUTPUT:
\[\displaystyle
\tag{\%{}t39}\label{t39} 
3\int_{-1}^{4}{\left. \frac{3x+13}{5}-2dx\right.}=\left( \frac{45}{2}\right) \mbox{}
\]
%%%%%%%%%%%%%%%

\textbf{Graphics}



\noindent
%%%%%%%%%%%%%%%
%%% INPUT:
\begin{minipage}[t]{8ex}\color{red}\bf
(\%{}i40) 
\end{minipage}
\begin{minipage}[t]{\textwidth}\color{blue}\tt
wxdraw2d(proportional\_axes=xy,xrange=[-2,5],yrange=[1,6],\\
         color=red,apply(parametric,append(C\_1,[t,0,1])),\\
         color=blue,apply(parametric,append(C\_2,[t,0,1])),\\
         color=green,apply(parametric,append(C\_3,[t,0,1])),\\
         color=black,label(["C\_1",1.5,1.9]),
         label(["C\_2",4.1,3.5],["C\_3",1.9,3.8]),\\
         color=black,label(["(-1,2)",-1,1.9]),
         label(["(4,2)",4,1.9],["(4,5)",4,5.1]))\$
\end{minipage}
%%% OUTPUT:
\[\displaystyle
\tag{\%{}t40}\label{t40} 
\includegraphics[width=.95\linewidth,height=.80\textheight,keepaspectratio]{Green's Theorem_img/Green's Theorem_6}\mbox{}
\]
%%%%%%%%%%%%%%%
\pagebreak


Find area of ellipse $\dfrac{x^2}{a^2}+\dfrac{y^2}{b^2}=1$



\noindent
%%%%%%%%%%%%%%%
%%% INPUT:
\begin{minipage}[t]{8ex}\color{red}\bf
(\%{}i41) 
\end{minipage}
\begin{minipage}[t]{\textwidth}\color{blue}\tt
kill(labels,a,b)\$
\end{minipage}


\noindent
%%%%%%%%%%%%%%%
%%% INPUT:
\begin{minipage}[t]{8ex}\color{red}\bf
(\%{}i1) 
\end{minipage}
\begin{minipage}[t]{\textwidth}\color{blue}\tt
assume(a\ensuremath{>}0,b\ensuremath{>}0)\$
\end{minipage}


\noindent
%%%%%%%%%%%%%%%
%%% INPUT:
\begin{minipage}[t]{8ex}\color{red}\bf
(\%{}i2) 
\end{minipage}
\begin{minipage}[t]{\textwidth}\color{blue}\tt
declare([a,b],constant)\$
\end{minipage}


\noindent
%%%%%%%%%%%%%%%
%%% INPUT:
\begin{minipage}[t]{8ex}\color{red}\bf
(\%{}i3) 
\end{minipage}
\begin{minipage}[t]{\textwidth}\color{blue}\tt
E:x\ensuremath{^2}/a\ensuremath{^2}+y\ensuremath{^2}/b\ensuremath{^2}=1\$
\end{minipage}


\noindent
%%%%%%%%%%%%%%%
%%% INPUT:
\begin{minipage}[t]{8ex}\color{red}\bf
(\%{}i4) 
\end{minipage}
\begin{minipage}[t]{\textwidth}\color{blue}\tt
sol:solve(E,y);
\end{minipage}
%%% OUTPUT:
\[\displaystyle
\tag{sol}\label{sol}
\left[y=-\frac{b\,\sqrt{{{a}^{2}}-{{x}^{2}}}}{a},y=\frac{b\,\sqrt{{{a}^{2}}-{{x}^{2}}}}{a}\right]\mbox{}
\]
%%%%%%%%%%%%%%%


\noindent
%%%%%%%%%%%%%%%
%%% INPUT:
\begin{minipage}[t]{8ex}\color{red}\bf
(\%{}i5) 
\end{minipage}
\begin{minipage}[t]{\textwidth}\color{blue}\tt
I:'integrate('integrate(1,y,0,rhs(sol[2])),x,-a,a)+\\
  'integrate('integrate(1,y,rhs(sol[1]),0),x,-a,a)\$
\end{minipage}


\noindent
%%%%%%%%%%%%%%%
%%% INPUT:
\begin{minipage}[t]{8ex}\color{red}\bf
(\%{}i6) 
\end{minipage}
\begin{minipage}[t]{\textwidth}\color{blue}\tt
ldisplay(I=box(ev(I,integrate)))\$
\end{minipage}
%%% OUTPUT:
\[\displaystyle
\tag{\%{}t6}\label{t6} 
\frac{2b\,\int_{-a}^{a}{\left. \sqrt{{{a}^{2}}-{{x}^{2}}}dx\right.}}{a}=\left( \ensuremath{\pi} ab\right) \mbox{}
\]
%%%%%%%%%%%%%%%

\textbf{Vector field} $\vec{F}_1\in\mathbb{R}^2$



\noindent
%%%%%%%%%%%%%%%
%%% INPUT:
\begin{minipage}[t]{8ex}\color{red}\bf
(\%{}i7) 
\end{minipage}
\begin{minipage}[t]{\textwidth}\color{blue}\tt
ldisplay(F\_1:[0,x])\$
\end{minipage}
%%% OUTPUT:
\[\displaystyle
\tag{\%{}t7}\label{t7} 
{{F}_{1}}=[0,x]\mbox{}
\]
%%%%%%%%%%%%%%%

$\nabla\times\vec{F}_1\in\mathbb{R}^2$



\noindent
%%%%%%%%%%%%%%%
%%% INPUT:
\begin{minipage}[t]{8ex}\color{red}\bf
(\%{}i8) 
\end{minipage}
\begin{minipage}[t]{\textwidth}\color{blue}\tt
ldisplay(curlF\_1:ev(express(curl(F\_1)),diff))\$
\end{minipage}
%%% OUTPUT:
\[\displaystyle
\tag{\%{}t8}\label{t8} 
{{\mathit{curlF}}_{1}}=1\mbox{}
\]
%%%%%%%%%%%%%%%

\textbf{Vector field} $\vec{F}_2\in\mathbb{R}^2$



\noindent
%%%%%%%%%%%%%%%
%%% INPUT:
\begin{minipage}[t]{8ex}\color{red}\bf
(\%{}i9) 
\end{minipage}
\begin{minipage}[t]{\textwidth}\color{blue}\tt
ldisplay(F\_2:[-y,0])\$
\end{minipage}
%%% OUTPUT:
\[\displaystyle
\tag{\%{}t9}\label{t9} 
{{F}_{2}}=[-y,0]\mbox{}
\]
%%%%%%%%%%%%%%%

$\nabla\times\vec{F}_2\in\mathbb{R}^2$



\noindent
%%%%%%%%%%%%%%%
%%% INPUT:
\begin{minipage}[t]{8ex}\color{red}\bf
(\%{}i10) 
\end{minipage}
\begin{minipage}[t]{\textwidth}\color{blue}\tt
ldisplay(curlF\_2:ev(express(curl(F\_2)),diff))\$
\end{minipage}
%%% OUTPUT:
\[\displaystyle
\tag{\%{}t10}\label{t10} 
{{\mathit{curlF}}_{2}}=1\mbox{}
\]
%%%%%%%%%%%%%%%
\pagebreak


\textbf{Vector field} $\vec{F}_3\in\mathbb{R}^2$



\noindent
%%%%%%%%%%%%%%%
%%% INPUT:
\begin{minipage}[t]{8ex}\color{red}\bf
(\%{}i11) 
\end{minipage}
\begin{minipage}[t]{\textwidth}\color{blue}\tt
ldisplay(F\_3:\ensuremath{\frac{1}{2}}*(F\_1+F\_2))\$
\end{minipage}
%%% OUTPUT:
\[\displaystyle
\tag{\%{}t11}\label{t11} 
{{F}_{3}}=\left[-\frac{y}{2},\frac{x}{2}\right]\mbox{}
\]
%%%%%%%%%%%%%%%

$\nabla\times\vec{F}_3\in\mathbb{R}^2$



\noindent
%%%%%%%%%%%%%%%
%%% INPUT:
\begin{minipage}[t]{8ex}\color{red}\bf
(\%{}i12) 
\end{minipage}
\begin{minipage}[t]{\textwidth}\color{blue}\tt
ldisplay(curlF\_3:ev(express(curl(F\_3)),diff))\$
\end{minipage}
%%% OUTPUT:
\[\displaystyle
\tag{\%{}t12}\label{t12} 
{{\mathit{curlF}}_{3}}=1\mbox{}
\]
%%%%%%%%%%%%%%%

\textbf{Work form} $\alpha_3\in\mathcal{A}^1(\mathbb{R}^2)$



\noindent
%%%%%%%%%%%%%%%
%%% INPUT:
\begin{minipage}[t]{8ex}\color{red}\bf
(\%{}i13) 
\end{minipage}
\begin{minipage}[t]{\textwidth}\color{blue}\tt
ldisplay(\ensuremath{\alpha}\_3:F\_3.cartan\_basis)\$
\end{minipage}
%%% OUTPUT:
\[\displaystyle
\tag{\%{}t13}\label{t13} 
{{\mathit{\ensuremath{\alpha}}}_{3}}=\frac{x\,\mathit{dy}}{2}-\frac{y\,\mathit{dx}}{2}\mbox{}
\]
%%%%%%%%%%%%%%%

$\mathrm{d}\alpha_3\in\mathcal{A}^2(\mathbb{R}^2)$



\noindent
%%%%%%%%%%%%%%%
%%% INPUT:
\begin{minipage}[t]{8ex}\color{red}\bf
(\%{}i14) 
\end{minipage}
\begin{minipage}[t]{\textwidth}\color{blue}\tt
ldisplay(d\ensuremath{\alpha}\_3:ext\_diff(\ensuremath{\alpha}\_3))\$
\end{minipage}
%%% OUTPUT:
\[\displaystyle
\tag{\%{}t14}\label{t14} 
{{\mathit{d\ensuremath{\alpha}}}_{3}}=\mathit{dx}\,\mathit{dy}\mbox{}
\]
%%%%%%%%%%%%%%%


\noindent
%%%%%%%%%%%%%%%
%%% INPUT:
\begin{minipage}[t]{8ex}\color{red}\bf
(\%{}i15) 
\end{minipage}
\begin{minipage}[t]{\textwidth}\color{blue}\tt
d\ensuremath{\alpha}\_3/apply("*",cartan\_basis);
\end{minipage}
%%% OUTPUT:
\[\displaystyle
\tag{\%{}o15}\label{o15} 
1\mbox{}
\]
%%%%%%%%%%%%%%%

$\nabla\cdot\vec{F}_3\in\mathbb{R}$



\noindent
%%%%%%%%%%%%%%%
%%% INPUT:
\begin{minipage}[t]{8ex}\color{red}\bf
(\%{}i16) 
\end{minipage}
\begin{minipage}[t]{\textwidth}\color{blue}\tt
ldisplay(divF\_3:ev(express(div(F\_3)),diff))\$
\end{minipage}
%%% OUTPUT:
\[\displaystyle
\tag{\%{}t16}\label{t16} 
{{\mathit{divF}}_{3}}=0\mbox{}
\]
%%%%%%%%%%%%%%%

\textbf{Flux form} $\beta_3\in\mathcal{A}^1(\mathbb{R}^2)$



\noindent
%%%%%%%%%%%%%%%
%%% INPUT:
\begin{minipage}[t]{8ex}\color{red}\bf
(\%{}i17) 
\end{minipage}
\begin{minipage}[t]{\textwidth}\color{blue}\tt
ldisplay(\ensuremath{\beta}\_3:first(F\_3)*cartan\_basis[2]-second(F\_3)*cartan\_basis[1])\$
\end{minipage}
%%% OUTPUT:
\[\displaystyle
\tag{\%{}t17}\label{t17} 
{{\mathit{\ensuremath{\beta}}}_{3}}=-\frac{y\,\mathit{dy}}{2}-\frac{x\,\mathit{dx}}{2}\mbox{}
\]
%%%%%%%%%%%%%%%

$\mathrm{d}\beta_3\in\mathcal{A}^2(\mathbb{R}^2)$



\noindent
%%%%%%%%%%%%%%%
%%% INPUT:
\begin{minipage}[t]{8ex}\color{red}\bf
(\%{}i18) 
\end{minipage}
\begin{minipage}[t]{\textwidth}\color{blue}\tt
ldisplay(d\ensuremath{\beta}\_3:edit(ext\_diff(\ensuremath{\beta}\_3)))\$
\end{minipage}
%%% OUTPUT:
\[\displaystyle
\tag{\%{}t18}\label{t18} 
{{\mathit{d\ensuremath{\beta}}}_{3}}=0\mbox{}
\]
%%%%%%%%%%%%%%%


\noindent
%%%%%%%%%%%%%%%
%%% INPUT:
\begin{minipage}[t]{8ex}\color{red}\bf
(\%{}i19) 
\end{minipage}
\begin{minipage}[t]{\textwidth}\color{blue}\tt
d\ensuremath{\beta}\_3/apply("*",cartan\_basis);
\end{minipage}
%%% OUTPUT:
\[\displaystyle
\tag{\%{}o19}\label{o19} 
0\mbox{}
\]
%%%%%%%%%%%%%%%
\pagebreak


$$A(D)=\iint_D\mathrm{d}A=\dfrac{1}{2}\oint_C -y\,\mathrm{d}x+x\,\mathrm{d}y$$


\textbf{Parametrize the ellipse}



\noindent
%%%%%%%%%%%%%%%
%%% INPUT:
\begin{minipage}[t]{8ex}\color{red}\bf
(\%{}i20) 
\end{minipage}
\begin{minipage}[t]{\textwidth}\color{blue}\tt
assume(0\ensuremath{\leq}t,t\ensuremath{\leq}2*\ensuremath{\pi})\$
\end{minipage}


\noindent
%%%%%%%%%%%%%%%
%%% INPUT:
\begin{minipage}[t]{8ex}\color{red}\bf
(\%{}i21) 
\end{minipage}
\begin{minipage}[t]{\textwidth}\color{blue}\tt
ldisplay(r:[a*cos(t),b*sin(t)])\$
\end{minipage}
%%% OUTPUT:
\[\displaystyle
\tag{\%{}t21}\label{t21} 
r=[a\,\cos{(t)},b\,\sin{(t)}]\mbox{}
\]
%%%%%%%%%%%%%%%

\textbf{Verify this is an ellipse}



\noindent
%%%%%%%%%%%%%%%
%%% INPUT:
\begin{minipage}[t]{8ex}\color{red}\bf
(\%{}i22) 
\end{minipage}
\begin{minipage}[t]{\textwidth}\color{blue}\tt
is(trigsimp(subst(map("=",\ensuremath{\zeta},r),E)));
\end{minipage}
%%% OUTPUT:
\[\displaystyle
\tag{\%{}o22}\label{o22} 
\mbox{true}\mbox{}
\]
%%%%%%%%%%%%%%%

\textbf{Derivative of the curve} $\vec{r}$



\noindent
%%%%%%%%%%%%%%%
%%% INPUT:
\begin{minipage}[t]{8ex}\color{red}\bf
(\%{}i23) 
\end{minipage}
\begin{minipage}[t]{\textwidth}\color{blue}\tt
ldisplay(r\ensuremath{\backslash}':diff(r,t))\$
\end{minipage}
%%% OUTPUT:
\[\displaystyle
\tag{\%{}t23}\label{t23} 
\mathit{r'}=[-a\,\sin{(t)},b\,\cos{(t)}]\mbox{}
\]
%%%%%%%%%%%%%%%

$\vec{F}_3\circ\vec{r}$



\noindent
%%%%%%%%%%%%%%%
%%% INPUT:
\begin{minipage}[t]{8ex}\color{red}\bf
(\%{}i24) 
\end{minipage}
\begin{minipage}[t]{\textwidth}\color{blue}\tt
ldisplay(For\_3:subst(map("=",\ensuremath{\zeta},r),F\_3))\$
\end{minipage}
%%% OUTPUT:
\[\displaystyle
\tag{\%{}t24}\label{t24} 
{{\mathit{For}}_{3}}=\left[-\frac{b\,\sin{(t)}}{2},\frac{a\,\cos{(t)}}{2}\right]\mbox{}
\]
%%%%%%%%%%%%%%%

$\vec{F}_3\cdot\vec{r}^{\prime}\in\mathbb{R}$



\noindent
%%%%%%%%%%%%%%%
%%% INPUT:
\begin{minipage}[t]{8ex}\color{red}\bf
(\%{}i25) 
\end{minipage}
\begin{minipage}[t]{\textwidth}\color{blue}\tt
ldisplay(T\_3:trigsimp(For\_3.r\ensuremath{\backslash}'))\$
\end{minipage}
%%% OUTPUT:
\[\displaystyle
\tag{\%{}t25}\label{t25} 
{{T}_{3}}=\frac{ab}{2}\mbox{}
\]
%%%%%%%%%%%%%%%

\textbf{Pullback} $\vec{r}^\ast\alpha_3\in\mathcal{A}^1(\mathbb{R}^2)$



\noindent
%%%%%%%%%%%%%%%
%%% INPUT:
\begin{minipage}[t]{8ex}\color{red}\bf
(\%{}i26) 
\end{minipage}
\begin{minipage}[t]{\textwidth}\color{blue}\tt
ldisplay(P\_3:trigsimp(r\ensuremath{\backslash}'|subst(map("=",\ensuremath{\zeta},r),\ensuremath{\alpha}\_3)))\$
\end{minipage}
%%% OUTPUT:
\[\displaystyle
\tag{\%{}t26}\label{t26} 
{{P}_{3}}=\frac{ab}{2}\mbox{}
\]
%%%%%%%%%%%%%%%

\textbf{Line integral} $I_3$



\noindent
%%%%%%%%%%%%%%%
%%% INPUT:
\begin{minipage}[t]{8ex}\color{red}\bf
(\%{}i27) 
\end{minipage}
\begin{minipage}[t]{\textwidth}\color{blue}\tt
ldisplay(I\_3:box('integrate(T\_3,t,0,2*\ensuremath{\pi})))\$
\end{minipage}
%%% OUTPUT:
\[\displaystyle
\tag{\%{}t27}\label{t27} 
{{I}_{3}}=\left( \ensuremath{\pi} ab\right) \mbox{}
\]
%%%%%%%%%%%%%%%

\textbf{Clean up}



\noindent
%%%%%%%%%%%%%%%
%%% INPUT:
\begin{minipage}[t]{8ex}\color{red}\bf
(\%{}i29) 
\end{minipage}
\begin{minipage}[t]{\textwidth}\color{blue}\tt
forget(a\ensuremath{>}0,b\ensuremath{>}0)\$\\
forget(0\ensuremath{\leq}t,t\ensuremath{\leq}2*\ensuremath{\pi})\$
\end{minipage}

\section{Vector forms of Green's Theorem}


Based on Michael Penn Video
\href{https://www.youtube.com/watch?v=-ys51OICh_c}
{Vector forms of Green's Theorem}



\noindent
%%%%%%%%%%%%%%%
%%% INPUT:
\begin{minipage}[t]{8ex}\color{red}\bf
(\%{}i30) 
\end{minipage}
\begin{minipage}[t]{\textwidth}\color{blue}\tt
kill(labels,t,x,y,z,P,Q)\$
\end{minipage}

\textbf{Define the space} $\mathbb{R}^3$



\noindent
%%%%%%%%%%%%%%%
%%% INPUT:
\begin{minipage}[t]{8ex}\color{red}\bf
(\%{}i1) 
\end{minipage}
\begin{minipage}[t]{\textwidth}\color{blue}\tt
\ensuremath{\zeta}:[x,y,z]\$
\end{minipage}


\noindent
%%%%%%%%%%%%%%%
%%% INPUT:
\begin{minipage}[t]{8ex}\color{red}\bf
(\%{}i2) 
\end{minipage}
\begin{minipage}[t]{\textwidth}\color{blue}\tt
scalefactors(\ensuremath{\zeta})\$
\end{minipage}


\noindent
%%%%%%%%%%%%%%%
%%% INPUT:
\begin{minipage}[t]{8ex}\color{red}\bf
(\%{}i3) 
\end{minipage}
\begin{minipage}[t]{\textwidth}\color{blue}\tt
init\_cartan(\ensuremath{\zeta})\$
\end{minipage}
\pagebreak


\textbf{Vector field} $\vec{F}\in\mathbb{R}^2$



\noindent
%%%%%%%%%%%%%%%
%%% INPUT:
\begin{minipage}[t]{8ex}\color{red}\bf
(\%{}i4) 
\end{minipage}
\begin{minipage}[t]{\textwidth}\color{blue}\tt
ldisplay(F:[x\ensuremath{^2}*y,x*z,z\ensuremath{^3}])\$
\end{minipage}
%%% OUTPUT:
\[\displaystyle
\tag{\%{}t4}\label{t4} 
F=[{{x}^{2}}y,xz,{{z}^{3}}]\mbox{}
\]
%%%%%%%%%%%%%%%

\textbf{3D Direction field}



\noindent
%%%%%%%%%%%%%%%
%%% INPUT:
\begin{minipage}[t]{8ex}\color{red}\bf
(\%{}i6) 
\end{minipage}
\begin{minipage}[t]{\textwidth}\color{blue}\tt
/* vector origins are {(x,y,z)| x,y=1,...,5}  */\\
coord:setify(makelist(k,k,0,5))\$\\
points3d:listify(cartesian\_product(coord,coord,coord))\$
\end{minipage}


\noindent
%%%%%%%%%%%%%%%
%%% INPUT:
\begin{minipage}[t]{8ex}\color{red}\bf
(\%{}i8) 
\end{minipage}
\begin{minipage}[t]{\textwidth}\color{blue}\tt
/* compute vectors at the given points  */\\
define(vf3d(x,y,z),vector(\ensuremath{\zeta},F))\$\\
vect3:makelist(vf3d(k[1],k[2],k[3]),k,points3d)\$
\end{minipage}


\noindent
%%%%%%%%%%%%%%%
%%% INPUT:
\begin{minipage}[t]{8ex}\color{red}\bf
(\%{}i9) 
\end{minipage}
\begin{minipage}[t]{\textwidth}\color{blue}\tt
wxdraw3d([head\_length=0.1,color=blue,head\_angle=25,unit\_vectors=true],vect3)\$
\end{minipage}
%%% OUTPUT:
\[\displaystyle
\tag{\%{}t9}\label{t9} 
\includegraphics[width=.95\linewidth,height=.80\textheight,keepaspectratio]{Green's Theorem_img/Green's Theorem_7}\mbox{}
\]
%%%%%%%%%%%%%%%
\pagebreak


$\nabla\times\vec{F}\in\mathbb{R}^3$



\noindent
%%%%%%%%%%%%%%%
%%% INPUT:
\begin{minipage}[t]{8ex}\color{red}\bf
(\%{}i10) 
\end{minipage}
\begin{minipage}[t]{\textwidth}\color{blue}\tt
ldisplay(curlF:ev(express(curl(F)),diff))\$
\end{minipage}
%%% OUTPUT:
\[\displaystyle
\tag{\%{}t10}\label{t10} 
\mathit{curlF}=[-x,0,z-{{x}^{2}}]\mbox{}
\]
%%%%%%%%%%%%%%%

\textbf{Work form} $\alpha\in\mathcal{A}^1(\mathbb{R}^3)$



\noindent
%%%%%%%%%%%%%%%
%%% INPUT:
\begin{minipage}[t]{8ex}\color{red}\bf
(\%{}i11) 
\end{minipage}
\begin{minipage}[t]{\textwidth}\color{blue}\tt
ldisplay(\ensuremath{\alpha}:F.cartan\_basis)\$
\end{minipage}
%%% OUTPUT:
\[\displaystyle
\tag{\%{}t11}\label{t11} 
\mathit{\ensuremath{\alpha}}={{z}^{3}}\,\mathit{dz}+xz\,\mathit{dy}+{{x}^{2}}y\,\mathit{dx}\mbox{}
\]
%%%%%%%%%%%%%%%

$\mathrm{d}\alpha\in\mathcal{A}^2(\mathbb{R}^3)$



\noindent
%%%%%%%%%%%%%%%
%%% INPUT:
\begin{minipage}[t]{8ex}\color{red}\bf
(\%{}i12) 
\end{minipage}
\begin{minipage}[t]{\textwidth}\color{blue}\tt
ldisplay(d\ensuremath{\alpha}:edit(ext\_diff(\ensuremath{\alpha})))\$
\end{minipage}
%%% OUTPUT:
\[\displaystyle
\tag{\%{}t12}\label{t12} 
\mathit{d\ensuremath{\alpha}}=\left( z-{{x}^{2}}\right) \,\mathit{dx}\,\mathit{dy}-x\,\mathit{dy}\,\mathit{dz}\mbox{}
\]
%%%%%%%%%%%%%%%

$\nabla\cdot\vec{F}\in\mathbb{R}$



\noindent
%%%%%%%%%%%%%%%
%%% INPUT:
\begin{minipage}[t]{8ex}\color{red}\bf
(\%{}i13) 
\end{minipage}
\begin{minipage}[t]{\textwidth}\color{blue}\tt
ldisplay(divF:ev(express(div(F)),diff))\$
\end{minipage}
%%% OUTPUT:
\[\displaystyle
\tag{\%{}t13}\label{t13} 
\mathit{divF}=3{{z}^{2}}+2xy\mbox{}
\]
%%%%%%%%%%%%%%%

\textbf{Flux form} $\beta\in\mathcal{A}^2(\mathbb{R}^3)$



\noindent
%%%%%%%%%%%%%%%
%%% INPUT:
\begin{minipage}[t]{8ex}\color{red}\bf
(\%{}i14) 
\end{minipage}
\begin{minipage}[t]{\textwidth}\color{blue}\tt
ldisplay(\ensuremath{\beta}:F[1]*cartan\_basis[2]\ensuremath{\sim }cartan\_basis[3]+\\
           F[2]*cartan\_basis[3]\ensuremath{\sim }cartan\_basis[1]+\\
           F[3]*cartan\_basis[1]\ensuremath{\sim }cartan\_basis[2])\$
\end{minipage}
%%% OUTPUT:
\[\displaystyle
\tag{\%{}t14}\label{t14} 
\mathit{\ensuremath{\beta}}={{x}^{2}}y\,\mathit{dy}\,\mathit{dz}-xz\,\mathit{dx}\,\mathit{dz}+{{z}^{3}}\,\mathit{dx}\,\mathit{dy}\mbox{}
\]
%%%%%%%%%%%%%%%

$\mathrm{d}\beta\in\mathcal{A}^2(\mathbb{R}^3)$



\noindent
%%%%%%%%%%%%%%%
%%% INPUT:
\begin{minipage}[t]{8ex}\color{red}\bf
(\%{}i15) 
\end{minipage}
\begin{minipage}[t]{\textwidth}\color{blue}\tt
ldisplay(d\ensuremath{\beta}:edit(ext\_diff(\ensuremath{\beta})))\$
\end{minipage}
%%% OUTPUT:
\[\displaystyle
\tag{\%{}t15}\label{t15} 
\mathit{d\ensuremath{\beta}}=\left( 3{{z}^{2}}+2xy\right) \,\mathit{dx}\,\mathit{dy}\,\mathit{dz}\mbox{}
\]
%%%%%%%%%%%%%%%


\noindent
%%%%%%%%%%%%%%%
%%% INPUT:
\begin{minipage}[t]{8ex}\color{red}\bf
(\%{}i16) 
\end{minipage}
\begin{minipage}[t]{\textwidth}\color{blue}\tt
d\ensuremath{\beta}/apply("*",cartan\_basis);
\end{minipage}
%%% OUTPUT:
\[\displaystyle
\tag{\%{}o16}\label{o16} 
3{{z}^{2}}+2xy\mbox{}
\]
%%%%%%%%%%%%%%%
\pagebreak


Based on Opentextbc Website
\href{https://opentextbc.ca/calculusv3openstax/chapter/greens-theorem}
{41 Green’s Theorem}

Based on Openstax Website
\href{https://openstax.org/books/calculus-volume-3/pages/6-4-greens-theorem}
{6.4 Green’s Theorem}


\textbf{Circulation Form of Green’s Theorem}

$$\oint_C\vec{F}\,\mathrm{d}\vec{r}=
\oint_C P\,\mathrm{d}x+Q\,\mathrm{d}y=
\iint_D\left({Q_x-P_y}\right)\,\mathrm{d}A$$


\textbf{Flux Form of Green’s Theorem}

$$\oint_C\vec{F}\cdot\vec{N}\,\mathrm{d}s=
\iint_D\left({P_x+Q_y}\right)\,\mathrm{d}A$$


\textbf{Define the space} $\mathbb{R}^2$



\noindent
%%%%%%%%%%%%%%%
%%% INPUT:
\begin{minipage}[t]{8ex}\color{red}\bf
(\%{}i17) 
\end{minipage}
\begin{minipage}[t]{\textwidth}\color{blue}\tt
\ensuremath{\zeta}:[x,y]\$
\end{minipage}


\noindent
%%%%%%%%%%%%%%%
%%% INPUT:
\begin{minipage}[t]{8ex}\color{red}\bf
(\%{}i18) 
\end{minipage}
\begin{minipage}[t]{\textwidth}\color{blue}\tt
scalefactors(\ensuremath{\zeta})\$
\end{minipage}


\noindent
%%%%%%%%%%%%%%%
%%% INPUT:
\begin{minipage}[t]{8ex}\color{red}\bf
(\%{}i19) 
\end{minipage}
\begin{minipage}[t]{\textwidth}\color{blue}\tt
init\_cartan(\ensuremath{\zeta})\$
\end{minipage}

\textbf{Vector field} $\vec{F}\in\mathbb{R}^2$



\noindent
%%%%%%%%%%%%%%%
%%% INPUT:
\begin{minipage}[t]{8ex}\color{red}\bf
(\%{}i20) 
\end{minipage}
\begin{minipage}[t]{\textwidth}\color{blue}\tt
F:[P,Q]\$
\end{minipage}


\noindent
%%%%%%%%%%%%%%%
%%% INPUT:
\begin{minipage}[t]{8ex}\color{red}\bf
(\%{}i21) 
\end{minipage}
\begin{minipage}[t]{\textwidth}\color{blue}\tt
depends(F,\ensuremath{\zeta})\$
\end{minipage}
\pagebreak


\textbf{Version 1 of Green's theorem (Circulation Form)}
$$\oint_C\vec{F}\,\mathrm{d}\vec{r}=
\iint_D\left({\nabla\times\vec{F}}\right)\,\mathrm{d}A$$


$\nabla\times\vec{F}\in\mathbb{R}^2$



\noindent
%%%%%%%%%%%%%%%
%%% INPUT:
\begin{minipage}[t]{8ex}\color{red}\bf
(\%{}i22) 
\end{minipage}
\begin{minipage}[t]{\textwidth}\color{blue}\tt
ldisplay(curlF:ev(express(curl(F)),diff))\$
\end{minipage}
%%% OUTPUT:
\[\displaystyle
\tag{\%{}t22}\label{t22} 
\mathit{curlF}={{Q}_{x}}-{{P}_{y}}\mbox{}
\]
%%%%%%%%%%%%%%%

\textbf{Work form} $\alpha\in\mathcal{A}^1(\mathbb{R}^2)$



\noindent
%%%%%%%%%%%%%%%
%%% INPUT:
\begin{minipage}[t]{8ex}\color{red}\bf
(\%{}i23) 
\end{minipage}
\begin{minipage}[t]{\textwidth}\color{blue}\tt
ldisplay(\ensuremath{\alpha}:F.cartan\_basis)\$
\end{minipage}
%%% OUTPUT:
\[\displaystyle
\tag{\%{}t23}\label{t23} 
\mathit{\ensuremath{\alpha}}=Q\,\mathit{dy}+P\,\mathit{dx}\mbox{}
\]
%%%%%%%%%%%%%%%

$\mathrm{d}\alpha\in\mathcal{A}^2(\mathbb{R}^2)$



\noindent
%%%%%%%%%%%%%%%
%%% INPUT:
\begin{minipage}[t]{8ex}\color{red}\bf
(\%{}i24) 
\end{minipage}
\begin{minipage}[t]{\textwidth}\color{blue}\tt
ldisplay(d\ensuremath{\alpha}:edit(ext\_diff(\ensuremath{\alpha})))\$
\end{minipage}
%%% OUTPUT:
\[\displaystyle
\tag{\%{}t24}\label{t24} 
\mathit{d\ensuremath{\alpha}}=\left( {{Q}_{x}}-{{P}_{y}}\right) \,\mathit{dx}\,\mathit{dy}\mbox{}
\]
%%%%%%%%%%%%%%%

\textbf{Version 2 of Green's theorem (Flux Form)}
$$\oint_C\left({\vec{F}\cdot\hat{n}}\right)\,\mathrm{d}s=
\iint_D\left({\nabla\cdot\vec{F}}\right)\,\mathrm{d}A$$


$\nabla\cdot\vec{F}\in\mathbb{R}$



\noindent
%%%%%%%%%%%%%%%
%%% INPUT:
\begin{minipage}[t]{8ex}\color{red}\bf
(\%{}i25) 
\end{minipage}
\begin{minipage}[t]{\textwidth}\color{blue}\tt
ldisplay(divF:ev(express(div(F)),diff))\$
\end{minipage}
%%% OUTPUT:
\[\displaystyle
\tag{\%{}t25}\label{t25} 
\mathit{divF}={{Q}_{y}}+{{P}_{x}}\mbox{}
\]
%%%%%%%%%%%%%%%

\textbf{Flux form} $\beta\in\mathcal{A}^2(\mathbb{R}^2)$



\noindent
%%%%%%%%%%%%%%%
%%% INPUT:
\begin{minipage}[t]{8ex}\color{red}\bf
(\%{}i26) 
\end{minipage}
\begin{minipage}[t]{\textwidth}\color{blue}\tt
ldisplay(\ensuremath{\beta}:F[1]*cartan\_basis[2]-F[2]*cartan\_basis[1])\$
\end{minipage}
%%% OUTPUT:
\[\displaystyle
\tag{\%{}t26}\label{t26} 
\mathit{\ensuremath{\beta}}=P\,\mathit{dy}-Q\,\mathit{dx}\mbox{}
\]
%%%%%%%%%%%%%%%

$\mathrm{d}\beta\in\mathcal{A}^2(\mathbb{R}^2)$



\noindent
%%%%%%%%%%%%%%%
%%% INPUT:
\begin{minipage}[t]{8ex}\color{red}\bf
(\%{}i27) 
\end{minipage}
\begin{minipage}[t]{\textwidth}\color{blue}\tt
ldisplay(d\ensuremath{\beta}:edit(ext\_diff(\ensuremath{\beta})))\$
\end{minipage}
%%% OUTPUT:
\[\displaystyle
\tag{\%{}t27}\label{t27} 
\mathit{d\ensuremath{\beta}}=\left( {{Q}_{y}}+{{P}_{x}}\right) \,\mathit{dx}\,\mathit{dy}\mbox{}
\]
%%%%%%%%%%%%%%%


\noindent
%%%%%%%%%%%%%%%
%%% INPUT:
\begin{minipage}[t]{8ex}\color{red}\bf
(\%{}i28) 
\end{minipage}
\begin{minipage}[t]{\textwidth}\color{blue}\tt
d\ensuremath{\beta}/apply("*",cartan\_basis);
\end{minipage}
%%% OUTPUT:
\[\displaystyle
\tag{\%{}o28}\label{o28} 
{{Q}_{y}}+{{P}_{x}}\mbox{}
\]
%%%%%%%%%%%%%%%
\pagebreak


\section{When Green's theorem doesn't apply}


Based on Michael Penn Video
\href{https://www.youtube.com/watch?v=9859d7wGxVU}
{When Green's theorem doesn't apply}


Find all possible values of $\oint_C\vec{F}\cdot\mathrm{d}\vec{r}$ where
$C$ satisfies all the conditions.



\noindent
%%%%%%%%%%%%%%%
%%% INPUT:
\begin{minipage}[t]{8ex}\color{red}\bf
(\%{}i29) 
\end{minipage}
\begin{minipage}[t]{\textwidth}\color{blue}\tt
kill(labels,t,x,y)\$
\end{minipage}

\textbf{Define the space} $\mathbb{R}^2$



\noindent
%%%%%%%%%%%%%%%
%%% INPUT:
\begin{minipage}[t]{8ex}\color{red}\bf
(\%{}i1) 
\end{minipage}
\begin{minipage}[t]{\textwidth}\color{blue}\tt
\ensuremath{\zeta}:[x,y]\$
\end{minipage}


\noindent
%%%%%%%%%%%%%%%
%%% INPUT:
\begin{minipage}[t]{8ex}\color{red}\bf
(\%{}i2) 
\end{minipage}
\begin{minipage}[t]{\textwidth}\color{blue}\tt
scalefactors(\ensuremath{\zeta})\$
\end{minipage}


\noindent
%%%%%%%%%%%%%%%
%%% INPUT:
\begin{minipage}[t]{8ex}\color{red}\bf
(\%{}i3) 
\end{minipage}
\begin{minipage}[t]{\textwidth}\color{blue}\tt
init\_cartan(\ensuremath{\zeta})\$
\end{minipage}

\textbf{Parameters}



\noindent
%%%%%%%%%%%%%%%
%%% INPUT:
\begin{minipage}[t]{8ex}\color{red}\bf
(\%{}i4) 
\end{minipage}
\begin{minipage}[t]{\textwidth}\color{blue}\tt
assume(a\ensuremath{>}0)\$
\end{minipage}


\noindent
%%%%%%%%%%%%%%%
%%% INPUT:
\begin{minipage}[t]{8ex}\color{red}\bf
(\%{}i5) 
\end{minipage}
\begin{minipage}[t]{\textwidth}\color{blue}\tt
declare(a,constant)\$
\end{minipage}


\noindent
%%%%%%%%%%%%%%%
%%% INPUT:
\begin{minipage}[t]{8ex}\color{red}\bf
(\%{}i6) 
\end{minipage}
\begin{minipage}[t]{\textwidth}\color{blue}\tt
params:[a=1]\$
\end{minipage}
\pagebreak


\textbf{Vector field} $\vec{F}\in\mathbb{R}^2$



\noindent
%%%%%%%%%%%%%%%
%%% INPUT:
\begin{minipage}[t]{8ex}\color{red}\bf
(\%{}i7) 
\end{minipage}
\begin{minipage}[t]{\textwidth}\color{blue}\tt
ldisplay(F:1/(x\ensuremath{^2}+y\ensuremath{^2})*[y,-x])\$
\end{minipage}
%%% OUTPUT:
\[\displaystyle
\tag{\%{}t7}\label{t7} 
F=\left[\frac{y}{{{y}^{2}}+{{x}^{2}}},-\frac{x}{{{y}^{2}}+{{x}^{2}}}\right]\mbox{}
\]
%%%%%%%%%%%%%%%

\textbf{2D Direction field}



\noindent
%%%%%%%%%%%%%%%
%%% INPUT:
\begin{minipage}[t]{8ex}\color{red}\bf
(\%{}i8) 
\end{minipage}
\begin{minipage}[t]{\textwidth}\color{blue}\tt
wxdrawdf(F,[x,-5,5],[y,-5,5])\$
\end{minipage}
%%% OUTPUT:
\[\displaystyle
\tag{\%{}t8}\label{t8} 
\includegraphics[width=.95\linewidth,height=.80\textheight,keepaspectratio]{Green's Theorem_img/Green's Theorem_8}\mbox{}
\]
%%%%%%%%%%%%%%%
\pagebreak


$\nabla\times\vec{F}\in\mathbb{R}^2$



\noindent
%%%%%%%%%%%%%%%
%%% INPUT:
\begin{minipage}[t]{8ex}\color{red}\bf
(\%{}i9) 
\end{minipage}
\begin{minipage}[t]{\textwidth}\color{blue}\tt
ldisplay(curlF:ratsimp(ev(express(curl(F)),diff)))\$
\end{minipage}
%%% OUTPUT:
\[\displaystyle
\tag{\%{}t9}\label{t9} 
\mathit{curlF}=0\mbox{}
\]
%%%%%%%%%%%%%%%

\textbf{Potential}



\noindent
%%%%%%%%%%%%%%%
%%% INPUT:
\begin{minipage}[t]{8ex}\color{red}\bf
(\%{}i10) 
\end{minipage}
\begin{minipage}[t]{\textwidth}\color{blue}\tt
ldisplay(\ensuremath{\phi}:potential(F))\$
\end{minipage}
%%% OUTPUT:
\[\displaystyle
\tag{\%{}t10}\label{t10} 
\mathit{\ensuremath{\phi}}=\operatorname{atan}\left( \frac{x}{y}\right) \mbox{}
\]
%%%%%%%%%%%%%%%

\textbf{Work form} $\alpha\in\mathcal{A}^1(\mathbb{R}^2)$



\noindent
%%%%%%%%%%%%%%%
%%% INPUT:
\begin{minipage}[t]{8ex}\color{red}\bf
(\%{}i11) 
\end{minipage}
\begin{minipage}[t]{\textwidth}\color{blue}\tt
ldisplay(\ensuremath{\alpha}:factor(F.cartan\_basis))\$
\end{minipage}
%%% OUTPUT:
\[\displaystyle
\tag{\%{}t11}\label{t11} 
\mathit{\ensuremath{\alpha}}=-\frac{x\,\mathit{dy}-y\,\mathit{dx}}{{{y}^{2}}+{{x}^{2}}}\mbox{}
\]
%%%%%%%%%%%%%%%

$\mathrm{d}\alpha\in\mathcal{A}^2(\mathbb{R}^2)$



\noindent
%%%%%%%%%%%%%%%
%%% INPUT:
\begin{minipage}[t]{8ex}\color{red}\bf
(\%{}i12) 
\end{minipage}
\begin{minipage}[t]{\textwidth}\color{blue}\tt
ldisplay(d\ensuremath{\alpha}:ratsimp(edit(ext\_diff(\ensuremath{\alpha}))))\$
\end{minipage}
%%% OUTPUT:
\[\displaystyle
\tag{\%{}t12}\label{t12} 
\mathit{d\ensuremath{\alpha}}=0\mbox{}
\]
%%%%%%%%%%%%%%%

$\nabla\cdot\vec{F}\in\mathbb{R}$



\noindent
%%%%%%%%%%%%%%%
%%% INPUT:
\begin{minipage}[t]{8ex}\color{red}\bf
(\%{}i13) 
\end{minipage}
\begin{minipage}[t]{\textwidth}\color{blue}\tt
ldisplay(divF:ev(express(div(F)),diff))\$
\end{minipage}
%%% OUTPUT:
\[\displaystyle
\tag{\%{}t13}\label{t13} 
\mathit{divF}=0\mbox{}
\]
%%%%%%%%%%%%%%%

\textbf{Flux form} $\beta\in\mathcal{A}^1(\mathbb{R}^2)$



\noindent
%%%%%%%%%%%%%%%
%%% INPUT:
\begin{minipage}[t]{8ex}\color{red}\bf
(\%{}i14) 
\end{minipage}
\begin{minipage}[t]{\textwidth}\color{blue}\tt
ldisplay(\ensuremath{\beta}:factor(F[1]*cartan\_basis[2]-F[2]*cartan\_basis[1]))\$
\end{minipage}
%%% OUTPUT:
\[\displaystyle
\tag{\%{}t14}\label{t14} 
\mathit{\ensuremath{\beta}}=\frac{y\,\mathit{dy}+x\,\mathit{dx}}{{{y}^{2}}+{{x}^{2}}}\mbox{}
\]
%%%%%%%%%%%%%%%

$\mathrm{d}\beta\in\mathcal{A}^2(\mathbb{R}^2)$



\noindent
%%%%%%%%%%%%%%%
%%% INPUT:
\begin{minipage}[t]{8ex}\color{red}\bf
(\%{}i15) 
\end{minipage}
\begin{minipage}[t]{\textwidth}\color{blue}\tt
ldisplay(d\ensuremath{\beta}:edit(ext\_diff(\ensuremath{\beta})))\$
\end{minipage}
%%% OUTPUT:
\[\displaystyle
\tag{\%{}t15}\label{t15} 
\mathit{d\ensuremath{\beta}}=0\mbox{}
\]
%%%%%%%%%%%%%%%
\pagebreak


\textbf{Curve} $\vec{r}_a\in\mathbb{R}^2$



\noindent
%%%%%%%%%%%%%%%
%%% INPUT:
\begin{minipage}[t]{8ex}\color{red}\bf
(\%{}i16) 
\end{minipage}
\begin{minipage}[t]{\textwidth}\color{blue}\tt
ldisplay(r\_a:[a*cos(t),a*sin(t)])\$
\end{minipage}
%%% OUTPUT:
\[\displaystyle
\tag{\%{}t16}\label{t16} 
{{r}_{a}}=[a\,\cos{(t)},a\,\sin{(t)}]\mbox{}
\]
%%%%%%%%%%%%%%%

\textbf{Derivative of the curve} $\vec{r}_a$



\noindent
%%%%%%%%%%%%%%%
%%% INPUT:
\begin{minipage}[t]{8ex}\color{red}\bf
(\%{}i17) 
\end{minipage}
\begin{minipage}[t]{\textwidth}\color{blue}\tt
ldisplay(r\ensuremath{\backslash}'\_a:diff(r\_a,t))\$
\end{minipage}
%%% OUTPUT:
\[\displaystyle
\tag{\%{}t17}\label{t17} 
{{\mathit{r'}}_{a}}=[-a\,\sin{(t)},a\,\cos{(t)}]\mbox{}
\]
%%%%%%%%%%%%%%%

$\vec{F}\circ\vec{r}_a$



\noindent
%%%%%%%%%%%%%%%
%%% INPUT:
\begin{minipage}[t]{8ex}\color{red}\bf
(\%{}i18) 
\end{minipage}
\begin{minipage}[t]{\textwidth}\color{blue}\tt
ldisplay(For\_a:trigsimp(subst(map("=",\ensuremath{\zeta},r\_a),F)))\$
\end{minipage}
%%% OUTPUT:
\[\displaystyle
\tag{\%{}t18}\label{t18} 
{{\mathit{For}}_{a}}=\left[\frac{\sin{(t)}}{a},-\frac{\cos{(t)}}{a}\right]\mbox{}
\]
%%%%%%%%%%%%%%%

$\vec{F}\cdot\vec{r}^{\prime}_a\in\mathbb{R}$



\noindent
%%%%%%%%%%%%%%%
%%% INPUT:
\begin{minipage}[t]{8ex}\color{red}\bf
(\%{}i19) 
\end{minipage}
\begin{minipage}[t]{\textwidth}\color{blue}\tt
ldisplay(T\_a:trigsimp(For\_a.r\ensuremath{\backslash}'\_a))\$
\end{minipage}
%%% OUTPUT:
\[\displaystyle
\tag{\%{}t19}\label{t19} 
{{T}_{a}}=-1\mbox{}
\]
%%%%%%%%%%%%%%%

\textbf{Pullback} $\vec{r}^\ast_a\alpha\in\mathcal{A}^1(\mathbb{R}^2)$



\noindent
%%%%%%%%%%%%%%%
%%% INPUT:
\begin{minipage}[t]{8ex}\color{red}\bf
(\%{}i20) 
\end{minipage}
\begin{minipage}[t]{\textwidth}\color{blue}\tt
ldisplay(P\_a:trigsimp(r\ensuremath{\backslash}'\_a|subst(map("=",\ensuremath{\zeta},r\_a),\ensuremath{\alpha})))\$
\end{minipage}
%%% OUTPUT:
\[\displaystyle
\tag{\%{}t20}\label{t20} 
{{P}_{a}}=-1\mbox{}
\]
%%%%%%%%%%%%%%%

\textbf{Line integral} $I_a$



\noindent
%%%%%%%%%%%%%%%
%%% INPUT:
\begin{minipage}[t]{8ex}\color{red}\bf
(\%{}i21) 
\end{minipage}
\begin{minipage}[t]{\textwidth}\color{blue}\tt
ldisplay(I\_a:'integrate(T\_a,t,0,2*\ensuremath{\pi}))\$
\end{minipage}
%%% OUTPUT:
\[\displaystyle
\tag{\%{}t21}\label{t21} 
{{I}_{a}}=-2\ensuremath{\pi} \mbox{}
\]
%%%%%%%%%%%%%%%

\textbf{Clean up}



\noindent
%%%%%%%%%%%%%%%
%%% INPUT:
\begin{minipage}[t]{8ex}\color{red}\bf
(\%{}i22) 
\end{minipage}
\begin{minipage}[t]{\textwidth}\color{blue}\tt
forget(a\ensuremath{>}0)\$
\end{minipage}
\end{document}
