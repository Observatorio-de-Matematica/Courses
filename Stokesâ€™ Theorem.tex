\documentclass{article}

%% Created with wxMaxima 16.04.2

\setlength{\parskip}{\medskipamount}
\setlength{\parindent}{0pt}
\usepackage[utf8]{inputenc}
\DeclareUnicodeCharacter{00B5}{\ensuremath{\mu}}
\usepackage{graphicx}
\usepackage{color}
\usepackage{amsmath}
\usepackage{ifthen}
\newsavebox{\picturebox}
\newlength{\pictureboxwidth}
\newlength{\pictureboxheight}
\newcommand{\includeimage}[1]{
    \savebox{\picturebox}{\includegraphics{#1}}
    \settoheight{\pictureboxheight}{\usebox{\picturebox}}
    \settowidth{\pictureboxwidth}{\usebox{\picturebox}}
    \ifthenelse{\lengthtest{\pictureboxwidth > .95\linewidth}}
    {
        \includegraphics[width=.95\linewidth,height=.80\textheight,keepaspectratio]{#1}
    }
    {
        \ifthenelse{\lengthtest{\pictureboxheight>.80\textheight}}
        {
            \includegraphics[width=.95\linewidth,height=.80\textheight,keepaspectratio]{#1}
            
        }
        {
            \includegraphics{#1}
        }
    }
}
\newlength{\thislabelwidth}
\DeclareMathOperator{\abs}{abs}
\usepackage{animate} % This package is required because the wxMaxima configuration option
                      % "Export animations to TeX" was enabled when this file was generated.

\definecolor{labelcolor}{RGB}{100,0,0}

\usepackage{fullpage}
\usepackage{amssymb}
\usepackage{enumerate}
\usepackage[bookmarks=false,pdfstartview={FitH},colorlinks=true,urlcolor=blue]{hyperref}
\usepackage{bookmark}
\usepackage{mathtools}

\begin{document}

\pagebreak{}
{\Huge {\sc Stokes' Theorem}}
\setcounter{section}{0}
\setcounter{subsection}{0}
\setcounter{figure}{0}


\hypersetup{pdfauthor={Daniel Volinski},
            pdftitle={Stokes' Theorem},
            pdfsubject={Vector Calculus},
            pdfkeywords={Mathemation}}

Based on Mathemation Video
\href{https://www.youtube.com/watch?v=8POUN2hc5E0}
{Stokes' Theorem - Examples I}

Written by Daniel Volinski at \href{mailto:danielvolinski@yahoo.es}{danielvolinski@yahoo.es}



\noindent
%%%%%%%%%%%%%%%
%%% INPUT:
\begin{minipage}[t]{8ex}\color{red}\bf
(\%{}i2) 
\end{minipage}
\begin{minipage}[t]{\textwidth}\color{blue}\tt
info:build\_info()\$info\ensuremath{@}version;
\end{minipage}
%%% OUTPUT:
\[\displaystyle
\tag{\%{}o2}\label{o2} 
\mbox{}
\]5.38.1



\noindent
%%%%%%%%%%%%%%%
%%% INPUT:
\begin{minipage}[t]{8ex}\color{red}\bf
(\%{}i2) 
\end{minipage}
\begin{minipage}[t]{\textwidth}\color{blue}\tt
reset()\$kill(all)\$
\end{minipage}


\noindent
%%%%%%%%%%%%%%%
%%% INPUT:
\begin{minipage}[t]{8ex}\color{red}\bf
(\%{}i1) 
\end{minipage}
\begin{minipage}[t]{\textwidth}\color{blue}\tt
derivabbrev:true\$
\end{minipage}


\noindent
%%%%%%%%%%%%%%%
%%% INPUT:
\begin{minipage}[t]{8ex}\color{red}\bf
(\%{}i2) 
\end{minipage}
\begin{minipage}[t]{\textwidth}\color{blue}\tt
ratprint:false\$
\end{minipage}


\noindent
%%%%%%%%%%%%%%%
%%% INPUT:
\begin{minipage}[t]{8ex}\color{red}\bf
(\%{}i3) 
\end{minipage}
\begin{minipage}[t]{\textwidth}\color{blue}\tt
fpprintprec:5\$
\end{minipage}


\noindent
%%%%%%%%%%%%%%%
%%% INPUT:
\begin{minipage}[t]{8ex}\color{red}\bf
(\%{}i4) 
\end{minipage}
\begin{minipage}[t]{\textwidth}\color{blue}\tt
load(linearalgebra)\$
\end{minipage}


\noindent
%%%%%%%%%%%%%%%
%%% INPUT:
\begin{minipage}[t]{8ex}\color{red}\bf
(\%{}i5) 
\end{minipage}
\begin{minipage}[t]{\textwidth}\color{blue}\tt
if get('draw,'version)=false then load(draw)\$
\end{minipage}
%%% OUTPUT:
%%%%%%%%%%%%%%%


\noindent
%%%%%%%%%%%%%%%
%%% INPUT:
\begin{minipage}[t]{8ex}\color{red}\bf
(\%{}i6) 
\end{minipage}
\begin{minipage}[t]{\textwidth}\color{blue}\tt
wxplot\_size:[1024,768]\$
\end{minipage}


\noindent
%%%%%%%%%%%%%%%
%%% INPUT:
\begin{minipage}[t]{8ex}\color{red}\bf
(\%{}i7) 
\end{minipage}
\begin{minipage}[t]{\textwidth}\color{blue}\tt
if get('drawdf,'version)=false then load(drawdf)\$
\end{minipage}


\noindent
%%%%%%%%%%%%%%%
%%% INPUT:
\begin{minipage}[t]{8ex}\color{red}\bf
(\%{}i8) 
\end{minipage}
\begin{minipage}[t]{\textwidth}\color{blue}\tt
set\_draw\_defaults(xtics=1,ytics=1,ztics=1,xyplane=0,nticks=100,\\
                  xaxis=true,xaxis\_type=dots,xaxis\_width=3,\\
                  yaxis=true,yaxis\_type=dots,yaxis\_width=3,\\
                  zaxis=true,zaxis\_type=dots,zaxis\_width=3,\\
                  background\_color=light\_gray)\$
\end{minipage}


\noindent
%%%%%%%%%%%%%%%
%%% INPUT:
\begin{minipage}[t]{8ex}\color{red}\bf
(\%{}i9) 
\end{minipage}
\begin{minipage}[t]{\textwidth}\color{blue}\tt
if get('vect,'version)=false then load(vect)\$
\end{minipage}


\noindent
%%%%%%%%%%%%%%%
%%% INPUT:
\begin{minipage}[t]{8ex}\color{red}\bf
(\%{}i10) 
\end{minipage}
\begin{minipage}[t]{\textwidth}\color{blue}\tt
norm(u):=block(ratsimp(radcan(\ensuremath{\sqrt{}}(u.u))))\$
\end{minipage}


\noindent
%%%%%%%%%%%%%%%
%%% INPUT:
\begin{minipage}[t]{8ex}\color{red}\bf
(\%{}i11) 
\end{minipage}
\begin{minipage}[t]{\textwidth}\color{blue}\tt
normalize(v):=block(v/norm(v))\$
\end{minipage}


\noindent
%%%%%%%%%%%%%%%
%%% INPUT:
\begin{minipage}[t]{8ex}\color{red}\bf
(\%{}i12) 
\end{minipage}
\begin{minipage}[t]{\textwidth}\color{blue}\tt
angle(u,v):=block([junk:radcan(\ensuremath{\sqrt{}}((u.u)*(v.v)))],acos(u.v/junk))\$
\end{minipage}


\noindent
%%%%%%%%%%%%%%%
%%% INPUT:
\begin{minipage}[t]{8ex}\color{red}\bf
(\%{}i13) 
\end{minipage}
\begin{minipage}[t]{\textwidth}\color{blue}\tt
mycross(va,vb):=[va[2]*vb[3]-va[3]*vb[2],va[3]*vb[1]-va[1]*vb[3],va[1]*vb[2]-va[2]*vb[1]]\$
\end{minipage}


\noindent
%%%%%%%%%%%%%%%
%%% INPUT:
\begin{minipage}[t]{8ex}\color{red}\bf
(\%{}i14) 
\end{minipage}
\begin{minipage}[t]{\textwidth}\color{blue}\tt
if get('cartan,'version)=false then load(cartan)\$
\end{minipage}


\noindent
%%%%%%%%%%%%%%%
%%% INPUT:
\begin{minipage}[t]{8ex}\color{red}\bf
(\%{}i15) 
\end{minipage}
\begin{minipage}[t]{\textwidth}\color{blue}\tt
declare(trigsimp,evfun)\$
\end{minipage}

\textbf{Stokes' Theorem}
$$\oint_C\vec{F}\cdot\mathrm{d}\vec{r}=\iint_S(\nabla\times
\vec{F})\cdot\mathrm{d}\vec{s}$$

\pagebreak


Let $\vec{F}(x,y,z)=\langle{e^z,x\,y\,z,x^3}\rangle$ and let
$C$ be the path of straight line segments shown down below.
Evaluate $\int_C\vec{F}\cdot\mathrm{d}\vec{r}$.


\textbf{Define the space} $\mathbb{R}^3$



\noindent
%%%%%%%%%%%%%%%
%%% INPUT:
\begin{minipage}[t]{8ex}\color{red}\bf
(\%{}i16) 
\end{minipage}
\begin{minipage}[t]{\textwidth}\color{blue}\tt
\ensuremath{\zeta}:[x,y,z]\$
\end{minipage}


\noindent
%%%%%%%%%%%%%%%
%%% INPUT:
\begin{minipage}[t]{8ex}\color{red}\bf
(\%{}i17) 
\end{minipage}
\begin{minipage}[t]{\textwidth}\color{blue}\tt
dim:length(\ensuremath{\zeta})\$
\end{minipage}


\noindent
%%%%%%%%%%%%%%%
%%% INPUT:
\begin{minipage}[t]{8ex}\color{red}\bf
(\%{}i18) 
\end{minipage}
\begin{minipage}[t]{\textwidth}\color{blue}\tt
scalefactors(\ensuremath{\zeta})\$
\end{minipage}


\noindent
%%%%%%%%%%%%%%%
%%% INPUT:
\begin{minipage}[t]{8ex}\color{red}\bf
(\%{}i19) 
\end{minipage}
\begin{minipage}[t]{\textwidth}\color{blue}\tt
init\_cartan(\ensuremath{\zeta})\$
\end{minipage}

\textbf{Vector field} $\vec{F}\in\mathbb{R}^3$



\noindent
%%%%%%%%%%%%%%%
%%% INPUT:
\begin{minipage}[t]{8ex}\color{red}\bf
(\%{}i20) 
\end{minipage}
\begin{minipage}[t]{\textwidth}\color{blue}\tt
ldisplay(F:[0,x*z,-x*y])\$
\end{minipage}
%%% OUTPUT:
\[\displaystyle
\tag{\%{}t20}\label{t20} 
F=[0,xz,-xy]\mbox{}
\]
%%%%%%%%%%%%%%%


\noindent
%%%%%%%%%%%%%%%
%%% INPUT:
\begin{minipage}[t]{8ex}\color{red}\bf
(\%{}i21) 
\end{minipage}
\begin{minipage}[t]{\textwidth}\color{blue}\tt
ldisplay(F:[exp(z),x*y*z,x\ensuremath{^3}])\$
\end{minipage}
%%% OUTPUT:
\[\displaystyle
\tag{\%{}t21}\label{t21} 
F=[{{e}^{z}},xyz,{{x}^{3}}]\mbox{}
\]
%%%%%%%%%%%%%%%

\textbf{3D Direction field}



\noindent
%%%%%%%%%%%%%%%
%%% INPUT:
\begin{minipage}[t]{8ex}\color{red}\bf
(\%{}i23) 
\end{minipage}
\begin{minipage}[t]{\textwidth}\color{blue}\tt
/* vector origins are {(x,y,z)| x,y=1,...,5}  */\\
coord:setify(makelist(k,k,0,4))\$\\
points3d:listify(cartesian\_product(coord,coord,coord))\$
\end{minipage}


\noindent
%%%%%%%%%%%%%%%
%%% INPUT:
\begin{minipage}[t]{8ex}\color{red}\bf
(\%{}i25) 
\end{minipage}
\begin{minipage}[t]{\textwidth}\color{blue}\tt
/* compute vectors at the given points  */\\
define(vf3d(x,y,z),vector(\ensuremath{\zeta},F))\$\\
vect3:makelist(vf3d(k[1],k[2],k[3]),k,points3d)\$
\end{minipage}


\noindent
%%%%%%%%%%%%%%%
%%% INPUT:
\begin{minipage}[t]{8ex}\color{red}\bf
(\%{}i26) 
\end{minipage}
\begin{minipage}[t]{\textwidth}\color{blue}\tt
wxdraw3d([head\_length=0.1,color=blue,head\_angle=25,unit\_vectors=true],vect3)\$
\end{minipage}
%%% OUTPUT:
\[\displaystyle
\tag{\%{}t26}\label{t26} 
\includegraphics[width=.95\linewidth,height=.80\textheight,keepaspectratio]{Stokes’ Theorem_img/Stokes’ Theorem_1}\mbox{}
\]
%%%%%%%%%%%%%%%
\pagebreak


\textbf{Calculate} $\nabla\times\vec{F}\in\mathbb{R}^3$



\noindent
%%%%%%%%%%%%%%%
%%% INPUT:
\begin{minipage}[t]{8ex}\color{red}\bf
(\%{}i27) 
\end{minipage}
\begin{minipage}[t]{\textwidth}\color{blue}\tt
ldisplay(curlF:ev(express(curl(F)),diff))\$
\end{minipage}
%%% OUTPUT:
\[\displaystyle
\tag{\%{}t27}\label{t27} 
\mathit{curlF}=[-xy,{{e}^{z}}-3{{x}^{2}},yz]\mbox{}
\]
%%%%%%%%%%%%%%%

\textbf{Calculate} $\nabla\cdot\vec{F}\in\mathbb{R}$



\noindent
%%%%%%%%%%%%%%%
%%% INPUT:
\begin{minipage}[t]{8ex}\color{red}\bf
(\%{}i28) 
\end{minipage}
\begin{minipage}[t]{\textwidth}\color{blue}\tt
ldisplay(divF:ev(express(div(F)),diff))\$
\end{minipage}
%%% OUTPUT:
\[\displaystyle
\tag{\%{}t28}\label{t28} 
\mathit{divF}=xz\mbox{}
\]
%%%%%%%%%%%%%%%

\textbf{Work form} $\vec{F}^\flat=\alpha\in\mathcal{A}^1(\mathbb{R}^3)$



\noindent
%%%%%%%%%%%%%%%
%%% INPUT:
\begin{minipage}[t]{8ex}\color{red}\bf
(\%{}i29) 
\end{minipage}
\begin{minipage}[t]{\textwidth}\color{blue}\tt
ldisplay(\ensuremath{\alpha}:edit(F.cartan\_basis))\$
\end{minipage}
%%% OUTPUT:
\[\displaystyle
\tag{\%{}t29}\label{t29} 
\mathit{\ensuremath{\alpha}}={{x}^{3}}\,\mathit{dz}+xyz\,\mathit{dy}+{{e}^{z}}\,\mathit{dx}\mbox{}
\]
%%%%%%%%%%%%%%%

\textbf{Calculate} $\mathrm{d}\alpha\in\mathcal{A}^2(\mathbb{R}^3)$



\noindent
%%%%%%%%%%%%%%%
%%% INPUT:
\begin{minipage}[t]{8ex}\color{red}\bf
(\%{}i30) 
\end{minipage}
\begin{minipage}[t]{\textwidth}\color{blue}\tt
ldisplay(d\ensuremath{\alpha}:edit(ext\_diff(\ensuremath{\alpha})))\$
\end{minipage}
%%% OUTPUT:
\[\displaystyle
\tag{\%{}t30}\label{t30} 
\mathit{d\ensuremath{\alpha}}=-xy\,\mathit{dy}\,\mathit{dz}+\left( 3{{x}^{2}}-{{e}^{z}}\right) \,\mathit{dx}\,\mathit{dz}+yz\,\mathit{dx}\,\mathit{dy}\mbox{}
\]
%%%%%%%%%%%%%%%

\textbf{Flux form} $\beta\in\mathcal{A}^2(\mathbb{R}^3)$



\noindent
%%%%%%%%%%%%%%%
%%% INPUT:
\begin{minipage}[t]{8ex}\color{red}\bf
(\%{}i31) 
\end{minipage}
\begin{minipage}[t]{\textwidth}\color{blue}\tt
ldisplay(\ensuremath{\beta}:F[1]*cartan\_basis[2]\ensuremath{\sim }cartan\_basis[3]+\\
           F[2]*cartan\_basis[3]\ensuremath{\sim }cartan\_basis[1]+\\
           F[3]*cartan\_basis[1]\ensuremath{\sim }cartan\_basis[2])\$
\end{minipage}
%%% OUTPUT:
\[\displaystyle
\tag{\%{}t31}\label{t31} 
\mathit{\ensuremath{\beta}}={{e}^{z}}\,\mathit{dy}\,\mathit{dz}-xyz\,\mathit{dx}\,\mathit{dz}+{{x}^{3}}\,\mathit{dx}\,\mathit{dy}\mbox{}
\]
%%%%%%%%%%%%%%%


\noindent
%%%%%%%%%%%%%%%
%%% INPUT:
\begin{minipage}[t]{8ex}\color{red}\bf
(\%{}i32) 
\end{minipage}
\begin{minipage}[t]{\textwidth}\color{blue}\tt
ldisplay(\ensuremath{\omega}:factor(edit(\ensuremath{\beta}\ensuremath{\sim }dx+\ensuremath{\beta}\ensuremath{\sim }dy+\ensuremath{\beta}\ensuremath{\sim }dz)))\$
\end{minipage}
%%% OUTPUT:
\[\displaystyle
\tag{\%{}t32}\label{t32} 
\mathit{\ensuremath{\omega}}=\left( {{e}^{z}}+xyz+{{x}^{3}}\right) \,\mathit{dx}\,\mathit{dy}\,\mathit{dz}\mbox{}
\]
%%%%%%%%%%%%%%%

\textbf{Calculate} $\mathrm{d}\beta\in\mathcal{A}^3(\mathbb{R}^3)$



\noindent
%%%%%%%%%%%%%%%
%%% INPUT:
\begin{minipage}[t]{8ex}\color{red}\bf
(\%{}i33) 
\end{minipage}
\begin{minipage}[t]{\textwidth}\color{blue}\tt
ldisplay(d\ensuremath{\beta}:edit(ext\_diff(\ensuremath{\beta})))\$
\end{minipage}
%%% OUTPUT:
\[\displaystyle
\tag{\%{}t33}\label{t33} 
\mathit{d\ensuremath{\beta}}=xz\,\mathit{dx}\,\mathit{dy}\,\mathit{dz}\mbox{}
\]
%%%%%%%%%%%%%%%


\noindent
%%%%%%%%%%%%%%%
%%% INPUT:
\begin{minipage}[t]{8ex}\color{red}\bf
(\%{}i34) 
\end{minipage}
\begin{minipage}[t]{\textwidth}\color{blue}\tt
diff(\ensuremath{\zeta},z)|(diff(\ensuremath{\zeta},y)|(diff(\ensuremath{\zeta},x)|d\ensuremath{\beta}));
\end{minipage}
%%% OUTPUT:
\[\displaystyle
\tag{\%{}o34}\label{o34} 
xz\mbox{}
\]
%%%%%%%%%%%%%%%
\pagebreak


\textbf{End Points}



\noindent
%%%%%%%%%%%%%%%
%%% INPUT:
\begin{minipage}[t]{8ex}\color{red}\bf
(\%{}i38) 
\end{minipage}
\begin{minipage}[t]{\textwidth}\color{blue}\tt
A:[1,0,0]\$B:[1,2,0]\$P:[0,2,1]\$Q:[0,0,1]\$
\end{minipage}

\textbf{Trajectories and their derivatives}



\noindent
%%%%%%%%%%%%%%%
%%% INPUT:
\begin{minipage}[t]{8ex}\color{red}\bf
(\%{}i46) 
\end{minipage}
\begin{minipage}[t]{\textwidth}\color{blue}\tt
C\_1:A*(1-t)+B*t\$C\ensuremath{\backslash}'\_1:diff(C\_1,t)\$\\
C\_2:B*(1-t)+P*t\$C\ensuremath{\backslash}'\_2:diff(C\_2,t)\$\\
C\_3:P*(1-t)+Q*t\$C\ensuremath{\backslash}'\_3:diff(C\_3,t)\$\\
C\_4:Q*(1-t)+A*t\$C\ensuremath{\backslash}'\_4:diff(C\_4,t)\$
\end{minipage}

\textbf{Line integrals according to Vector Calculus}



\noindent
%%%%%%%%%%%%%%%
%%% INPUT:
\begin{minipage}[t]{8ex}\color{red}\bf
(\%{}i50) 
\end{minipage}
\begin{minipage}[t]{\textwidth}\color{blue}\tt
I\_1:'integrate(ev(F,map("=",\ensuremath{\zeta},C\_1)).C\ensuremath{\backslash}'\_1,t,0,1)\$\\
I\_2:'integrate(ev(F,map("=",\ensuremath{\zeta},C\_2)).C\ensuremath{\backslash}'\_2,t,0,1)\$\\
I\_3:'integrate(ev(F,map("=",\ensuremath{\zeta},C\_3)).C\ensuremath{\backslash}'\_3,t,0,1)\$\\
I\_4:'integrate(ev(F,map("=",\ensuremath{\zeta},C\_4)).C\ensuremath{\backslash}'\_4,t,0,1)\$
\end{minipage}

\textbf{Total line integral according to Vector Calculus}



\noindent
%%%%%%%%%%%%%%%
%%% INPUT:
\begin{minipage}[t]{8ex}\color{red}\bf
(\%{}i51) 
\end{minipage}
\begin{minipage}[t]{\textwidth}\color{blue}\tt
ldisplay(I\_1+I\_2+I\_3+I\_4=box(ev(I\_1+I\_2+I\_3+I\_4,integrate)))\$
\end{minipage}
%%% OUTPUT:
\[\displaystyle
\tag{\%{}t51}\label{t51} 
\int_{0}^{1}{\left. {{\left( 1-t\right) }^{3}}-{{e}^{t}}dt\right.}+\int_{0}^{1}{\left. {{e}^{1-t}}-{{t}^{3}}dt\right.}=(0)\mbox{}
\]
%%%%%%%%%%%%%%%

\textbf{Line integrals according to Differential Forms}



\noindent
%%%%%%%%%%%%%%%
%%% INPUT:
\begin{minipage}[t]{8ex}\color{red}\bf
(\%{}i55) 
\end{minipage}
\begin{minipage}[t]{\textwidth}\color{blue}\tt
I\_1:'integrate(C\ensuremath{\backslash}'\_1|ev(\ensuremath{\alpha},map("=",\ensuremath{\zeta},C\_1)),t,0,1)\$\\
I\_2:'integrate(C\ensuremath{\backslash}'\_2|ev(\ensuremath{\alpha},map("=",\ensuremath{\zeta},C\_2)),t,0,1)\$\\
I\_3:'integrate(C\ensuremath{\backslash}'\_3|ev(\ensuremath{\alpha},map("=",\ensuremath{\zeta},C\_3)),t,0,1)\$\\
I\_4:'integrate(C\ensuremath{\backslash}'\_4|ev(\ensuremath{\alpha},map("=",\ensuremath{\zeta},C\_4)),t,0,1)\$
\end{minipage}

\textbf{Total line integral according to Differential Forms}



\noindent
%%%%%%%%%%%%%%%
%%% INPUT:
\begin{minipage}[t]{8ex}\color{red}\bf
(\%{}i56) 
\end{minipage}
\begin{minipage}[t]{\textwidth}\color{blue}\tt
ldisplay(I\_1+I\_2+I\_3+I\_4=box(ev(I\_1+I\_2+I\_3+I\_4,integrate)))\$
\end{minipage}
%%% OUTPUT:
\[\displaystyle
\tag{\%{}t56}\label{t56} 
\int_{0}^{1}{\left. -{{e}^{t}}-{{t}^{3}}+3{{t}^{2}}-3t+1dt\right.}+\int_{0}^{1}{\left. {{e}^{1-t}}-{{t}^{3}}dt\right.}=(0)\mbox{}
\]
%%%%%%%%%%%%%%%
\pagebreak


\textbf{Surface} $\vec{S}\in\mathbb{R}^3$



\noindent
%%%%%%%%%%%%%%%
%%% INPUT:
\begin{minipage}[t]{8ex}\color{red}\bf
(\%{}i57) 
\end{minipage}
\begin{minipage}[t]{\textwidth}\color{blue}\tt
S:[x,y,1-x]\$
\end{minipage}

\textbf{Normal} $\vec{N}\in\mathbb{R}^3$



\noindent
%%%%%%%%%%%%%%%
%%% INPUT:
\begin{minipage}[t]{8ex}\color{red}\bf
(\%{}i58) 
\end{minipage}
\begin{minipage}[t]{\textwidth}\color{blue}\tt
ldisplay(N:mycross(diff(S,x),diff(S,y)))\$
\end{minipage}
%%% OUTPUT:
\[\displaystyle
\tag{\%{}t58}\label{t58} 
N=[1,0,1]\mbox{}
\]
%%%%%%%%%%%%%%%

\textbf{Calculate} $(\nabla\times\vec{F})\circ\vec{S}$



\noindent
%%%%%%%%%%%%%%%
%%% INPUT:
\begin{minipage}[t]{8ex}\color{red}\bf
(\%{}i59) 
\end{minipage}
\begin{minipage}[t]{\textwidth}\color{blue}\tt
ldisplay(curlFoS:subst(map("=",\ensuremath{\zeta},S),curlF))\$
\end{minipage}
%%% OUTPUT:
\[\displaystyle
\tag{\%{}t59}\label{t59} 
\mathit{curlFoS}=[-xy,{{e}^{1-x}}-3{{x}^{2}},\left( 1-x\right) y]\mbox{}
\]
%%%%%%%%%%%%%%%

\textbf{Integrand according to Vector Calculus}



\noindent
%%%%%%%%%%%%%%%
%%% INPUT:
\begin{minipage}[t]{8ex}\color{red}\bf
(\%{}i60) 
\end{minipage}
\begin{minipage}[t]{\textwidth}\color{blue}\tt
ldisplay(integrand:expand(curlFoS.N))\$
\end{minipage}
%%% OUTPUT:
\[\displaystyle
\tag{\%{}t60}\label{t60} 
\mathit{integrand}=y-2xy\mbox{}
\]
%%%%%%%%%%%%%%%

\textbf{Integrand according to Differential Forms}



\noindent
%%%%%%%%%%%%%%%
%%% INPUT:
\begin{minipage}[t]{8ex}\color{red}\bf
(\%{}i61) 
\end{minipage}
\begin{minipage}[t]{\textwidth}\color{blue}\tt
ldisplay(integrand:diff(S,y)|(diff(S,x)|ev(d\ensuremath{\alpha},map("=",\ensuremath{\zeta},S))))\$
\end{minipage}
%%% OUTPUT:
\[\displaystyle
\tag{\%{}t61}\label{t61} 
\mathit{integrand}=y-2xy\mbox{}
\]
%%%%%%%%%%%%%%%

\textbf{Surface integral}



\noindent
%%%%%%%%%%%%%%%
%%% INPUT:
\begin{minipage}[t]{8ex}\color{red}\bf
(\%{}i62) 
\end{minipage}
\begin{minipage}[t]{\textwidth}\color{blue}\tt
I:'integrate('integrate(integrand,y,0,2),x,0,1)\$
\end{minipage}


\noindent
%%%%%%%%%%%%%%%
%%% INPUT:
\begin{minipage}[t]{8ex}\color{red}\bf
(\%{}i63) 
\end{minipage}
\begin{minipage}[t]{\textwidth}\color{blue}\tt
ldisplay(I=box(ev(I,integrate)))\$
\end{minipage}
%%% OUTPUT:
\[\displaystyle
\tag{\%{}t63}\label{t63} 
\int_{0}^{1}{\left. \int_{0}^{2}{\left. y-2xydy\right.}dx\right.}=(0)\mbox{}
\]
%%%%%%%%%%%%%%%
\pagebreak


\textbf{Graphics}



\noindent
%%%%%%%%%%%%%%%
%%% INPUT:
\begin{minipage}[t]{8ex}\color{red}\bf
(\%{}i64) 
\end{minipage}
\begin{minipage}[t]{\textwidth}\color{blue}\tt
wxdraw3d(nticks=100,proportional\_axes=xyz,line\_width=3,\\
         font="Courier-Bold",font\_size=20,view=[65,140],\\
         color=cyan,\\
         apply(parametric\_surface,append(S,[x,0,1,y,0,2])),\\
         color=blue,\\
         apply(parametric,append(C\_1,[t,0,1])),\\
         apply(parametric,append(C\_2,[t,0,1])),\\
         color=red,\\
         apply(parametric,append(C\_3,[t,0,1])),\\
         apply(parametric,append(C\_4,[t,0,1])),\\
         color=black,\\
         apply(label,[append(["A"],A)]),\\
         apply(label,[append(["B"],B)]),\\
         apply(label,[append(["P"],P)]),\\
         apply(label,[append(["Q"],Q)]))\$
\end{minipage}
%%% OUTPUT:
\[\displaystyle
\tag{\%{}t64}\label{t64} 
\includegraphics[width=.95\linewidth,height=.80\textheight,keepaspectratio]{Stokes’ Theorem_img/Stokes’ Theorem_2}\mbox{}
\]
%%%%%%%%%%%%%%%
\end{document}
